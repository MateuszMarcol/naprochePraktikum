\documentclass{article}

\usepackage[utf8]{inputenc}
\usepackage[english]{babel}
\usepackage[foundations]{../lib/tex/naproche}


\title{Sum of an Arithmetic Series}
\author{}
\date{}

\begin{document}
\maketitle

\begin{forthel}
    [prove off][check off]

    [readtex \path{100_Theorems.ftl.tex}]

    [prove on][check on]
  \end{forthel}

\section{* The Number of Subsets of a Set}

\begin{forthel}
\begin{definition}
Let $X$ be a finite set.
$|X|$ is a natural number.
\end{definition}

\begin{axiom}
Let $X$ be a finite set. Let $n$ be a natural number.
If there exists a function $f$ such that $f$ is a bijection between $X$ and $\Seq{1}{n}$ then $|X|=n$.
\end{axiom}

\begin{definition}
Let $X$ be a finite set.
The number of elements in $X$ is $|X|$.
\end{definition}

%\begin{definition}
%Let $Y$ be a set such that $Y$ is nonempty. Let $y$ be an element of $Y$.
%$\{y\}_Y$ is $\{x \in Y \mid x = y\}$.
%\end{definition}
%\begin{lemma}
%Let $Y$ be a nonempty set. Let $y$ be an element of $Y$.
%Then $\{y\}_Y$ is a set.
%\end{lemma}

\begin{theorem}
Let $X$ be a set such that $X$ is finite. Let $n$ be a natural number such that $|X|=n$.
Then the number of elements in the powerset of $X$ is $2^{n}$.
\end{theorem}
%\begin{proof}[by induction on $n$]
%$n$ is a natural number.
%Case $n = 0$. $X$ is the empty set. The number of elements in the powerset of $X$ is $1$. $2^{0} = 1$.
%Take a natural number $m$ such that $m + 1 = n$. $m$ is inductively smaller than$n$. 
%Let $Y$ be a set such that the number of elements in $Y$ is equal to $n$.
%Assume $y$ is a value of $Y$. Then $\{y\}_Y$ is a subset of $Y$. Let $X$ be equal to $(Y\setminus \{y\}_Y)$.
%The number of elements in $X$ is equal to $m$. Thus the number of elements in the powerset of $X$ is equal to $2^{m}$.
%$\pow(X)$ is a nonempty set.
%Let $x$ be an element of $\pow(X)$. Then $x \cup \{y\}_Y$ is an element of $\pow(Y)$.
%Let $M$ be a set. %Assume forall $z \in M$ exists an $x \in \pow(X)$ such that $z = x \cup \{y\}_Y$.%is equal to $\{(x \cup \{y\}_Y) \in \pow(Y) \mid x \in \pow(X)\}$. 
%The number of elements in $M$ is equal to the number of elements in the powerset of $X$.
%$X$ and $M$ are disjoint.
%The powerset of $Y$ is the union of the powerset of $X$ and $M$.
%Thus the number of elements in $Y$ is equal to $2^{m} + 2^{m}$. $2^{m} + 2^{m}= 2 \cdot 2^{m} = 2^{m+1} = 2^{n}$. end.
%\end{proof}
\end{forthel}


\end{document}