\documentclass{article}

\usepackage[utf8]{inputenc}
\usepackage[english]{babel}
\usepackage{xurl}
\usepackage{amsfonts}
\usepackage{../lib/tex/naproche}[nonumbers]

\title{Russell's Paradox}
\author{Marcel Schütz}
\date{2021}

\begin{document}
  \pagenumbering{gobble}

  \maketitle

  Russell's paradox is the assertion that every set theory that contains an
  unrestricted comprehension principle leads to contradictions.
  The following is a formalization of this fact expressed in the language of
  \textit{sets} and \textit{classes}.

  \begin{forthel}
    [readtex \path{preliminaries.ftl.tex}]

    \begin{theorem}[Russell]
      If every class is a set then we have a contradiction.
    \end{theorem}
    \begin{proof}
      Assume that every class is a set.
      Define \[ R = \class{x | \text{$x$ is a set such that $x \notin x$}}. \]
      Then $R$ is a set.
      Hence $R \in R$ iff $R \notin R$.
      Contradiction.
    \end{proof}
  \end{forthel}
\end{document}
