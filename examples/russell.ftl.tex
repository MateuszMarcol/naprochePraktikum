\documentclass{article}

\usepackage[utf8]{inputenc}
\usepackage[english]{babel}
\usepackage[nonumbers]{../lib/tex/naproche}

\title{Russell's Paradox}
\author{\Naproche formalization: Marcel Schütz}
\date{2021}

\begin{document}
  \pagenumbering{gobble}

  \maketitle

  Russell's paradox is the assertion that every set (or class) theory that
  contains an unrestricted comprehension principle leads to contradictions.
  The following is a formalization of this fact expressed in \Naproche's
  built-in language of \textit{classes} and \textit{objects}.

  \begin{forthel}
    \begin{theorem}[Russell]
      Assume that every class is an object.
      Then we have a contradiction.
    \end{theorem}
    \begin{proof}
      Define \[ R = \class{x | \text{$x$ is a class such that $x \notin x$}}. \]
      Then $R \in R$ iff $R \notin R$.
      Contradiction.
    \end{proof}
  \end{forthel}
\end{document}
