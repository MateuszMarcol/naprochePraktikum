\documentclass{article}

\usepackage[utf8]{inputenc}
\usepackage[english]{babel}
\usepackage{xurl}
\usepackage{csquotes}
\usepackage[nonumbers, settheory]{../lib/tex/naproche}

\title{Paradoxes}
\author{Marcel Schütz}
\date{2022}

\begin{document}
  \maketitle

  \noindent This is a small collection of some famous (mathematical) paradoxes.
  On mid-range hardware \Naproche needs approximately 7 minutes to check it
  (including checking the library files it depends on, which takes by far the
  largest part of the time).

  \begin{forthel}
    [readtex \path{set-theory/sections/02_ordinals.ftl.tex}]
  \end{forthel}

  \section*{The Drinker Paradox}

  The Drinker Paradox is a principle of classical predicate logic popularised by
  the logician Raymond Smullyan in his 1978 book \textit{What Is the Name of
  this Book?} which can be stated as

  \begin{quotation}
    \noindent There is someone in the pub such that, if he is drinking then
    everyone in the pub is drinking.
  \end{quotation}

  \noindent We can formalize and prove this assertion in \Naproche:

  \begin{forthel}
    \begin{signature}
      A person is a notion.
    \end{signature}

    \begin{signature}
      Let $P$ be a person.
      $P$ is drinking is an atom.
    \end{signature}

    \begin{signature}
      The pub is an object.
    \end{signature}

    \begin{signature}
      Let $P$ be a person and $A$ be an object.
      $P$ is inside $A$ is a relation.
    \end{signature}

    \begin{theorem}
      Assume that there is a person inside the pub.
      Then there is a person $P$ such that $P$ is inside the pub and if $P$ is
      drinking then every person inside the pub is drinking.
      %Then there is a person $P$ inside the pub such that if $P$ is drinking
      %then every person inside the pub is drinking.
    \end{theorem}
    \begin{proof}[by case analysis]
      Case every person inside the pub is drinking.
        Choose a person $P$ inside the pub.
        Then $P$ is drinking and every person inside the pub is drinking.
        Hence if $P$ is drinking then every person inside the pub is drinking.
      End.

      %Case there is a person inside the pub that is not drinking.
      Case there is a person that is inside the pub and not drinking.
        %Consider a person $P$ inside the pub that is not drinking.
        Consider a person $P$ that is inside the pub and not drinking.
        Then if $P$ is drinking then every person inside the pub is drinking.
      End.
    \end{proof}
  \end{forthel}


  \section*{Russell's Paradox}

  Russell's Paradox is a set-theoretical paradox discovered by Bertrand Russell
  in 1901 which shows that every set theory with an unrestricted comprehension
  principle leads to contradictions.
  Using \Naproche's ontology of classes and sets and its inbuilt axiom of class
  comprehension we can formalize this paradox as follows.

  \begin{forthel}
    \begin{theorem}
      If every class is a set then we have a contradiction.
    \end{theorem}
    \begin{proof}
      Define $R = \class{x | \text{$x$ is a set such that $x \notin x$}}$.
      Assume that every class is a set.
      Then $R$ is a set.
      Hence $R \in R$ iff $R \notin R$.
      Contradiction.
    \end{proof}
  \end{forthel}


  \section*{The Barber Paradox}

  The Barber Paradox is a puzzle illustrating Russell's Paradox, formulated by
  Bertrand Russell in 1918 with the following words:

  \begin{quotation}
    \noindent You can define the barber as \enquote{one who shaves all those,
    and those only, who do not shave themselves}.
    The question is, does the barber shave himself?
  \end{quotation}

  \noindent We can show in \Naproche that the existence of such a barber would
  be contradictory.

  \begin{forthel}
    \begin{signature}
      A barber is a person.
    \end{signature}

    \begin{signature}
      Let $B, P$ be persons.
      $B$ shaves $P$ is an atom.
    \end{signature}

    Let $P$ shaves himself stand for $P$ shaves $P$.

    \begin{theorem}
      Let $B$ be a barber such that for any person $P$ $B$ shaves $P$ iff $P$
      does not shave himself.
      Then we have a contradiction.
    \end{theorem}
    \begin{proof}
      $B$ shaves himself iff $B$ does not shave himself.
    \end{proof}
  \end{forthel}


  \section*{The Burali-Forti Paradox}

  The Burali-Forti paradox, named after Cesare Burali-Forti, demonstates that
  the assumption of the collection $\Ord$ of all all ordinal numbers being a set
  leads to contradictions.

  \begin{forthel}
    \begin{theorem}
      If $\Ord$ is a set then we have a contradiction.
    \end{theorem}
    \begin{proof}
      $\Ord$ is transitive and every element of $\Ord$ is transitive.
      Assume that $\Ord$ is a set.
      Then $\Ord$ is an ordinal.
      Hence $\Ord \in \Ord$.
      Contradiction.
    \end{proof}
  \end{forthel}
\end{document}
