\documentclass{article}

\usepackage[utf8]{inputenc}
\usepackage[english]{babel}
\usepackage{xurl}
\usepackage[nonumbers]{../lib/tex/naproche}

\title{Paradoxes}
\author{\Naproche formalization: Marcel Schütz}
\date{2022}

\begin{document}
  \maketitle

  \begin{forthel}
    [readtex \path{axioms.ftl.tex}]

    [readtex \path{macros.ftl.tex}]

    [readtex \path{vocabulary.ftl.tex}]
  \end{forthel}


  \section*{The Barber Paradox}

  \begin{forthel}
    \begin{signature}
      A person is a notion.
    \end{signature}

    Let a barber denote a person.

    \begin{signature}
      Let $B$ be a barber and $P$ be a person.
      $B$ shaves $P$ is a relation.
    \end{signature}

    Let $P$ shaves himself stand for $P$ shaves $P$.

    \begin{theorem}
      Let $B$ be a barber such that for any person $P$ $B$ shaves $P$ iff $P$
      does not shave himself.
      Then we have a contradiction.
    \end{theorem}
    \begin{proof}
      $B$ shaves himself iff $B$ does not shave himself.
    \end{proof}
  \end{forthel}


  \section*{The Drinker Paradox}

  \begin{forthel}
    \begin{signature}
      Let $P$ be a person.
      $P$ is drinking is a relation.
    \end{signature}

    \begin{signature}
      The pub is an object.
    \end{signature}

    \begin{signature}
      Let $P$ be a person and $A$ be an object.
      $P$ is inside $A$ is a relation.
    \end{signature}

    \begin{theorem}
      Assume that there is a person inside the pub.
      Then there is a person $P$ such that $P$ is inside the pub and if $P$ is
      drinking then every person inside the pub is drinking.
      %Then there is a person $P$ inside the pub such that if $P$ is drinking
      %then every person inside the pub is drinking.
    \end{theorem}
    \begin{proof}
      Case every person inside the pub is drinking.
        Choose a person $P$ inside the pub.
        Then $P$ is drinking and every person inside the pub is drinking.
        Hence if $P$ is drinking then every person inside the pub is drinking.
      End.

      %Case there is a person inside the pub that is not drinking.
      Case there is a person that is inside the pub and not drinking.
        %Consider a person $P$ inside the pub that is not drinking.
        Consider a person $P$ that is inside the pub and not drinking.
        Then if $P$ is drinking then every person inside the pub is drinking.
      End.
    \end{proof}
  \end{forthel}


  \section*{Russell's Paradox}

  \begin{forthel}
    \begin{theorem}
      Assume that every class is a set.
      Then we have a contradiction.
    \end{theorem}
    \begin{proof}
      Define $R = \class{x | \text{$x$ is a set such that $x \notin x$}}$.
      Then $R \in R$ iff $R \notin R$.
      Contradiction.
    \end{proof}
  \end{forthel}


  \section*{The Burali-Forti Paradox}

  \begin{forthel}
    \begin{definition}
      Let $x$ be a set.
      $x$ is transitive iff every element of $x$ is a set and for all $y \in x$
      and all $z \in y$ we have $z \in x$.
    \end{definition}

    \begin{definition}
      An ordinal number is a transitive set $\alpha$ such that every element of
      $\alpha$ is a transitive set.
    \end{definition}

    \begin{definition}
      Let $\alpha, \beta$ be ordinal numbers.
      $\alpha < \beta$ iff $\alpha \in \beta$.
    \end{definition}

    \begin{theorem}
      Let $\alpha$ be the collection of all ordinal numbers.
      Assume that $\alpha$ is a set.
      Then $\alpha$ is an ordinal number such that $\alpha < \alpha$.
    \end{theorem}
    \begin{proof}
      (1) $\alpha$ is transitive. \\
      Proof.
        Let $\beta \in \alpha$ and $\gamma \in \beta$.
        Then $\beta$ is an ordinal number.
        Hence $\gamma$ is a transitive set.

        Let us show that every element of $\gamma$ is a transitive set.
          Let $x \in \gamma$.
          Then $x \in \beta$.
          Indeed $\beta$ is transitive.
          Thus $x$ is a transitive set.
          Indeed every element of $\beta$ is a transitive set.
        End.

        Therefore $\gamma$ is an ordinal number.
        Consequently $\gamma \in \alpha$.
      Qed.

      (2) Every set that is contained in $\alpha$ is transitive.
      Indeed every element of $\alpha$ is an ordinal number and every ordinal
      number is transitive.

      Therefore $\alpha$ is an ordinal number.
      Consequently $\alpha < \alpha$.
    \end{proof}
  \end{forthel}
\end{document}
