\documentclass{article}

\usepackage[utf8]{inputenc}
\usepackage[english]{babel}
\usepackage[foundations]{../lib/tex/naproche}


\title{Irrationality of the Square Root of 2}
\author{}
\date{}



\begin{document}
\maketitle

\begin{forthel}
    [prove off][check off]

    [readtex \path{100_Theorems.ftl.tex}]

    [prove on][check on]
  \end{forthel}

\section{* The Irrationality of the Square Root of 2}



\begin{forthel}

Let $p$ denote a prime number.
Let $n$ denote an integer.


\begin{proposition}
    $q^{2} = p$ for no rational number $q$.
\end{proposition}
\begin{proof}[by contradiction]
    Assume the contrary.
    Take a positive rational number $q$ such that $p = q^{2}$.
    Take coprime natural numbers $m,n$ such that $m \cdot q = n$.
    Then $(m \cdot q) \cdot n = n^{2}$. 
    Then  $(m \cdot q) \cdot n = (m \cdot q) \cdot (m \cdot q) = ((m \cdot q) \cdot m) \cdot q$.
    $((m \cdot q) \cdot m) \cdot q  = (m^{2} \cdot q) \cdot q = m^{2} \cdot q^{2}$.
    $m^{2} \cdot q^{2} = n^{2}.$
    Then $p \cdot m^{2} = n^{2}$.
    $p$ divides $n^{2}$. Therefore $p$ divides $n$.
    Take a natural number $k$ such that $n = k \cdot p$.
    Then $p \cdot m^{2} = p \cdot (k \cdot n)$.
    Therefore $m \cdot m$ is equal to $p \cdot k^{2}$.
    Hence $p$ divides $m$.
    Contradiction.
\end{proof}
\end{forthel}

\end{document}