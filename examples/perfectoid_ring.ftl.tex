\documentclass{article}

\usepackage[utf8]{inputenc}
\usepackage[english]{babel}
\usepackage{amssymb, mathtools}
\usepackage{xurl}

\usepackage{../lib/tex/naproche}

\title{A Shallow Formalization of Perfectoid Rings in the
\Naproche Natural Language Proof Assistant}
\author{Peter Koepke\\
University of Bonn}

\newcommand{\card}[1]{\operatorname{card}(#1)}

\begin{document}

\maketitle

We define \textit{perfectoid rings} in \Naproche with the main intention to approximate
the language of definitions and
to the mathematical \textit{language} of Definition 3.1 in
Peter Scholze, \textit{Étale cohomology of diamonds} and of other material leading up
to that definition.
This requires to iteratively formalize
defining notions, down to undefined basic notions.
The formalization is shallow as we do not prove well-definedness or lemmas about
newly defined notions. As an example, for the definition of topologically nilpotent
we introduce the notation $x^{n,S}$ for an $n$-th power in a ring $S$, but there are
no inductive definitions or axioms about this term except that it should be an element
of $S$. Also the natural numbers $n$ carry no arithmetic structure.
Such properties would have to be axiomatized in detail for a deeper formalization.
We think, however, that the undefined notions in this text are basic enough so that an
extension to a complete formalization is possible.

As a further simplification we circumvent certain constructions like quotients of rings,
and we express a property like
\[ \Phi^{R} : R^o / \omega \cong R^o / \omega^{p,R}. \]
via congruences mod $\omega$ or mod $\omega^{p,R}$ resp.. Note that we
have to repeat the variable $R$ for a ring throughout the document since
we do not have automatic type elaboration as in Lean. This should be
provided in future versions of \Naproche.

This text is written as a file
\texttt{perfectoidring.ftl.tex}
in the \LaTeX{} dialect of \Naproche and
typeset by pdf-\LaTeX. The formal parts are printed on a grey background. Everything else
is interpreted as commentary that is not subject to the logical checking by the system.
\texttt{perfectoidring.ftl.tex}
can be checked in the \Naproche program by opening the file in Isabelle (see
\texttt{https://isabelle.in.tum.de/}).
The main burden is the ``type checking'' of statements, i.e., that the preconditions
for complex terms are satisfied. These checks can take several seconds and involve
first-order proving since we do not use a standard type theory with fast type checking
algorithms.


\section{Sets and natural numbers}

A somewhat odd collection of notations and properties for classes and sets is
built into \Naproche.
Further premises needed in the formalization are imported from:

\begin{itemize}
  \item \url{$NAPROCHE/examples/basic-notions/sets-and-functions/01_sets/01_sets.ftl.tex}
  \item \url{$NAPROCHE/examples/basic-notions/natural-numbers/03_divisibility/04_primes.ftl.tex}
\end{itemize}

\typeout{
  \begin{forthel}
    [readtex basic-notions/sets-and-functions/sections/01_sets/01_sets.ftl.tex]
    [readtex basic-notions/natural-numbers/sections/03_divisibility/04_primes.ftl.tex]
  \end{forthel}
}

\begin{forthel}
  Let $n,m$ denote natural numbers.
\end{forthel}


\section{Rings}

We work with commutative rings with 1's. Since we
do not really do arithmetic in this text, we can take the natural numbers
$0$ and $1$ as the neutral elements of all occuring rings, and
we can even assume that every natural number is a ring element.

\begin{forthel}
  \begin{signature}
    A ring is a set.
  \end{signature}

  Let $S$ denote a ring.

  \begin{axiom}
    Every natural number is an element of $S$.
  \end{axiom}

  \begin{signature}
    Let $x \in S$.
    $-^{S} x$ is an element of $S$.
  \end{signature}

  \begin{signature}
    Let $x,y \in S$.
    $x +^{S} y$ is an element of $S$.
  \end{signature}

  \begin{signature}
    Let $x,y \in S$.
    $x \cdot^{S} y$ is an element of $S$.
  \end{signature}

  \begin{axiom}
    $(x +^{S} y) +^{S} z = x +^{S} (y +^{S} z)$ for all $x,y,z \in S$.
  \end{axiom}

  \begin{axiom}
    $x +^{S} 0 = x$ for all $x \in S$.
  \end{axiom}

  \begin{axiom}
    $x +^{S} -^{S} x = 0$ for all $x \in S$.
  \end{axiom}

  \begin{axiom}
    $x +^{S} y = y +^{S} x$ for all $x,y \in S$.
  \end{axiom}

  \begin{axiom}
    $(x \cdot^{S} y) \cdot^{S} z = x \cdot^{S} (y \cdot^{S} z)$ for all $x,y,z \in S$.
  \end{axiom}

  \begin{axiom}
    $x \cdot^{S} 1 = x$ for all $x \in S$.
  \end{axiom}

  \begin{axiom}
    $x \cdot^{S} y = y \cdot^{S} x$ for all $x,y \in S$.
  \end{axiom}

  \begin{axiom}
    $(x +^{S} y) \cdot^{S} z = (x \cdot^{S} z) +^{S} (y \cdot^{S} z)$ for all $x,y,z \in S$.
  \end{axiom}
\end{forthel}

As an example we formalize a trivial lemma and its proof.

\begin{forthel}
  \begin{lemma}
    $ 0 \cdot^{S} x = 0$ for all $x \in S$.
  \end{lemma}
  \begin{proof}
    Let $x \in S$.
    $(0 \cdot^{S} x) +^{S} x = (0 \cdot^{S} x) +^{S} (1 \cdot^{S} x) = (0 +^{S} 1) \cdot^{S} x =  1 \cdot^{S} x = x $.
    $ 0 \cdot^{S} x = (0 \cdot^{S} x) +^{S} (-^{S}x +^{S} x) = 0$.
  \end{proof}

  \begin{definition}
    A subring of $S$ is a subset $T$ of $S$ such that

    $0, 1 \in T$ and

    (for all $x,y \in S$ $x +^{S} y \in S$) and

    (for all $x \in S$ $-^{S} x \in S$) and

    (for all $x,y \in S$ $x +^{S} y \in S$).
  \end{definition}

  \begin{lemma}
    $S$ is a subring of $S$.
  \end{lemma}

  \begin{definition}
    A unit in $S$ is an element $x$ of $S$ such that $x \cdot^{S} y = 1$ for some element $y$ of $S$.
  \end{definition}

  \begin{definition}
    Let $x,y \in S$.
    Let $X$ be a subset of $S$.
    $x$ divides $y$ in $X$ within $S$ iff $x \cdot^{S} z = y$ for some element $z$ of $X$.
  \end{definition}

  Let $x | y$ in $X$ within $S$ stand for $x$ divides $y$ in $X$ within $S$.

  Let $x -^{S} y$ stand for $x +^{S} -^{S} y$.

  Let $x \equiv y$ mod $a,X$ within $S$ stand for $a$ divides $x-^{S} y$ in $X$ within $S$.
\end{forthel}

It suffices to introduce powers of ring elements formally:

\begin{forthel}
  \begin{signature}
    Let $x \in S$.
    $x^{n,S}$ is an element of $S$.
  \end{signature}
\end{forthel}


\section{Topological rings}

\begin{forthel}
  \begin{signature}
    A topological ring is a ring.
  \end{signature}

  Let $R$ denote a topological ring.

  \begin{signature}
    Assume that $X$ is a subset of $R$.
    $X$ is open in $R$ is an atom.
  \end{signature}

  \begin{axiom}
    $\emptyset$, $R$ are open in $R$.
  \end{axiom}

  \begin{axiom}
    Let $X,Y$ be subsets of $R$ that are open in $R$.
    Then $X \cap Y$ is open in $R$.
  \end{axiom}

  \begin{axiom}
    Let $Z$ be a set such that every element of $Z$ is a subset of $R$ that is open in $R$.
    Then $\bigcup Z$ is open in $R$.
  \end{axiom}

  \begin{axiom}
    For every subset $C$ of $R$ that is open in $R$ there exist subsets $A,B$ of $R$ such that
    $A,B$ are open in $R$ and ($x +^{R} y \in C$ where  $x \in A$ and $y \in B$).
  \end{axiom}

  \begin{axiom}
    For every subset $C$ of $R$ that is open in $R$ there exist subsets $A,B$ of $R$ such that
    $A,B$ are open in $R$ and ($x \cdot^{R} y \in C$ where $x \in A$ and $y \in B$).
  \end{axiom}

  \begin{definition}
    Let $x \in R$.
    An open neighborhood of $x$ in $R$ is a subset $U$ of $R$ such that $U$ is open in $R$ and $x \in U$.
  \end{definition}

  \begin{definition}
    $R$ is Hausdorff iff for every $x,y \in R$ there is an open neighborhood $A$ of $x$ in $R$ such that
    there is an open neighborhood $B$ of $y$ in $R$ such that $A \cap B = \emptyset$.
  \end{definition}

  \begin{definition}
    Let $x$ be an element of $R$.
    $x$ is topologically nilpotent in $R$ iff for every open neighborhood $U$ of $0$ in $R$
    there exists a natural number $N$ such that $x^{n,R} \in U$ for all $n$ such that $n \geq N$.
  \end{definition}

  \begin{definition}
    A pseudouniformizer in $R$ is an element of $R$ that is a unit in $R$ and topologically nilpotent in $R$.
  \end{definition}
\end{forthel}


\section{Boundedness}

See Wedhorn, p.36

\begin{forthel}
  \begin{definition}
    Assume that $B$ is a subset of $R$.
    $B$ is bounded in $R$ iff for every open neighborhood $U$ of $0$ in $R$
    there exists an open neighborhood $V$ of $0$ in $R$ such that $v \cdot^{R} b \in U$ where $v \in V$ and $b \in B$.
  \end{definition}

  \begin{definition}
    A bounded subset of $R$ is a subset of $R$ that is bounded in $R$.
  \end{definition}

  \begin{lemma}
    Every subset of every bounded subset of $R$ is a bounded subset of $R$.
  \end{lemma}
\end{forthel}

The corresponding lemma and its proof in \verb+power_bounded.lean+ read:

\begin{verbatim}
lemma subset {S1 S2 : set R} (h : S1 \subset S2) (H : is_bounded S2) :
is_bounded S1 :=
begin
  intros U hU,
  rcases H U hU with <V, hV1, hV2>,
  use [V, hV1],
  intros v hv b hb,
  exact hV2 _ hv _ (h hb),
end
\end{verbatim}

\begin{forthel}
  \begin{definition}
    Let $x \in R$.
    $x^{\mathbb{N};R} = \{ x^{n,R} \mid \text{$n$ is a natural number} \}$.
  \end{definition}

  \begin{axiom}
    $x^{\mathbb{N};R}$ is a set for every $x \in R$.
  \end{axiom}

  \begin{definition}
    Let $x \in R$.
    $x$ is powerbounded in $R$ iff $x^{\mathbb{N};R}$ is bounded in $R$.
  \end{definition}
\end{forthel}



\section{Completion of topological rings, Wedhorn 5.4}

\begin{forthel}
  \begin{definition}
    A filter basis on $R$ is a nonempty set $B$ such that
    every element of $B$ is a nonempty subset of $R$ and for all elements $A1,A2$ of $B$
    there exists an element $A$ of $B$ such that  $A \subset A1 \cap A2$.
  \end{definition}

  \begin{definition}
    Let $B$ be a filter basis on $R$ and $x \in R$.
    $B$ converges to $x$ in $R$ iff for every open neighborhood $U$ of $x$ in $R$
    there exists an element $A$ of $B$ such that $A \subset U$.
  \end{definition}

  \begin{definition}
    Let $B$ be a filter basis on $R$.
    $B$ is Cauchy in $R$ iff for every open neighborhood $U$ of $0$ in $R$ there exists
    $A \in B$ such that $x +^{R} (-^{R} y) \in U$ for all $x,y \in A$.
  \end{definition}

  \begin{definition}
    $R$ is complete iff $R$ is Hausdorff and every filter basis on $R$ that is Cauchy in $R$
    converges to some $x \in R$ in $R$.
  \end{definition}
\end{forthel}


\section{Tate rings}

\begin{forthel}
  % Scholze, p.14
  \begin{definition}
    A Tate ring is a topological ring $R$ such that
    there exists a subring $R0$ of $R$ that is open in $R$ and bounded in $R$
    and there exists an element of $R$ that is a unit in $R$
    and topologically nilpotent in $R$.
  \end{definition}

  Let $R$ denote a Tate ring.
\end{forthel}


\section{Frobenius}

Normally we would use (set-theoretic) quotients by principal ideals in $R^o$
for the definition of perfectoid rings, and then Frobenius-style maps on
those quotient rings.
Instead we define a global map $\Phi^R$, $x \mapsto x^{p,R}$ and consider the
question whether it induces maps between such quotients. This
can be expressed by calculations modulo the generators of the ideals.
A complete formalization would have to prove certain well-definednesses
which we assume for simplicity.

\begin{forthel}
  \begin{definition}
    $R^o$ is the class of elements $x$ of $R$ such that $x$ is powerbounded in $R$.
  \end{definition}

  \begin{axiom}
    $R^o$ is a set and a subset of $R$.
  \end{axiom}

  \begin{signature}
    $p$ is a prime number.
  \end{signature}

  \begin{definition}
    Let $x \in R$.
    $\Phi^{R}(x) = x^{p,R}$.
  \end{definition}

  \begin{definition}
    Let $S,T \subset R$.
    Let $a \in S$ and $b \in T$.
    \[ \Phi^{R} : S / a \cong T / b \] iff

    (for every $x,y \in S$ if $x \equiv y$ mod $a,S$ within $R$ then $x^{p,R} \equiv y^{p,R}$ mod $b,T$ within $R$) and

    (for every $x,y \in S$ $ (x +^{R} y)^{p,R} \equiv x^{p,R} +^{R} y^{p,R}$ mod $b,T$ within $R$) and

    (for every $x,y \in S$ $ (x \cdot^{R} y)^{p,R} \equiv x^{p,R} \cdot^{R} y^{p,R}$ mod $b,T$ within $R$) and

    (for every $x,y \in S$ if
    $x^{p,R} \equiv y^{p,R}$ mod $b,T$ within $R$ then $x \equiv y$ mod $a,S$ within $R$) and

    (for every $z \in T$ there exists $x \in S$ such that $z \equiv x^{p,R}$ mod $b,T$ within $R$).
  \end{definition}


  \begin{definition}
    $R$ is uniform iff $R^o$ is bounded in $R$.
  \end{definition}

  \begin{axiom}
    Let $\omega$ be a pseudouniformizer in $R$.
    Then $\omega, \omega^{p,R} \in R^o$.
  \end{axiom}
\end{forthel}


\section{Perfectoid rings}

\begin{forthel}
  \begin{definition}
    $R$ is perfectoid iff $R$ is complete and uniform and there exists a pseudouniformizer $\omega$ in $R$
    such that $\omega^{p,R} | p$ in $R^o$ within $R$ and \[ \Phi^{R} : R^o / \omega \cong R^o / \omega^{p,R}. \]
  \end{definition}
\end{forthel}

\end{document}
