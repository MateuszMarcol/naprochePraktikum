\documentclass{article}

\usepackage[utf8]{inputenc}
\usepackage[english]{babel}
\usepackage{url}
\usepackage{../lib/tex/naproche}
\usepackage{../lib/tex/basicnotions}

\title{The Cantor-Schröder-Bernstein Theorem}
\author{\Naproche formalization: \vspace{0.5em} \\
Alexander Holz (2019), \\
Marcel Schütz (2019 - 2021)}
\date{}

\begin{document}
  \pagenumbering{gobble}

  \maketitle

  This is a formalization of the proof of the \textit{Cantor-Schröder-Bernstein
  Theorem}, i.e. of the fact that two sets are equipollent if they can be
  embedded into each other, based on some version of the \textit{Knaster-Tarski
  Fixed Point Theorem}.

  \begin{forthel}
    [readtex \path{set-theory/sections/02_functions/05_functions-and-set-systems.ftl.tex}]
  \end{forthel}

  \begin{forthel}
    [readtex \path{set-theory/sections/02_functions/06_equipollency.ftl.tex}]
  \end{forthel}

  \begin{forthel}
    \begin{theorem}[Cantor Schroeder Bernstein]
      Let $x,y$ be sets.
      $x$ and $y$ are equipollent iff there exists a function from $x$ into $y$ and there exists a function from $y$ into $x$.
    \end{theorem}
    \begin{proof}
      Case $x$ and $y$ are equipollent.
        Take a bijection $f$ between $x$ and $y$.
        Then $f^{-1}$ is a bijection between $y$ and $x$.
        Hence $f$ is a function from $x$ into $y$ and $f^{-1}$ is a function from $y$ into $x$.
      End.

      Case there exists a function from $x$ into $y$ and there exists a function from $y$ into $x$.
        Take a function $f$ from $x$ into $y$.
        Take a function $g$ from $y$ into $x$.
        We have $y \setminus f[a] \subseteq y$ for any $a \in \pow(x)$.

        (1) Define $h(a) = x \setminus g[y \setminus f[a]]$ for $a \in \pow(x)$.

        $h$ is a function from $\pow(x)$ to $\pow(x)$.
        Indeed $h(a)$ is a subset of $x$ for each subset $a$ of $x$.

        Let us show that $h$ preserves subsets.
          Let $a,b$ be subsets of $x$.
          Assume $a \subseteq b$.
          Then $f[a] \subseteq f[b]$.
          Hence $y \setminus f[b] \subseteq y \setminus f[a]$.
          Thus $g[y \setminus f[b]] \subseteq g[y \setminus f[a]]$ (by \ref{SetTheory_02_02_889945}).
          Indeed $y \setminus f[b]$ and $y \setminus f[a]$ are subsets of $y$.
          Therefore $x \setminus g[y \setminus f[a]] \subseteq x \setminus g[y \setminus f[b]]$.
          Consequently $h[a] \subseteq h[b]$.
        End.

        Hence we can take a fixed point $c$ of $h$.

        (2) Define $F(u) = f(u)$ for $u \in c$.

        We have $c = h(c)$ iff $x \setminus c = g[y \setminus f[c]]$.
        $g^{-1}$ is a bijection between $\range(g)$ and $y$.
        Thus $x \setminus c = g[y \setminus f[c]] \subseteq \range(g)$.
        Therefore $x \setminus c$ is a subset of $\dom(g^{-1})$.

        (3) Define $G(u) = g^{-1}(u)$ for $u \in x \setminus c$.

        $F$ is a bijection between $c$ and $\range(F)$.
        $G$ is a bijection between $x \setminus c$ and $\range(G)$.

        Define \[ H(u) =
          \begin{cases}
            F(u) & : u \in c \\
            G(u) & : u \notin c
          \end{cases} \]
        for $u \in x$.

        Let us show that $H$ is a function to $y$.
          $\dom(H)$ is a set and every value of $H$ is an object.
          Hence $H$ is a function.

          Let us show that every value of $H$ lies in $y$.
            Let $v$ be a value of $H$.
            Take $u \in x$ such that $H(u) = v$.
            If $u \in c$ then $v = H(u) = F(u) = f(u) \in y$.
            If $u \notin c$ then $v = H(u) = G(u) = g^{-1}(u) \in y$.
          End.
        End.

        Let us show that every element of $y$ is a value of $H$.
          Let $v \in y$.

          Case $v \in f[c]$.
            Take $u \in c$ such that $f(u) = v$.
            Then $F(u) = v$.
          End.

          Case $v \notin f[c]$.
            Then $v \in y \setminus f[c]$.
            Hence $g(v) \in g[y \setminus f[c]]$.
            Thus $g(v) \in x \setminus h(c)$.
            We have $g(v) \in x \setminus c$.
            Therefore we can take $u \in x \setminus c$ such that $G(u) = v$.
            Then $v = H(u)$.
          End.
        End.

        Let us show that $H$ is one to one.
          Let $u,v \in \dom(H)$.
          Assume $u \neq v$.

          Case $u,v \in c$.
            Then $H(u) = F(u)$ and $H(v) = F(v)$.
            We have $F(u) \neq F(v)$.
            Hence $H(u) \neq H(v)$.
          End.

          Case $u,v \notin c$.
            Then $H(u) = G(u)$ and $H(v) = G(v)$.
            We have $G(u) \neq G(v)$.
            Hence $H(u) \neq H(v)$.
          End.

          Case $u \in c$ and $v \notin c$.
            Then $H(u) = F(u)$ and $H(v) = G(v)$.
            Hence $v \in g[y \setminus f[c]]$.
            We have $G(v) \in y \setminus F[c]$.
            Thus $G(v) \neq F(u)$.
          End.

          Case $u \notin c$ and $v \in c$.
            Then $H(u) = G(u)$ and $H(v) = F(v)$.
            Hence $u \in g[y \setminus f[c]]$.
            We have $G(u) \in y \setminus f[c]$.
            Thus $G(u) \neq F(v)$.
          End.
        End.

        Hence $H$ is a bijection between $x$ and $y$.
      End.
    \end{proof}
  \end{forthel}
\end{document}
