\documentclass[../../set-theory.ftl.tex]{subfiles}

\begin{document}

  \section{Sets}

  \begin{forthel}
    [readtex \path{vocabulary.ftl.tex}]
  \end{forthel}

  \begin{forthel}
    [readtex \path{macros.ftl.tex}]
  \end{forthel}

  \begin{forthel}
    Let $x,y,z$ denote sets.
  \end{forthel}

  \begin{forthel}
    \begin{axiom}\label{SetTheory_01_01_729563}
      $x$ is a class.
    \end{axiom}

    \begin{axiom}\label{SetTheory_01_01_603161}
      $x$ is an element.
    \end{axiom}

    \begin{axiom}\label{SetTheory_01_01_617091}
      Let $u$ be an element of $x$.
      Then $u$ is an element.
    \end{axiom}
  \end{forthel}


  \subsection{Subsets}

  \begin{forthel}
    \begin{definition}
      A subset of $x$ is a set $y$ such that every element of $y$ is an element of $x$.
    \end{definition}

    Let $y \subseteq x$ stand for $y$ is a subset of $x$.
    Let $y \subset x$ stand for $y \subseteq x$.
    Let a superset of $x$ stand for a set $y$ such that $x \subseteq y$.
    Let $y \supseteq x$ stand for $y$ is a superset of $x$.
    Let $y \supset x$ stand for $y \supseteq x$.

    \begin{definition}
      A proper subset of $x$ is a subset of $x$ that is not equal to $x$.
    \end{definition}

    Let $y \subsetneq x$ stand for $x$ is a proper subset of $x$.
    Let a proper superset of $x$ stand for a set $y$ such that $x \subsetneq y$.
    Let $y \supsetneq x$ stand for $y$ is a proper superset of $x$.

    \begin{proposition}\label{SetTheory_01_01_375611}
      $x \subseteq x$.
    \end{proposition}

    \begin{proposition}\label{SetTheory_01_01_726162}
      If $x \subseteq y$ and $y \subseteq z$ then $x \subseteq z$.
    \end{proposition}
  \end{forthel}


  \subsection{Set extensionality}

  \begin{forthel}
    \begin{axiom}[Set extensionality]\label{SetTheory_01_01_253735}
      If $x \subseteq y$ and $y \subseteq x$ then $x = y$.
    \end{axiom}
  \end{forthel}


  \subsection{Separation}

  \begin{forthel}
    \begin{axiom}[Separation]\label{SetTheory_01_01_240572}
      Let $C$ be a class and $x$ be a set.
      Assume that every element of $C$ is contained in $x$.
      Then $C$ is a set.
    \end{axiom}
  \end{forthel}


  \subsection{Set existence}

  \begin{forthel}
    \begin{axiom}[Set existence]\label{SetTheory_01_01_559689}
      There exists a set.
    \end{axiom}
  \end{forthel}


  \subsection{The empty set}

  \begin{forthel}
    \begin{definition}
      $x$ is empty iff $x$ has no elements.
    \end{definition}

    Let $x$ is nonempty stand for $x$ is not empty.

    \begin{lemma}
      There exists an empty set.
    \end{lemma}
    \begin{proof}
      Define $C = \class{u | \text{contradiction}}$.
      Take a set $x$ (by \nameref{SetTheory_01_01_559689}).
      Then every element of $C$ is contained in $x$.
      Hence $C$ is a set (by \nameref{SetTheory_01_01_240572}).
      $C$ has no element.
      Hence the thesis.
    \end{proof}

    \begin{lemma}
      If $x$ and $y$ are empty then $x = y$.
    \end{lemma}
    \begin{proof}
      Assume that $x$ and $y$ are empty.
      Then every element of $x$ is an element of $y$ and every element of $y$ is an element of $x$.
      Hence $x \subseteq y$ and $y \subseteq x$.
      Thus $x = y$.
    \end{proof}

    \begin{definition}
      $\emptyset$ is the empty set.
    \end{definition}

    Let $\set{}$ stand for $\emptyset$.
    Let the empty set stand for $\emptyset$.

    \begin{proposition}\label{SetTheory_01_01_656396}
      $\emptyset$ is a subset of every set.
    \end{proposition}
    \begin{proof}
      Let $x$ be a set.
      Then every element of $\emptyset$ is an element of $x$.
      Indeed $\emptyset$ has no element.
      Hence $\emptyset \subseteq x$.
    \end{proof}
  \end{forthel}


  \subsection{Pairing}

  \begin{forthel}
    \begin{axiom}[Pairing]\label{SetTheory_01_01_422528}
      Let $u,v$ be elements.
      There exists a set $z$ such that $z = \class{w | \text{$w = u$ or $w = v$}}$.
    \end{axiom}

    \begin{definition}
      Let $u,v$ be elements.
      $\set{u,v}$ is the set $z$ such that $z = \class{w | \text{$w = u$ or $w = v$}}$.
    \end{definition}

    Let the unordered pair of $u$ and $v$ stand for $\set{u,v}$.

    \begin{lemma}
      Let $u$ be an element.
      There exists a set $z$ such that $z = \class{w | w = u}$.
    \end{lemma}
    \begin{proof}
      Take $z = \set{u,u}$.
      Then $z = \class{w | w = u}$.
    \end{proof}

    \begin{definition}
      Let $u$ be an element.
      $\set{u}$ is the set $z$ such that $z = \class{w | w = u}$.
    \end{definition}

    Let the singleton set of $u$ stand for $\set{u}$.

    \begin{definition}
      A singleton set is a set $x$ such that $x = \set{u}$ for some element $u$.
    \end{definition}
  \end{forthel}


  \subsection{Set-systems}

  \begin{forthel}
    \begin{definition}
      A system of sets is a set $X$ such that every element of $X$ is a set.
    \end{definition}

    Let $X, Y, Z$ denote systems of sets.

    \begin{definition}
      A system of nonempty sets is a system of sets $X$ such that every element of $X$ is nonempty.
    \end{definition}

    \begin{proposition}\label{SetTheory_01_01_261697}
      $\set{x}$ is a system of sets.
    \end{proposition}

    \begin{proposition}\label{SetTheory_01_01_176500}
      $\set{x,y}$ is a system of sets.
    \end{proposition}

    \begin{definition}
      A system of subsets of $x$ is a set $X$ such that every element of $X$ is a subset of $x$.
    \end{definition}

    \begin{proposition}\label{SetTheory_01_01_366869}
      Every system of subsets of $x$ is a system of sets.
    \end{proposition}
  \end{forthel}


  \subsection{Intersections}

  \begin{forthel}
    \begin{lemma}
      Let $x$ be a nonempty system of sets.
      Then there exists a set $z$ such that $z = \class{u | \text{$u$ is contained in every element of $x$}}$.
    \end{lemma}
    \begin{proof}
      Take an element $y$ of $x$.
      Then $y$ is a set.
      (1) Define $z = \class{u | \text{$u$ is contained in every element of $x$}}$.
      Every element of $z$ is contained in $y$.
      Hence $z$ is a set.
      Therefore the thesis (by 1).
    \end{proof}

    \begin{definition}
      Let $x$ be a nonempty system of sets.
      $\bigcap x$ is the set $z$ such that $z = \class{u | \text{$u$ is contained in every element of $x$}}$.
    \end{definition}

    Let the intersection over $x$ stand for $\bigcap x$.

    \begin{lemma}
      Let $x,y$ be sets.
      Then there exists a set $z$ such that $z = \class{u | \text{$u \in x$ and $u \in y$}}$.
    \end{lemma}
    \begin{proof}
      Take $z = \bigcap \set{x,y}$.
      Then \[ z = \class{u | \text{$u$ is contained in every element of $\set{x,y}$}}. \]
      Hence $z = \class{u | \text{$u \in x$ and $u \in y$}}$.
    \end{proof}

    \begin{definition}
      $x \cap y$ is the set $z$ such that $z = \class{u | \text{$u \in x$ and $u \in y$}}$.
    \end{definition}

    Let the intersection of $x$ and $y$ stand for $x \cap y$.

    \begin{proposition}\label{SetTheory_01_01_220491}
      $\bigcap \set{x,y} = x \cap y$.
    \end{proposition}
    \begin{proof}
      Let us show that $\bigcap \set{x,y} \subseteq x \cap y$.
        Let $u \in \bigcap \set{x,y}$.
        Then $u$ is an element of every element of $\set{x,y}$.
        Hence $u \in x$ and $u \in y$.
        Thus $u \in x \cap y$.
      End.

      Let us show that $x \cap y \subseteq \bigcap \set{x,y}$.
        Let $u \in x \cap y$.
        Then $u \in x$ and $u \in y$.
        Hence $u$ is an element of every element of $\set{x,y}$.
        Thus $u \in \bigcap \set{x,y}$.
      End.
    \end{proof}

    \begin{corollary}\label{SetTheory_01_01_485484}
      $\bigcap \set{x} = x$.
    \end{corollary}
    \begin{proof}
      $\bigcap \set{x} = \bigcap \set{x,x} = x \cap x = x$.
    \end{proof}

    \begin{proposition}\label{SetTheory_01_01_517087}
      Let $x$ be a nonempty system of sets.
      Then $y \subseteq \bigcap x$ iff $y$ is a subset of every element of $x$.
    \end{proposition}
    \begin{proof}
      Case $y \subseteq \bigcap x$.
        Let $z$ be an element of $x$.
        Let $u \in y$.
        Then $u \in \bigcap x$.
        Hence $u \in z$.
      End.

      Case $y$ is a subset of every element of $x$.
        Let $u \in y$.
        Then $u \in z$ for all sets $z$ such that $z \in x$.
        Hence $u \in \bigcap x$.
      End.
    \end{proof}

    \begin{definition}
      $x$ and $y$ are disjoint iff $x \cap y = \emptyset$.
    \end{definition}

    \begin{proposition}\label{SetTheory_01_01_300845}
      If $x$ and $y$ are disjoint then $y$ and $x$ are disjoint.
    \end{proposition}
    \begin{proof}
      Assume that $x$ and $y$ are disjoint.
      Then $x \cap y$ is empty.
      Hence there is no element $u$ such that $u \in x$ and $u \in y$.
      Thus $y \cap x$ is empty.
      Therefore $y$ and $x$ are disjoint.
    \end{proof}
  \end{forthel}


  \subsection{Unions}

  \begin{forthel}
    \begin{axiom}[Union]\label{SetTheory_01_01_709019}
      Let $x$ be a system of sets.
      Then there exists a set $z$ such that $z = \class{u | \text{$u$ is contained in some element of $x$}}$.
    \end{axiom}

    \begin{definition}
      Let $x$ be a system of sets.
      $\bigcup x$ is the set $z$ such that $z = \class{u | \text{$u$ is contained in some element of $x$}}$.
    \end{definition}

    Let the union over $x$ stand for $\bigcup x$.

    \begin{lemma}
      Let $x,y$ be sets.
      Then there exists a set $z$ such that $z = \class{u | \text{$u \in x$ or $u \in y$}}$.
    \end{lemma}
    \begin{proof}
      Take $z = \bigcup \set{x,y}$.
      Then \[ z = \class{u | \text{$u$ is contained in some element of $\set{x,y}$}}. \]
      Hence $z = \class{u | \text{$u \in x$ or $u \in y$}}$.
    \end{proof}

    \begin{definition}
      $x \cup y$ is the set $z$ such that $z = \class{w | \text{$w \in x$ or $w \in y$}}$.
    \end{definition}

    Let the union of $x$ and $y$ stand for $x \cup y$.

    \begin{proposition}\label{SetTheory_01_01_519005}
      $\bigcup \set{x,y} = x \cup y$.
    \end{proposition}
    \begin{proof}
      Let us show that $\bigcup \set{x,y} \subseteq x \cup y$.
        Let $u \in \bigcup \set{x,y}$.
        Then $u$ is an element of some element of $\set{x,y}$.
        Hence $u \in x$ or $u \in y$.
        Thus $u \in x \cup y$.
      End.

      Let us show that $x \cup y \subseteq \bigcup \set{x,y}$.
        Let $u \in x \cup y$.
        Then $u \in x$ or $u \in y$.
        Hence $u$ is an element of some element of $\set{x,y}$.
        Thus $u \in \bigcup \set{x,y}$.
      End.
    \end{proof}

    \begin{corollary}\label{SetTheory_01_01_820534}
      $\bigcup \set{x} = x$.
    \end{corollary}
    \begin{proof}
      $\bigcup \set{x} = \bigcup \set{x,x} = x \cup x = x$.
    \end{proof}

    \begin{proposition}\label{SetTheory_01_01_251673}
      Let $x$ be a system of sets.
      Then $\bigcup x \subseteq y$ iff every element of $x$ is a subset of $y$.
    \end{proposition}
    \begin{proof}
      Case $\bigcup x \subseteq y$.
        Let $z$ be an element of $x$.
        Let $u \in z$.
        Then $u$ is an element of some element of $x$.
        Hence $u \in \bigcup x$.
        Thus $u \in y$.
      End.

      Case every element of $x$ is a subset of $y$.
        Let $u \in \bigcup x$.
        Take a set $z$ such that $z \in x$ and $u \in z$.
        Then $z$ is a subset of $y$.
        Hence $u \in y$.
      End.
    \end{proof}

    \begin{proposition}\label{SetTheory_01_01_675114}
      $\bigcup \emptyset = \emptyset$.
    \end{proposition}
    \begin{proof}
      $\emptyset$ has no elements.
      Hence there is no $x \in \emptyset$ that has an element.
      Thus $\bigcup \emptyset$ is empty.
      Therefore $\bigcup \emptyset = \emptyset$.
    \end{proof}
  \end{forthel}


  \subsection{Complements}

  \begin{forthel}
    \begin{lemma}
      Let $x,y$ be sets.
      There exists a set $z$ such that $z = \class{w | \text{$w \in x$ and $w \notin y$}}$.
    \end{lemma}
    \begin{proof}
      Define $z = \class{w | \text{$w \in x$ and $w \notin y$}}$.
      Then every element of $z$ is contained in $x$.
      Hence $z$ is a set (by \nameref{SetTheory_01_01_240572}).
    \end{proof}

    \begin{definition}
      $x \setminus y$ is the set such that $x \setminus y = \class{w | \text{$w \in x$ and $w \notin y$}}$.
    \end{definition}

    Let the complement of $y$ in $x$ stand for $x \setminus y$.
  \end{forthel}


  \subsection{Computation laws}

  \begin{forthel}
    \begin{proposition}\label{SetTheory_01_01_830899}
      \[ x \cup y = y \cup x. \]
    \end{proposition}
    \begin{proof}
      Let us show that $x \cup y \subseteq y \cup x$.
        Let $u \in x \cup y$.
        Then $u \in x$ or $u \in y$.
        Hence $u \in y$ or $u \in x$.
        Thus $u \in y \cup x$.
      End.

      Let us show that $y \cup x \subseteq x \cup y$.
        Let $u \in y \cup x$.
        Then $u \in y$ or $u \in x$.
        Hence $u \in x$ or $u \in y$.
        Thus $u \in x \cup y$.
      End.
    \end{proof}


    \begin{proposition}\label{SetTheory_01_01_728823}
      \[ x \cap y = y \cap x. \]
    \end{proposition}
    \begin{proof}
      Let us show that $x \cap y \subseteq y \cap x$.
        Let $u \in x \cap y$.
        Then $u \in x$ and $u \in y$.
        Hence $u \in y$ and $u \in x$.
        Thus $u \in y \cap x$.
      End.

      Let us show that $y \cap x \subseteq x \cap y$.
        Let $u \in y \cap x$.
        Then $u \in y$ and $u \in x$.
        Hence $u \in x$ and $u \in y$.
        Thus $u \in x \cap y$.
      End.
    \end{proof}

    \begin{proposition}\label{SetTheory_01_01_665069}
      \[ ((x \cup y) \cup z) = x \cup (y \cup z). \]
      % Note again that we have to put the LHS in parentheses.
    \end{proposition}
    \begin{proof}
      Let us show that $((x \cup y) \cup z) \subseteq x \cup (y \cup z)$.
        Let $u \in (x \cup y) \cup z$.
        Then $u \in x \cup y$ or $u \in z$.
        Hence $u \in x$ or $u \in y$ or $u \in z$.
        Thus $u \in x$ or $u \in (y \cup z)$.
        Therefore $u \in x \cup (y \cup z)$.
      End.

      Let us show that $x \cup (y \cup z) \subseteq (x \cup y) \cup z$.
        Let $u \in x \cup (y \cup z)$.
        Then $u \in x$ or $u \in y \cup z$.
        Hence $u \in x$ or $u \in y$ or $u \in z$.
        Thus $u \in x \cup y$ or $u \in z$.
        Therefore $u \in (x \cup y) \cup z$.
      End.
    \end{proof}


    \begin{proposition}\label{SetTheory_01_01_368359}
      \[ ((x \cap y) \cap z) = x \cap (y \cap z). \]
      % And again parentheses are needed on the LHS.
    \end{proposition}
    \begin{proof}
      Let us show that $((x \cap y) \cap z) \subseteq x \cap (y \cap z)$.
        Let $u \in (x \cap y) \cap z$.
        Then $u \in x \cap y$ and $u \in z$.
        Hence $u \in x$ and $u \in y$ and $u \in z$.
        Thus $u \in x$ and $u \in (y \cap z)$.
        Therefore$ u \in x \cap (y \cap z)$.
      End.

      Let us show that $x \cap (y \cap z) \subseteq (x \cap y) \cap z$.
        Let $u \in x \cap (y \cap z)$.
        Then $u \in x$ and $u \in y \cap z$.
        Hence $u \in x$ and $u \in y$ and $u \in z$.
        Thus $u \in x \cap y$ and $u \in z$.
        Therefore$ u \in (x \cap y) \cap z$.
      End.
    \end{proof}

    \begin{proposition}\label{SetTheory_01_01_106755}
      \[ x \cap (y \cup z) = (x \cap y) \cup (x \cap z). \]
    \end{proposition}
    \begin{proof}
      Let us show that $x \cap (y \cup z) \subseteq (x \cap y) \cup (x \cap z)$.
        Let $u \in x \cap (y \cup z)$.
        Then $u \in x$ and $u \in y \cup z$.
        Hence $u \in x$ and ($u \in y$ or $u \in z$).
        Thus ($u \in x$ and $u \in y$) or ($u \in x$ and $u \in z$).
        Therefore $u \in x \cap y$ or $u \in x \cap z$.
        Hence $u \in (x \cap y) \cup (x \cap z)$.
      End.

      Let us show that $((x \cap y) \cup (x \cap z)) \subseteq x \cap (y \cup z)$.
        Let $u \in (x \cap y) \cup (x \cap z)$.
        Then $u \in x \cap y$ or $u \in x \cap z$.
        Hence ($u \in x$ and $u \in y$) or ($u \in x$ and $u \in z$).
        Thus $u \in x$ and ($u \in y$ or $u \in z$).
        Therefore $u \in x$ and $u \in y \cup z$.
        Hence$ u \in x \cap (y \cup z)$.
      End.
    \end{proof}

    \begin{proposition}\label{SetTheory_01_01_836290}
      \[ x \cup (y \cap z) = (x \cup y) \cap (x \cup z). \]
    \end{proposition}
    \begin{proof}
      Let us show that $x \cup (y \cap z) \subseteq (x \cup y) \cap (x \cup z)$.
        Let $u \in x \cup (y \cap z)$.
        Then $u \in x$ or $u \in y \cap z$.
        Hence $u \in x$ or ($u \in y$ and $u \in z$).
        Thus ($u \in x$ or $u \in y$) and ($u \in$ x or $u \in z$).
        Therefore $u \in x \cup y$ and $u \in x \cup z$.
        Hence $u \in (x \cup y) \cap (x \cup z)$.
      End.

      Let us show that $((x \cup y) \cap (x \cup z)) \subseteq x \cup (y \cap z)$.
        Let $u \in (x \cup y) \cap (x \cup z)$.
        Then $u \in x \cup y$ and $u \in x \cup z$.
        Hence ($u \in x$ or $u \in y$) and ($u \in x$ or $u \in z$).
        Thus $u \in x$ or ($u \in y$ and $u \in z$).
        Therefore $u \in x$ or $u \in y \cap z$.
        Hence $u \in x \cup (y \cap z)$.
      End.
    \end{proof}

    \begin{proposition}\label{SetTheory_01_01_496190}
      \[ x \cup x = x. \]
    \end{proposition}
    \begin{proof}
      $x \cup x = \class{u | \text{$u \in x$ or $u \in x$}}$.
      Hence $x \cup x = \class{u | u \in x}$.
      Thus $x \cup x = x$.
    \end{proof}


    \begin{proposition}\label{SetTheory_01_01_783425}
      \[ x \cap x = x. \]
    \end{proposition}
    \begin{proof}
      $x \cap x = \class{u | \text{$u \in x$ and $u \in x$}}$.
      Hence $x \cap x = \class{u | u \in x}$.
      Thus $x \cap x = x$.
    \end{proof}

    \begin{proposition}\label{SetTheory_01_01_339365}
      \[ x \setminus (y \cap z) = (x \setminus y) \cup (x \setminus z). \]
    \end{proposition}
    \begin{proof}
      Let us show that $x \setminus (y \cap z) \subseteq (x \setminus y) \cup (x \setminus z)$.
        Let $u \in x \setminus (y \cap z)$.
        Then $u \in x$ and $u \notin y \cap z$.
        Hence it is wrong that ($u \in y$ and $u \in z$).
        Thus $u \notin y$ or $u \notin z$.
        Therefore $u \in x$ and ($u \notin y$ or $u \notin z$).
        Then ($u \in x$ and $u \notin y$) or ($u \in x$ and $u \notin z$).
        Hence $u \in x \setminus y$ or $u \in x \setminus z$.
        Thus $u \in (x \setminus y) \cup (x \setminus z)$.
      End.

      Let us show that $((x \setminus y) \cup (x \setminus z)) \subseteq x \setminus (y \cap z)$.
        Let $u \in (x \setminus y) \cup (x \setminus z)$.
        Then $u \in x \setminus y$ or $u \in x \setminus z$.
        Hence ($u \in x$ and $u \notin y$) or ($u \in x$ and $u \notin z$).
        Thus $u \in x$ and ($u \notin y$ or $u \notin z$).
        Therefore $u \in x$ and not ($u \in y$ and $u \in z$).
        Then $u \in x$ and not $u \in y \cap z$.
        Hence $u \in x \setminus (y \cap z)$.
      End.
    \end{proof}


    \begin{proposition}\label{SetTheory_01_01_403962}
      \[ x \setminus (y \cup z) = (x \setminus y) \cap (x \setminus z). \]
    \end{proposition}
    \begin{proof}
      Let us show that $x \setminus (y \cup z) \subseteq (x \setminus y) \cap (x \setminus z)$.
        Let $u \in x \setminus (y \cup z)$.
        Then $u \in x$ and $u \notin y \cup z$.
        Hence it is wrong that ($u \in y$ or $u \in z$).
        Thus $u \notin y$ and $u \notin z$.
        Therefore $u \in x$ and ($u \notin y$ and $u \notin z$).
        Then ($u \in x$ and $u \notin y$) and ($u \in x$ and $u \notin z$).
        Hence $u \in x \setminus y$ and $u \in x \setminus z$.
        Thus $u \in (x \setminus y) \cap (x \setminus z)$.
      End.

      Let us show that $((x \setminus y) \cap (x \setminus z)) \subseteq
      x \setminus (y \cup z)$.
        Let $u \in (x \setminus y) \cap (x \setminus z)$.
        Then $u \in x \setminus y$ and $u \in x \setminus z$.
        Hence ($u \in x$ and $u \notin y$) and ($u \in x$ and $u \notin z$).
        Thus $u \in x$ and ($u \notin y$ and $u \notin z$).
        Therefore $u \in x$ and not ($u \in y$ or $u \in z$).
        Then $u \in x$ and not $u \in y \cup z$.
        Hence $u \in x \setminus (y \cup z)$.
      End.
    \end{proof}

    \begin{proposition}\label{SetTheory_01_01_628970}
      \[ x \subseteq x \cup y. \]
    \end{proposition}
    \begin{proof}
      Let $u \in x$.
      Then $u \in x$ or $u \in y$.
      Hence $u \in x \cup y$.
    \end{proof}


    \begin{proposition}\label{SetTheory_01_01_368515}
      \[ x \cap y \subseteq x. \]
    \end{proposition}
    \begin{proof}
      Let $u \in x \cap y$.
      Then $u \in x$ and $u \in y$.
      Hence $u \in x$.
    \end{proof}


    \begin{proposition}\label{SetTheory_01_01_591527}
      \[ x \subseteq y \iff x \cup y = y. \]
    \end{proposition}
    \begin{proof}
      Case $x \subseteq y$.

        Let us show that $x \cup y \subseteq y$.
          Let $u \in x \cup y$.
          Then $u \in x$ or $u \in y$.
          If $u \in x$ then $u \in y$.
          Hence $u \in y$.
        End.

        Let us show that $y \subseteq x \cup y$.
          Let $u \in y$.
          Then $u \in x$ or $u \in y$.
          Hence $u \in x \cup y$.
        End.
      End.

      Case $x \cup y = y$.
        Let $u \in x$.
        Then $u \in x$ or $u \in y$.
        Hence $u \in x \cup y = y$.
      End.
    \end{proof}


    \begin{proposition}\label{SetTheory_01_01_681535}
      \[ x \subseteq y \iff x \cap y = x. \]
    \end{proposition}
    \begin{proof}
      Case $x \subseteq y$.

        Let us show that $x \cap y \subseteq x$.
          Let $u \in x \cap y$.
          Then $u \in x$ and $u \in y$.
          Hence $u \in x$.
        End.

        Let us show that $x \subseteq x \cap y$.
          Let $u \in x$.
          Then $u \in y$.
          Hence $u \in x$ and $u \in y$.
          Thus $u \in x \cap y$.
        End.
      End.

      Case $x \cap y = x$.
        Let $u \in x$.
        Then $u \in x \cap y$.
        Hence $u \in x$ and $u \in y$.
        Thus $u \in y$.
      End.
    \end{proof}

    \begin{proposition}\label{SetTheory_01_01_402739}
      \[ x \setminus x = \emptyset. \]
    \end{proposition}
    \begin{proof}
      $x \setminus x$ has no elements.
      Indeed $x \setminus x = \class{u | \text{$u \in x$ and $u \notin x$}}$.
      Hence the thesis.
    \end{proof}


    \begin{proposition}\label{SetTheory_01_01_661163}
      \[ x \setminus \emptyset = x. \]
    \end{proposition}
    \begin{proof}
      $x \setminus \emptyset = \class{u | \text{$u \in x$ and $u \notin \emptyset$}}$.
      No element is an element of $\emptyset$.
      Hence $x \setminus \emptyset = \class{u | u \in x}$.
      Then we have the thesis.
    \end{proof}

    \begin{proposition}\label{SetTheory_01_01_408438}
      \[ x \setminus (x \setminus y) = x \cap y. \]
    \end{proposition}
    \begin{proof}
      Let us show that $x \setminus (x \setminus y) \subseteq x \cap y$.
        Let $u \in x \setminus (x \setminus y)$.
        Then $u \in x$ and $u \notin x \setminus y$.
        Hence $u \notin x$ or $u \in y$.
        Thus $u \in y$.
        Therefore $u \in x \cap y$.
      End.

      Let us show that $x \cap y \subseteq x \setminus (x \setminus y)$.
        Let $u \in x \cap y$.
        Then $u \in x$ and $u \in y$.
        Hence $u \notin x$ or $u \in y$.
        Thus $u \notin x \setminus y$.
        Therefore $u \in x \setminus (x \setminus y)$.
      End.
    \end{proof}


    \begin{proposition}\label{SetTheory_01_01_185130}
      \[ y \subseteq x \iff x \setminus (x \setminus y) = y. \]
    \end{proposition}
    \begin{proof}
      Case $y \subseteq x$. Obvious.

      Case $x \setminus (x \setminus y) = y$.
        Then every element of $y$ is an element of $x \setminus
        (x \setminus y)$.
        Thus every element of $y$ is an element of $x$.
        Then we have the thesis.
      End.
    \end{proof}


    \begin{proposition}\label{SetTheory_01_01_878796}
      \[ x \cap (y \setminus z) = (x \cap y) \setminus (x \cap z). \]
    \end{proposition}
    \begin{proof}
      Let us show that $x \cap (y \setminus z) \subseteq (x \cap y) \setminus (x \cap z)$.
        Let $u \in x \cap (y \setminus z)$.
        Then $u \in x$ and $u \in y \setminus z$.
        Hence $u \in x$ and $u \in y$.
        Thus $u \in x \cap y$ and $u \notin z$.
        Therefore $u \notin x \cap z$.
        Then we have $u \in (x \cap y) \setminus (x \cap z)$.
      End.

      Let us show that $((x \cap y) \setminus (x \cap z)) \subseteq x \cap (y \setminus z)$.
        Let $u \in (x \cap y) \setminus (x \cap z)$.
        Then $u \in x$ and $u \in y$.
        $u \notin x \cap z$.
        Hence $u \notin z$.
        Thus $u \in y \setminus z$.
        Therefore $u \in x \cap (y \setminus z)$.
      End.
    \end{proof}
  \end{forthel}
\end{document}
