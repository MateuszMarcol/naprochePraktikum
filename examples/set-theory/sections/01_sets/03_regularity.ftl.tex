\documentclass[../../set-theory.ftl.tex]{subfiles}

\begin{document}
  \begin{comment}
    \begin{forthel}
      % Uncomment for debugging:

      %[prove off][check off]
      %[readtex set-theory/sections/01_sets/01_sets.ftl.tex]
      %[readtex set-theory/sections/01_sets/02_powerset.ftl.tex]
      %[prove on][check on]
    \end{forthel}
  \end{comment}


  \section{The axiom of regularity}

  \begin{forthel}
    [readtex \path{set-theory/sections/01_sets/01_sets.ftl.tex}]
  \end{forthel}

  \begin{forthel}
    Let $u,v,w$ denote objects.
    Let $x,y,z$ denote sets.
  \end{forthel}

  \noindent The \textit{axiom of regularity} (or \textit{axiom of foundation})
  states that every non-empty set has a $\in$-minimal element.

  \begin{forthel}
    \begin{axiom}[Regularity]\label{SetTheory_01_03_283644}
      Every nonempty set $x$ that contains some set contains some set $y$ such that $x$ and $y$ are disjoint.
    \end{axiom}
  \end{forthel}

  \noindent As a consequence we get that no set can contain itself.
  Moreover, this allows us to show that there exists no universal set, i.e. that
  \enquote{the set of all sets} does not exist.

  \begin{forthel}
    \begin{proposition}\label{SetTheory_01_03_877283}
      No set $x$ is an element of $x$.
    \end{proposition}
    \begin{proof}
      Assume the contrary.
      Take a set $x$ such that $x \in x$.
      We can take an element $y$ of $\set{x}$ such that $\set{x}$ and $y$ are disjoint (by \nameref{SetTheory_01_03_283644}).
      Indeed $\set{x}$ contains some set.
      Then $y = x$.
      Hence $\set{x}$ and $x$ are disjoint.
      Contradiction.
      Indeed $x \in \set{x}$ and $x \in x$.
    \end{proof}

    \begin{corollary}\label{SetTheory_01_03_722484}
      There is no set that contains every set.
    \end{corollary}
    \begin{proof}
      Assume the contrary.
      Take a set $V$ that contains every set.
      Then $V$ is an element of $V$.
      Contradiction.
    \end{proof}

    \begin{proposition}\label{SetTheory_01_03_512352}
      There exist no sets $x,y$ such that $x \in y$ and $y \in x$.
    \end{proposition}
    \begin{proof}
      Assume the contrary.
      Take sets $x,y$ such that $x \in y$ and $y \in x$.
      Consider an element $z$ of $\set{x,y}$ such that $\set{x,y}$ and $z$ are disjoint (by \nameref{SetTheory_01_03_283644}).
      Indeed $\set{x,y}$ contains some set.
      We have $z = x$ or $z = y$.

      Case $z = x$.
        Then $x$ and $\set{x,y}$ are disjoint.
        Hence $y \notin x$.
        Contradiction.
      End.

      Case $z = y$.
        Then $y$ and $\set{x,y}$ are disjoint.
        Hence $x \notin y$.
        Contradiction.
      End.
    \end{proof}
  \end{forthel}
\end{document}
