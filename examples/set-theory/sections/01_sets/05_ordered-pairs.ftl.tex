\documentclass[../../set-theory.ftl.tex]{subfiles}

\begin{document}
  \section{Ordered pairs}

  \begin{forthel}
    [readtex \path{set-theory/sections/01_sets/01_sets.ftl.tex}]
  \end{forthel}

  \begin{forthel}
    Let $u,v,w,u',v',w'$ denote objects.
    Let $x,y,z,x',y',z'$ denote sets.
  \end{forthel}

  \noindent In this paragraph we introduce the \textit{ordered pair} of two
  objects, following the definition proposed by Kuratowski.

  \vspace{1em}\noindent \textcolor{gray}{Note that Naproche has ordered pairs
  already built in.
  Thus we have to formulate the definition of them as an axiom.}

  \begin{forthel}
    \begin{axiom}
      $(u,v) = \set{\set{u}, \set{u,v}}$.
    \end{axiom}

    \begin{proposition}\label{SetTheory_01_05_366682}
      Let $u,v$ be objects.
      Then $(u,v)$ is an object.
    \end{proposition}
    \begin{proof}
      $\set{u}$ and $\set{u,v}$ are objects.
      Hence $\set{\set{u}, \set{u,v}}$ is an object.
      We have $(u,v) = \set{\set{u}, \set{u,v}}$.
      Thus $(u,v)$ is an object.
    \end{proof}
  \end{forthel}

  \noindent The central property of ordered pairs is that two of them agree if
  they agree on each component.

  \begin{forthel}
    \begin{proposition}\label{SetTheory_01_05_270653}
      If $(u,v) = (u',v')$ then $u = u'$ and $v = v'$.
    \end{proposition}
    \begin{proof}
      Assume $(u,v) = (u',v')$.
      (1) Then $\set{\set{u}, \set{u,v}} = \set{\set{u'}, \set{u',v'}}$.
      Hence ($\set{u} = \set{u'}$ or $\set{u} = \set{u',v'}$) and ($\set{u,v} = \set{u'}$ or $\set{u,v} = \set{u',v'}$).
      Thus ($\set{u} = \set{u'}$ and ($\set{u,v} = \set{u'}$ or $\set{u,v} = \set{u',v'}$)) or ($\set{u} = \set{u',v'}$ and ($\set{u,v} = \set{u'}$ or $\set{u,v} = \set{u',v'}$)).

      Case $\set{u} = \set{u'}$ and ($\set{u,v} = \set{u'}$ or $\set{u,v} = \set{u',v'}$).
        We have $\set{u} = \set{u'}$.
        Hence $u = u'$.

        Case $\set{u,v} = \set{u'}$.
          Then $u = u'= v$.
          Hence $\set{\set{u}, \set{u,u}} = \set{\set{u}, \set{u,v'}}$ (by 1).
          Thus $\set{\set{u}} = \set{\set{u}, \set{u,v'}}$.
          Therefore $\set{u} = \set{u,v'}$.
          Consequently $v' = u = v$.
        End.

        Case $\set{u,v} = \set{u',v'}$.
          Then $\set{u,v} = \set{u,v'}$.
          Hence $v = v'$.
        End.
      End.

      Case $\set{u} = \set{u',v'}$ and ($\set{u,v} = \set{u'}$ or $\set{u,v} = \set{u',v'}$).
        We have $\set{u} = \set{u',v'}$.
        Hence $u = u'$.

        Case $\set{u,v} = \set{u'}$.
          Then $u = v = u'$.
          Hence $v = v'$.
        End.

        Case $\set{u,v} = \set{u',v'}$.
          Then $\set{u,v} = \set{u,v'}$.
          Hence $v = v'$.
        End.
      End.
    \end{proof}

    \begin{definition}
      A pair is an object $x$ such that $x = (u,v)$ for some objects $u,v$.
    \end{definition}

    Let an ordered pair stand for a pair.

    \begin{definition}
      Let $x$ be a pair.
      The first component of $x$ is the object $u$ such that $x = (u,v)$ for some object $v$.
    \end{definition}

    Let the first entry of $x$ stand for the first component of $x$.

    \begin{definition}
      Let $x$ be a pair.
      The second component of $x$ is the object $v$ such that $x = (u,v)$ for some object $u$.
    \end{definition}

    Let the second entry of $x$ stand for the second component of $x$.

    \begin{lemma}\label{SetTheory_01_05_160632}
      Let $x$ be a pair.
      Let $u$ be the first component of $x$ and $v$ be the second component of $x$.
      Then $x = (u,v)$.
    \end{lemma}

    \begin{lemma}\label{SetTheory_01_05_204699}
      Let $x,y$ be pairs.
      Assume that the first component of $x$ agrees with the first component of $y$ and the second component of $x$ agrees with the second component of $y$.
      Then $x = y$.
    \end{lemma}
  \end{forthel}
\end{document}
