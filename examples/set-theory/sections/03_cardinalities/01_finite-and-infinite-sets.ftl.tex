\documentclass[../../set-theory.tex]{subfiles}

\begin{document}
  \section{Finite and infinite sets}

  \begin{forthel}
    [readtex \path{set-theory/sections/02_functions/06_equipollency.ftl.tex}]
  \end{forthel}

  \begin{forthel}
    [readtex \path{arithmetic/sections/02_ordering/01_ordering.ftl.tex}]
  \end{forthel}

  \begin{forthel}
    Let $n, m, k$ denote natural numbers.
    Let $x, y, z$ denote sets.
  \end{forthel}


  \subsection{Initial segments of the natural numbers}

  \begin{forthel}
    \begin{lemma}\label{SetTheory_03_01_609328}
      Let $n,m$ be natural numbers.
      Then there exists a set $x$ such that $x = \class{k \in \mathbb{N} | n \leq k \leq m}$.
    \end{lemma}
    \begin{proof}
      (1) Define \[ P = \class{l \in \mathbb{N} | \text{there exists a set $x$ such that $x = \class{k \in \mathbb{N} | k \leq l}$}}. \]

      (BASE CASE) $P$ contains $0$.
      Proof.
        Take $x = \set{0}$.
        Every natural number that is less than or equal to $0$ is equal to $0$.
        Hence $x = \class{k \in \mathbb{N} | k \leq 0}$.
        Thus $0 \in P$.
      Qed.

      (INDUCTION STEP) For all natural numbers $l$ we have $l \in P \implies l + 1 \in P$. \\
      Proof.
        Let $l$ be a natural number.
        Assume $l \in P$.
        [prover vampire]
        Then we can take a set $x'$ such that $x' = \class{k \in \mathbb{N} | k \leq l}$.
        [prover eprover]
        Take $x = x' \cup \set{l + 1}$.
        For all natural numbers $k$ we have $k \leq l + 1$ iff $k \leq l$ or $k = l + 1$.
        Hence $x = \class{k \in \mathbb{N} | k \leq l + 1}$.
        Thus $l + 1 \in P$ (by 1).
        Indeed $x$ is a set.
      Qed.

      Hence every natural number is contained in $P$.
      Thus we can take a set $x'$ such that $x' = \class{k \in \mathbb{N} | k \leq m}$ (by 1).
      Define $x = \class{k \in \mathbb{N} | \text{$k \in x'$ and $n \leq k$}}$.
      Then every element of $x$ is contained in $x'$.
      Hence $x$ is a set (by \nameref{SetTheory_01_01_240572}).
      We have $x = \class{k \in \mathbb{N} | n \leq k \leq m}$.
    \end{proof}

    \begin{definition}
      Let $n,m$ be natural numbers.
      $\set{n, \dots, m}$ is the set $x$ such that $x = \class{k \in \mathbb{N} | n \leq k \leq m}$.
    \end{definition}

    \begin{proposition}\label{SetTheory_03_01_510576}
      Assume $n \leq m$.
      Then $\set{1, \dots, n} \subseteq \set{1, \dots, m}$.
    \end{proposition}
    \begin{proof}
      Let $k \in \set{1, \dots, n}$.
      Then $1 \leq k \leq n$.
      Hence $1 \leq k \leq m$.
      Thus $k \in \set{1, \dots, m}$.
    \end{proof}

    \begin{proposition}\label{SetTheory_03_01_792549}
      $\set{n, \dots, m} = \emptyset$ iff $m < n$.
    \end{proposition}
    \begin{proof}
      Case $\set{n, \dots, m} = \emptyset$.
        Then there is no natural number $k$ such that $n \leq k \leq m$.
        Hence for every natural number $k$ we have $k < n$ or $m < k$.
        Thus $m < n$ or $m < m$.
        Consequently $n > m$.
      End.

      Case $m < n$.
        Assume that $\set{n, \dots, m}$ is nonempty.
        Take a natural number $k$ such that $n \leq k \leq m$.
        Then $n \leq m$.
        Hence $m$ is not less than $n$.
        Contradiction.
      End.
    \end{proof}

    \begin{corollary}\label{SetTheory_03_01_614247}
      Let $n$ be nonzero.
      Then $\set{n, \dots, 0} = \emptyset$.
    \end{corollary}

    \begin{proposition}\label{SetTheory_03_01_240037}
      $\set{n, \dots, m + 1} \setminus \set{m + 1} = \set{n, \dots, m}$.
    \end{proposition}
    \begin{proof}
      Let us show that $\set{n, \dots, m + 1} \setminus \set{m + 1} \subseteq \set{n, \dots, m}$.
        Let $k \in \set{n, \dots, m + 1} \setminus \set{m + 1}$.
        Then $k \in \set{n, \dots, m + 1}$ and $k \notin \set{m + 1}$.
        Hence $n \leq k \leq m + 1$ and $k \neq m + 1$.
        Thus $n \leq k \leq m$.
        Therefore $k \in \set{n, \dots, m}$.
      End.

      Let us show that $\set{n, \dots, m} \subseteq \set{n, \dots, m + 1} \setminus \set{m + 1}$.
        Let $k \in \set{n, \dots, m}$.
        Then $n \leq k \leq m$.
        Hence $n \leq k \leq m + 1$ and $k \neq m + 1$.
        Thus $k \in \set{n, \dots, m + 1}$ and $k \notin \set{m + 1}$.
        Therefore $k \in \set{n, \dots, m + 1} \setminus \set{m + 1}$.
      End.
    \end{proof}

    \begin{proposition}
      For all natural numbers $n,m$ if $\set{1, \dots, n}$ and $\set{1, \dots, m}$ are equipollent then $n = m$.
    \end{proposition}
    \begin{proof}
      Define \[ P = \class{n \in \mathbb{N} | \classtext{for all natural numbers $m$ if $\set{1, \dots, n}$ and $\set{1, \dots, m}$ are equipollent then $n = m$}}. \]

      (BASE CASE) $P$ contains $0$. \\
      Proof.
        Let us show that for all natural numbers $m$ if $\set{1, \dots, 0}$ and $\set{1, \dots, m}$ are equipollent then $m = 0$.
          Let $m$ be a natural number.
          Assume that $\set{1, \dots, 0}$ and $\set{1, \dots, m}$ are equipollent.
          We have $\set{1, \dots, 0} = \emptyset$.
          Hence $\set{1, \dots, m}$ and $\emptyset$ are equipollent.
          Thus $\set{1, \dots, m} = \emptyset$.
          Therefore $m = 0$.
        End.
      Qed.

      (INDUCTION STEP) For all natural numbers $n$ we have $n \in P \implies n + 1 \in P$. \\
      Proof.
        Let $n$ be a natural number.
        Assume $n \in P$.

        Define \[ Q = \class{m \in \mathbb{N} | \classtext{if $n + 1 \neq m$ then $\set{1, \dots, m}$ and $\set{1, \dots, n + 1}$ are not equipollent}}. \]

        (BASE CASE) $Q$ contains $0$. \\
        Proof.
          Case $n = 0$. Obvious.

          Case $n \neq 0$.
            We have $\set{1, \dots, 0} = \emptyset$.
            $n + 1 > 1$.
            Hence $\set{1, \dots, n + 1} \neq \emptyset$.
            Thus $\set{1, \dots, 0}$ and $\set{1, \dots, n + 1}$ are not equipollent.
            Therefore $0 \in Q$.
          End.
        Qed.

        (INDUCTION STEP) For all natural numbers $m$ we have $m \in Q \implies m + 1 \in Q$. \\
        Proof.
          Let $m$ be a natural number.
          Assume $m \in Q$.

          Let us show that if $n + 1 \neq m + 1$ then $\set{1, \dots, m + 1}$ and $\set{1, \dots, n + 1}$ are not equipollent.
            Assume $n + 1 \neq m + 1$.
            Then $n \neq m$.
            Hence $\set{1, \dots, n}$ and $\set{1, \dots, m}$ are not equipollent.
            Indeed $n \in P$.
            Suppose that $\set{1, \dots, n + 1}$ and $\set{1, \dots, m + 1}$ are equipollent.
            Take a bijection $f$ between $\set{1, \dots, n + 1}$ and $\set{1, \dots, m + 1}$.

            If $n + 1 = f^{-1}(m + 1)$ then $f(n + 1) = m + 1$.
            Define \[ g(k) =
              \begin{cases}
                m + 1    & : k = n + 1 \\
                f(n + 1) & : k = f^{-1}(m + 1) \\
                f(k)     & : \text{$k \neq n + 1$ and $k \neq f^{-1}(m + 1)$}
              \end{cases} \]
            for $k \in \set{1, \dots, n + 1}$.

            (1) $g$ is a function to $\set{1, \dots, m + 1}$. \\
            Proof.
              $g$ is a function.
              Let $k \in \set{1, \dots, n + 1}$.

              Let us show that $g(k) \in \set{1, \dots, m + 1}$.
                Case $k = n + 1$.
                  Then $g(k) = m + 1 \in \set{1, \dots, m + 1}$.
                End.

                Case $k = f^{-1}(m + 1)$.
                  Then $g(k) = f(n + 1) \in \set{1, \dots, m + 1}$.
                End.

                Case $k \neq n + 1$ and $k \neq f^{-1}(m + 1)$.
                  Then $g(k) = f(k) \in \set{1, \dots, m + 1}$.
                End.
              End.
            Qed.

            (2) $g$ is a function onto $\set{1, \dots, m + 1}$. \\
            Proof.
              Let us show that every element of $\set{1, \dots, m + 1}$ is a value of $g$.
                Let $l \in \set{1, \dots, m + 1}$.

                Case $l = m + 1$.
                  Then $l = m + 1 = g(n + 1)$.
                End.

                Case $l = f(n + 1)$.
                  Then $l = f(n + 1) = g(f^{-1}(m + 1))$.
                End.

                Case $l \neq m + 1$ and $l \neq f(n + 1)$.
                  Take $k = f^{-1}(l)$.
                  $k \neq n + 1$.
                  Indeed if $k = n + 1$ then $f(n + 1) = f(k) = f(f^{-1}(l)) = l$.
                  $k \neq f^{-1}(m + 1)$.
                  Indeed if $k = f^{-1}(m + 1)$ then $m + 1 = f(f^{-1}(m + 1)) = f(k) = f(f^{-1}(l)) = l$.
                  Hence $g(k) = f(k) = f(f^{-1}(l)) = l$.
                End.
              End.

              Thus $g$ is a function onto $\set{1, \dots, m + 1}$ (by \ref{SetTheory_02_01_195739}, 1).
            Qed.

            (3) $g$ is one to one. \\
            Proof.
              Let $k,k' \in \set{1, \dots, n + 1}$.
              Assume $g(k) = g(k')$.

              Let us show that $k = k'$.
                Case $k, k' = n + 1$. Trivial.

                Case $k = n + 1$ and $k' = f^{-1}(m + 1)$.
                  Then $g(k) = m + 1$ and $g(k') = f(n + 1)$.
                  Hence $k' = f^{-1}(m + 1) = f^{-1}(g(k)) = f^{-1}(g(k')) = f^{-1}(f(n + 1)) = n + 1 = k$.
                End.

                Case $k = n + 1$ and $k' \neq n + 1, f^{-1}(m + 1)$.
                  Then $g(k) = m + 1$ and $g(k') = f(k')$.
                  Hence $f(k') = g(k') = g(k) = m + 1$.
                  Thus $f(k) = f(n + 1) = g(f^{-1}(m + 1)) = g(f^{-1}(f(k'))) = g(k')= f(k')$.
                  Therefore $k = k'$.
                End.

                Case $k = f^{-1}(m + 1)$ and $k' = n + 1$.
                  Then $g(k) = f(n + 1)$ and $g(k') = m + 1$.
                  Hence $f(k) = f(f^{-1}(m + 1)) = m + 1 = g(k') = g(k) = f(n + 1) = f(k')$.
                  Thus $k = k'$.
                End.

                Case $k, k' = f^{-1}(m + 1)$. Trivial.

                Case $k = f^{-1}(m + 1)$ and $k' \neq n + 1, f^{-1}(m + 1)$.
                  Then $g(k) = f(n + 1)$ and $g(k') = f(k')$.
                  Hence $f(k') = g(k') = g(k) = f(n + 1)$.
                  Thus $k' = n + 1$.
                  Indeed $f$ is one to one and $k', n + 1$ belong to the domain of $f$.
                  Contradiction.
                End.

                Case $k \neq n + 1, f^{-1}(m + 1)$ and $k' = n + 1$.
                  Then $g(k) = f(k)$ and $g(k') = m + 1$.
                  Hence $f(k) = g(k) = g(k') = m + 1$.
                  Thus $f(k') = f(n + 1) = g(f^{-1}(m + 1)) = g(f^{-1}(f(k))) = g(k) = f(k)$.
                  Therefore $k' = k$.
                End.

                Case $k \neq n + 1, f^{-1}(m + 1)$ and $k' = f^{-1}(m + 1)$.
                  Then $g(k) = f(k)$ and $g(k') = f(n + 1)$.
                  Hence $f(k) = g(k) = g(k') = f(n + 1)$.
                  Thus $k = n + 1$.
                  Indeed $f$ is one to one and $k, n + 1$ belong to the domain of $f$.
                  Contradiction.
                End.

                Case $k, k' \neq n + 1, f^{-1}(m + 1)$.
                  Then $g(k) = f(k)$ and $g(k') = f(k')$.
                  Hence $f(k) = g(k) = g(k') = f(k')$.
                  Thus $k = k'$.
                  Indeed $f$ is one to one and $k, k'$ belong to the domain of $f$.
                End.
              End.
            Qed.

            Thus $g$ is a bijection between $\set{1, \dots, n + 1}$ and $\set{1, \dots, m + 1}$ (by 2, 3).
            Take $h = g \restr \set{1, \dots, n}$.
            Then $h$ is a one to one function of $\set{1, \dots, n}$ (by \ref{SetTheory_02_01_507691}).
            Indeed $\set{1, \dots, n} \subseteq \set{1, \dots, n + 1}$.

            Let us show that every element of $\set{1, \dots, m + 1} \setminus \set{g(n + 1)}$ is a value of $h$.
              Let $k \in \set{1, \dots, m + 1} \setminus \set{g(n + 1)}$.
              Then $k \in \set{1, \dots, m + 1}$.
              Take $l = g^{-1}(k)$.
              Hence $l \in \set{1, \dots, n + 1}$.
              We have $l = g^{-1}(k) \neq n + 1$.
              Indeed if $g^{-1}(k) = n + 1$ then $k = g(g^{-1}(k)) = g(n + 1) \notin \set{1, \dots, m + 1} \setminus \set{g(n + 1)}$.
              Thus $l \in \set{1, \dots, n}$.
              We have $k = g(g^{-1}(k)) = g(l) = h(l)$.
              Therefore $k$ is a value of $h$.
            End.

            Let us show that $h$ is a function to $\set{1, \dots, m + 1} \setminus \set{g(n + 1)}$.
              Let $k \in \set{1, \dots, n}$.
              Then $h(k) = g(k) \in \set{1, \dots, m + 1}$.
              Indeed $g$ is a function to $\set{1, \dots, m + 1}$ and $k \in \dom(g)$.
              We have $h(k) \neq g(n + 1)$.
              Indeed if $h(k) = g(n + 1)$ then $k = g^{-1}(g(k)) = g^{-1}(h(k)) = g^{-1}(g(n + 1)) = n + 1 \notin \set{1, \dots, n}$.
              Thus $h(k) \notin \set{g(n + 1)}$.
              Therefore $h(k) \in \set{1, \dots, m + 1} \setminus \set{g(n + 1)}$.
            End.

            Then $h$ is a function onto $\set{1, \dots, m + 1} \setminus \set{g(n + 1)}$.
            Hence $h$ is a bijection between $\set{1, \dots, n}$ and $\set{1, \dots, m + 1} \setminus \set{g(n + 1)}$.
            We have $g(n + 1) = m + 1$.
            Thus $\set{1, \dots, m + 1} \setminus \set{g(n + 1)} = \set{1, \dots, m}$.
            Therefore $h$ is a bijection between $\set{1, \dots, n}$ and $\set{1, \dots, m}$.
            Consequently $\set{1, \dots, n}$ and $\set{1, \dots, m}$ are equipollent.
            Contradiction.
          End.
        Qed.
      Qed.
    \end{proof}
  \end{forthel}


  \subsection{Finite sets}

  \begin{forthel}
    \begin{definition}
      $x$ has $n$ elements iff $x$ is equipollent to $\set{1, \dots, n}$.
    \end{definition}

    Let $x$ has exactly $n$ elements stand for $x$ has $n$ elements.
    Let $\# x = n$ stand for $x$ has $n$ elements.
    Let $\# x \neq n$ stand for it is wrong that $x$ has $n$ elements.

    \begin{definition}
      $x$ is finite iff there exists a natural number $n$ such that $x$ has $n$ elements.
    \end{definition}

    Let $\# x < \infty$ stand for $x$ is finite.

    \begin{definition}
      A system of finite sets is a system of set $X$ such that every element of $X$ is finite.
    \end{definition}

    \begin{definition}
      $x$ has less than $n$ elements iff $x$ has $m$ elements for some natural number $m$ such that $m < n$.
    \end{definition}

    Let $\# x < n$ stand for $x$ has less than $n$ elements.
    Let $x$ does not have less than $n$ elements stand for it is wrong that $x$ has less than $n$ elements.

    \begin{definition}
      $x$ has at most $n$ elements iff $x$ has $m$ elements for some natural number $m$ such that $m \leq n$.
    \end{definition}

    Let $\# x \leq n$ stand for $x$ has at most $n$ elements.
    Let $x$ does not have at most $n$ elements stand for it is wrong that $x$ has at most $n$ elements.

    \begin{definition}
      $x$ has more than $n$ elements iff $x$ does not have at most $n$ elements.
    \end{definition}

    Let $\# x > n$ stand for $x$ has more than $n$ elements.
    Let $x$ does not have more than $n$ elements stand for it is wrong that $x$ has more than $n$ elements.

    \begin{definition}
      $x$ has at least $n$ elements iff $x$ does not have less than $n$ elements.
    \end{definition}

    Let $\# x \geq n$ stand for $x$ has at least $n$ elements.
  \end{forthel}


  \subsection{Properties of finite cardinalities}

  \begin{forthel}
    \begin{proposition}
      \[ (\text{$\# x = n$ and $\# x = m$}) \implies n = m. \]
    \end{proposition}
    \begin{proof}
      Assume $\# x = n$ and $\# x = m$.
      Then $x$ is equipollent to $\set{1, \dots, n}$ and $x$ is equipollent to $\set{1, \dots, m}$.
      Hence $\set{1, \dots, n}$ and $\set{1, \dots, m}$ are equipollent.
      Thus $n = m$.
    \end{proof}

    \begin{proposition}
      \[ (\text{$\# x < n$ and $n \leq m$}) \implies \# x < m. \]
    \end{proposition}
    \begin{proof}
      Assume $\# x < n$ and $n \leq m$.
      Take a natural number $k$ such that $k < n$ and $\# x = k$.
      Then $k < m$.
      Hence $\# x < m$.
    \end{proof}

    \begin{proposition}
      \[ \# x \leq n \iff (\text{$\# x = n$ or $\# x < n$}). \]
    \end{proposition}
    \begin{proof}
      Case $\# x \leq n$.
        Take a natural number $m$ such that $m \leq n$ and $\# x = m$.
        Then $m = n$ or $m < n$.
        If $m = n$ then $\# x = n$.
        If $m < n$ then $\# x < n$.
      End.

      Case $\# x = n$ or $\# x < n$.
        Case $\# x = n$.
          We have $n \leq n$.
          Hence $\# x \leq n$.
        End.

        Case $\# x < n$.
          Take a natural number $m$ such that $m < n$ and $\# x = m$.
          Then $m \leq n$.
          Hence $\# x \leq n$.
        End.
      End.
    \end{proof}

    \begin{proposition}
      \[ (\text{$\# x \leq n$ and $n \leq m$}) \implies \# x \leq m. \]
    \end{proposition}
    \begin{proof}
      Assume $\# x \leq n$ and $n \leq m$.
      Take a natural number $k$ such that $k \leq n$ and $\# x = k$.
      Then $k \leq m$.
      Hence $\# x \leq m$.
    \end{proof}

    \begin{proposition}
      \[ (\text{$\# x > n$ and $n \geq m$}) \implies \# x > m. \]
    \end{proposition}
    \begin{proof}
      Assume $\# x > n$ and $n \geq m$.
      Assume that it is wrong that $\# x > m$.
      Then $\# x \leq m$.
      Take a natural number $k$ such that $k \leq m$ and $\# x = k$.
      Then $k \leq n$.
      Hence $\# x \leq n$.
      Contradiction.
    \end{proof}

    \begin{proposition}
      \[ \# x \geq n \iff (\text{$\# x = n$ or $\# x > n$}). \]
    \end{proposition}
    \begin{proof}
      Case $\# x \geq n$.
        Then it is wrong that $\# x < n$.
        Assume that it is wrong that $\# x = n$ or $\# x > n$.
        Then $\# x \neq n$ and $\# x \leq n$.
        Hence $\# x < n$.
        Contradiction.
      End.

      Case $\# x = n$ or $\# x > n$.
        Assume that it is wrong that $\# x \geq n$.
        Then $\# x < n$.
        Hence we can take a natural number $m$ such that $m < n$ and $\# x = m$.

        Case $\# x = n$.
          Then $m = n$.
          Contradiction.
        End.

        Case $\# x > n$.
          We have $m \leq n$.
          Hence $\# x \leq n$.
          Contradiction.
        End.
      End.
    \end{proof}

    \begin{proposition}
      \[ (\text{$\# x \geq n$ and $n \geq m$}) \implies \# x \geq m. \]
    \end{proposition}
    \begin{proof}
      Assume $\# x \geq n$ and $n \geq m$.
      Then $\# x = n$ or $\# x > n$.
      We have $n = m$ or $n > m$.

      Case $n = m$.
        Then $\# x = m$ or $\# x > m$.
        Hence $\# x \geq m$.
      End.

      Case $n > m$.
        Then $\# x > m$.
        Hence $\# x \geq m$.
      End.
    \end{proof}

    \begin{proposition}
      $x$ is finite iff $\# x \leq n$ for some natural number $n$.
    \end{proposition}
    \begin{proof}
      Case $x$ is finite.
        Take a natural number $n$ such that $\# x = n$.
        We have $n \leq n$.
        Hence $\# x \leq n$.
      End.

      Case $\# x \leq n$ for some natural number $n$.
        Consider a natural number $n$ such that $\# x \leq n$.
        Take a natural number $m$ such that $m \leq n$ and $\# x = m$.
        Then $x$ is finite.
      End.
    \end{proof}

    \begin{corollary}
      $x$ is finite iff $\# x < n$ for some natural number $n$.
    \end{corollary}
    \begin{proof}
      Case $x$ is finite.
        Take a natural number $n$ such that $\# x = n$.
        Then $\# x < n + 1$.
      End.

      Case $\# x < n$ for some natural number $n$.
        Then $\# x \leq n$ for some natural number $n$.
        Hence $x$ is finite.
      End.
    \end{proof}
  \end{forthel}


  \subsection{Properties of finite sets}

  \begin{forthel}
    \begin{proposition}\label{SetTheory_03_01_835809}
      $x$ is empty iff $x$ has zero elements.
    \end{proposition}
    \begin{proof}
      Case $x$ is empty.
        Then $x = \emptyset$.
        We have $\set{1, \dots, 0} = \emptyset$.
        Then $\id_{\emptyset}$ is a bijection between $x$ and $\set{1, \dots, 0}$.
        Hence $x$ and $\set{1, \dots, 0}$ are equipollent.
        Thus $x$ has zero elements.
      End.

      Case $x$ has zero elements.
        Take a bijection $f$ between $x$ and $\set{1, \dots, 0}$.
        We have $\set{1, \dots, 0} = \emptyset$.
        Hence $x$ is empty.
        Indeed if $x$ is nonempty then $\emptyset$ contains $f(u)$ for some $u \in x$.
      End.
    \end{proof}

    \begin{corollary}\label{SetTheory_03_01_750555}
      $\emptyset$ has zero elements.
    \end{corollary}

    \begin{proposition}\label{SetTheory_03_01_758768}
      Let $u$ be an element.
      Then $\set{u}$ has exactly one element.
    \end{proposition}
    \begin{proof}
      Define $f(w) = 1$ for $w \in \set{u}$.

      (1) $f$ is one to one.
      Indeed for all $a,b \in \set{u}$ we have $f(a) = f(u) = f(b)$.

      (2) $f$ is a function onto $\set{1}$.
      Indeed every element of $\set{1}$ is equal to $1$ and $f(u) = 1$.

      Hence $f$ is a bijection between $\set{u}$ and $\set{1}$.
      We have $\set{1} = \set{1, \dots, 1}$.
      Thus $\set{u}$ has exactly one element.
    \end{proof}

    \begin{proposition}\label{SetTheory_03_01_105900}
      Let $u,v$ be distinct elements.
      Then $\set{u,v}$ has exactly two elements.
    \end{proposition}
    \begin{proof}
      Define \[ f(w) =
        \begin{cases}
          1 & w = u \\
          2 & w = v
        \end{cases} \]
      for $w \in \set{u,v}$.

      (1) $f$ is one to one.
      Proof.
        Let $a,b \in \set{u,v}$.
        Assume $a \neq b$.
        Then ($a = u$ and $b = v$) or ($a = v$ and $b = u$).

        Case $a = u$ and $b = v$.
          Then $f(a) = 1$ and $f(b) = 2$.
          Hence $f(a) \neq f(b)$.
        End.

        Case $a = v$ and $b = u$.
          Then $f(a) = 2$ and $f(b) = 1$.
          Hence $f(a) \neq f(b)$.
        End.
      Qed.

      Let us show that every element of $\set{1, 2}$ is a value of $f$.
        Let $b \in \set{1,2}$.
        Then $b = 1$ or $b = 2$.
        If $b = 1$ then $b = f(u)$.
        If $b = 2$ then $b = f(v)$.
        Hence $b$ is a value of $f$.
      Qed.

      Let us show that $f$ is a function to $\set{1,2}$.
        Let $a \in \set{u,v}$.
        Then $a = u$ or $a = v$.
        If $a = u$ then $f(a) = 1 \in \set{1,2}$.
        If $a = v$ then $f(a) = 2 \in \set{1,2}$.
        Thus $f(a) \in \set{1,2}$.
      End.

      (2) Hence $f$ is a function onto $\set{1,2}$.

      Thus $f$ is a bijection between $\set{u,v}$ and $\set{1,2}$ (by 1, 2).
      We have $\set{1,2} = \set{1, \dots, 2}$.
      Therefore $\set{u,v}$ has exactly two elements.
    \end{proof}
  \end{forthel}
\end{document}
