\documentclass[../../set-theory.ftl.tex]{subfiles}

\begin{document}
  \section{Equipollency}

  \begin{forthel}
    [readtex \path{set-theory/sections/02_functions/03_invertible-functions.ftl.tex}]
  \end{forthel}

  \begin{forthel}
  Let $u,v,w$ denote elements.
  Let $x,y,z$ denote sets.
  Let $f,g,h$ denote functions.
  \end{forthel}


  \begin{forthel}
    \begin{definition}
      $x$ and $y$ are equipollent iff there exists a bijection between $x$ and $y$.
    \end{definition}

    Let $x$ and $y$ are equipotent stand for $x$ and $y$ are equipollent.

    \begin{proposition}\label{SetTheory_02_06_639059}
      $x$ and $x$ are equipollent.
    \end{proposition}
    \begin{proof}
      $\id{x}$ is a bijection between $x$ and $x$.
    \end{proof}

    \begin{proposition}\label{SetTheory_02_06_467393}
      If $x$ and $y$ are equipollent then $y$ and $x$ are equipollent.
    \end{proposition}
    \begin{proof}
      Assume that $x$ and $y$ are equipollent.
      Take a bijection $f$ between $x$ and $y$.
      Then $f^{-1}$ is a bijection between $y$ and $x$.
      Hence $y$ and $x$ are equipollent.
    \end{proof}

    \begin{proposition}\label{SetTheory_02_06_956273}
      If $x$ and $y$ are equipollent and $y$ and $z$ are equipollent then $x$ and $z$ are equipollent.
    \end{proposition}
    \begin{proof}
      Assume that $x$ and $y$ are equipollent and $y$ and $z$ are equipollent.
      Take a bijection $f$ between $x$ and $y$.
      Take a bijection $g$ between $y$ and $z$.
      Then $g \circ f$ is a bijection between $x$ and $z$.
      Hence $x$ and $z$ are equipollent.
    \end{proof}

    \begin{proposition}\label{SetTheory_02_06_430789}
      $x$ and $\emptyset$ are equipollent iff $x$ is empty.
    \end{proposition}
    \begin{proof}
      Case $x$ and $\emptyset$ are equipollent.
        Take a bijection $f$ between $x$ and $\emptyset$.
        Assume that $x$ is nonempty.
        Take an element $u$ of $x$.
        Then $f(u) \in \emptyset$.
        Contradiction.
      End.

      Case $x$ is empty.
        Then $x = \emptyset$.
        Hence $x$ and $\emptyset$ are equipollent.
      End.
    \end{proof}
  \end{forthel}
\end{document}
