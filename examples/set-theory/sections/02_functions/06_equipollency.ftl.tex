\documentclass[../../set-theory.ftl.tex]{subfiles}

\begin{document}
  \begin{comment}
    \begin{forthel}
      % Uncomment for debugging:

      %[prove off][check off]
      %[readtex set-theory/sections/01_sets/01_sets.ftl.tex]
      %[readtex set-theory/sections/01_sets/02_powerset.ftl.tex]
      %[readtex set-theory/sections/01_sets/03_regularity.ftl.tex]
      %[readtex set-theory/sections/01_sets/04_symmetric-difference.ftl.tex]
      %[readtex set-theory/sections/01_sets/05_ordered-pairs.ftl.tex]
      %[readtex set-theory/sections/01_sets/06_cartesian-product.ftl.tex]
      %[readtex set-theory/sections/02_functions/01_functions.ftl.tex]
      %[readtex set-theory/sections/02_functions/02_image-and-preimage.ftl.tex]
      %[readtex set-theory/sections/02_functions/03_invertible-functions.ftl.tex]
      %[readtex set-theory/sections/02_functions/04_functions-and-symmetric-difference.ftl.tex]
      %[readtex set-theory/sections/02_functions/05_functions-and-set-systems.ftl.tex]
      %[prove on][check on]
    \end{forthel}
  \end{comment}


  \section{Equipollency}

  \begin{forthel}
    [readtex \path{set-theory/sections/02_functions/03_invertible-functions.ftl.tex}]
  \end{forthel}

  \begin{forthel}
    Let $u,v,w$ denote objects.
    Let $x,y,z$ denote sets.
    Let $f,g,h$ denote functions.
  \end{forthel}

  \noindent We conclude this part about functions by introducing the notion of
  \textit{equipollency}:
  Two sets $x,y$ being equipollent expresses the idea of $x$ and $y$ having the
  same number of elements.

  \begin{forthel}
    \begin{definition}
      $x$ is equipollent to $y$ iff there exists a bijection between $x$ and $y$.
    \end{definition}

    Let $x$ is equpotent to $y$ stand for $x$ and $y$ are equipollent.

    \begin{proposition}\label{SetTheory_02_06_639059}
      $x$ and $x$ are equipollent.
    \end{proposition}
    \begin{proof}
      $\id_{x}$ is a bijection between $x$ and $x$.
    \end{proof}

    \begin{proposition}\label{SetTheory_02_06_467393}
      If $x$ and $y$ are equipollent then $y$ and $x$ are equipollent.
    \end{proposition}
    \begin{proof}
      Assume that $x$ and $y$ are equipollent.
      Take a bijection $f$ between $x$ and $y$.
      Then $f^{-1}$ is a bijection between $y$ and $x$.
      Hence $y$ and $x$ are equipollent.
    \end{proof}

    \begin{proposition}\label{SetTheory_02_06_956273}
      If $x$ and $y$ are equipollent and $y$ and $z$ are equipollent then $x$ and $z$ are equipollent.
    \end{proposition}
    \begin{proof}
      Assume that $x$ and $y$ are equipollent and $y$ and $z$ are equipollent.
      Take a bijection $f$ between $x$ and $y$.
      Take a bijection $g$ between $y$ and $z$.
      Then $g \circ f$ is a bijection between $x$ and $z$.
      Hence $x$ and $z$ are equipollent.
    \end{proof}

    \begin{proposition}\label{SetTheory_02_06_430789}
      $x$ and $\emptyset$ are equipollent iff $x$ is empty.
    \end{proposition}
    \begin{proof}
      Case $x$ and $\emptyset$ are equipollent.
        Take a bijection $f$ between $x$ and $\emptyset$.
        Assume that $x$ is nonempty.
        Take an element $u$ of $x$.
        Then $f(u) \in \emptyset$.
        Contradiction.
      End.

      Case $x$ is empty.
        Then $x = \emptyset$.
        Hence $x$ and $\emptyset$ are equipollent.
      End.
    \end{proof}
  \end{forthel}
\end{document}
