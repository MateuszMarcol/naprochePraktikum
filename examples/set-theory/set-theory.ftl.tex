\documentclass{article}

\usepackage[english]{babel}
\usepackage{amssymb}
\usepackage{xurl}
\usepackage{hyperref}
\usepackage{subfiles}
\usepackage{../../lib/tex/naproche}
\usepackage{../../lib/tex/basic-notions}

\hypersetup{
  colorlinks=true,
  linkcolor=blue,
  urlcolor=blue
}

\title{Set Theory}
\author{Marcel Schütz}
\date{2021}

\begin{document}
  \maketitle

  \begin{abstract}
    This is a formalization of some ZF-like set theory.
    It introduces the common operations on sets like unions, intersections,
    complements, powersets, symmetric differences and Cartesian products.
    For all these operations, detailled proofs of their algebraic computation
    laws are given.
    Moreover, basic notions concerning functions, like images, preimages and
    invertability, are provided, again with detailed proofs of how they behave
    towards the mentioned operations on sets, which finally leads to the
    definition of equipollency.
  \end{abstract}

  \tableofcontents

  \newpage
  \part{Sets}

  \subfile{sections/01_sets/01_sets.ftl.tex}
  \begin{comment}
    \begin{forthel}
      [readtex \path{set-theory/sections/01_sets/01_sets.ftl.tex}]
    \end{forthel}
  \end{comment}

  \newpage
  \subfile{sections/01_sets/02_powerset.ftl.tex}
  \begin{comment}
    \begin{forthel}
      [readtex \path{set-theory/sections/01_sets/02_powerset.ftl.tex}]
    \end{forthel}
  \end{comment}

  \newpage
  \subfile{sections/01_sets/03_regularity.ftl.tex}
  \begin{comment}
    \begin{forthel}
      [readtex \path{set-theory/sections/01_sets/03_regularity.ftl.tex}]
    \end{forthel}
  \end{comment}

  \newpage
  \subfile{sections/01_sets/04_symmetric-difference.ftl.tex}
  \begin{comment}
    \begin{forthel}
      [readtex \path{set-theory/sections/01_sets/04_symmetric-difference.ftl.tex}]
    \end{forthel}
  \end{comment}

  \newpage
  \subfile{sections/01_sets/05_ordered-pairs.ftl.tex}
  \begin{comment}
    \begin{forthel}
      [readtex \path{set-theory/sections/01_sets/05_ordered-pairs.ftl.tex}]
    \end{forthel}
  \end{comment}

  \newpage
  \subfile{sections/01_sets/06_cartesian-product.ftl.tex}
  \begin{comment}
    \begin{forthel}
      [readtex \path{set-theory/sections/01_sets/06_cartesian-product.ftl.tex}]
    \end{forthel}
  \end{comment}


  \newpage
  \part{Functions}

  \subfile{sections/02_functions/01_functions.ftl.tex}
  \begin{comment}
    \begin{forthel}
      [readtex \path{set-theory/sections/02_functions/01_functions.ftl.tex}]
    \end{forthel}
  \end{comment}

  \newpage
  \subfile{sections/02_functions/02_image-and-preimage.ftl.tex}
  \begin{comment}
    \begin{forthel}
      [readtex \path{set-theory/sections/02_functions/02_image-and-preimage.ftl.tex}]
    \end{forthel}
  \end{comment}

  \newpage
  \subfile{sections/02_functions/03_invertible-functions.ftl.tex}
  \begin{comment}
    \begin{forthel}
      [readtex \path{set-theory/sections/02_functions/03_invertible-functions.ftl.tex}]
    \end{forthel}
  \end{comment}

  \newpage
  \subfile{sections/02_functions/04_functions-and-symmetric-difference.ftl.tex}
  \begin{comment}
    \begin{forthel}
      [readtex \path{set-theory/sections/02_functions/04_functions-and-symmetric-difference.ftl.tex}]
    \end{forthel}
  \end{comment}

  \newpage
  \subfile{sections/02_functions/05_functions-and-set-systems.ftl.tex}
  \begin{comment}
    \begin{forthel}
      [readtex \path{set-theory/sections/02_functions/05_functions-and-set-systems.ftl.tex}]
    \end{forthel}
  \end{comment}

  \newpage
  \subfile{sections/02_functions/06_equipollency.ftl.tex}
  \begin{comment}
    \begin{forthel}
      [readtex \path{set-theory/sections/02_functions/06_equipollency.ftl.tex}]
    \end{forthel}
  \end{comment}
\end{document}
