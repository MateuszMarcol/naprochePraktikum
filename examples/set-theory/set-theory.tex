\documentclass{article}

\usepackage[english]{babel}
\usepackage{amssymb}
\usepackage{xurl}
\usepackage{hyperref}
\usepackage{subfiles}
\usepackage{csquotes}
\usepackage{../../lib/tex/naproche}
\usepackage{../../lib/tex/basic-notions}

\hypersetup{
  colorlinks=true,
  linkcolor=blue,
  urlcolor=blue
}

\title{Set Theory}
\author{Marcel Schütz}
\date{2021}

\begin{document}
  \maketitle

  \begin{abstract}
    This is a formalization of some ZF-like set theory.
    It introduces the common operations on sets like unions, intersections,
    complements, powersets, symmetric differences and Cartesian products and
    presents detailled proofs of their algebraic properties.
    Moreover, basic notions concerning functions like images, preimages and
    invertibility are provided, again with detailed proofs of their computation
    laws, up to the definition of equipollency.

    It can either be regarded as an independent collection of contents from
    basic undergraduate mathematics or serve as the basis for more
    sophisticated formalizations.
  \end{abstract}

  \tableofcontents

  \newpage
  \part{Sets}

  \subfile{sections/01_sets/01_sets.ftl.tex}

  \subfile{sections/01_sets/02_powerset.ftl.tex}

  \subfile{sections/01_sets/03_regularity.ftl.tex}

  \subfile{sections/01_sets/04_symmetric-difference.ftl.tex}

  \subfile{sections/01_sets/05_ordered-pairs.ftl.tex}

  \subfile{sections/01_sets/06_cartesian-product.ftl.tex}


  \newpage
  \part{Functions}

  \subfile{sections/02_functions/01_functions.ftl.tex}

  \subfile{sections/02_functions/02_image-and-preimage.ftl.tex}

  \subfile{sections/02_functions/03_invertible-functions.ftl.tex}

  \subfile{sections/02_functions/04_functions-and-symmetric-difference.ftl.tex}

  \subfile{sections/02_functions/05_functions-and-set-systems.ftl.tex}

  \subfile{sections/02_functions/06_equipollency.ftl.tex}
\end{document}
