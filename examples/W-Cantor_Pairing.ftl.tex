                                            
\documentclass{article}
\usepackage[utf8]{inputenc}
\usepackage[english]{babel}
\usepackage[foundations]{../lib/tex/naproche}

\title{Cantor Pairing}
\author{Peter Koepke}
\date{}

\begin{document}\begin{forthel}

[check off][prove off]
[readtex \path{100_Theorems.ftl.tex}]
[prove on][check on]

\begin{lemma}
Let $m,n$ be natural numbers. Then $m+n$ is a natural number.
\end{lemma}

\begin{signature}
$CP$ is a function $f$ such that 
$f : \mathbb{N} \times \mathbb{N} \rightarrow \mathbb{N}$
and \[f(x,y) = \frac{(x+y)\cdot((x+y)+1)}{2} + x\]
 for all $x,y \in \mathbb{N}$.
\end{signature}

\begin{lemma}
For all natural numbers $n$
\[\frac{n \cdot (n+1)}{2} \in \mathbb{N}.\]
\end{lemma}
\begin{proof}[by induction]
Let $n$ be a natural number.

Case $n=0$. Trivial.

Then $n \neq 0$. Take a natural number $m$ such that $n = m + 1$.
$m$ is inductively smaller than $n$ and
$\frac{m \cdot (m+1)}{2} \in \mathbb{N}$.

$\frac{n \cdot (n+1)}{2} =$
%
$\frac{(m+1) \cdot ((m+1)+1)}{2} =$
%
$\frac{(m+1) \cdot (m+2)}{2} =$
%
$\frac{((m+1) \cdot m) + ((m+1) \cdot 2)}{2} =$
%
$\frac{(m+1) \cdot m}{2} + \frac{(m+1) \cdot 2}{2} =$
%
$\frac{(m+1) \cdot m}{2} + (m+1) \in \mathbb{N}$.
\end{proof}

\begin{lemma}
Every natural number is a value of $CP$.
\end{lemma}
\begin{proof}[by induction]
Let $n$ be a natural number.

Case $n = 0$. $(0,0) \in \mathbb{N} \times \mathbb{N}$.
$CP(0,0) = \frac{(0+0)\cdot((0+0)+1)}{2} + 0 = 0$. Trivial.

Then $n \neq 0$. Take a natural number $m$ such that $n = m + 1$.
$m$ is inductively smaller than $n$.
$m$ is a value of $CP$.
Take natural numbers $x,y$ such that
$m = CP(x,y)$.

Case $y \neq 0$. Then $x+1, y-1 \in \mathbb{N}$. $(x+1,y-1) \in \mathbb{N} \times \mathbb{N}$.
$(x+1)+(y-1)=x+y$. 
Then $CP(x+1,y-1) = 
\frac{((x+1)+(y-1))\cdot(((x+1)+(y-1))+1)}{2} + (x + 1) =$

$\frac{(x+y)\cdot((x+y)+1)}{2} + (x + 1) = $

$(\frac{(x+y)\cdot((x+y)+1)}{2} + x) + 1 =$

$CP(x,y) + 1 = m + 1 = n.$
qed.

Then $y = 0$.

$(0+(x+1))+1 = x + 2$.

Then $CP(0,x+1) = \frac{(0+(x+1)) \cdot ((0+(x+1))+1)}{2} + 0 =$

$\frac{(x+1) \cdot (x+2)}{2} =$

$\frac{((x+1) \cdot x) + ((x+1) \cdot 2)}{2} = $

$\frac{(x+1) \cdot x}{2} + \frac{(x+1) \cdot 2}{2} =$

$\frac{(x+1) \cdot x}{2} + (x+1) =$

$(\frac{(x+1) \cdot x}{2} + x) +1 =$

$CP(x,0) + 1 = m + 1 = n$.
\end{proof}

\begin{lemma}
$CP$ is a surjection from $\mathbb{N} \times \mathbb{N}$ onto $\mathbb{N}$.
\end{lemma}

Let $x,y,u,v$ denote natural numbers.

\begin{lemma}
Every natural number is a real number.
\end{lemma}

\begin{lemma}
Let $u,v$ be natural numbers such that $u < v$. Then $u + 1 \leq v$.
\end{lemma}
\begin{proof}[by contradiction]
Assume the contrary.
Then $u < v < u+1$
and $0 = u - u < v - u < (u+1) - u = 1$.
Contradiction. Indeed $u - v \in \mathbb{Z}$.

\end{proof}

\begin{lemma}
Let $b,c,d$ be real numbers and $b \leq c \leq d$.
Then $b \leq d$.
\end{lemma}



\begin{lemma}
Let $(x,y), (u,v) \in \mathbb{N} \times \mathbb{N}$ and $x+y < u+v$. Then 
$CP(x,y),CP(u,v)$ are real numbers and $CP(x,y) < CP(u,v)$.
\end{lemma}
\begin{proof}
$x+y, u+v$ are natural numbers.
Let $z = x + y$. Let $w = u + v$.

$z+1 \leq w$ and $(z+1)+1 \leq w+1$.

$(z+1)\cdot(z+2) \leq w \cdot (z+2) \leq w \cdot(w+1)$.
[timelimit 30]
$(z+1)\cdot(z+2) \leq w \cdot(w+1)$. 
[timelimit 3]
(1)$z \cdot(z+3) < (z+1)\cdot(z+2)$.
Proof.
$ z \cdot(z+3) = (z \cdot z) + (z \cdot 3)
< ((z \cdot z) + (z \cdot 3)) + 2
= ((z \cdot z) + ((1 \cdot z) + (z \cdot 2))) + 2 
= (((z \cdot z) + (1 \cdot z)) + (z \cdot 2)) + 2
= ((z \cdot z) + (1 \cdot z)) + ((z \cdot 2) + (1 \cdot 2))
= ((z +1) \cdot z) + ((z+1) \cdot 2)
= (z+1)\cdot(z+2)$.
Qed.

$CP(x,y) = \frac{z\cdot(z+1)}{2} + x \leq$

$\frac{z\cdot(z+1)}{2} + z =$

$\frac{z\cdot(z+1)}{2} + \frac{z \cdot 2}{2} =$

$\frac{(z\cdot(z+1)) + (z \cdot 2)}{2} =$

$\frac{z \cdot((z+1) + 2)}{2} =$

$\frac{z \cdot (z+3)}{2} <$

%$\frac{(z+1)\cdot(z+2)}{2}$.

$\frac{(z+1)\cdot(z+2)}{2} \leq$

$\frac{w \cdot(w+1)}{2} \leq$

$\frac{w \cdot(w+1)}{2} + u = CP(u,v)$.

[timelimit 3]
$CP(x,y) < CP(u,v)$.
[timelimit 3]

\end{proof}



\end{forthel}
\end{document}