\documentclass{article}

\usepackage[utf8]{inputenc}
\usepackage[british]{babel}
\usepackage{amsfonts}
\usepackage{xurl}
\usepackage{csquotes}
\usepackage[nonumbers, foundations, arithmetic]{../lib/tex/naproche}

\newcommand{\N}{\mathrm{N}}
\newcommand{\Int}{\mathbb{Z}}
\newcommand{\Prim}{\mathbb{P}}

\title{Furstenberg's proof of the infinitude of primes}
\author{\Naproche formalization: \vspace{0.5em} \\
Andrei Paskevich (2007), \\
Marcel Schütz (2021)}
\date{}


\begin{document}
  \maketitle

  Furstenberg's proof of the infinitude of primes is a topological proof of the
  fact that there are infinitely many primes.
  It was published 1955 while Furstenberg was still an undergraduate
  student.\footnote{Furstenberg, Harry. \enquote{On the Infinitude of Primes}.
  The American Mathematical Monthly, vol. 62, no. 5, 1955, pp. 353–353}
  The following formalization of his proof is based on the notions of natural
  numbers and sets provided by the following files:

  \begin{forthel}
    [readtex \path{arithmetic/sections/03_divisibility/03_modular-arithmetic.ftl.tex}]
  \end{forthel}

  \begin{forthel}
    [readtex \path{arithmetic/sections/03_divisibility/04_primes.ftl.tex}]
  \end{forthel}

  \begin{forthel}
    [readtex \path{foundations/sections/10_sets.ftl.tex}]
  \end{forthel}

  The central idea of Furstenberg's proof is to define a certain topology on
  $\Nat$ from the properties of which we can deduce that the set of
  primes is infinite.\footnote{Actually, Furstenberg's proof makes use of a
  topology on $\Int$. But this topology can as well be restricted to
  $\Nat$ without substantially changing the proof.}
  To do so, let us begin with a few preliminaries about natural numbers.

  \begin{forthel}
    \begin{axiom}
      $\Nat$ is a set.
    \end{axiom}

    Let $n,m,k$ denote natural numbers.
    Let $p,q$ denote nonzero natural numbers.

    \begin{definition}
      Let $A$ be a subset of $\Nat$.
      $A^{\complement} = \Nat \setminus A$.
    \end{definition}

    Let the complement of $A$ stand for $A^{\complement}$.

    \begin{lemma}
      The complement of any subset of $\Nat$ is a subset of $\Nat$.
    \end{lemma}
  \end{forthel}

  Towards a suitable topology on $\Nat$ let us define \textit{arithmetic
  sequences} $\N_{n, q}$ on $\Nat$.

  \begin{forthel}
    \begin{definition}
      $\N_{n, q} = \{ m \in \Nat \mid m \equiv n \pmod{q} \}$.
    \end{definition}
  \end{forthel}

  This allows us to define the \textit{evenly spaced natural number
  topology} on $\Nat$, whose open sets are defined as follows.

  \begin{forthel}
    \begin{definition}
      Let $U$ be a subset of $\Nat$.
      $U$ is open iff for any $n \in U$ there exists a $q$ such that
      $\N_{n, q} \subseteq U$.
    \end{definition}

    \begin{definition}
      A system of open sets is a system of sets $S$ such that every element of
      $S$ is an open subset of $\Nat$.
    \end{definition}
  \end{forthel}

  We can show that the open sets form a topology on $\Nat$.

  \begin{forthel}
    \begin{lemma}
      $\Nat$ and $\emptyset$ are open.
    \end{lemma}

    \begin{lemma}
      Let $U,V$ be open subsets of $\Nat$.
      Then $U \cap V$ is open.
    \end{lemma}
    \begin{proof}
      $U \cap V$ is a subset of $\Nat$.
      Let $n \in U \cap V$.
      Take $q$ such that $\N_{n, q} \subseteq U$.
      Take $p$ such that $\N_{n, p} \subseteq V$.

      Let us show that $\N_{n, p \cdot q} \subseteq U \cap V$.
        Let $m \in \N_{n, p \cdot q}$.
        We have $m \equiv n \pmod{p \cdot q}$.
        Hence $m \equiv n \pmod{p}$ and $m \equiv n \pmod{q}$.
        Thus $m \in \N_{n, p}$ and $m \in \N_{n, q}$.
        Therefore $m \in U$ and $m \in V$.
        Consequently $m \in U \cap V$.
      End.
    \end{proof}

    \begin{lemma}
      Let $S$ be a system of open sets.
      Then $\bigcup S$ is an open set.
    \end{lemma}
    \begin{proof}
      Let $n \in \bigcup S$.
      Take a set $M$ such that $n \in M \in S$.
      Consider a $q$ such that $\N_{n, q} \subseteq M$.
      Then $\N_{n, q} \subseteq \bigcup S$.
    \end{proof}
  \end{forthel}

  Now that we have a topology of open sets on $\Nat$, we can continue
  with a characterization of closed sets.
  Their key property is that they are closed under \textit{finite} unions.
  Since we cannot provide a proper definition of finiteness in the context of
  this formalization, we cannot prove this closedness condition.
  All we can do is to prove that the union of \textit{two} closed sets remains
  closed.
  Having shown this little fact we will introduce the notion of finiteness
  axiomatically and state that every finite union of closed sets is indeed
  closed.
  Actually this condition is all we need to know about finiteness to prove that
  there are infinitely many primes.

  \begin{forthel}
    \begin{definition}
      Let $A$ be a subset of $\Nat$.
      $A$ is closed iff $A^{\complement}$ is open.
    \end{definition}

    \begin{definition}
      A system of closed sets is a system of sets $S$ such that every element of
      $S$ is a closed subset of $\Nat$.
    \end{definition}

    \begin{lemma}
      Let $A,B$ be closed subsets of $\Nat$.
      Then $A \cup B$ is closed.
    \end{lemma}
    \begin{proof}
      % The LHS of the following equation must be put in parentheses due to a
      % bug in Naproche's parser.
      We have $((A \cup B)^{\complement}) = A^{\complement} \cap B^{\complement}$.
      $A^{\complement}$ and $B^{\complement}$ are open.
      Hence $A^{\complement} \cap B^{\complement}$ is open.
      Thus $A \cup B$ is closed.
    \end{proof}

    \begin{signature}
      Let $X$ be a class.
      $X$ is finite is an atom.
    \end{signature}

    Let $X$ is infinite stand for $X$ is not finite.

    \begin{axiom}
      Let $S$ be a finite system of closed sets.
      Then $\bigcup S$ is closed.
    \end{axiom}
  \end{forthel}

  An important step towards Furstenberg's proof is to show that arithmetic
  sequences are closed.

  \begin{forthel}
    \begin{lemma}
      $\N_{n, q}$ is a closed subset of $\Nat$.
    \end{lemma}
    \begin{proof}[by contradiction]
      Let $m \in (\N_{n, q})^{\complement}$.

      Let us show that $\N_{m, q} \subseteq (\N_{n, q})^{\complement}$.
        Let $k \in \N_{m, q}$.
        Assume $k \notin (\N_{n, q})^{\complement}$.
        Then $k \equiv m \pmod{q}$ and $n \equiv k \pmod{q}$.
        Hence $m \equiv n \pmod{q}$.
        Therefore $m \in \N_{n, q}$.
        Contradiction.
      End.
    \end{proof}
  \end{forthel}

  Finally, to show that there are infinitely many primes we identify a prime
  number $p$ with the arithmetic sequence $\N_{0, p}$.
  In fact we could prove that there exists a bijection between the set of all
  prime numbers and the set of all arithmetic sequences of this form.
  But this would go beyond the scope of this formalization.

  \begin{forthel}
    \begin{definition}
      $\Prim = \{ \N_{0, p} \mid \text{$p$ is a prime number} \}$.
    \end{definition}

    \begin{lemma}
      $\Prim$ is a set.
    \end{lemma}
    \begin{proof}
      Every element of $\Prim$ is a subset of $\Nat$.
      Hence every element of $\Prim$ is contained in $\pow(\Nat)$.
      Thus $\Prim$ is a set.
    \end{proof}

    \begin{theorem}[Furstenberg]
      $\Prim$ is infinite.
    \end{theorem}
    \begin{proof}[by contradiction]

      Let us show that for any natural number $n$ $n$ belongs to
      $\bigcup \Prim$ iff $n$ has a prime divisor.
        Let $n$ be a natural number.

        If $n$ has a prime divisor then $n$ belongs to $\bigcup \Prim$. \\
        Proof.
          Assume $n$ has a prime divisor.
          Take a prime divisor $p$ of $n$.
          We have $\N_{0, p} \in \Prim$.
          Hence $n \in \N_{0, p}$.
        Qed.

        If $n$ belongs to $\bigcup \Prim$ then $n$ has a prime divisor. \\
        Proof.
          Assume that $n$ belongs to $\bigcup \Prim$.
          Take a prime number $r$ such that $n \in \N_{0, r}$.
          Hence $n \equiv 0 \pmod{r}$.
          Thus $n \mod r = 0 \mod r = 0$.
          Therefore $r$ is a prime divisor of $n$.
        Qed.
      End.

      Thus we have $(\bigcup \Prim)^{\complement} = \set{1}$.
      Indeed any natural number having no prime divisor is equal to $1$.
      Assume that $\Prim$ is finite.
      Then $\bigcup \Prim$ is closed and
      $(\bigcup \Prim)^{\complement}$ is open.
      Indeed every element of $\Prim$ is a closed set.
      Take a $p$ such that $\N_{1, p} \subseteq (\bigcup \Prim)^{\complement}$.
      $1 + p$ is an element of $\N_{1, p}$.
      Indeed $1 + p \equiv 1 \pmod{p}$.
      $1 + p$ is not equal to $1$.
      Hence $1 + p \notin (\bigcup \Prim)^{\complement}$.
      Contradiction.
    \end{proof}
  \end{forthel}
\end{document}
