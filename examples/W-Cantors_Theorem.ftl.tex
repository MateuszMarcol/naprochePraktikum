\documentclass{article}
\usepackage[english]{babel}
\usepackage{../lib/tex/naproche}

\begin{document}

\section{Cantor's Theorem (Wiedijk \# 63)}

In this document we give a proof of Cantor's Theorem:

  \begin{theorem*}
    There is no surjection from a set onto its powerset.
  \end{theorem*}

  We need to provide certain definitions concerning surjective
  functions and the notion of powerset.
\begin{forthel}
[prove off]
[readtex \path{100_Theorems.ftl.tex}]
[prove on]
    \begin{theorem*}[Cantor]
      Let $M$ be a set. Then there is 
      no surjection from $M$ onto the powerset of $M$.
    \end{theorem*}
    \begin{proof}
      Assume that $f$ is
      a surjection $f$ from $M$ onto the powerset of $M$.
      % The value of $f$ at any element of $M$ is a set.
      Define \[ N = \class{x \in M | \text{$x$ is not an element of $f(x)$}}. \]
      %$N$ is a subset of $M$.
      Take an element $z$ of $M$ such that $f(z) = N$.
      Then \[ z \in N \iff z \notin f(z) = N. \]
      Contradiction.
    \end{proof}
\end{forthel}

\end{document}