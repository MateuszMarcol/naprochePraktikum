\documentclass[../../set-theory/set-theory.tex]{subfiles}

\begin{document}
  \chapter{Sets}\label{chapter:sets}

  \filename{foundations/sections/10_sets.ftl.tex}

  \begin{forthel}
    %[prove off][check off]

    [readtex \path{foundations/sections/09_invertible-maps.ftl.tex}]

    %[prove on][check on]
  \end{forthel}


  \section{Sub- and supersets}

  \begin{forthel}
    \begin{definition}\printlabel{FOUNDATIONS_10_5530582838673408}
      A proper class is a class that is not a set.
    \end{definition}
  \end{forthel}

  \begin{forthel}
    \begin{definition}\printlabel{FOUNDATIONS_10_1346889551183872}
      Let $A$ be a class.
      A subset of $A$ is a subclass of $A$ that is a set.
    \end{definition}

    Let a superset of $A$ stand for a superclass of $A$ that is a set.
    Let a proper subset of $A$ stand for a proper subclass of $A$ that is a set.
    Let a proper superset of $A$ stand for a proper superclass of $A$ that is a
    set.
  \end{forthel}


  \section{Powerclasses}

  \begin{forthel}
    \begin{definition}\printlabel{FOUNDATIONS_10_1448589907722240}
      Let $A$ be a class.
      The powerclass of $A$ is
      \[ \class{ x | \text{$x$ is a subset of $A$} }. \]
    \end{definition}

    Let $\pow(A)$ stand for the powerclass of $A$.
  \end{forthel}


  \section{Systems of sets}

  \begin{forthel}
    \begin{definition}\printlabel{FOUNDATIONS_10_5805323570905088}
      A system of sets is a class $X$ such that every element of $X$ is a set.
    \end{definition}
  \end{forthel}

  \begin{forthel}
    \begin{definition}\printlabel{FOUNDATIONS_10_1631952387964928}
      A system of nonempty sets is a class $X$ such that every element of $X$ is
      a nonempty set.
    \end{definition}
  \end{forthel}

  \begin{forthel}
    \begin{definition}\printlabel{FOUNDATIONS_10_943381479948288}
      Let $A$ be a class.
      A system of subsets of $A$ is a class $X$ such that every element of $X$
      is a subset of $A$.
    \end{definition}
  \end{forthel}

  \begin{forthel}
    \begin{proposition}\printlabel{FOUNDATIONS_10_8268633648136192}
      Let $A$ be a class.
      Then $\emptyset$ is a system of subsets of $A$.
    \end{proposition}
  \end{forthel}

  \begin{forthel}
    \begin{proposition}\printlabel{FOUNDATIONS_10_7546016869908480}
      Let $A$ be a class.
      Then $\pow(A)$ is a system of subsets of $A$.
    \end{proposition}
  \end{forthel}

  \begin{forthel}
    \begin{proposition}
      Let $X, Y$ be systems of sets.
      Then $X \cup Y$ is a system of sets.
    \end{proposition}
  \end{forthel}

  \begin{forthel}
    \begin{proposition}
      Let $X, Y$ be systems of sets.
      Then $X \cap Y$ is a system of sets.
    \end{proposition}
  \end{forthel}

  \begin{forthel}
    \begin{proposition}
      Let $X, Y$ be systems of sets.
      Then $X \setminus Y$ is a system of sets.
    \end{proposition}
  \end{forthel}


  \section{Unions}

  \begin{forthel}
    \begin{definition}\printlabel{FOUNDATIONS_10_541772562300928}
      Let $X$ be a system of sets.
      The union over $X$ is
      \[ \class{ a | \text{$a \in x$ for some $x \in X$} }. \]
    \end{definition}

    Let $\bigcup X$ stand for the union over $X$.
  \end{forthel}

  \begin{forthel}
    \begin{proposition}\printlabel{FOUNDATIONS_10_4872701241982976}
      \[ \bigcup \emptyset = \emptyset. \]
    \end{proposition}
    \begin{proof}
      $\bigcup \emptyset = \class{ a | \text{$a \in x$ for some
      $x \in \emptyset$} }$.
      $\emptyset$ has no elements.
      Hence there is no object $a$ such that $a \in x$ for some
      $x \in \emptyset$.
      Thus $\bigcup \emptyset = \emptyset$.
    \end{proof}
  \end{forthel}

  \begin{forthel}
    \begin{proposition}\printlabel{FOUNDATIONS_10_2559541585641472}
      Let $x, y$ be sets.
      Then \[ \bigcup \set{x, y} = x \cup y. \]
    \end{proposition}
    \begin{proof}
      Let us show that $\bigcup \set{x, y} \subseteq x \cup y$.
        Let $a \in \bigcup \set{x, y}$.
        Then $a$ is contained in some element of $\set{x, y}$.
        Hence $a \in x$ or $a \in y$.
        Thus $a \in x \cup y$.
      End.

      Let us show that $x \cup y \subseteq \bigcup \set{x, y}$.
        Let $a \in x \cup y$.
        Then $a \in x$ or $a \in y$.
        Hence $a$ is contained in some element of $\set{x, y}$.
        Therefore $a \in \bigcup \set{x, y}$.
      End.
    \end{proof}
  \end{forthel}

  \begin{forthel}
    \begin{corollary}\printlabel{FOUNDATIONS_10_2157223832715264}
      Let $x$ be a set.
      Then \[ \bigcup \set{x} = x. \]
    \end{corollary}
  \end{forthel}


  \section{Intersections}

  \begin{forthel}
    \begin{definition}\printlabel{FOUNDATIONS_10_2659345095458816}
      Let $X$ be a system of sets.
      The intersection over $X$ is
      \[ \class{ a | \text{$a \in x$ for all $x \in X$} }. \]
    \end{definition}

    Let $\bigcap X$ stand for the intersection over $X$.
  \end{forthel}

  \begin{forthel}
    \begin{proposition}\printlabel{FOUNDATIONS_10_2809770322952192}
      $\bigcap \emptyset$ is the class of all objects.
    \end{proposition}
    \begin{proof}
      Define $V = \class{ x | \text{$x$ is an object} }$.
      We have $\bigcap \emptyset \subseteq V$.
      Indeed every element of $\bigcap \emptyset$ is an object.

      Let us show that $V \subseteq \bigcap \emptyset$.
        Let $a \in V$.
        Then $a$ is an object.
        For every $x \in \emptyset$ we have $a \in x$.
        Indeed $\emptyset$ has no elements.
        Thus $a \in \bigcap \emptyset$.
      End.
    \end{proof}
  \end{forthel}

  \begin{forthel}
    \begin{proposition}\printlabel{FOUNDATIONS_10_7851827447988224}
      Let $x, y$ be sets.
      Then \[ \bigcap \set{x, y} = x \cap y. \]
    \end{proposition}
    \begin{proof}
      Let us show that $\bigcap \set{x, y} \subseteq x \cap y$.
        Let $a \in \bigcap \set{x, y}$.
        Then $a$ is contained in every element of $\set{x, y}$.
        Hence $a \in x$ and $a \in y$.
        Thus $a \in x \cap y$.
      End.

      Let us show that $x \cap y \subseteq \bigcap \set{x, y}$.
        Let $a \in x \cap y$.
        Then $a \in x$ and $a \in y$.
        Hence $a$ is contained in every element of $\set{x, y}$.
        Therefore $a \in \bigcap \set{x, y}$.
      End.
    \end{proof}
  \end{forthel}

  \begin{forthel}
    \begin{corollary}\printlabel{FOUNDATIONS_10_7239895674257408}
      Let $x$ be a set.
      Then \[ \bigcap \set{x} = x. \]
    \end{corollary}
  \end{forthel}


  \section{Classes of functions}

  \begin{forthel}
    \begin{definition}\printlabel{FOUNDATIONS_10_5119110467813376}
      Let $x, y$ be sets.
      $[x \to y]$ is the class of all maps from $x$ to $y$.
    \end{definition}

    \begin{proposition}\printlabel{FOUNDATIONS_10_3702893448265728}
      Let $x, y$ be sets.
      Then every element of $[x \to y]$ is a function.
    \end{proposition}
  \end{forthel}


  \section{Axioms for mathematics}

  \begin{forthel}
    \begin{definition}
      Let $A$ be a class and $a$ be an object and $f$ be a map such that
      $A \subseteq \dom(f)$.
      $A$ is inductive regarding $a$ and $f$ iff $a \in A$ and for all $x \in A$
      we have $f(x) \in A$.
    \end{definition}
  \end{forthel}

  \begin{forthel}
    \begin{axiom}[Set existence]\printlabel{FOUNDATIONS_10_2362039748001792}
      There exists a set.
    \end{axiom}
  \end{forthel}

  \begin{forthel}
    \begin{axiom}[Separation]\printlabel{FOUNDATIONS_10_2263707272871936}
      Let $A$ be a class.
      If there exists a set $x$ such that every element of $A$ is contained in
      $x$ then $A$ is a set.
    \end{axiom}
  \end{forthel}

  \begin{forthel}
    \begin{axiom}[Pairing]\printlabel{FOUNDATIONS_10_7376893816864768}
      Let $a, b$ be objects.
      Then $\set{a, b}$ is a set.
    \end{axiom}
  \end{forthel}

  \begin{forthel}
    \begin{axiom}[Union]\printlabel{FOUNDATIONS_10_5536459412996096}
      Let $X$ be a system of sets.
      If $X$ is a set then $\bigcup X$ is a set.
    \end{axiom}
  \end{forthel}

  \begin{forthel}
    \begin{axiom}[Infinity]\printlabel{FOUNDATIONS_10_367388832825344}
      Let $A$ be a class and $a \in A$ and $f : A \to A$.
      Then there exists a subset of $A$ that is inductive regarding $a$ and $f$.
    \end{axiom}
  \end{forthel}

  \begin{forthel}
    \begin{axiom}[Powerset]\printlabel{FOUNDATIONS_10_5862230203564032}
      Let $x$ be a set.
      Then $\pow(x)$ is a set.
    \end{axiom}

    Let the powerset of $x$ stand for $\pow(x)$.
  \end{forthel}

  \begin{forthel}
    \begin{axiom}[Choice]\printlabel{FOUNDATIONS_10_1897613305577472}
      Let $X$ be a system of nonempty sets.
      Then there exists a map $f$ such that $\dom(f) = X$ and $f(x) \in x$ for
      any $x \in X$.
    \end{axiom}
  \end{forthel}

  \begin{forthel}
    \begin{axiom}[Foundation]\printlabel{FOUNDATIONS_10_1320008569323520}
      Let $X$ be a nonempty system of sets.
      Then $X$ has an element $x$ such that $X$ and $x$ are disjoint.
    \end{axiom}
  \end{forthel}

  \begin{forthel}
    \begin{axiom}[Replacement]\printlabel{FOUNDATIONS_10_8142956584239104}
      Let $f$ be a map and $x$ be a set.
      Then $f[x]$ is a set.
    \end{axiom}
  \end{forthel}

  \begin{forthel}
    \begin{axiom}[Function]\printlabel{FOUNDATIONS_10_7781693549182976}
      Let $f$ be a map.
      If $\dom(f)$ is a set then $f$ is a function.
    \end{axiom}
  \end{forthel}


  \section{Consequences of the axioms}

  \begin{forthel}
    \begin{proposition}\printlabel{FOUNDATIONS_10_5891530432708608}
      $\emptyset$ is a set.
    \end{proposition}
    \begin{proof}
      Take a set $x$ (by \cref{FOUNDATIONS_10_2362039748001792}).
      Define $A = \class{y \in x | y \neq y}$.
      Then $A$ is a set (by \cref{FOUNDATIONS_10_2263707272871936}).
      We have $A = \emptyset$.
      Hence $\emptyset$ is a set.
    \end{proof}
  \end{forthel}

  \begin{forthel}
    \begin{proposition}\printlabel{FOUNDATIONS_10_7556516257202176}
      Let $a$ be an object.
      Then $\set{a}$ is a set.
    \end{proposition}
  \end{forthel}

  \begin{forthel}
    \begin{corollary}\printlabel{FOUNDATIONS_10_8408517115379712}
      Let $A$ be a class that has a unique element.
      Then $A$ is a set.
    \end{corollary}
  \end{forthel}

  \begin{forthel}
    \begin{proposition}\printlabel{FOUNDATIONS_10_4052198354845696}
      Let $x, y$ be sets.
      Then $x \cup y$ is a set.
    \end{proposition}
    \begin{proof}
      Take $X = \set{x,y}$.
      Then $X$ is a set.
      Hence $\bigcup X$ is a set (by \cref{FOUNDATIONS_10_5536459412996096}).
      Indeed $X$ is a system of sets.
      We have $x \cup y = \bigcup X$.
      Thus $x \cup y$ is a set.
    \end{proof}
  \end{forthel}

  \begin{forthel}
    \begin{proposition}\printlabel{FOUNDATIONS_10_4475839687163904}
      Let $x, y$ be sets.
      Then $x \cap y$ is a set.
    \end{proposition}
    \begin{proof}
      We have $x \cap y \subseteq x$.
      Hence $x \cap y$ is a set (by \cref{FOUNDATIONS_10_2263707272871936}).
    \end{proof}
  \end{forthel}

  \begin{forthel}
    \begin{proposition}\printlabel{FOUNDATIONS_10_7795203882614784}
      Let $x, y$ be sets.
      Then $x \setminus y$ is a set.
    \end{proposition}
    \begin{proof}
      We have $x \setminus y \subseteq x$.
      Hence $x \setminus y$ is a set (by \cref{FOUNDATIONS_10_2263707272871936}).
    \end{proof}
  \end{forthel}

  \begin{forthel}
    \begin{proposition}\printlabel{FOUNDATIONS_10_4458706448154624}
      Let $x, y$ be sets.
      Then $x \times y$ is a set.
    \end{proposition}
    \begin{proof}
      $\set{a}$ and $\set{a,b}$ are sets for each $a \in x$ and each $b \in y$.
      Define $P = \class{\set{\set{a}, \set{a,b}} | \text{$a \in x$ and
      $b \in y$}}$.

      (1) $P$ is a set. \\
      Proof.
        Let us show that $P \subseteq \pow(\pow(x \cup y))$.
          Let $p \in P$.
          Consider $a \in x$ and $b \in y$ such that
          $p = \set{\set{a},\set{a,b}}$.
          Then $a, b \in x \cup y$.
          Hence $\set{a}, \set{a,b} \in \pow(x \cup y)$.
          Thus $\set{\set{a}, \set{a,b}} \in \pow(\pow(x \cup y))$.
        End.

        $x \cup y$ is a set.
        Consequently $\pow(\pow(x \cup y))$ is a set (by
        \cref{FOUNDATIONS_10_5862230203564032}).
        Therefore $P$ is a set (by \cref{FOUNDATIONS_10_2263707272871936}).
      Qed.

      Define $l(p) =$ ``choose $a \in x$, choose $b \in y$ such that
      $p = \set{\set{a}, \set{a,b}}$ in $a$'' for $p \in P$.
      Define $r(p) =$ ``choose $a \in x$, choose $b \in y$ such that
      $p = \set{\set{a}, \set{a,b}}$ in $b$'' for $p \in P$.

      Define $f(p) = (l(p), r(p))$ for $p \in P$.

      Let us show that for any objects $u, u', v, v'$ if
      $\set{\set{u}, \set{u,v}} = \set{\set{u'}, \set{u',v'}}$ then $u = u'$ and
      $v = v'$.
        Let $u, u', v, v'$ be objects.
        Assume $\set{\set{u}, \set{u,v}} = \set{\set{u'}, \set{u',v'}}$.
        Then ($\set{u} = \set{u'}$ or $\set{u} = \set{u',v'}$) and
        ($\set{u,v} = \set{u'}$ or $\set{u,v} = \set{u',v'}$).
        Thus ($\set{u} = \set{u'}$ and ($\set{u,v} = \set{u'}$ or
        $\set{u,v} = \set{u',v'}$)) or ($\set{u} = \set{u',v'}$ and
        ($\set{u,v} = \set{u'}$ or $\set{u,v} = \set{u',v'}$)).

        Case $\set{u} = \set{u'}$ and ($\set{u,v} = \set{u'}$ or
          $\set{u,v} = \set{u',v'}$).
          We have $\set{u} = \set{u'}$.
          Hence $u = u'$.

          Case $\set{u,v} = \set{u'}$.
            Then $u = u'= v$.
            Hence $\set{\set{u}, \set{u,u}} = \set{\set{u}, \set{u,v'}}$ (by 1).
            Thus $\set{\set{u}} = \set{\set{u}, \set{u,v'}}$.
            Therefore $\set{u} = \set{u,v'}$.
            Consequently $v' = u = v$.
          End.

          Case $\set{u,v} = \set{u',v'}$.
            Then $\set{u,v} = \set{u,v'}$.
            Hence $v = v'$.
          End.
        End.

        Case $\set{u} = \set{u',v'}$ and ($\set{u,v} = \set{u'}$ or
        $\set{u,v} = \set{u',v'}$).
          We have $\set{u} = \set{u',v'}$.
          Hence $u = u'$.

          Case $\set{u,v} = \set{u'}$.
            Then $u = v = u'$.
            Hence $v = v'$.
          End.

          Case $\set{u,v} = \set{u',v'}$.
            Then $\set{u,v} = \set{u,v'}$.
            Hence $v = v'$.
          End.
        End.
      End.

      Let us show that for any $a \in x$ and any $b \in y$ we have
      $f(\set{\set{a}, \set{a,b}}) = (a,b)$.
        Let $a \in x$ and $b \in y$.
        Take $p = \set{\set{a}, \set{a,b}}$.
        Then $p$ is a set.
        Then we can choose $a' \in x$ and $b' \in y$ such that $p =
        \set{\set{a'}, \set{a',b'}}$ and $l(p) = a'$.
        Then $a = a'$ and $b = b'$.
        Hence $l(p) = a$.
        Choose $a'' \in x$ and $b'' \in y$ such that $p =
        \set{\set{a''}, \set{a'',b''}}$ and $r(p) = b''$.
        Then $a = a''$ and $b = b''$.
        Thus $r(p) = b$.
        Therefore $f(p) = (a,b)$.
      End.

      (2) $x \times y = f[P]$. \\
      Proof.
        For all $p \in P$ we have $l(p) \in x$ and $r(p) \in y$.
        Hence $f(p) \in x \times y$ for all $p \in P$.
        Therefore $f[P] \subseteq x \times y$.

        Let us show that $x \times y \subseteq f[P]$.
          Let $z \in x \times y$.
          Take $a \in x$ and $b \in y$ such that $z = (a,b)$.
          Then $(a,b) = f(\set{\set{a}, \set{a,b}})$.
          Hence there exists a $p \in P$ such that $(a,b) = f(p)$.
          Thus $(a,b) \in f[P]$.
        End.

        Consequently $x \times y = f[P]$.
      Qed.

      Thus $x \times y$ is the image of some set under some map.
      Therefore $x \times y$ is a set (by \cref{FOUNDATIONS_10_8142956584239104}).
    \end{proof}
  \end{forthel}

  \begin{forthel}
    \begin{proposition}\printlabel{FOUNDATIONS_10_5486815207227392}
      Let $X$ be a nonempty system of sets.
      Then $\bigcap X$ is a set.
    \end{proposition}
    \begin{proof}
      Take an element $x$ of $X$.
      Then $\bigcap X \subseteq x$.
      Hence $\bigcap X$ is a set (by \cref{FOUNDATIONS_10_2263707272871936}).
    \end{proof}
  \end{forthel}

  \begin{forthel}
    \begin{proposition}\printlabel{FOUNDATIONS_10_7598384349184000}
      Let $f$ be a map such that $\dom(f)$ is a set.
      Then $\range(f)$ is a set.
    \end{proposition}
    \begin{proof}
      $\range(f) = f_{*}(\dom(f))$ and $f_{*}(\dom(f))$ is a set.
      Hence $\range(f)$ is a set (by \cref{FOUNDATIONS_10_8142956584239104}).
    \end{proof}
  \end{forthel}

  \begin{forthel}
    \begin{proposition}\printlabel{FOUNDATIONS_10_8631339572002816}
      Let $A$ be a class and $x$ be a set.
      Assume that there exists a map from $A$ into $x$.
      Then $A$ is a set.
    \end{proposition}
    \begin{proof}
      Consider a map $f$ from $A$ into $x$.
      Then $f^{-1}$ is a bijection between $\range(f)$ and $A$.
      $\range(f)$ is a set and $A$ is the image of $\range(f)$ under $f^{-1}$.
      Thus $A$ is a set (by \cref{FOUNDATIONS_10_8142956584239104}).
    \end{proof}
  \end{forthel}

  \begin{forthel}
    \begin{proposition}\printlabel{FOUNDATIONS_10_8812282138066944}
      There exist no sets $x, y$ such that $x \in y$ and $y \in x$.
    \end{proposition}
    \begin{proof}
      Assume the contrary.
      Take sets $x,y$ such that $x \in y$ and $y \in x$.
      Consider an element $z$ of $\set{x,y}$ such that $\set{x,y}$ and $z$ are
      disjoint (by \cref{FOUNDATIONS_10_1320008569323520}).
      Indeed $\set{x,y}$ is a nonempty system of sets.
      Then we have $z = x$ or $z = y$.

      Case $z = x$.
        Then $x$ and $\set{x,y}$ are disjoint.
        Hence $y \notin x$.
        Contradiction.
      End.

      Case $z = y$.
        Then $y$ and $\set{x,y}$ are disjoint.
        Hence $x \notin y$.
        Contradiction.
      End.
    \end{proof}
  \end{forthel}

  \begin{forthel}
    \begin{corollary}\printlabel{FOUNDATIONS_10_3086917813927936}
      Let $x$ be a set.
      Then $x \notin x$.
    \end{corollary}
  \end{forthel}

  \begin{forthel}
    \begin{proposition}\printlabel{FOUNDATIONS_10_4105036244189184}
      Let $x, y$ be sets.
      Then $[x \to y]$ is a set.
    \end{proposition}
    \begin{proof}
      Define $R = \{ F \in \pow(x \times y) \mid$ (for all $a \in x$ there
      exists a $b \in y$ such that $(a,b) \in F$) and for all $a \in x$ and
      all $b, b' \in y$ such that $(a,b), (a,b') \in F$ we have $b = b' \}$.

      [skipfail on] % Proof task seems to be too difficult for Naproche %!!
      Define $h(F) = \fun a \in x.$ choose $b \in y$ such that
      $(a,b) \in F$ in $b$ for $F \in R$.
      [skipfail off]

      Let us show that $[x \to y] \subseteq \range(h)$.
        Let $f \in [x \to y]$.
        Define $F = \{ (a,f(a)) \mid a \in x \}$.

        Then $F \in R$. \\
        Proof.
          Define $g(a) = (a,f(a))$ for $a \in x$.
          Then $F = \range(g)$.
          Hence $F$ is a set.
          Thus $F \in \pow(x \times y)$.
          Indeed $F \subseteq x \times y$.

          (1) For all $a \in x$ there exists a $b \in y$ such that $(a,b) \in F$.

          (2) For all $a \in x$ and all $b, b' \in y$ such that $(a,b),
          (a,b') \in F$ we have $b = b'$.
        End.

        We have $\dom(f) = x = \dom(h(F))$.
        For each $a \in x$ we have $h(F)(a) = f(a)$.
        Hence $f = h(F)$.
        Thus $f \in \range(h)$.
      End.

      Therefore $[x \to y]$ is a set.
      Indeed Then $R$ is a set.
    \end{proof}
  \end{forthel}
\end{document}
