\documentclass[../set-theory.tex]{subfiles}

\begin{document}
  \chapter{Invertible maps and involutions}\label{invertible-maps}

  \readftl{foundations/sections/09_invertible-maps.ftl.tex}

  \begin{forthel}
    %[prove off][check off]

    [readtex \path{foundations/sections/08_injections-surjections-bijections.ftl.tex}]

    %[prove on][check on]
  \end{forthel}


  \section{Invertible maps}

  \begin{forthel}
    \begin{definition}\printlabel{SET_THEORY_09_7776974319648768}
      Let $f$ be a map.
      An inverse of $f$ is a map $g$ from $\range(f)$ to $\dom(f)$ such that
      \[ f(a) = b \iff g(b) = a \]
      for all $a \in \dom(f)$ and all $b \in \dom(g)$.
    \end{definition}
  \end{forthel}

  \begin{forthel}
    \begin{definition}\printlabel{SET_THEORY_09_3430350086733824}
      Let $f$ be a map.
      $f$ is invertible iff $f$ has an inverse.
    \end{definition}
  \end{forthel}

  \begin{forthel}
    \begin{lemma}\printlabel{SET_THEORY_09_5108611793551360}
      Let $f$ be a map and $g, g'$ be inverses of $f$.
      Then $g = g'$.
    \end{lemma}
    \begin{proof}
      We have $\dom(g) = \range(f) = \dom(g')$.

      Let us show that $g(b) = g'(b)$ for all $b \in \range(f)$.
        Let $b \in \range(f)$.
        Take $a = g'(b)$.
        Then $g(b) = a$ iff $f(a) = b$.
        We have $f(a) = b$ iff $g'(b) = a$.
        Thus $g(b) = g'(b)$.
      End.
    \end{proof}
  \end{forthel}

  \begin{forthel}
    \begin{definition}\printlabel{SET_THEORY_09_6458627204317184}
      Let $f$ be an invertible map.
      $f^{-1}$ is the inverse of $f$.
    \end{definition}

    Let $f$ is involutory stand for $f$ is the inverse of $f$.
    Let $f$ is selfinverse stand for $f$ is the inverse of $f$.
  \end{forthel}


  \section{Some basic facts about invertible maps}

  \begin{forthel}
    \begin{proposition}\printlabel{SET_THEORY_09_7840743571849216}
      Let $A, B$ be classes and $f : A \onto B$ and $g : B \onto A$.
      Then $g$ is the inverse of $f$ iff $g \circ f = \id_{A}$ and $f \circ g =
      \id_{B}$.
    \end{proposition}
    \begin{proof}
      Case $g$ is the inverse of $f$.
        We have $\dom(g \circ f) = \dom(f) = A = \dom(\id_{A})$.
        For all $a \in A$ we have $(g \circ f)(a) = g(f(a)) = a$.
        Hence $g \circ f = \id_{A}$.

        We have $\dom(f \circ g) = \dom(g) = B = \dom(\id_{B})$.
        For all $b \in B$ we have $(f \circ g)(b) = f(g(b)) = b$.
        Hence $f \circ g = \id_{B}$.
      End.

      Case $g \circ f = \id_{A}$ and $f \circ g = \id_{B}$.
        Then $\dom(g) = B = \range(f)$ and $\range(g) = A = \dom(f)$.
        Let $a \in \dom(f)$ and $b \in \dom(g)$.
        If $f(a) = b$ then $g(b) = g(f(a)) = (g \circ f)(a) = \id_{A}(a) = a$.
        If $g(b) = a$ then $f(a) = f(g(b)) = (f \circ g)(b) = \id_{B}(b) = b$.
        Hence $f(a) = b$ iff $g(b) = a$.
      End.
    \end{proof}
  \end{forthel}

  \begin{forthel}
    \begin{proposition}\printlabel{SET_THEORY_09_8414736098000896}
      Let $A, B$ be classes and $f : A \onto B$.
      Assume that $f$ is invertible.
      Then $f^{-1}$ is an invertible map from $B$ onto $A$ such that
      \[ (f^{-1})^{-1} = f. \]
    \end{proposition}
    \begin{proof}
      $f^{-1}$ is a map from $B$ to $A$.
      Indeed $\range(f) = B$ and $\dom(f) = A$.
      $f^{-1}$ is a map onto $A$.
      Indeed for any $a \in A$ we have $f^{-1}(f(a)) = a$.
      $f^{-1}$ is the inverse of $f$.
      Thus $f \circ f^{-1} = \id_{B}$ and $f^{-1} \circ f = \id_{A}$.
      Therefore $f$ is the inverse of $f^{-1}$.
    \end{proof}
  \end{forthel}

  \begin{forthel}
    \begin{proposition}\printlabel{SET_THEORY_09_4577560740495360}
      Let $A, B$ be classes and $f : A \onto B$.
      Assume that $f$ is invertible.
      Then \[ f \circ f^{-1} = \id_{B} \] and \[ f^{-1} \circ f = \id_{A}. \]
    \end{proposition}
    \begin{proof}
      $f^{-1}$ is a map from $B$ onto $A$ .
      $f^{-1}$ is the inverse of $f$.
    \end{proof}
  \end{forthel}

  \begin{forthel}
    \begin{proposition}\printlabel{SET_THEORY_09_4606651604664320}
      Let $A, B$ be classes and $f : A \onto B$ and $a \in A$.
      Assume that $f$ is invertible.
      Then \[ f^{-1}(f(a)) = a. \]
    \end{proposition}
    \begin{proof}
      We have $f^{-1}(f(a)) = (f^{-1} \circ f)(a) = \id_{A}(a) = a$.
    \end{proof}

    \begin{proposition}
      Let $A, B$ be classes and $f : A \onto B$ and $b \in B$.
      Assume that $f$ is invertible.
      Then \[ f(f^{-1}(b)) = b. \]
    \end{proposition}
    \begin{proof}
      We have $f(f^{-1}(b)) = (f \circ f^{-1})(b) = \id_{B}(b) = b$.
    \end{proof}
  \end{forthel}

  \begin{forthel}
    \begin{proposition}\printlabel{SET_THEORY_09_7619151963095040}
      Let $A, B, C$ be classes and $f : A \onto B$ and $g : B \onto C$.
      Assume that $f$ and $g$ are invertible.
      Then $g \circ f$ is invertible and
      \[ (g \circ f)^{-1} = f^{-1} \circ g^{-1}. \]
    \end{proposition}
    \begin{proof}
      $f^{-1}$ is a map from $B$ onto $A$.
      $g^{-1}$ is a map from $C$ onto $B$.
      Take $h = f^{-1} \circ g^{-1}$.
      Then $h$ is a map from $C$ onto $A$.
      $g \circ f$ is a map from $A$ to $C$.

      Let us show that $((g \circ f) \circ h) = \id_{C}$.
        We have $f \circ (f^{-1} \circ g^{-1}) = (f \circ f^{-1}) \circ g^{-1}$.
        Indeed $f \circ (f^{-1} \circ g^{-1})$ and $(f \circ f^{-1}) \circ g^{-1}$ are maps from $C$.
        $f \circ h$ is a map from $C$ to $B$.
        Hence
        \[ (g \circ f) \circ h = \]
        \[ g \circ (f \circ h) = \]
        \[ g \circ (f \circ (f^{-1} \circ g^{-1})) = \]
        \[ g \circ ((f \circ f^{-1}) \circ g^{-1}) = \]
        \[ g \circ (\id_{B} \circ g^{-1}) = \]
        \[ g \circ g^{-1} = \]
        \[ \id_{C}. \]
      End.

      Let us show that $h \circ (g \circ f) = \id_{A}$.
        We have $(f^{-1} \circ g^{-1}) \circ g = f^{-1} \circ (g^{-1} \circ g)$.
        $g \circ f$ is a map from $A$ to $C$.
        Hence
        \[ h \circ (g \circ f) = \]
        \[ (h \circ g) \circ f = \]
        \[ ((f^{-1} \circ g^{-1}) \circ g) \circ f = \]
        \[ (f^{-1} \circ (g^{-1} \circ g)) \circ f = \]
        \[ (f^{-1} \circ \id_{B}) \circ f = \]
        \[ f^{-1} \circ f = \]
        \[ \id_{A}. \]
      End.

      Thus $h$ is the inverse of $g \circ f$.
      Indeed $g \circ f$ is a map from $A$ onto $C$ and $h$ is a map from $C$ onto $A$.
    \end{proof}
  \end{forthel}

  \begin{forthel}
    \begin{proposition}\printlabel{SET_THEORY_09_6374884963778560}
      Let $A, B$ be classes and $f : A \onto B$ and $X \subseteq A$.
      Assume that $f$ is invertible.
      Then $f \restriction X$ is invertible and
      \[ (f\restriction X)^{-1} = f^{-1} \restriction (f_{*}(X)). \]
    \end{proposition}
    \begin{proof}
      $f \restriction X$ is a map from $X$ onto $f_{*}(X)$.
      Take $g = f^{-1} \restriction (f_{*}(X))$.
      Then $g$ is a map from $f_{*}(X)$.

      Let us show that $X \subseteq \range(g)$.
        Let $a \in X$.
        Then $f(a) \in f_{*}(X)$.
        Hence $g(f(a)) = f^{-1}(f(a)) = a$.
        Thus $a$ is a value of $g$.
      End.

      Let us show that $\range(g) \subseteq X$.
        Let $a \in \range(g)$.
        Take $b \in f_{*}(X)$ such that $a = g(b)$.
        Take $c \in X$ such that $b = f(c)$.
        Then $a = (f^{-1} \restriction (f_{*}(X)))(b) = f^{-1}(b) = f^{-1}(f(c)) = c$.
        Hence $a \in X$.
      End.

      Hence $\range(g) = X$.
      Thus $g$ is a map onto $X$.

      Let us show that $g((f \restriction X)(a)) = a$ for all $a \in X$.
        Let $a \in X$.
        Then $g((f \restriction X)(a)) = g(f(a)) = (f^{-1} \restriction (f_{*}(X)))(f(a)) = f^{-1}(f(a)) = a$.
      End.

      Let us show that $((f \restriction X)(g(b))) = b$ for all $b \in f_{*}(X)$.
        Let $b \in f_{*}(X)$.
        Take $a \in X$ such that $b = f(a)$.
        We have $g(b) = g(f(a)) = (f^{-1} \restriction (f_{*}(X)))(f(a)) = f^{-1}(f(a)) = a$.
        Hence $(f \restriction X)(g(b)) = (f \restriction X)(a) = f(a) = b$.
      End.

      Thus $g \circ (f \restriction X) = \id_{X}$ and $(f \restriction X) \circ g = \id_{f_{*}(X)}$.
      Therefore $g$ is the inverse of $f \restriction X$.
    \end{proof}
  \end{forthel}

  \begin{forthel}
    \begin{proposition}\printlabel{SET_THEORY_09_7726021377785856}
      Let $A, B$ be classes and $f : A \onto B$ and $Y \subseteq B$.
      Assume that $f$ is invertible.
      Then \[ f^{*}(Y) = (f^{-1})_{*}(Y). \]
    \end{proposition}
    \begin{proof}
      We have $(f^{-1})_{*}(Y) = \class{ f^{-1}(b) | b \in Y }$ and $f^{*}(Y) = \class{ a \in A | f(a) \in Y }$.

      Let us show that $f^{*}(Y) \subseteq (f^{-1})_{*}(Y)$.
        Let $a \in f^{*}(Y)$.
        Take $b \in Y$ such that $b = f(a)$.
        Then $f^{-1}(b) = f^{-1}(f(a)) = a$.
        Hence $a \in (f^{-1})_{*}(Y)$.
      End.

      Let us show that $f^{-1}_{*}(Y) \subseteq f^{*}(Y)$.
        Let $a \in f^{-1}_{*}(Y)$.
        Take $b \in Y$ such that $a = f^{-1}(b)$.
        Then $f(a) = f(f^{-1}(b)) = b$.
        Hence $a \in f^{*}(Y)$.
      End.
    \end{proof}
  \end{forthel}

  \begin{forthel}
    \begin{corollary}\printlabel{SET_THEORY_09_8607784268464128}
      Let $A, B$ be classes and $f : A \onto B$ and $b \in B$.
      Assume that $f$ is invertible.
      Then \[ f^{*}(\set{b}) = \set{f^{-1}(b)}. \]
    \end{corollary}
    \begin{proof}
      $f^{*}(\set{b}) = f^{-1}_{*}(\set{b})$.
      We have $f^{-1}_{*}(\set{b}) = \class{ f^{-1}(c) | c \in \set{b} }$.
      Hence $f^{-1}_{*}(\set{b}) = \set{f^{-1}(b)}$.
    \end{proof}
  \end{forthel}

  \begin{forthel}
    \begin{proposition}\printlabel{SET_THEORY_09_6777575974109184}
      Let $A, B$ be classes and $f : A \onto B$.
      Then $f$ is invertible iff $f$ is injective.
    \end{proposition}
    \begin{proof}
      Case $f$ is invertible.
        Let $a, b \in A$.
        Assume $f(a) = f(b)$.
        Then $a = f^{-1}(f(a)) = f^{-1}(f(b)) = b$.
      End.

      Case $f$ is injective.
        Define $g(b) =$ ``choose $a \in A$ such that $f(a) = b$ in $a$'' for
        $b \in B$.
        Then $g$ is a map from $B$ to $A$.
        For all $a \in A$ we have $a = g(f(a))$.
        Hence $g$ is a map from $B$ onto $A$.
        For all $a \in A$ we have $g(f(a)) = a$.
        For all $b \in B$ we have $f(g(b)) = b$.
        Hence $g$ is the inverse of $f$.
      End.
    \end{proof}
  \end{forthel}

  \begin{forthel}
    \begin{corollary}\printlabel{SET_THEORY_09_5708971514003456}
      Let $A, B$ be classes and $f : A \onto B$.
      Assume that $f$ is invertible.
      Then $f^{-1}$ is a bijection between $B$ and $A$.
    \end{corollary}
    \begin{proof}
      $f^{-1}$ is a map from $B$ onto $A$.
      $f^{-1}$ is invertible.
      Hence $f^{-1}$ is injective.
      Thus $f^{-1}$ is a map from $B$ into $A$.
      Therefore $f^{-1}$ is a bijection between $B$ and $A$.
    \end{proof}
  \end{forthel}


  \section{Involutions}

  \begin{forthel}
    \begin{definition}\printlabel{SET_THEORY_09_7282039688527872}
      Let $A$ be a class.
      An involution on $A$ is a selfinverse map $f$ on $A$.
    \end{definition}
  \end{forthel}

  \begin{forthel}
    \begin{proposition}\printlabel{SET_THEORY_09_7944474185433088}
      Let $A$ be a class.
      $\id_{A}$ is an involution on $A$.
    \end{proposition}
    \begin{proof}
      We have $\id_{A} \circ \id_{A} = \id_{A}$.
      Hence $\id_{A}$ is selfinverse.
    \end{proof}
  \end{forthel}

  \begin{forthel}
    \begin{proposition}\printlabel{SET_THEORY_09_6897019612299264}
      Let $A$ be a class and $f, g$ be involutions on $A$.
      Then $g \circ f$ is an involution on $A$ iff $g \circ f = f \circ g$.
    \end{proposition}
    \begin{proof}
      Case $g \circ f$ is an involution on $A$.
        Then $(g \circ f)^{-1} = f^{-1} \circ g^{-1} = f \circ g$.
      End.

      Case $g \circ f = f \circ g$.
        $f \circ f$, $f \circ g$ and $f \circ g$ are maps on $A$.
        Hence
        \[ (g \circ f) \circ (g \circ f) = \]
        \[ (g \circ f) \circ (f \circ g) = \]
        \[ ((g \circ f) \circ f) \circ g = \]
        \[ (g \circ (f \circ f)) \circ g = \]
        \[ (g \circ \id_{A}) \circ g = \]
        \[ g \circ g = \]
        \[ \id_{A}. \]
        Thus $g \circ f$ is selfinverse.
      End.
    \end{proof}
  \end{forthel}

  \begin{forthel}
    \begin{corollary}\printlabel{SET_THEORY_09_5958206868160512}
      Let $A$ be a class and $f$ be an involutions on $A$.
      Then $f \circ f$ is an involution on $A$.
    \end{corollary}
  \end{forthel}

  \begin{forthel}
    \begin{proposition}\printlabel{SET_THEORY_09_2314262743613440}
      Let $A$ be a class and $f$ be an involution on $A$.
      Then $f$ is a permutation of $A$.
    \end{proposition}
    \begin{proof}
      $f$ is an invertible map from $A$ onto $A$.
      Hence $f$ is a bijection between $A$ and $A$.
      Thus $f$ is a permutation of $A$.
    \end{proof}
  \end{forthel}
\end{document}
