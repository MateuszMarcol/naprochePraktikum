\documentclass[../../set-theory/set-theory.tex]{subfiles}

\begin{document}
  \chapter{Classes}\label{chapter:classes}

  \filename{foundations/sections/01_classes.ftl.tex}


  \section{Preliminaries}

  \begin{forthel}
    [readtex \path{axioms.ftl.tex}]

    [readtex \path{macros.ftl.tex}]

    [readtex \path{vocabulary.ftl.tex}]
  \end{forthel}


  \section{Sub- and superclasses}

  \begin{forthel}
    \begin{definition}\printlabel{FOUNDATIONS_01_3275578358628352}
      Let $A$ be a class.
      A subclass of $A$ is a class $B$ such that every element of $B$ is an
      element of $A$.
    \end{definition}

    Let $B \subseteq A$ stand for $B$ is a subclass of $A$.
    Let $B \subset A$ stand for $B \subseteq A$.

    Let a superclass of $B$ stand for a class $A$ such that $B \subseteq A$.
    Let $B \supseteq A$ stand for $B$ is a superclass of $A$.
    Let $B \supset A$ stand for $B \subseteq A$.

    Let a proper subclass of $A$ stand for a subclass $B$ of $A$ such that
    $B \neq A$.
    Let $B \subsetneq A$ stand for $B$ is a proper subclass of $A$.

    Let a proper superclass of $B$ stand for a superclass $A$ of $B$ such that
    $A \neq B$.
    Let $B \supsetneq A$ stand for $B$ is a proper superclass of $A$.

    Let $A$ includes $B$ stand for $B \subseteq A$.
    Let $B$ is included in $A$ stand for $B \subseteq A$.
  \end{forthel}

  \begin{forthel}
    \begin{proposition}\printlabel{FOUNDATIONS_01_5994555614691328}
      Let $A$ be a class.
      Then \[ A \subseteq A. \]
    \end{proposition}
    \begin{proof}
      Every element of $A$ is contained in $A$.
      Therefore $A \subseteq A$.
    \end{proof}
  \end{forthel}

  \begin{forthel}
    \begin{proposition}\printlabel{FOUNDATIONS_01_3939677545431040}
      Let $A, B, C$ be classes.
      Then \[ (\text{$A \subseteq B$ and $B \subseteq C$}) \implies
      A \subseteq C. \]
    \end{proposition}
    \begin{proof}
      Assume $A \subseteq B$ and $B \subseteq C$.
      Then every element of $A$ is contained in $B$ and every element of $B$ is
      contained in $C$.
      Hence every element of $A$ is contained in $C$.
      Thus $A \subseteq C$.
    \end{proof}
  \end{forthel}

  \begin{forthel}
    \begin{proposition}\printlabel{FOUNDATIONS_01_7159957847801856}
      Let $A, B$ be classes.
      Then \[ (\text{$A \subseteq B$ and $B \subseteq A$}) \implies A = B. \]
    \end{proposition}
    \begin{proof}
      Assume $A \subseteq B$ and $B \subseteq A$.
      Then every element of $A$ is contained in $B$ and every element of $B$ is
      contained in $A$.
      Hence $A = B$.
    \end{proof}
  \end{forthel}


  \section{The empty class}

  \begin{forthel}
    \begin{definition}\printlabel{FOUNDATIONS_01_6252477624090624}
      Let $A$ be a class.
      $A$ is empty iff $A$ has no elements.
    \end{definition}

    Let $A$ is nonempty stand for $A$ is not empty.
  \end{forthel}

  \begin{forthel}
    \begin{definition}\printlabel{FOUNDATIONS_01_7939928493129728}
      \[ \emptyset = \class{ x | x \neq x }. \]
    \end{definition}
  \end{forthel}

  \begin{forthel}
    \begin{proposition}\printlabel{FOUNDATIONS_01_2263153161273344}
      Let $A$ be a class.
      $A$ is empty iff $A = \emptyset$.
    \end{proposition}
    \begin{proof}
      We can show that $\emptyset$ is empty.
      Indeed any element $x$ of $\emptyset$ is nonequal to $x$.
      Hence if $A = \emptyset$ then $A$ is empty.
      If $A$ is empty then $A$ and $\emptyset$ have no elements.
      Hence if $A$ is empty then $A \subseteq \emptyset$ and
      $\emptyset \subseteq A$.
      Thus if $A$ is empty then $A = \emptyset$.
    \end{proof}
  \end{forthel}

  \begin{forthel}
    \begin{corollary}\printlabel{FOUNDATIONS_01_1495141426659328}
      $\emptyset$ is empty.
    \end{corollary}
  \end{forthel}

  \begin{forthel}
    \begin{corollary}\printlabel{FOUNDATIONS_01_6931785090859008}
      Let $A$ be a class.
      Then \[ \emptyset \subseteq A. \]
    \end{corollary}
    \begin{proof}
      $\emptyset$ has no elements.
      Hence every element of $\emptyset$ is contained in $A$.
    \end{proof}
  \end{forthel}


  \section{Unordered pairs}

  \begin{forthel}
    \begin{definition}\printlabel{FOUNDATIONS_01_3471035364016128}
      Let $a, b$ be objects.
      The unordered pair of $a$ and $b$ is
      \[ \class{ x | \text{$x = a$ or $x = b$} }. \]
    \end{definition}

    Let $\set{a, b}$ stand for the unordered pair of $a$ and $b$.
  \end{forthel}

  \begin{forthel}
    \begin{definition}\printlabel{FOUNDATIONS_01_1160414603771904}
      Let $a$ be an object.
      The singleton class of $a$ is
      \[ \class{ x | x = a }. \]
    \end{definition}

    Let $\set{a}$ stand for the singleton class of $a$.
  \end{forthel}

  \begin{forthel}
    \begin{proposition}\printlabel{FOUNDATIONS_01_6125259604361216}
      Let $a, a', b, b'$ be objects.
      Assume $\set{a, b} = \set{a', b'}$.
      Then ($a = a'$ and $b = b'$) or ($a = b'$ and $b = a'$).
    \end{proposition}
    \begin{proof}
      We have $a = a'$ or $a = b'$.
      If $a = a'$ then $b = b'$.
      If $a = b'$ then $b = a'$.
      Hence ($a = a'$ and $b = b'$) or ($a = b'$ and $b = a'$).
    \end{proof}
  \end{forthel}

  \begin{forthel}
    \begin{corollary}\printlabel{FOUNDATIONS_01_6954678910713856}
      Let $a, a'$ be objects.
      Then \[ \set{a} = \set{a'} \implies a = a'. \]
    \end{corollary}
  \end{forthel}

  \begin{forthel}
    \begin{definition}
      Let $A$ be a class.
      A unique element of $A$ is an element $a$ of $A$ such that for each
      $x \in A$ we have $x = a$.
    \end{definition}
  \end{forthel}

  \begin{forthel}
    \begin{proposition}
      Let $A$ be a class.
      Then $A$ has a unique element iff $A = \set{a}$ for some object $a$.
    \end{proposition}
  \end{forthel}


  \section{Unions, intersections, complements}

  \begin{forthel}
    \begin{definition}\printlabel{FOUNDATIONS_01_2159753924968448}
      Let $A, B$ be classes.
      The union of $A$ and $B$ is
      \[ \class{ x | \text{$x \in A$ or $x \in B$} }. \]
    \end{definition}

    Let $A \cup B$ stand for the union of $A$ and $B$.
  \end{forthel}

  \begin{forthel}
    \begin{definition}\printlabel{FOUNDATIONS_01_5744033011859456}
      Let $A, B$ be classes.
      The intersection of $A$ and $B$ is
      \[ \class{ x | \text{$x \in A$ and $x \in B$} }. \]
    \end{definition}

    Let $A \cap B$ stand for the intersection of $A$ and $B$.
  \end{forthel}

  \begin{forthel}
    \begin{definition}\printlabel{FOUNDATIONS_01_7620345041256448}
      Let $A, B$ be classes.
      The complement of $B$ in $A$ is
      \[ \class{ x | \text{$x \in A$ and $x \notin B$} }. \]
    \end{definition}

    Let $A \setminus B$ stand for the complement of $B$ in $A$.
  \end{forthel}


  \section{Disjoint classes}

  \begin{forthel}
    \begin{definition}\printlabel{FOUNDATIONS_01_4981913324355584}
      Let $A, B$ be classes.
      $A$ and $B$ are disjoint iff $A$ and $B$ have no common elements.
    \end{definition}
  \end{forthel}

  \begin{forthel}
    \begin{proposition}\printlabel{FOUNDATIONS_01_1211191546347520}
      Let $A, B$ be classes.
      Then $A$ and $B$ are disjoint iff $A \cap B$ is empty.
    \end{proposition}
  \end{forthel}
\end{document}
