\documentclass[../set-theory.tex]{subfiles}

\begin{document}
  \chapter{Surjections, injections and bijections}\label{injections-surjections-bijections}

  \readftl{foundations/sections/08_injections-surjections-bijections.ftl.tex}

  \begin{forthel}
    %[prove off][check off]

    [readtex \path{foundations/sections/06_maps.ftl.tex}]

    %[prove on][check on]
  \end{forthel}


  \section{Surjective maps}

  \begin{forthel}
    \begin{definition}\printlabel{FOUNDATIONS_08_8681187805495296}
      Let $B$ be a class.
      A map onto $B$ is a map $f$ such that $\range(f) = B$.
    \end{definition}

    Let $f$ surjects onto $B$ stand for $\range(f) = B$.
  \end{forthel}

  \begin{forthel}
    \begin{definition}\printlabel{FOUNDATIONS_08_4611935769198592}
      Let $A, B$ be classes.
      A map from $A$ onto $B$ is a map from $A$ that is a map onto $B$.
    \end{definition}

    Let $f: A \onto B$ stand for $f$ is a map from $A$ onto $B$.
    Let $f$ maps $A$ onto $B$ stand for $f$ is a map from $A$ onto $B$.
  \end{forthel}

  \begin{forthel}
    \begin{proposition}\printlabel{FOUNDATIONS_08_1974205941809152}
      Let $B$ be a class and $f$ be a map to $B$.
      $f$ is a map onto $B$ iff every element of $B$ is a value of $f$.
    \end{proposition}
    \begin{proof}
      Case $f$ is a map onto $B$.
        Then $B = \range(f)$.
        Let $b$ be an element of $B$.
        Then $b \in \range(f)$.
        Hence $b$ is a value of $f$.
      End.

      Case every element of $B$ is a value of $f$.
        Let us show that $B \subseteq \range(f)$.
          Let $b \in B$.
          Then $b$ is a value of $f$.
          Hence $b \in \range(f)$.
        End.

        Let us show that $\range(f) \subseteq B$.
          Let $b \in \range(f)$.
          Then $b$ is a value of $f$.
          Hence $b \in B$.
        End.
      End.
    \end{proof}
  \end{forthel}


  \section{Injective maps}

  \begin{forthel}
    \begin{definition}\printlabel{FOUNDATIONS_08_605931408719872}
      Let $f$ be a map.
      $f$ is injective iff for all $a, a' \in \dom(f)$ if $f(a) = f(a')$ then $a = a'$.
    \end{definition}
  \end{forthel}

  \begin{forthel}
    \begin{definition}\printlabel{FOUNDATIONS_08_7264375979114496}
      Let $B$ be a class.
      A map into $B$ is an injective map to $B$.
    \end{definition}
  \end{forthel}

  \begin{forthel}
    \begin{definition}\printlabel{FOUNDATIONS_08_5754157419986944}
      Let $A, B$ be classes.
      A map from $A$ into $B$ is a map from $A$ that is a map into $B$.
    \end{definition}

    Let $f: A \into B$ stand for $f$ is a map from $A$ into $B$.
    Let $f$ maps $A$ into $B$ stand for $f$ is a map from $A$ into $B$.
  \end{forthel}


  \section{Bijective maps}

  \begin{forthel}
    \begin{definition}\printlabel{FOUNDATIONS_08_3356670992318464}
      Let $A, B$ be classes.
      A bijection between $A$ and $B$ is an injective map from $A$ onto $B$.
    \end{definition}

    Let a bijection from $A$ to $B$ stand for a bijection between $A$ and $B$.
  \end{forthel}

  \begin{forthel}
    \begin{proposition}\printlabel{FOUNDATIONS_08_60881194975232}
      Let $A, B$ be classes and $f : A \into B$.
      Then $f$ is a bijection between $A$ and $\range(f)$.
    \end{proposition}
    \begin{proof}
      $f$ is injective and $f$ is a map from $A$ onto $\range(f)$.
      Hence $f$ is a bijection between $A$ and $\range(f)$.
    \end{proof}
  \end{forthel}

  \begin{forthel}
    \begin{definition}\printlabel{FOUNDATIONS_08_8188451318923264}
      Let $A$ be a class.
      A permutation of $A$ is a bijection between $A$ and $A$.
    \end{definition}
  \end{forthel}


  \section{Some basic facts}

  \begin{forthel}
    \begin{proposition}\printlabel{FOUNDATIONS_08_7883784041005056}
      Let $A$ be a class.
      Then $\id_{A}$ is a permutation of $A$.
    \end{proposition}
    \begin{proof}
      (1) $\id_{A}$ is a map on $A$.

      (2) $\id_{A}$ is a map onto $A$. \\
      Proof.
        Let $a \in A$.
        Then $a = \id_{A}(a)$.
        Hence $a \in \range(\id_{A})$.
      Qed.

      (3) $\id_{A}$ is a map into $A$. \\
      Proof.
        Let $a, a' \in A$.
        Assume $\id_{A}(a) = \id_{A}(a')$.
        Then $a = a'$.
      Qed.
    \end{proof}
  \end{forthel}

  \begin{forthel}
    \begin{proposition}\printlabel{FOUNDATIONS_08_8542698338254848}
      Let $A, B, C$ be classes and $f : A \onto B$ and $g : B \onto C$.
      Then $g \circ f$ is a map from $A$ onto $C$.
    \end{proposition}
    \begin{proof}
      $g \circ f$ is a map from $A$.

      Let us show that $g \circ f$ is a map onto $C$.
        Let $c \in C$.
        Take $b \in B$ such that $c = g(b)$.
        Take $a \in A$ such that $b = f(a)$.
        Then $c = g(f(a)) = (g \circ f)(a)$.
      End.
    \end{proof}
  \end{forthel}

  \begin{forthel}
    \begin{proposition}\printlabel{FOUNDATIONS_08_3367836856614912}
      Let $A, B, C$ be classes and $f : A \into B$ and $g : B \into C$.
      Then $g \circ f$ is a map from $A$ into $C$.
    \end{proposition}
    \begin{proof}
      $g \circ f$ is a map from $A$.

      Let us show that $g \circ f$ is injective.
        Let $a, a' \in A$.
        Assume $(g \circ f)(a) = (g \circ f)(a')$.
        Then $g(f(a)) = g(f(a'))$.
        Hence $f(a) = f(a')$.
        Indeed $f(a), f(a') \in B$.
        Thus $a = a'$.
      End.
    \end{proof}
  \end{forthel}

  \begin{forthel}
    \begin{corollary}\printlabel{FOUNDATIONS_08_6435206693126144}
      Let $A, B, C$ be classes.
      Let $f$ be a bijection between $A$ and $B$ and $g$ be a bijection between $B$ and $C$.
      Then $g \circ f$ is a bijection between $A$ and $C$.
    \end{corollary}
  \end{forthel}

  \begin{forthel}
    \begin{proposition}\printlabel{FOUNDATIONS_08_2621531811217408}
      Let $A, B$ be classes and $f : A \into B$ and $X \subseteq A$.
      Then $f \restriction X$ is injective.
    \end{proposition}
    \begin{proof}
      Let $a, a' \in X$.
      Assume $(f \restriction X)(a) = (f \restriction X)(a')$.
      Then $f(a) = f(a')$.
      Hence $a = a'$.
    \end{proof}
  \end{forthel}

  \begin{forthel}
    \begin{proposition}\printlabel{FOUNDATIONS_08_647446231252992}
      Let $A, B$ be classes and $f : A \into B$ and $X \subseteq A$.
      Then $f \restriction X$ is a bijection between $X$ and $f_{*}(X)$.
    \end{proposition}
  \end{forthel}

  \begin{forthel}
    \begin{corollary}\printlabel{FOUNDATIONS_08_8159443759923200}
      Let $A, B$ be classes and $f : A \into B$.
      Then $f$ is a bijection between $A$ and $f_{*}(A)$.
    \end{corollary}
  \end{forthel}
\end{document}
