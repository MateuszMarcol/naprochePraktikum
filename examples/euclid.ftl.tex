\documentclass[11pt]{article}
\usepackage{amssymb}
\usepackage{../lib/tex/naproche}

\author{Peter Koepke\\University of Bonn}

\title{\textbf{An Introduction to $\mathbb{N}$aproche, 
Along a Formalization of Euclid's Proof 
of the Infinitude of Primes}}

\begin{document}

\newcommand{\val}[2]{#1_{#2}}
\newcommand{\Prod}[3]{#1_{#2} \cdots #1_{#3}}
\newcommand{\Seq}[2]{\{#1,\dots,#2\}}
\newcommand{\Set}[3]{\{#1_{#2},\dots,#1_{#3}\}}
\newcommand{\Primes}{\mathbb{P}}
\newcommand{\Naproche}{$\mathbb{N}$aproche}

\maketitle

\section{About This Document}
\subsection{Running \Naproche{} in Isabelle}
This pdf-document can be read as an introduction to the \Naproche{} natural
proof assistant. It contains a proof of the infinitude of primes, typeset on a 
light-grey background, and explains various features of the ForTheL input
language to \Naproche{} with examples from that proof.

Simulataneously, the \LaTeX{} source \verb+Introduction.ftl.tex+ of this document 
is a ForTheL text which proof-checks successfully in the \Naproche{} prover.
The source text can be loaded into the Isabelle Proof Interactive 
Development Environment (Isabelle-PIDE), in which the text can be edited
and continuously checked for correctness. Isabelle 2021 contains \Naproche{} as
a component which is activated when files with the extension .ftl.tex are edited.

Note that the checking process consists of proving a large number of explicit 
and implicit first-order properties of the ForTheL text by the automatic 
theorem prover E. The power of E and of the overall \Naproche{} system
strongly depends on hardware performance. It may be necessary to supply
more proof details to get a text checked. On an older laptop 
with an Intel Pentium N3710 processor \Naproche{} takes 
around 75 seconds to check this text.

\subsection{The ForTheL Language and \LaTeX}
Since only content in

\verb_\_\verb_begin{forthel} ... _\verb_\_\verb_end{forthel}_ 

\noindent environments is passed to \Naproche{} one can write arbitrary material
outside those environments (like these explanatory comments). To experiment with
\Naproche{} one can edit the forthel environments. For this it may be convenient to
deactivate most of these environments by replacing the outer 
\verb_\_\verb_begin_ and \verb_\_\verb_end_ by
\verb_begin_ and \verb_end_.

By default, \Naproche{} continuously checks the active buffer for correctness
as a ForTheL text (parsing) and for logical correctness (proof checking). 
Checking results are displayed in the jedit Output window.
Further feedback is given by coloured highlighting of the buffer and by
ballons when hovering over sentences with the mouse pointer.

ForTheL files with a proper \LaTeX preamble and a begin document 
/ end document structuring can be immediately compiled by \TeX/\LaTeX
once the forthel environment is defined. This has, e.g. been defined in the style
file \verb_naproche_ which we are using for this document. This package
also provides the gray background for ForTheL.  

\subsection{An Example}

We demonstrate this principle with a small technical section 
mainly concerned with notation. The \verb+ [...]+ commands in
this section give some parser commands to \Naproche{} which we shall 
discuss later.

The source code of the following section is:
\begin{verbatim}
\section{Notation}

\begin{forthel}


%\begin{lemma} Contradiction. \end{lemma}

[printprover on]

[synonym number/-s]
[synonym divide/-s]
[synonym set/-s] [synonym element/-s] 
[synonym belong/-s] [synonym subset/-s]

\end{forthel}
\end{verbatim}

Here, \verb+\section{Notation}+ is an ordinary \LaTeX
section command. The rest of the snippet is in a
{\verb_begin{forthel} ... end{forthel}_} environment
which is marked in the \LaTeX output by 
a gray background:

\section{Notation}

\begin{forthel}

[printprover on][dump on]

[synonym number/-s]
[synonym divide/-s]
[synonym set/-s] [synonym element/-s] 
[synonym belong/-s] [synonym subset/-s]

\end{forthel}

\section{General Remarks}

The \Naproche{} system (Natural Proof Checking)
is a proof assistant that 
accepts input texts 
in a controlled natural language for mathematics, and checks
them for logical correctness. \Naproche{} accepts
two dialects of the ForTheL language (Formula Theory Language):
a classic one close to the language of the original SAD system,
indicated by the .ftl file ending,
and a LaTeX-oriented version with .ftl.tex file ending.
In this note we use the latter version, which directly allows
mathematical typesetting with \LaTeX.

Text processing by \Naproche{} translates accepted texts in a natural 
mathematical language into a formal logic format which is 
then handed over to further processing in the formal logic.
In the present \Naproche{} system, the logic format is classical
first-order predicate logic and the processing is mainly by
repeated sledgehammer-like proving using the eprover automatic
theorem prover (ATP).
Other logic formats like Lean or possibly Isabelle's Isar
language are under consideration.

\subsection{Natural Language Processing}

The ForTheL input text is thus interpreted in the target logic
and also proposes proof methods to be used by the reasoner
and the ATP. ForTheL leverages a number of natural language
mechanisms to capture formal content in a compact, user-friendly
and natural way. This corresponds to usual natural language
features, where the phrase ``white horse that belongs 
to Mary'' with its adjective, noun and relative sentence 
corresponds to a first-order statement like
$$horse(x) \wedge white(x) \wedge property-of(x,Mary)$$
with a (hidden) variable $x$, predicates $horse()$, 
$white()$, and $property-of( , )$, and a constant $Mary$.
\Naproche{} extracts this formal context whilst reading
the input sentence by sentence. Previous sentences provide the context
of already introduced language components, in which the
new sentence is to be interpreted. 

\subsection{Axiomatic Approach}

The \Naproche{} system
comes with a minimal set of in-built mathematical notions.
Usually on has to explicitly extend the first-order
language through Signature and Definition commands and through
Axioms. Then Lemmas and Theorems can be postulated and proved
with familiar proof structures. In the following this procedure
is explained along a standard proof of the infinitude of 
prime numbers:
\begin{itemize}
\item
set up a language and axioms for natural number arithmetic;
\item
define divisibility and prime natural numbers;
\item
introduce some set theory so that one can define
finite sets, sequences and products.
\end{itemize}
Finally, a checked natural language proof of Euclid's theorem
can be carried out in this axiomatic setup.

\section{Natural Numbers - Introducing the Language of Arithmetic}

A first-order language is determined by its (symbols for)
relations, functions, and constants. We want to introduce 
the functions $+$ and $*$ of addition and multiplication
(of natural numbers)
and the constants $0$ and $1$. Since we shall 
later consider other domains as well, like sets and functions,
we capture those domains by relations. The type ``natural number''
of ordinary mathematical discourse will be modeled by an
internal unary relation symbol
\verb+aNaturalNumber+, and the arithmetic functions and 
quantifiers will be
restricted to the extension of the unary relation symbol.
So the (weak) type system of ordinary mathematical language
is modeled by a system of first-order predicates. These
types do not follow a strict ``type theory'' with specific
mathematical laws but they are still powerful enough
to organize the universe of mathematics.

Here is ForTheL code for introducing the type, or rather
notion, of natural numbers, the constants $0$ and $1$
and the operations of $+$ and $*$.

In order to be able to form classes and sets of natural numbers 
we agree that natural numbers are small objects.
 
\begin{forthel}

Let $x$ is small stand for $x$ is setsized.

\begin{signature}  A natural number is a small object.
\end{signature}

Let $i,k,l,m,n,p,q,r$ denote natural numbers.

\begin{lemma} i is small. \end{lemma}

\begin{signature} $0$ is a natural number.
\end{signature}

Let $x$ is nonzero stand for $x \neq 0$.

\begin{signature} $1$ is a nonzero natural number.
\end{signature}

\begin{signature} $m + n$ is a natural number.
\end{signature}

\begin{signature} $m * n$ is a natural number.
\end{signature}
\end{forthel}

\subsection{First-Order Translation}
The first-order translation
of sentences in the ForTheL code can be found in the output
window of jedit or hovering the mouse
over the sentence:
\begin{small}
\begin{verbatim}
1. forall v0 ((HeadTerm :: aNaturalNumber(v0)) implies 
     (truth and isSetsized(v0)))
2. forall v0 ((HeadTerm :: v0 = 0) implies aNaturalNumber(v0)) 
3. forall v0 ((HeadTerm :: v0 = 1) implies 
     (aNaturalNumber(v0) and not v0 = 0)) 
4. (aNaturalNumber(m) and aNaturalNumber(n)) 
5. forall v0 ((HeadTerm :: v0 = m+n) implies aNaturalNumber(v0)) 
6. (aNaturalNumber(m) and aNaturalNumber(n)) 
7. forall v0 ((HeadTerm :: v0 = m*n) implies aNaturalNumber(v0)) 
\end{verbatim}
\end{small}
In these formulas we see the newly introduced first-order symbols:

\verb_aNaturalNumber(v0), 0, 1, +, *_.

The first-order translations follow a certain idiom which
is favourable for the overall processing. Formula 1 is
exhibits
the new symbol marked by the tag \verb+HeadTerm+. Similarly
formula 2 emphasizes the symbol \verb+0+ which would not have been
the case in the equivalent \verb+aNaturalNumber(0)+.
Note that 5 and 7 both have the premises

\verb+(aNaturalNumber(m) and aNaturalNumber(n))+

\noindent for the two arguments of the operations. These premises have to
be proved before the operations can reasonably be applied within
proofs.

\subsection{Some ForTheL Commands and Keywords}

Let us now go through the natural language phrases used to 
reach this translation. New first-order symbols are spawned by
Signature commands. The new notion comes before the keyword ``is''
after which the new notion is classified as a new type (``is a notion'')
or as a member of of an existing type (``is a natural number'').

The phrase before ``is'' is read as a new language pattern that
the parser learns. A pattern has some word tokens, 
like ``natural'',
``number'', or some symbolic tokens, like 
``$0$'', ``$1$'', ``$+$'', ``$*$''.
In between those tokens a pattern may have holes for the insertion
of terms, which in the Signature command are indicated by previously
introduced variables, like ``$m$'' or ``$n$''. These were introduced in
the parser command ``Let $i,k,l,m,n,p,q,r$ denote natural numbers.''
Thereafter, $m$ and $n$ are variables which are ``pretyped'' to be
natural numbers. With that,
\begin{signature} $m + n$ is a natural number.
\end{signature}
has the ``double translation''
\begin{verbatim}
(aNaturalNumber(m) and aNaturalNumber(n))
forall v0 ((HeadTerm :: v0 = m+n) implies aNaturalNumber(v0))
\end{verbatim}
where the first (or more) formulas are premises and the last contains the
newly introduced symbol.

We can also qualify the typing on the right-hand side of the 
``is'' keyword
by first-order formulae. In our example, we have introduced a pattern for
a first-order formula by the parser command 
``Let $x$ is nonzero stand for $x \neq 0$.'' 
This formula is then applied
as an adjective in the next Signature command
\begin{signature} $1$ is a nonzero natural number.
\end{signature}
Note that some natural language processing is also taking place:
``nonzero'' is introduced within the phrase ``$x$ is nonzero'' 
in an
adjective position. So in the Signature command, ``nonzero'' 
can be
used as an adjective which modifies ``natural number''. 
The first-order
effect of this is a conjunction
\begin{verbatim}
3. forall v0 ((HeadTerm :: v0 = 1) implies 
     (aNaturalNumber(v0) and not v0 = 0)) 
\end{verbatim}
The equality ``$=$'' and inequality ``$\neq$'' are predefined 
phrases with corresponding first-order symbols.

\subsection{``Grammar''}

Note that we have also used the ``linguistic'' commands of the 
notation section: the command \verb+[synonym number/numbers]+ 
identifies the tokens ``number'' and ``numbers'', providing the
plural form. The command can be abbreviated to
\verb+[synonym number/-s]+. This is a simple linguistic
``hack'' which allows grammatically correct forms. But it
also allows wrong ones, and it is up to the user to make the right
choices. 

\section{Natural Numbers - Postulating Axioms}

We now have to introduce axioms for our abstract first-order structure.
Axiom are ForTheL statements written in axiom environments.
For arithmetic we use self-explanatory symbolic formulas.
There are many ways of axiomatizing the natural numbers in order
to be able to prove our final goal: the infinitude of
primes. Here we axiomatize the natural numbers as
a sort of commutative ``half-ring'' with $1$.

\begin{forthel}
\begin{axiom} $m + n = n + m$.
\end{axiom}

\begin{axiom} $(m + n) + l = m + (n + l)$.
\end{axiom}

\begin{axiom}  $m + 0 = m = 0 + m$.
\end{axiom}

\begin{axiom} $m * n = n * m$.
\end{axiom}

\begin{axiom} $(m * n) * l = m * (n * l)$.
\end{axiom}

\begin{axiom} $m * 1 = m = 1 * m$.
\end{axiom}

\begin{axiom} $m * 0 = 0 = 0 * m$.
\end{axiom}

\begin{axiom} $m * (n + l) = (m * n) + (m * l)$ and
                $(n + l) * m = (n * m) + (l * m)$.
\end{axiom}

\begin{axiom} If $l + m = l + n$ or $m + l = n + l$ 
then $m = n$.
\end{axiom}

\begin{axiom} Assume that $l$ is nonzero.
If $l * m = l * n$ or $m * l = n * l$ then $m = n$.
\end{axiom}

\begin{axiom} If $m + n = 0$ then $m = 0$ and $n = 0$.
\end{axiom}

\end{forthel}

Axioms - like Signatures - are toplevel sections which consist of
$n + 1$ statements. The first $n$ are assumption statements 
(``Assume ...'', ``Let ...'')
under which the final statement is postulated. Note that previous
pretypings of variables also act like assumptions.

\section{The Natural Order - Defining \\Relations and Functions}

Definitions extend the first-order language by defined symbols
as in the following examples concerning the ordering
of the natural numbers. A definition corresponds to a 
Signature command in which a symbol is introduced plus
an Axiom containing the defining property.

\begin{forthel}

\begin{definition} $m \leq n$ iff 
there exists a natural number $l$ such that  
$m + l = n$.
\end{definition}

Let $m < n$ stand for $m \leq n$ and $m \neq n$.

\begin{definition} Assume that $n \leq m$.
$m - n$ is a natural number $l$ such that $n + l  = m$.
\end{definition}

\end{forthel}

The first definition defines the binary relation $\leq$ by an
``iff'' equivalence. This is followed by a purely 
syntactic definition of $<$. $m < n$ is simply an abbreviation
for another formula. The abbreviation is already expanded,
possibly recursively, by the parser. The third definition
defines the binary difference function $-$.

\subsection{Axiomatic Content in Definitions}

Definitions of functions and constants usually contain 
implicit postulates corresponding to the existence
and uniquess-properties of function values and constants. In case of
the function definition the condition for $l$ should be
satisfiable by a unique natural number. This is however not
checked by \Naproche, so that the well-definedness of the
function is the user's responsibility. If the function
definition were non-unique we would have a contradictory
system of assumptions. Consider, e.g., the wrong definition

\begin{definition} Assume that $n \leq m$.
$m - n$ is a natural number $l$ such that $n = m$.
\end{definition}

The first-order translation would be

\begin{verbatim}
(aNaturalNumber(m) and aNaturalNumber(n))
n\leq m
forall v0 ((HeadTerm :: v0 = m-n) 
   iff (aNaturalNumber(v0) and n = m))
\end{verbatim}

Every number fits the defining equivalence provided that $m = n$.
But then $0 = 0 - 0 = 1$, contradiction. 

For relation definitions, these problems do not arise.

\section{Lemmas and Theorems}

After setting up the axiomatics we proceed to claim and prove
properties. Claims together with the accumulated axioms will
be given to the background ATP (= eprover). Many basic
propositions can be proved by the ATP without further intervention.
The following three lemmas show that $\leq$ is a partial order:

\begin{forthel}

\begin{lemma} $m \leq m$.
\end{lemma}

\begin{lemma} If $m \leq n \leq m$ 
then $m = n$.
\end{lemma}

\begin{lemma} If $m \leq n \leq l$ 
then  $m \leq l$.
\end{lemma}
\end{forthel}

\subsection{Eprover in the Background}
These lemmas were checked correct by \Naproche{} without explicit proofs.
We can look at the task given to the ATP by putting 
a [dump on] command in the beginning of the ForTheL parts of the 
document and looking for the dump of the provertask in the
output window. The task is written in the first-order
logic language TPTP which is a standard input language for ATPs.
Observe that all previous Signature, Axiom and Definition
environments can be found as premises of the
conjecture $m \leq m$.
\begin{scriptsize}
\begin{verbatim}
[Translation] (line 409 of ...
aNaturalNumber(m) 
[Translation] (line 409 of ...
m\leq m 
[Reasoner] (line 409 of ...
goal:  m \leq m . 
[Main] (line 409 of ...
fof(m_,hypothesis,$true).
fof(m_,hypothesis,aNaturalNumber(sz0)).
fof(m_,hypothesis,(aNaturalNumber(sz1) & ( ~ (sz1 = sz0)))).
fof(m_,hypothesis,( ! [W0] : ( ! [W1] : ((aNaturalNumber(W0) & aNaturalNumber(W1)) 
  => aNaturalNumber(sdtpldt(W0,W1)))))).
fof(m_,hypothesis,( ! [W0] : ( ! [W1] : ((aNaturalNumber(W0) & aNaturalNumber(W1)) 
  => aNaturalNumber(sdtasdt(W0,W1)))))).
fof(m_,hypothesis,( ! [W0] : ( ! [W1] : ((aNaturalNumber(W0) & aNaturalNumber(W1)) 
 => (sdtpldt(W0,W1) = sdtpldt(W1,W0)))))).
fof(m_,hypothesis,( ! [W0] : ( ! [W1] : ( ! [W2] : (((aNaturalNumber(W0) & 
  aNaturalNumber(W1)) & aNaturalNumber(W2)) 
  => (sdtpldt(sdtpldt(W1,W2),W0) = sdtpldt(W1,sdtpldt(W2,W0)))))))).
fof(m_,hypothesis,( ! [W0] : (aNaturalNumber(W0) 
  => ((sdtpldt(W0,sz0) = W0) & (W0 = sdtpldt(sz0,W0)))))).
fof(m_,hypothesis,( ! [W0] : ( ! [W1] : ((aNaturalNumber(W0) & aNaturalNumber(W1)) 
  => (sdtasdt(W0,W1) = sdtasdt(W1,W0)))))).
fof(m_,hypothesis,( ! [W0] : ( ! [W1] : ( ! [W2] : (((aNaturalNumber(W0) & 
  aNaturalNumber(W1)) & aNaturalNumber(W2)) 
  => (sdtasdt(sdtasdt(W1,W2),W0) = sdtasdt(W1,sdtasdt(W2,W0)))))))).
fof(m_,hypothesis,( ! [W0] : (aNaturalNumber(W0) 
  => ((sdtasdt(W0,sz1) = W0) & (W0 = sdtasdt(sz1,W0)))))).
fof(m_,hypothesis,( ! [W0] : (aNaturalNumber(W0) 
  => ((sdtasdt(W0,sz0) = sz0) & (sz0 = sdtasdt(sz0,W0)))))).
fof(m_,hypothesis,( ! [W0] : ( ! [W1] : ( ! [W2] : (((aNaturalNumber(W0) 
  & aNaturalNumber(W1)) & aNaturalNumber(W2)) 
  => ((sdtasdt(W1,sdtpldt(W2,W0)) = sdtpldt(sdtasdt(W1,W2),sdtasdt(W1,W0))) 
  & (sdtasdt(sdtpldt(W2,W0),W1) = sdtpldt(sdtasdt(W2,W1),sdtasdt(W0,W1))))))))).
fof(m_,hypothesis,( ! [W0] : ( ! [W1] : ( ! [W2] : (((aNaturalNumber(W0) 
  & aNaturalNumber(W1)) & aNaturalNumber(W2)) 
  => (((sdtpldt(W0,W1) = sdtpldt(W0,W2)) 
  | (sdtpldt(W1,W0) = sdtpldt(W2,W0))) => (W1 = W2))))))).
fof(m_,hypothesis,( ! [W0] : (aNaturalNumber(W0) => (( ~ (W0 = sz0)) 
  => ( ! [W1] : ( ! [W2] : ((aNaturalNumber(W1) & aNaturalNumber(W2)) 
  => (((sdtasdt(W0,W1) = sdtasdt(W0,W2)) 
  | (sdtasdt(W1,W0) = sdtasdt(W2,W0))) => (W1 = W2))))))))).
fof(m_,hypothesis,( ! [W0] : ( ! [W1] : ((aNaturalNumber(W0) & aNaturalNumber(W1)) 
  => ((sdtpldt(W0,W1) = sz0) => ((W0 = sz0) & (W1 = sz0))))))).
fof(m_,hypothesis,( ! [W0] : ( ! [W1] : ((aNaturalNumber(W0) & aNaturalNumber(W1)) 
  => (sdtbszlzezqdt(W0,W1) 
  <=> ( ? [W2] : (aNaturalNumber(W2) & (sdtpldt(W0,W2) = W1)))))))).
fof(m_,hypothesis,( ! [W0] : ( ! [W1] : ((aNaturalNumber(W0) & aNaturalNumber(W1)) 
  => (sdtbszlzezqdt(W1,W0) => ((aNaturalNumber(sdtmndt(W0,W1)) 
  & (sdtpldt(W1,sdtmndt(W0,W1)) = W0)) & ( ! [W2] : ((aNaturalNumber(W2) 
  & (sdtpldt(W1,W2) = W0)) => (W2 = sdtmndt(W0,W1)))))))))).
fof(m__,hypothesis,aNaturalNumber(xm)).
\end{verbatim}
\end{scriptsize}

\subsection{Testing for Contradictions}

It is quite common to accidentally introduce trivial inconsistencies
in formalizations. Not just by function definitions, but also
because some marginal cases outside the main argument have not
been treated right. E.g., although the number $0$ is quite 
uninteresting for the study of prime numbers, we still have to
deal with $0$-cases or explicitly request that terms are
nonzero. If a text with non-trivial mathematical content checks
unexpectedly fast then one should become suspicious.

To find inconsistencies it is helpful to try to prove
\begin{lemma} Contradiction. \end{lemma}
in various places of a text. If the lemma is validated by 
\Naproche{} then one has to investigate further. In the source text
of this document one finds an Contradiction-Lemma which is commented
out by \verb+%+. This can be quickly activated for a contradiction
check. It can also be used to force rechecking of the text: uncomment
the lemma and then comment it again; this will lead to rechecking at
least from the position of the lemma onwards.  


\section{Linear and Discrete Orders}

We need more axiomatic assumptions for the ordering of the natural numbers. The axioms
so far do not guarantee that the ordering is linear. Also we want
a ``discrete'' order with nothing strictly between 
$0$ and $1$. So we continue:

\begin{forthel}

\begin{axiom} $m \leq n$ or $n < m$.
\end{axiom}

\begin{lemma} Assume that $l < n$.
  Then $m + l < m + n$ and $l + m < n + m$.
\end{lemma}

\begin{lemma} Assume that $m$ is nonzero and $l < n$.
  Then $m * l < m * n$ and $l * m < n * m$.
\end{lemma}

\begin{axiom} $n = 0$ or $n = 1$ or $1 < n$.
\end{axiom}

\begin{lemma} If $m \neq 0$ then $n \leq n * m$.
\end{lemma}
\end{forthel}

\section{Induction}

\Naproche{} has inherited an elegant treatment of induction
from the SAD system. \Naproche{} has a special binary relation
symbol $\prec$ for a universal inductive relation: if at any
point $m$ property $P$ is inherited at $m$ provided all
$\prec$-predecessors of $m$ satisfy $P$, then $P$ holds everywhere.

So to prove that $P$ holds universally, it suffices to prove
the ``inheritance'' along $\prec$. This modification of proof tasks
is already carried out by the parser when it comes across the
keyword ``proof by induction''. This will be demonstrated in a 
later proof in this Introduction.

\begin{forthel}
\begin{axiom} If $n < m$ then $n \prec m$.
\end{axiom}
\end{forthel}

From this axiom one can derive Peano axioms for the 
natural numbers. On the other hand, with some simple
axioms about the successor operation $+1$ and with $\prec$
one could have derived all the above structural axioms.

\section{Division}

We now get to notions that are crucial for the
study of divisors and prime numbers: 

\begin{forthel}

\begin{definition}
  $n$ divides $m$ iff for some $l$ $m = n * l$.
\end{definition}

Let $x | y$ denote $x$ divides $y$.
Let a divisor of $x$ denote a natural number 
that divides $x$.
\end{forthel}

The definition is similar to the definition of $\leq$.
Note, however, the possible syntactic variations:
``there exists a natural number $l$ such that  
$m + l = n$'', ``for some $l$ $m = n * l$''. It is also possible
to put the quantifier after the property:
``$n$ divides $m$ iff $m = n * l$ for some $l$''.

Natural language has many mechanisms for putting
information into sentences in a compact, un-formalistic way.
Un-formalistic means, e.g., that natural language
does not generally allow brackets (...) in speech. ForTheL 
implements several of these natural language mechanisms.
Although the language has evolved, 
``The syntax and semantics of the ForTheL language'' by Andrei
Paskevich is still a good guide to most constructs of the
language.

\section{An Interactive Proof}

We shall later need a technical lemma on divisibility:

\begin{lemma} Let $l | m$ and $l | m + n$. Then $l | n$.
\end{lemma}

On the computer I am using, \Naproche{} does not find
a proof on its own: depending on some default timeouts the
proof search is abandoned, and the goal $l | n$ fails.
In Isabelle-Naproche this is signaled in the output
window, and the failed goal gets an underlining in red.

So the user has to ``interactively'' supply a proof, which in a first
approximation is a list
of statements which leads up to the claim, and which
\Naproche{}'s ATP is able to prove successively.
Proof statements can also introduce assumptions
and new variables to the argument, and they can
structure the proof.

\begin{forthel}
\begin{lemma} Let $l | m$ and $l | m + n$. Then $l | n$.
\end{lemma}

\begin{proof}
Assume that $l$ is nonzero.
Take $p$ such that $m = l * p$. 
Take $q$ such that $m + n = l * q$.

Let us show that 
$p \leq q$.
Proof by contradiction.
Assume the contrary. Then $q < p$.
$m+n = l * q < l * p = m$.
Contradiction. qed.

Take $r = q - p$.
We have $(l * p) + (l * r) = l * q = m + n = (l * p) + n$.
Hence $n = l * r$.
\end{proof}

\end{forthel}

When \Naproche{} encounters a statement immediately followed by an 
explicit proof 
then \Naproche{} defers proving the statement and first goes through
the proof. Since proofs may contain subproofs, this process may
take place recursively.

Proofs of a ``toplevel'' Lemma or Theorem use the

\verb+\begin{proof}...\end{proof}+ 

\noindent environment well-known from \LaTeX.
In our proof there is also a ``lowlevel'' proof of $p \leq q$
indicated by ``Let us show that''. Let us discuss some aspects of the
proof:
\begin{itemize}
\item Most sentences in a proof are statements, or statements
extended by certain constructs that organize the flow of the argument.
\item ``Assume that $l$ is nonzero.'' is an assumption that introduces
the premise ``$l$ is nonzero'' to the argument. Instead of ``Assume that''
one could also use variants like 
[let us | we can] (assume | presume | suppose) [that].
\item ``Take $p$ such that $m = l * p$.'' introduces a new variable
$p$ with a specific property to the argument. To verify this construct 
the prover has to show the existence of some object satisfying the
property. Again there are variants:
[let us | we can] (choose | take | consider).
\item ``Let us show that $p \leq q$.'' claims that the statement
$p \leq q$ holds and announces a subsequent proof.
Alternatives: [let us | we can] (prove | show | demonstrate) (that).
\item ``Proof by contradiction'' denotes the start of an indirect
proof. It is recommended to explicitly mark indirect proof.
Note that in the example this is a ``lowlevel'' proof that
uses a simple 

\verb+Proof [by ...](.) ... (qed. | end.)+

\noindent environment instead of the \LaTeX proof environment.
\item Other proof methods are ``by cases'' and ``by induction''.
\item ``Assume the contrary.'': The contrary is the negation
of the current thesis which in this case is the statement claimed
just before. ``thesis'' denotes the current thesis, ``contradiction''
stands for ``false''.
\item ``Then $q < p$.'': Words like ``then'', ``hence'', ``thus'',
``therefore'', ``consequently'' are filler words which are redundant
for \Naproche{} but may help human readers to understand the text.
\item ``$m+n = l * q < l * p = m$'': binary relations like ``$=$''
or ``$<$'' can be chained. The statement means the conjunction
of the single relations. These will be checked from
left to right.
\item ``Contradiction. qed.'': The indirect proof has reached the
desired contradiction, and that proof environment is closed by
``qed.''.
\end{itemize}

\Naproche{} is able to prove the next lemma without an
explicit proof in the text.

\begin{forthel}

\begin{lemma} Let $m | n \neq 0$. Then $m \leq n$.
\end{lemma}

\end{forthel}

\section{Primes}

Prime numbers are defined as usual. Indeed we define
the adjective ``prime'' which will enable us
to write ``prime natural number'' or ``prime divisor''.

\begin{forthel}

Let $x$ is nontrivial stand for $x \neq 0$ and $x \neq 1$.

\begin{definition}
$n$ is prime iff $n$ is nontrivial and
    for every divisor $m$ of $n$ $m = 1$ or $m = n$.
\end{definition}

\end{forthel}
\subsection{Proof by Induction}
The following lemma obviously holds by induction: either
$k$ is prime itself, or $k$ has a divisor strictly
between $1$ and $k$; by induction that divisor has a prime
divisor which is also a prime divisor of $k$.

\begin{forthel}

\begin{lemma} Every nontrivial $k$ has a prime divisor.
\end{lemma}
\begin{proof} Proof by induction.
\end{proof}

\end{forthel}

The phrase ``proof by induction'' invokes a general induction principle
for the relation $\prec$. To prove $\forall k \phi(k)$, it suffices to
prove:
$$\forall v_0 (\forall v_1 (v_1 \prec v_0 \rightarrow \phi(v_1))
\rightarrow \phi(v_0).$$
So ``proof by induction'' transforms the thesis into a new thesis:
\begin{footnotesize}
\begin{verbatim}
thesis: forall v0 ((aNaturalNumber(v0) and (not v0 = 0 and not v0 = 1)) 
implies ((InductionHypothesis :: forall v1 ((aNaturalNumber(v1) and 
(not v1 = 0 and not v1 = 1)) implies (iLess(v1,v0) 
implies exists v2 ((aNaturalNumber(v2) and doDivides(v2,v1)) 
and isPrime(v2))))) implies exists v1 
((aNaturalNumber(v1) and doDivides(v1,v0)) and isPrime(v1))))
\end{verbatim}
\end{footnotesize}
Note that internally,
\verb+iLess+ represents the relation $\prec$. Since we had postulated
axiomatically that $<$ is a subrelation of $\prec$, the induction principle
for $\prec$ implies a standard induction principle for the natural numbers. 

\section{Classes}

``sets'' and ``classes'' are built-in notions of 
\Naproche{}, as well as ``element of ...''.
``element of
$y$'' determines the type of
elements-of-$y$. Such ``of''-types lead to several 
linguistic modifications: one can quantify over
elements-of-$y$ like (for all | for some | no) (element of $y$);
$y$ has some element; etc.

Similarly, the subclass relation is given by the dependent
type of subclasses-of-$T$.

\begin{forthel}

Let $S,T$ stand for classes.
Let $x$ belongs to $S$ stand 
for $x$ is an element of $S$.

\begin{definition} A subclass of $S$ is a class $T$ 
such that every element of $T$ belongs to $S$.
\end{definition}

Let $T \subseteq S$ stand for $T$ is a subclass of $S$.
\end{forthel}

To avoid the classical antinomies of set theory, 
classes can only have ``small'' elements 
which in \Naproche{}'s terminology are ``setsized'';
both these adjective were identified earlier in the 
document.
We extend the built-in ontology of \Naproche{}
according to the following principles:
\begin{forthel}

\begin{axiom}
Every element of every class is small.
\end{axiom}

\begin{axiom}
Every set is a small class and every small class is a set.
\end{axiom}

\end{forthel}
Classes can be naturally formed in ForTheL: 

\begin{forthel}

\begin{definition} $\mathbb{N}$ is the 
class of natural numbers.
\end{definition}

\end{forthel}

The verbose form is equivalent to the 
use of abstraction terms:

\begin{definition} $\Seq{m}{n} = \{$ natural number $i | 
m \leq i \leq n \}$.
\end{definition}


\section{Finite Sequences and Products, using Intuitive ``...''-Notation}

This section demonstrates how some notation that is 
considered vague can be interpreted as formally
exact. Mathematics often uses ...-notations to
indicate arbitrary size finite or even infinite mathematical
objects.

From a \LaTeX standpoint, a notation like
$\Seq{m}{n}$ can canonically be seen as the printout
of a corresponding macro in the \LaTeX source. \Naproche{}
on the other hand accepts standard \LaTeX macros as
patterns, so that the macro can be a properly introduced
\Naproche{} pattern with a first-order definition.
In this way, intuitive and customary notation can be used
also as \Naproche{} input.

In the present text, $\Seq{m}{n}$ is printed from
a macro defined by:

\verb+\newcommand{\Seq}[2]{\{#1,\dots,#2\}}+.

This notation or macro can be given a precise semantics
by ForTheL definitions.

\begin{forthel}
\begin{definition} $\Seq{m}{n}$ is the class of
natural numbers $i$ such that $m \leq i \leq n$.
\end{definition}
\end{forthel}

So far there are basically no axioms for set formation, so
we postulate:
\begin{forthel}
\begin{axiom} $\Seq{m}{n}$ is a set. \end{axiom}

\end{forthel}

The macro for the $\Seq{m}{n}$- notation is visible in the \LaTeX
code:
\begin{verbatim}
\begin{definition} $\Seq{m}{n}$ is the class of
natural numbers $i$ such that $m \leq i \leq n$.
\end{definition}
\end{verbatim}

\subsection{Functions}

The notion of ``function'' and some related notations like the domain
of a function $F$ or the application $F(x)$ of a function
to an argument are provided by \Naproche{}.
We add an axiom about smallness of values:

\begin{forthel}

\begin{axiom}
Assume $F$ is a function and $x \in Dom(F)$. Then $F(x)$ is small.
\end{axiom}

\begin{definition} A sequence of length $n$ is a
function $F$ such that $Dom(F) = \Seq{1}{n}$.
\end{definition}

\end{forthel}

The members $F(i)$ of a sequence $F$ are often
written in an indexed notation $\val{f}{i}$
where this notation is defined by a macro

\verb+\newcommand{\val}[2]{#1_{#2}}+.

\noindent The ForTheL semantics is defined by:

\begin{forthel}
Let $\val{F}{i}$ stand for $F(i)$.

\begin{definition} Let $F$ be a sequence of length $n$.
$\Set{F}{1}{n} = \{ \val{F}{i} | i \in Dom(F)\}$.
\end{definition}

\end{forthel}

Dot notation is also used for iterations of all sorts.
For Euclid's theorem we shall want to consider products
of finitely many prime numbers. So we postulate
axiomatically:

\begin{forthel}

%\begin{signature} Let $F$ be a sequence of length $n$
%such that $\Set{F}{1}{n} \subseteq \mathbb{N}$.
%$\Prod{F}{1}{n}$ is a nonzero natural number.
%\end{signature}

%\begin{axiom} $\mathbb{N}$ is a set. \end{axiom}

\begin{signature} Let $F$ be a sequence of length $n$
such that $\Set{F}{1}{n} \subseteq \mathbb{N}$.
$\Prod{F}{1}{n}$ is a natural number.
\end{signature}

\begin{axiom}[Factorproperty] Let $F$ be a sequence of length $n$
such that $F(i)$ is a nonzero natural number for every $i \in Dom(F)$.
Then $\Prod{F}{1}{n}$ is nonzero and 
$F(i)$ divides $\Prod{F}{1}{n}$ for every $i \in Dom(F)$.
\end{axiom}


\end{forthel}

Note that we can name toplevel sections by single words like ``Factorproperty''
or numbers. These can be referenced later in the form ``(by Factorproperty)''.

\section{Finite and Infinite Sets}

Finite sequences readily allow a formalization
of finiteness for arbitrary sets and classes.

\begin{forthel}

\begin{definition} $S$ is finite iff 
$S = \Set{F}{1}{n}$ for some natural number $n$ and some function $F$ that is 
a sequence of length $n$.
\end{definition}

\begin{definition} $S$ is infinite iff $S$ is not finite.
\end{definition}

\end{forthel}

\section{Euclid's Theorem}

Now everything is in place for the proof that there
are infinitely many prime numbers.
\begin{forthel}

\begin{signature} $\Primes$ is the class of prime natural numbers.
\end{signature}

% \begin{lemma} Contradiction. \end{lemma}


\begin{theorem}[Euclid]
$\Primes$ is infinite.
\end{theorem}
\begin{proof}
Assume that $r$ is a natural number and 
$p$ is a sequence of length $r$ and
$\Set{p}{1}{r}$ is a subclass of $\Primes$.

%$\val{p}{i}$ is a nonzero natural number for every $i \in Dom(p)$.
(1) $\val{p}{i}$ is a nonzero natural number for every $i$ such
that $1 \leq i \leq r$.

Consider $n=\Prod{p}{1}{r}+1$.
Take a prime divisor $q$ of $n$.

Let us show that $q \neq \val{p}{i}$ for all $i$ such that 
$1 \leq i \leq r$.

Proof by contradiction.
Assume that $q=\val{p}{i}$ for some natural number $i$ such that
$1 \leq i \leq r$.
$q$ is a divisor of $n$ and $q$ is a divisor of $\Prod{p}{1}{r}$
(by Factorproperty, 1).
Thus $q$ divides $1$. Contradiction. qed.

Hence $\Set{p}{1}{r}$ is not the class of prime natural numbers.
\end{proof}
\end{forthel}

\end{document}