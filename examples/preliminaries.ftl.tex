\documentclass[11pt]{article}
\usepackage{amssymb}
\usepackage{../lib/tex/naproche}

\author{Peter Koepke, University of Bonn}

\title{A Kelley-Morse-Style Ontology with Objects}

\begin{document}

\newcommand{\val}[2]{#1_{#2}}
\newcommand{\Prod}[3]{#1_{#2} \cdots #1_{#3}}
\newcommand{\Seq}[2]{\{#1,\dots,#2\}}
\newcommand{\FinSet}[3]{\{#1_{#2},\dots,#1_{#3}\}}
\newcommand{\Primes}{\mathbb{P}}
\newcommand{\Naproche}{$\mathbb{N}$aproche}

\maketitle

\section{Basic Ideas}

This file provides an ontology for $\mathbb{N}$aproche which
is close to ordinary mathematical practices. There are
mathematical objects, and sets and classes of mathematical
objects. Sets are classes which are objects themselves and
can be used in further mathematical constructions. Functions
and maps map objects to objects, where functions are those
maps which are objects. Sets and functions are required to
satisfy separation and replacement properties known from the
set theories of Kelley-Morse or Zermelo-Fraenkel.

Modelling mathematical notions by objects corresponds
to the intuition that one does not expect internal set-theoretical
structurings of numbers, points, etc.. This is also advantageous
for automated proving since it prevents proof search to
dig into such internal structurings.

The ontology presented here is more hierarchical then previous
ad-hoc first-order axiomatizations in some of our formalizations. This
may lead to more complex and longer ontological checking tasks.
Controlling the complexitiy of automated proofs will be a major
issue for the further development of $\mathbb{N}$aproche.

\section{Importing Grammatical Forms}
We import singular/plural forms of words that will be used in
our formalizations. In the long run this should be replaced by
employing a proper English vocabulary.

\begin{forthel}
[read vocabulary.ftl]

[read macros.ftl]
\end{forthel}

\section{Sets and Classes}
The notions of \textit{classes} and \textit{sets} are already
provided by $\mathbb{N}$aproche. Classes can be defined by
abstraction terms $\{\dots\mid\dots\}$. Since these terms need to be
processed by the parser we cannot introduce them simply by first-order
definitions but have to implement them in the software.
That sets is are classes which are objects, or extensionality can be proved
from inbuilt assumptions, but it is important to reprove or restate such
facts so that they can be accessed by the ATP.

\begin{forthel}
Let $S,T$ denote classes. Let $X,Y$ denote sets.
Let $x$ denote objects.

\begin{lemma} Every set is a class. \end{lemma}

\begin{lemma} Every class that is an object is a set. \end{lemma}

\begin{axiom} Every element of every class is an object.
\end{axiom}

Let $M,N$ denote classes.

\begin{definition} The empty set is the set that has
no elements.
\end{definition}

\begin{definition}
A subclass of $S$ is a class $T$ such that every $x \in T$
belongs to $S$.
\end{definition}

Let $T \subseteq S$ stand for $T$ is a subclass of $S$.

\begin{lemma} [Class Extensionality] If $M$ is a subclass of $N$ and
$N$ is a subclass of $M$ then $M = N$.
\end{lemma}

\begin{definition}
A subset of $S$ is a set $X$ such that $X \subseteq S$.
\end{definition}

\begin{axiom} [Separation Axiom] Assume that $X$ is a set.
Let $T$ be a subclass of $X$. Then $T$ is a set.
\end{axiom}

\end{forthel}
\section{Pairs and Products}
Since we prefer objects over sets if possible, we do not work
with Kuratowski-style set-theoretical ordered pairs, but
axiomatize them as objects.

\begin{forthel}
\begin{axiom}
For any objects $a,b,c,d$ if $(a,b) = (c,d)$ then $a = c$
and $b = d$.
\end{axiom}

\begin{definition}
$M \times N = \{(x,y) | x$ is an element of $M$ and $y$
is an element of $N\}$.
\end{definition}

\begin{axiom}
$M \times N$ is a set.
\end{axiom}

\begin{lemma}
Let $x,y$ be objects.
If $(x,y)$ is an element of $M \times N$ then $x$ is an
element of $M$ and $y$ is an element of $N$.
\end{lemma}

\end{forthel}
\section{Functions and Maps}
The treatment of functions and maps is similar to that
of sets and classes.
\begin{forthel}

\begin{lemma} Every function is a map.
\end{lemma}

\begin{lemma} Every map that is an object is a function.
\end{lemma}

Let $f,g$ stand for maps.

\begin{lemma} The domain of every map is a class.
\end{lemma}

\begin{lemma} If $x \in Dom(f)$ then $f(x)$ is an object.
\end{lemma}

\begin{lemma} [Functional Extensionality] Assume $Dom(f)=Dom(g)$ and for every $x \in Dom(f)$
$f(x)=g(x)$. Then $f=g$.
\end{lemma}

\begin{definition}
Assume $M$ is a subclass of the domain of $f$.
$f[M] = \{ f(x) | x \in M \}$.
\end{definition}

\begin{definition}
$f : S \rightarrow T$ iff $Dom (f) = S$ and $f[S] \subseteq T$.
\end{definition}

\begin{axiom} [Replacement Axiom]
Assume $M$ is a subset of the domain of $f$.
Then $f[M]$ is a set.
\end{axiom}

\end{forthel}

\end{document}
