\documentclass{article}

\usepackage[utf8]{inputenc}
\usepackage[english]{babel}
\usepackage{../lib/tex/naproche}
\usepackage{amssymb}

\newcommand{\Prod}[2]{#1\times #2}
\newcommand{\Succ}[1]{#1^{+}}
\newcommand{\image}[2]{#1^{\to}[#2]}
\newcommand{\card}[1]{\left|#1\right|}
\newcommand{\surjects}{\twoheadrightarrow}

\title{Regularity of successor cardinals}
\author{Steffen Frerix}
\date{2018}

\begin{document}
  \pagenumbering{gobble}

  \maketitle

  \begin{forthel}
    [synonym cardinal/-s]
    [synonym ordinal/-s]

    Let $x, y, X, Y, M, N$ denote sets.

    \begin{axiom}
      For any objects $x,y,x',y'$ if $(x, y) = (x', y')$ then $x' = x$ and $y' = y$.
    \end{axiom}

    \begin{axiom}
      Let $X$ be a set.
      Let $x$ be an element of $X$. Then $x$ is setsized.
    \end{axiom}

    \begin{axiom}
      Let $x,y$ be setsized objects. Then $(x, y)$ is setsized.
    \end{axiom}

    \begin{axiom}
      Let $f$ be a function.
      Let $X$ be a set.
      Assume $\dom(f) = X$.
      Let $x$ be an element of $X$. Then $f(x)$ is setsized.
    \end{axiom}

    \begin{definition}
      $\Prod{X}{Y} = \{(x,y) \mid x \in X \wedge y \in Y\}$.
    \end{definition}

    \begin{axiom}
      $\Prod{X}{Y}$ is a set.
    \end{axiom}

    \begin{lemma}
      Let $x,y$ be objects.
      If $(x,y)$ is an element of $\Prod{M}{N}$
      then $x$ is an element of $M$ and $y$ is an element of $N$.
    \end{lemma}

    Let $f$ denote a function.

    \begin{definition}
      A subset of $M$ is a set $N$ such that every element of $N$ is an element of $M$.
    \end{definition}

    Let $M\subseteq N$ stand for $M$ is a subset of $N$.

    \begin{axiom}[Extensionality]
      If $M\subseteq N$ and $N\subseteq M$ then $M = N$.
    \end{axiom}

    \begin{definition}
      Assume $X\subseteq \dom(f)$.
      $\image{f}{X} = \{f(x) \mid x \in X\}$.
    \end{definition}

    \begin{axiom}
      Assume $X\subseteq \dom(f)$.
      $\image{f}{X}$ is a set.
    \end{axiom}

    Let $f$ is surjective from $X$ onto $Y$ stand for $\dom(f) = X$ and $\image{f}{X} = Y$.

    Let $f : X \surjects Y$ stand for $f$ is surjective from $X$ onto $Y$.

    \begin{signature}
      An ordinal is a set.
    \end{signature}

    Let $\alpha, \beta$ denote ordinals.

    \begin{axiom}
      Every element of $\alpha$ is an ordinal.
    \end{axiom}

    \begin{signature}
      A cardinal is an ordinal.
    \end{signature}

    Let $\kappa, \mu, \nu$ denote cardinals.

    \begin{signature}
      $\alpha < \beta$ is an atom.
    \end{signature}

    \begin{axiom}
      If $\alpha < \beta$ then $\alpha$ is an element of $\beta$.
    \end{axiom}

    Let $a \leq b$ stand for $a = b$ or $a < b$.

    \begin{definition}[Cofinality]
      Let $\kappa$ be a cardinal.
      Let $Y$ be a subset of $\kappa$.
      %Assume $M\subseteq \kappa$.
      $Y$ is cofinal in $\kappa$ iff
        for every element $x$ of $\kappa$ there exists an element $y$ of $Y$ such that $x < y$.
    \end{definition}

    Let a cofinal subset of $\kappa$ stand for a subset of $\kappa$ that is cofinal in $\kappa$.

    \begin{signature}[Cardinality]
      $\card{X}$ is a cardinal.
    \end{signature}

    \begin{axiom}[Existence of surjection]
      Assume $M$ has an element.
      $\card{M} \leq \card{N}$ iff there exists a function $f$ such that $\dom(f) = N$ and $\image{f}{N} = M$.
    \end{axiom}

    \begin{axiom}[Transitivity]
      %Let $M\in \kappa$.
      %Let $N$ be an element of $M$.
      %$N$ is an element of $\kappa$.
      Let $x\in y\in \kappa$.
      Then $x\in \kappa$.
    \end{axiom}

    \begin{axiom}
      $\card{\Prod{M}{M}} = \card{M}$.
    \end{axiom}

    \begin{axiom}
      $\card{\kappa} = \kappa$.
    \end{axiom}

    \begin{axiom}
      Let $N$ be a subset of $M$. $\card{N} \leq \card{M}$.
    \end{axiom}

    \begin{definition}
      $\kappa$ is regular iff $\card{x} = \kappa$ for every cofinal subset $x$ of $\kappa$.
    \end{definition}

    \begin{signature}
      $\Succ{\kappa}$ is a cardinal.
    \end{signature}

    \begin{axiom}
      $\alpha < \beta$ or $\beta < \alpha$ or $\beta = \alpha$.
    \end{axiom}

    \begin{axiom}
      $\kappa < \Succ{\kappa}$.
    \end{axiom}

    \begin{axiom}
      $\card{\alpha} \leq \kappa$ for every element $\alpha$ of $\Succ{\kappa}$.
    \end{axiom}

    \begin{axiom}
      For no cardinals $\mu, \nu$ we have $\mu < \nu$ and $\nu < \mu$.
    \end{axiom}

    \begin{axiom}
      There is no cardinal $\nu$ such that $\kappa < \nu < \Succ{\kappa}$.
    \end{axiom}

    \begin{definition}
      The empty set is a cardinal $\eta$ such that $\eta$
      is an element of (every ordinal that has an element).
    \end{definition}

    \begin{definition}
      The constant zero on $M$ is a function $f$ such that $\dom(f) = M$ and $f(x)$ is the empty set for every element $x$ of $M$.
    \end{definition}

    Let $0^M$ stand for the constant zero on $M$.

    \begin{theorem}
      $\Succ{\kappa}$ is regular.
    \end{theorem}
    \begin{proof}
      Proof by contradiction. Assume the contrary.
      Take a cofinal subset $x$ of $\Succ{\kappa}$ such that $\card{x} \neq \Succ{\kappa}$.
      Then $\card{x} \leq \kappa$.
      Take a function $f$ that is surjective from $\kappa$ onto $x$ (by existence of surjection).
      Indeed $x$ has an element and $\card{\kappa} = \kappa$.
      %
      Define
      $$
        g(z) =
          \begin{cases}
            \text{choose a function $h$ such that $h : \kappa \surjects z$ in $h$}
            & \text{$z$ has an element}
            \\
            \text{$0^\kappa$}
            & \text{$z$ has no element}
          \end{cases}
      $$
      for $z$ in $\Succ{\kappa}$.

      Define $h((\xi,\zeta)) = g(f(\xi))(\zeta)$ for $(\xi,\zeta)$ in $\Prod{\kappa}{\kappa}$.

      Let us show that $h$ is surjective from $\Prod{\kappa}{\kappa}$ onto $\Succ{\kappa}$.
      $\dom(h) = \Prod{\kappa}{\kappa}$.

        Every element of $\Succ{\kappa}$ is an element of $\image{h}{\Prod{\kappa}{\kappa}}$.
        Proof.
          Let $n$ be an element of $\Succ{\kappa}$.
          Take an element $\xi$ of $\kappa$ such that $n < f(\xi)$.
          Take an element $\zeta$ of $\kappa$ such that $g(f(\xi))(\zeta) = n$.
          Then $n = h((\xi,\zeta))$.
          Therefore the thesis.
          Indeed $(\xi,\zeta)$ is an element of $\Prod{\kappa}{\kappa}$.
        End.

        Every element of $\image{h}{\Prod{\kappa}{\kappa}}$ is an element of $\Succ{\kappa}$.
        proof.
          Let $n$ be an element of $\image{h}{\Prod{\kappa}{\kappa}}$.
          We can take elements $a,b$ of $\kappa$ such that $n = h((a,b))$.
          Then $n = g(f(a))(b)$. $f(a)$ is an element of $\Succ{\kappa}$.

          Case $f(a)$ has an element. Obvious.

          Case $f(a)$ has no element. Obvious.
        End.
      End.

      Therefore $\Succ{\kappa} \leq \kappa$. Contradiction.
    \end{proof}
  \end{forthel}

\end{document}
