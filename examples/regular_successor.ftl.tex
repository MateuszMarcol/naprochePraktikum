\documentclass{article}

\usepackage[utf8]{inputenc}
\usepackage[english]{babel}
\usepackage{../lib/tex/naproche}

\title{Regularity of successor cardinals}
\author{Steffen Frerix}
\date{2018}

\begin{document}
  \pagenumbering{gobble}

  \maketitle

  \begin{forthel}
    [synonym cardinal/-s]
    [synonym ordinal/-s]

    Let $M,N$ denote sets.

    \begin{axiom}
      For any objects $a,b,c,d$ if $(a,b) = (c,d)$ then $a = c$ and $b = d$.
    \end{axiom}

    \begin{axiom}
      Let $x$ be an element of $M$. $x$ is setsized.
    \end{axiom}

    \begin{axiom}
      Let $x,y$ be setsized objects. $(x, y)$ is setsized.
    \end{axiom}

    \begin{axiom}
      Let $f$ be a function. Let $M$ be a set. Assume $Dom(f) = M$.
      Let $x$ be an element of $M$. Then $f(x)$ is setsized.
    \end{axiom}

    \begin{definition}
      $Prod(M,N) = \{(x,y) | x "is an element of" M "and" y "is an element of" N\}$.
    \end{definition}

    \begin{axiom}
      $Prod(M, N)$ is a set.
    \end{axiom}

    \begin{lemma}
      Let $x,y$ be objects. If $(x,y)$ is an element of $Prod(M,N)$ then $x$ is an element of $M$ and $y$ is an element of $N$.
    \end{lemma}

    Let $f$ denote a function.

    \begin{definition}
      A subset of $M$ is a set $N$ such that every element of $N$ is an element of $M$.
    \end{definition}

    \begin{axiom}[Extensionality]
      If $M$ is a subset of $N$ and $N$ is a subset of $M$ then $M = N$.
    \end{axiom}

    \begin{definition}
      Assume $M$ is a subset of $Dom(f)$. $f^[M] = \{f(x) | x "is an element of" M\}$.
    \end{definition}

    \begin{axiom}
      Assume $M$ is a subset of $Dom(f)$. $f^[M]$ is a set.
    \end{axiom}

    Let $f$ is surjective from $M$ onto $N$ stand for $Dom(f) = M$ and $f^[M] = N$.

    \begin{signature}
      An ordinal is a set.
    \end{signature}

    Let $alpha, beta$ denote ordinals.

    \begin{axiom}
      Every element of $alpha$ is an ordinal.
    \end{axiom}

    \begin{signature}
      A cardinal is an ordinal.
    \end{signature}

    Let $A,B,C$ denote cardinals.

    \begin{signature}
      $alpha < beta$ is an atom.
    \end{signature}

    \begin{axiom}
      If $alpha < beta$ then $alpha$ is an element of $beta$.
    \end{axiom}

    Let $a =< b$ stand for $a = b$ or $a < b$.

    \begin{definition}
      Assume $M$ is a subset of $A$. $M$ is cofinal in $A$ iff for every element $x$ of $A$ there exists an element $y$ of $M$ such that $x < y$.
    \end{definition}

    Let a cofinal subset of $A$ stand for a subset of $A$ that is cofinal in $A$.

    \begin{signature}
      The cardinality of $M$ is a cardinal.
    \end{signature}

    Let $card(M)$ stand for the cardinality of $M$.

    \begin{axiom}[SurjExi]
      Assume $M$ has an element. $card(M) =< card(N)$ iff there exists a function $f$ such that $Dom(f) = N$ and $f^[N] = M$.
    \end{axiom}

    \begin{axiom}[Transitivity]
      Let $M$ be an element of $A$. Assume $N$ is an element of $M$. $N$ is an element of $A$.
    \end{axiom}

    \begin{axiom}
      $card(Prod(M,M)) = card(M)$.
    \end{axiom}

    \begin{axiom}
      $card(A) = A$.
    \end{axiom}

    \begin{axiom}
      Let $N$ be a subset of $M$. $card(N) =< card(M)$.
    \end{axiom}

    \begin{definition}
      $A$ is regular iff $card(M) = A$ for every cofinal subset $M$ of $A$.
    \end{definition}

    \begin{signature}
      $Succ(A)$ is a cardinal.
    \end{signature}

    \begin{axiom}
      $alpha < beta$ or $beta < alpha$ or $beta = alpha$.
    \end{axiom}

    \begin{axiom}
      $A < Succ(A)$.
    \end{axiom}

    \begin{axiom}
      $card(i) =< A$ for every element $i$ of $Succ(A)$.
    \end{axiom}

    \begin{axiom}
      For no cardinals $A,B$ we have $A < B$ and $B < A$.
    \end{axiom}

    \begin{axiom}
      There is no cardinal $B$ such that $A < B < Succ(A)$.
    \end{axiom}

    \begin{definition}
      The empty set is a cardinal $E$ such that $E$ is an element of (every ordinal that has an element).
    \end{definition}

    \begin{definition}
      The constant zero on $M$ is a function $f$ such that $Dom(f) = M$ and $f(x)$ is the empty set for every element $x$ of $M$.
    \end{definition}

    Let $0^M$ stand for the constant zero on $M$.

    \begin{theorem}
      $Succ(A)$ is regular.
    \end{theorem}
    \begin{proof}
      Proof by contradiction. Assume the contrary. Take a cofinal subset $x$ of
      $Succ(A)$ such that $card(x) != Succ(A)$. Then $card(x) =< A$. Take a function $f$ that is surjective from $A$ onto $x$ (by SurjExi). Indeed $x$ has an element and $card(A) = A$.

      Define $g(i) =$
        Case $i$ has an element -> Choose a function $h$ that is surjective from $A$ onto $i$ in $h$,
        Case $i$ has no element -> $0^A$
      for $i$ in $Succ(A)$.

      Define $h((xi,zeta)) = g(f(xi))(zeta)$ for $(xi,zeta)$ in $Prod(A,A)$.

      Let us show that $h$ is surjective from $Prod(A,A)$ onto $Succ(A)$. $Dom(h) = Prod(A,A)$.

        Every element of $Succ(A)$ is an element of $h^[Prod(A,A)]$.
        proof.
          Let $n$ be an element of $Succ(A)$. Take an element $xi$ of $A$ such that $n < f(xi)$. Take an element $zeta$ of $A$ such that $g(f(xi))(zeta) = n$. Then $n = h((xi,zeta))$. Therefore the thesis. Indeed $(xi,zeta)$ is an element of $Prod(A,A)$.
        end.

        Every element of $h^[Prod(A,A)]$ is an element of $Succ(A)$.
        proof.
          Let $n$ be an element of $h^[Prod(A,A)]$. We can take elements $a,b$ of $A$ such that $n = h((a,b))$. Then $n = g(f(a))(b)$. $f(a)$ is an element of $Succ(A)$.

          Case $f(a)$ has an element. Obvious (by Transitivity).

          Case $f(a)$ has no element. Obvious (by Transitivity).
        end.
      end.

      Therefore $Succ(A) =< A$. Contradiction.
    \end{proof}
  \end{forthel}

\end{document}
