\documentclass{article}

\usepackage[utf8]{inputenc}
\usepackage[english]{babel}
\usepackage{amssymb}
\usepackage{../lib/tex/naproche}

\title{Maximum modulus principle}
\author{Steffen Frerix (2018), Adrian De Lon (2021), Peter Koepke (2021)}
\date{}

\newcommand{\Ball}[2]{B_{#1}(#2)}
\newcommand{\image}[2]{#1^{\to}[#2]}

\begin{document}
  \pagenumbering{gobble}

  \maketitle


  \section{Introduction}

We formalize that the maximum modulus principle in complex analysis is
implied by the open mapping theorem and the identity theorem.

We use general set theoretic preliminaries:

\begin{forthel}
    [readtex preliminaries.ftl.tex]
\end{forthel}

\section{Real and complex numbers}

\begin{forthel}
\begin{signature}
      A complex number is a mathematical object.
\end{signature}

Let $z, w$ denote complex numbers.

\begin{definition}
$\mathbb{C}$ is the collection of all complex numbers.
\end{definition}

\begin{signature}
      A real number is a complex number.
\end{signature}

Let $x,y$ denote real numbers.

\begin{signature}
      $|z|$ is a real number.
\end{signature}

\begin{signature}
      $x$ is positive is an atom.
\end{signature}

Let $\varepsilon, \delta$ denote positive real numbers.

    \begin{signature}
      $x < y$ is an atom.
    \end{signature}
    Let $x > y$ stand for $y < x$.
    Let $x \leq y$ stand for $x = y$ or $x < y$.

    \begin{axiom}
      If $x < y$ then not $y < x$.
    \end{axiom}
  \end{forthel}

  \section{Open balls}

  \begin{forthel}

    \begin{signature}
      $\Ball{\varepsilon}{z}$ is a subset of $\mathbb{C}$ that contains $z$.
    \end{signature}

    \begin{axiom}
      $|z| < |w|$ for some element $w$ of $\Ball{\varepsilon}{z}$.
    \end{axiom}

Let $M$ denote a subset of $\mathbb{C}$.

    \begin{definition}
      $M$ is open iff for every element $z$ of $M$ there exists $\varepsilon$ such that
        $\Ball{\varepsilon}{z}$ is a subset of $M$.
    \end{definition}

    \begin{axiom}
      $\Ball{\varepsilon}{z}$ is open.
    \end{axiom}

\begin{signature} A region is an open subset of $\mathbb{C}$.
\end{signature}

\begin{signature} Let $M$ be a region. $M$ is simply connected is an atom.
\end{signature}
\end{forthel}

\section{Holomorphic functions}

\begin{forthel}
    \begin{signature}
      A holomorphic function is a function $f$ such that $Dom(f) \subseteq \mathbb{C}$
and $f[Dom(f)] \subseteq \mathbb{C}$.
    \end{signature}

Let $f$ denote a holomorphic function.

    \begin{definition}
      A local maximal point of $f$ is an element $z$ of the domain of $f$ such that there exists 
$\varepsilon$ such that $\Ball{\varepsilon}{z}$ is a subset of the domain of $f$ and 
$|f(w)| \leq |f(z)|$ for every element $w$ of $\Ball{\varepsilon}{z}$.
    \end{definition}

    \begin{definition}
      Let $U$ be a subset of the domain of $f$. $f$ is constant on $U$ iff there exists 
$z$ such that $f(w) = z$ for every element $w$ of $U$.
    \end{definition}

    Let $f$ is constant stand for $f$ is constant on the domain of $f$.

    \begin{axiom}
      Assume $f$ is a holomorphic function and $\Ball{\varepsilon}{z}$ is a subset of the domain of $f$.
      If $f$ is not constant on $\Ball{\varepsilon}{z}$
        then $f[\Ball{\varepsilon}{z}]$ is open.
    \end{axiom}

    \begin{axiom}[Identity theorem]
      Assume $f$ is a holomorphic function and the domain of $f$ is a region.
      Assume that $\Ball{\varepsilon}{z}$ is a subset of the domain of $f$.
      If $f$ is constant on $\Ball{\varepsilon}{z}$ then $f$ is constant.
    \end{axiom}

    \begin{proposition}[Maximum modulus principle]
      Assume $f$ is a holomorphic function and the domain of $f$ is a region.
      If $f$ has a local maximal point then $f$ is constant.
    \end{proposition}
    \begin{proof}
      Let $z$ be a local maximal point of $f$.
      Take $\varepsilon$ such that
        $\Ball{\varepsilon}{z}$ is a subset of $\dom(f)$
        and $|f(w)| \leq |f(z)|$ for every element $w$ of $\Ball{\varepsilon}{z}$.

      Let us show that $f$ is constant on $\Ball{\varepsilon}{z}$.
      Proof by contradiction.
        Assume the contrary.
        Then $f[\Ball{\varepsilon}{z}]$ is open.
        We can take $\delta$ such that
          $\Ball{\delta}{f(z)}$ is a subset of $f[\Ball{\varepsilon}{z}]$.
        Therefore there exists an element $w$ of $\Ball{\varepsilon}{z}$ such that
          $|f(z)| < |f(w)|$. Contradiction.
    	End.

      Hence $f$ is constant.
    \end{proof}
  \end{forthel}

\end{document}
