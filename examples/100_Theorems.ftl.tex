\documentclass{article}
\usepackage[english]{babel}
\usepackage{../lib/tex/naproche}

\begin{document}

\newcommand{\Prod}[3]{#1_{#2} \cdots #1_{#3}}
\newcommand{\Seq}[2]{\{#1,\dots,#2\}}
\newcommand{\FinSet}[3]{\{#1_{#2},\dots,#1_{#3}\}}
\newcommand{\Primes}{\mathbb{P}}
\newcommand{\inv}[1]{#1^{-1}}

\title{Sets and Numbers - A readable and proof-checked $\mathbb{N}$aproche
formalization,
including some of Wiedijk's "100 Theorems"}

\author{Peter Koepke, Mateusz Marcol and Patrick Schäfer}

\maketitle

\tableofcontents
\newpage

\section*{Introduction}

The \Naproche{} system (for Natural Proof Checking)
checks the logical correctness of texts
written in an input language which ideally reads
like common mathematical language. Proofs and their 
structures should resemble the
style of undergraduate textbooks. 
This text serves several purposes: to demonstrate the 
\Naproche{} concept including its
input language ForTheL (for Formula Theory Language);
to lay a logical and set-theoretical foundation for
number systems in a standard build-up of mathematical notions
with sets and classes, maps, relations and numbers;
to prove several theorems from the well-known list of 100 theorems
that F. Wiedijk has proposed as formalization benchmarks.
 
Foundationally, the ordered field $\mathbb{R}$ of real numbers
is postulated axiomatically. We then construe the structures of integer and
rational numbers as substructures of $\mathbb{R}$:
\[ \mathbb{R} \supseteq \mathbb{Q} \supseteq \mathbb{Z}
\supseteq \mathbb{N}. \]
This corresponds to the intuition of a continuous line of numbers with
distinguished elements $0$ and $1$, from which integer and rational numbers can
be constructed geometrically. Technically, this has the advantage 
that the real addition and multiplication
can be {\em restricted} to those substructures, instead of {\em extending} operations
to larger superstructures.

We see the present state of \Naproche{} and this paper as proof of 
concept for interactive theorem
proving with natural formalizations. To extend \Naproche{} to a powerful comprehensive
system like Mizar or Isabelle is in principle possible, given sufficient resources.
We think, however, that it is more promising to equip big existing
system with a natural language interface by adapting 
\Naproche's translation techniques
from natural language texts into the formal logics of those systems.

\section{Basic Ideas}

The natural proof assistant \Naproche is intended to
approximate and support ordinary mathematical practices.
\Naproche employs the controlled natural language ForTheL
as its input language. Some ForTheL notions are already
built into \Naproche, as well as some basic properties of
these notions.
There are
mathematical objects, and sets and classes of mathematical
objects. Sets are classes which are objects themselves and
can thus be used in further mathematical constructions. Functions
and maps map objects to objects, where functions are those
maps which are themselves objects.

Modelling mathematical notions by objects corresponds
to the intuition that numbers, points, etc. should
not have internal set-theoretical
structurings, contrary to purely set-theoretical
foundations of mathematics. This is also advantageous
for automated proving since it prevents proof searches to
dig into mathematically irrelevant internal structurings.

As in common mathematics, sets and functions are required to
satisfy separation and replacement properties known from the
set theories of Kelley-Morse or Zermelo-Fraenkel.

The ontology presented here is more comprehensive than
ad-hoc first-order axiomatizations in some previous
\Naproche formalizations. Sometimes
this results in more complex and longer ontological checking tasks.
Controlling the complexitiy of automated proofs will be a major
issue for the further development.


\section{Importing Some Mathematical Language}

We import singular/plural forms of words that will be used in
our formalizations (\path{examples/vocabulary.ftl.tex}).
In the long run this should be replaced by
employing a proper English vocabulary. We also
import some alternative formulations for
useful mathematical phrases (\path{examples/macros.ftl.tex}).

\begin{forthel}
  [readtex \path{axioms.ftl.tex}]

  [readtex \path{vocabulary.ftl.tex}]

  [readtex \path{macros.ftl.tex}]
\end{forthel}


\section{Sets and Classes}

The notions of \textit{classes} and \textit{sets} are already
provided by \Naproche. Classes are usually defined by
abstraction terms $\{ ... \mid ... \}$. Since such terms need to be
processed by the parser we cannot introduce them simply by first-order
definitions but have to implement them in the software.
That sets are classes which are objects, or extensionality can be proved
from inbuilt assumptions, but it is important to reprove or restate such
facts so that they can directly be accessed by the ATP.

\begin{forthel}
  Let $S, T$ denote classes.

  \begin{definition}
    The empty set is the set that has no elements.
  \end{definition}

  \begin{definition}
    A subclass of $S$ is a class $T$ such that every $x \in T$ belongs to $S$.
  \end{definition}

  Let $T \subseteq S$ stand for $T$ is a subclass of $S$.

  Let a proper subclass of $S$ stand for a subclass $T$ of $S$ such that $T \neq
  S$.

  \begin{lemma}[Class Extensionality]
    If $S$ is a subclass of $T$ and $T$ is a subclass of $S$ then $S = T$.
  \end{lemma}

  \begin{definition}
    A subset of $S$ is a set $X$ such that $X \subseteq S$.
  \end{definition}

  Let a proper subset of $S$ stand for a subset $X$ of $S$ such that $X \neq S$.

  \begin{axiom}[Separation Axiom]
    Assume that $X$ is a set.
    Let $T$ be a subclass of $X$.
    Then $T$ is a set.
  \end{axiom}

  \begin{definition}
    The intersection of $S$ and $T$ is $\class{x \in S | x \in T}$.
  \end{definition}

  Let $S \cap T$ stand for the intersection of $S$ and $T$.

  \begin{definition}
    The union of $S$ and $T$ is $\class{x | x \in S \vee x \in T}$.
  \end{definition}

  Let $S \cup T$ stand for the union of $S$ and $T$.

  \begin{definition}
    The set difference of $S$ and $T$ is $\class{x \in S | x \notin T}$.
  \end{definition}

  Let $S \setminus T$ stand for the set difference of $S$ and $T$.

  \begin{definition}
    $S$ is disjoint from $T$ iff there is no element of $S$ that is an element
    of $T$.
  \end{definition}

  \begin{definition}
    A family is a set $F$ such that every element of $F$ is a set.
  \end{definition}

  \begin{definition}
    A disjoint family is a family $F$ such that $X$ is disjoint from $Y$ for all
    nonequal elements $X, Y$ of $F$.
  \end{definition}

    \begin{definition*}
      Let $X$ be a set.
      The powerset of $X$ is the collection of subsets of $X$.
    \end{definition*}

Let $\pow(X)$ denote the powerset of $X$.

    \begin{axiom*}
      The powerset of any set is a set.
    \end{axiom*}

\end{forthel}


\section{Pairs and Products}

Since we prefer objects over sets if possible, we do not work
with Kuratowski-style set-theoretical ordered pairs, but
axiomatize them as objects.

\begin{forthel}
  \begin{axiom}
    For any objects $a, b, c, d$ if $(a,b) = (c,d)$ then $a = c$ and $b = d$.
  \end{axiom}

  \begin{definition}
    $S \times T = \class{(x,y) | \text{$x \in S$ and $y \in T$}}$.
  \end{definition}

  \begin{axiom}
    Let $X, Y$ be sets.
    Then $X \times Y$ is a set.
  \end{axiom}

  \begin{lemma}
    Let $x, y$ be objects.
    If $(x,y)$ is an element of $S \times T$ then $x$ is an element of $S$ and
    $y$ is an element of $T$.
  \end{lemma}
\end{forthel}


\section{Functions and Maps}

The treatment of functions and maps is similar to that
of sets and classes.

\begin{forthel}
  Let $f$ stand for a map.

\begin{definition}
Let $f$ be a map. A value of $f$ is an object $y$ such
that $f(x) = y$ for some $x \in \dom(f)$.
\end{definition}


  \begin{axiom}
    Assume that $\dom(f)$ is a set and $f(x)$ is an object for any
    $x \in \dom(f)$.
    Then $f$ is a function.
  \end{axiom}

  \begin{definition}
    Assume $S$ is a subclass of the domain of $f$.
    $f[S] = \class{f(x) | x \in S}$.
  \end{definition}

\begin{definition}
Let $f$ be a map. $\range(f) = f[\dom(f)]$.
\end{definition}


  Let the image of $f$ stand for $f[\dom(f)]$.

\begin{definition}
    A map from $S$ to $T$ is a map $f$ such that $\dom(f) = S$ and $f[S]
    \subseteq T$.
  \end{definition}

Let $f : S \rightarrow T$ stand for $f$ is a map from $S$ to $T$.

\begin{definition}
A surjection from $S$
onto $T$ is a map $f$ from $S$ to $T$ such that $f[S] = T$.
\end{definition}

\begin{definition}
Let $f$ be a map. $f$ is injective iff $f(x) \neq f(y)$ for all
distinct elements $x,y$ of $\dom(f)$.
\end{definition}

\begin{signature}
Let $f$ be an injective map.
$\inv{f}$ is the map $g$ such that $\dom(g) = \range(f)$
and for all $v \in \dom(g)$ $g(v) \in \dom(f)$ and $f(g(v))=v$.
\end{signature}

\begin{lemma}
Let $f$ be an injective map. Then 
$\inv{f} : \range(f) \rightarrow \dom(f)$.
\end{lemma}


\begin{definition}
A bijection between $S$ and $T$ is an injective surjection from
$S$ onto $T$.
\end{definition}

    \begin{definition*}
      Let $x, y$ be sets.
      $x$ and $y$ are equinumerous iff there exists a bijection between $x$ and
      $y$.
    \end{definition*}


\begin{lemma} Let $S, T$ be classes.
Let $f$ be a bijection between $S$ and $T$.
Then $\inv{f}$ is a bijection between $T$ and $S$.
\end{lemma}

  \begin{axiom}[Replacement Axiom]
    Let $X$ be a set.
    Assume that $X$ is a subset of the domain of $f$.
    Then $f[X]$ is a set.
  \end{axiom}

  Let $g$ retracts $f$ stand for $g(f(x)) = x$ for all elements $x$ of
  $\dom(f)$.
  Let $h$ sections $f$ stand for $f(h(y)) = y$ for all elements $y$ of
  $\dom(h)$.

  \begin{definition}
    $f : S \leftrightarrow T$ iff $f : S \rightarrow T$ and there exists a map
    $g$ such that $g : T \rightarrow S$ and $g$ retracts $f$ and $g$ sections
    $f$.
  \end{definition}

\begin{definition*}
      Let $X$ be a set.
      A function of $X$ is a function $f$ such that $\dom(f) = X$.
    \end{definition*}

    \begin{definition*}
      Let $f$ be a function and $Y$ be a set.
      $f$ surjects onto $Y$ iff $Y = \class{f(x) | x \in \dom(f)}$.
    \end{definition*}

    Let a surjective function from $X$ onto $Y$ stand for a function of $X$ that
    surjects onto $Y$.


Let $A,B$ stand for sets.
Let $x$ is in $A$ denote $x$ is an element of $A$.

\begin{definition}[1 3]
$A$ is nonempty iff $A$ has an element.
\end{definition}

\end{forthel}

\section{* Cantor's Theorem}

In this document we give a proof of Cantor's Theorem:

  \begin{theorem*}
    There is no surjection from a set onto its powerset.
  \end{theorem*}

  We need to provide certain definitions concerning surjective
  functions and the notion of powerset.
\begin{forthel}
\end{forthel}
    \begin{definition*}
      Let $X$ be a set.
      A function of $X$ is a function $f$ such that $\dom(f) = X$.
    \end{definition*}

    \begin{definition*}
      Let $f$ be a function and $Y$ be a set.
      $f$ surjects onto $Y$ iff $Y = \class{f(x) | x \in \dom(f)}$.
    \end{definition*}

    Let a surjective function from $X$ onto $Y$ stand for a function of $X$ that
    surjects onto $Y$.
  On this basis Cantor's theorem and its proof can be formalized as follows.
\begin{forthel}
    \begin{theorem*}[Cantor]
      Let $M$ be a set. Then there is 
      no surjective function from $M$ onto the powerset of $M$.
    \end{theorem*}
    \begin{proof}
      Assume the contrary.
      Take a surjective function $f$ from $M$ onto the powerset of $M$.
      % The value of $f$ at any element of $M$ is a set.
      Define \[ N = \class{x \in M | \text{$x$ is not an element of $f(x)$}}. \]
      %$N$ is a subset of $M$.
      Take an element $z$ of $M$ such that $f(z) = N$.
      Then \[ z \in N \iff z \notin f(z) = N. \]
      Contradiction.
    \end{proof}
\end{forthel}

In this document we give a proof of Cantor's Theorem:

  \begin{theorem*}
    There is no surjection from a set onto its powerset.
  \end{theorem*}

  We need to provide certain definitions concerning surjective
  functions and the notion of powerset.

    \begin{definition*}
      Let $X$ be a set.
      A function of $X$ is a function $f$ such that $\dom(f) = X$.
    \end{definition*}

    \begin{definition*}
      Let $f$ be a function and $Y$ be a set.
      $f$ surjects onto $Y$ iff $Y = \class{f(x) | x \in \dom(f)}$.
    \end{definition*}

    Let a surjective function from $X$ to $Y$ stand for a function of $X$ that
    surjects onto $Y$.

    \begin{definition*}
      Let $X$ be a set.
      The powerset of $X$ is the collection of subsets of $X$.
    \end{definition*}

    \begin{axiom*}
      The powerset of any set is a set.
    \end{axiom*}

  On this basis Cantor's theorem and its proof can be formalized as follows.

    \begin{theorem*}[Cantor]
      Let $M$ be a set.
      No function of $M$ surjects onto the powerset of $M$.
    \end{theorem*}
    \begin{proof}
      Assume the contrary.
      Take a surjective function $f$ from $M$ to the powerset of $M$.
      The value of $f$ at any element of $M$ is a set.
      Define \[ N = \class{x \in M | \text{$x$ is not an element of $f(x)$}}. \]
      $N$ is a subset of $M$.
      Consider an element $z$ of $M$ such that $f(z) = N$.
      Then \[ z \in N \iff z \notin f(z) = N. \]
      Contradiction.
    \end{proof}


\section{The Real Field}

\begin{forthel}
[synonym number/-s]

\begin{signature}
A real number is a mathematical object.
\end{signature}

\begin{definition}
$\mathbb{R}$ is the collection of real numbers.
\end{definition}
Let $x,y,z,w$ denote real numbers.

\begin{axiom}
$\mathbb{R}$ is a set.
\end{axiom}


\begin{signature}[1 12 A1]
$x + y$ is a real number.
\end{signature}
Let the sum of $x$ and $y$ stand for $x + y$.

\begin{signature}[1 12 M1]
$x \cdot y$ is a real number.
\end{signature}
Let the product of $x$ and $y$ denote $x \cdot y$.

\begin{signature}[1 5]
$x < y$ is an atom.
\end{signature}
Let $x > y$ stand for $y < x$.
Let $x \leq y$ stand for $x < y \vee x = y$.
Let $x \geq y$ stand for $y \leq x$.

\begin{axiom}[1 5 i]
$(x < y \wedge x \neq y) \wedge not y < x$
or $not x < y \wedge x = y \wedge not y < x$
or $not x < y \wedge x \neq y \wedge y < x$.
\end{axiom}

\begin{axiom}[1 5 ii]
If $x < y$ and $y < z$ then $x < z$.
\end{axiom}

\begin{proposition}
$x \leq y$ iff not $x > y$.
\end{proposition}

\begin{axiom}[1 12 A2]
$x + y = y + x$.
\end{axiom}

\begin{axiom}[1 12 A3]
$(x + y) + z = x + (y + z)$.
\end{axiom}

\begin{signature}[1 12 A4]
$0$ is a real number such that
for every real number $x$ $x + 0 = x$.
\end{signature}

\begin{signature}[1 12 A5]
$-x$ is a real number such that $x + (-x) = 0$.
\end{signature}

\begin{axiom}[1 12 M2]
$x \cdot y = y \cdot x$.
\end{axiom}

\begin{axiom}[1 12 M3]
$((x \cdot y)) \cdot z = x \cdot (y \cdot z)$.
\end{axiom}

\begin{signature}[1 12 M4]
$1$ is a real number such that $1 \neq 0$ and
for every real number $x$ $1 \cdot x = x$.
\end{signature}

\begin{signature}[1 12 M5]
Assume $x \neq 0$. $1/x$ is a real number
such that $x \cdot (1/x) = 1$.
\end{signature}

Let $x$ is nonzero stand for $x \neq 0$.

\begin{axiom}[1 12 D]
$x \cdot (y + z) = (x \cdot y) + (x \cdot z)$.
\end{axiom}

\begin{proposition}[Dist1]
$((y \cdot x)) + (z \cdot x) = (y + z) \cdot x$.
\end{proposition}

\begin{proposition}[1 14 a]
If $x + y = x + z$ then $y = z$.
\end{proposition}
\begin{proof}
Assume $x + y = x + z$.
Then \[ y = (-x+x) + y = -x + (x+y) = -x + (x+z) = (-x+x) + z = z. \]
\end{proof}

\begin{proposition}[1 14 b]
If $x + y = x$ then $y = 0$.
\end{proposition}

\begin{proposition}[1 14 c]
If $x + y = 0$ then $y = -x$.
\end{proposition}

\begin{proposition}[1 14 d]
$-(-x) = x$.
\end{proposition}


\begin{proposition}[1 15 a]
If $x$ is nonzero and $x \cdot y = x \cdot z$
then $y = z$.
\end{proposition}
\begin{proof}
Let $x$ be nonzero and $x \cdot y = x \cdot z$.
\[ y = 1 \cdot y = ((1/x) \cdot x) \cdot y = (1/x) \cdot (x \cdot y) =
(1/x) \cdot (x \cdot z) = ((1/x) \cdot x) \cdot z = 1 \cdot z = z. \]
\end{proof}

\begin{proposition}[1 15 b]
If $x$ is nonzero and $x \cdot y = x$ then $y = 1$.
\end{proposition}

\begin{proposition}[1 15 c]
If $x$ is nonzero and $x \cdot y = 1$ then $y = 1/x$.
\end{proposition}

\begin{proposition}[1 15 d]
If $x$ is nonzero then $1/(1/x) = x$.
\end{proposition}

\begin{proposition}[1 16 a]
$0 \cdot x = 0$.
\end{proposition}

\begin{proposition}[1 16 b]
If $x$ is nonzero and $y \neq 0$ then $x \cdot y \neq 0$.
\end{proposition}

\begin{proposition}[1 16 c]
$(-x) \cdot y = -(x \cdot y)$.
\end{proposition}
\begin{proof}
$((x \cdot y)) + ((-x \cdot y)) = (x + (-x)) \cdot y
= 0 \cdot y = 0$.
\end{proof}

\begin{proposition}
$-x = -1 \cdot x$.
\end{proposition}

\begin{proposition}[1 16d]
$(-x) \cdot (-y) = x \cdot y$.
\end{proposition}
\begin{proof}
$(-x)\cdot (-y) = -(x\cdot(-y)) = -((-y)\cdot x) =
-(-(y\cdot x)) = y\cdot x = x\cdot y$.
\end{proof}

Let $x - y$ stand for $x + (-y)$.
Let $\frac{x}{y}$ stand for $(x \cdot (1/y))$.

\begin{lemma} Let $z \neq 0$. Then
$x = \frac{z \cdot x}{z}$.
\end{lemma}

\begin{lemma}
$(1 - x) \cdot y = y - (x \cdot y)$.
\end{lemma}

\begin{lemma} Let $w \neq 0$. Then
$\frac{x - y}{w} + \frac{y - z}{w} = \frac{x - z}{w}$.
\end{lemma}
\begin{proof}
$(x - y) + (y - z) = ((x - y) + y) - z = x - z$.
\end{proof}

\end{forthel}

\section{Some Numbers}

\begin{forthel}

\begin{definition}
$2 = 1 + 1$.
\end{definition}

\begin{definition}
$3 = 2 + 1$.
\end{definition}

\begin{definition}
$4 = 3 + 1$.
\end{definition}

\begin{definition}
$5 = 4 + 1$.
\end{definition}

\begin{definition}
$6 = 5 + 1$.
\end{definition}

\begin{definition}
$7 = 6 + 1$.
\end{definition}

\begin{definition}
$8 = 7 + 1$.
\end{definition}

\begin{definition}
$9 = 8 + 1$.
\end{definition}

\begin{definition}
$10 = 9 + 1$.
\end{definition}
\end{forthel}
With these numbers we can prove some small "decimal arithmetic":
\begin{forthel}

\begin{lemma}
$2 + 5 = 7$.
\end{lemma}

\begin{lemma}
$2 \cdot 5 = 10$.
\end{lemma}

\begin{lemma}
$6 \cdot 4 = (2 \cdot 10) + 4$.
\end{lemma}


\end{forthel}

\section{The Real Ordered Field}

\begin{forthel}

\begin{axiom}[1 17 i]
If $y < z$ then $x + y < x + z$ and $y + x < z + x$.
\end{axiom}

\begin{axiom}[1 17 ii]
If $x > 0$ and $y > 0$ then $x \cdot y > 0$.
\end{axiom}

\begin{definition}
$x$ is positive iff $x > 0$.
\end{definition}

\begin{definition}
$x$ is negative iff $x < 0$.
\end{definition}

\begin{definition}
$x$ is nonnegative iff $x \geq 0$.
\end{definition}


\begin{proposition}[1 18 a]
$x > 0$ iff $-x < 0$.
\end{proposition}
\begin{proof}
If $x > 0$ then $0 = x + (-x) > 0 + (-x) = -x$.
If $-x < 0$ then $0 = x + (-x) < x + 0 = x$.
\end{proof}

\begin{proposition}[1 18 b]
If $x > 0$ and $y < z$ then
$x \cdot y < x \cdot z$.
\end{proposition}
\begin{proof}
Let $x > 0$ and $y < z$.
$z - y > y - y = 0$.
$x \cdot (z - y) > 0$.
$x \cdot z = (x \cdot (z - y)) + (x \cdot y)$.
$((x \cdot (z - y))) + (x \cdot y)  > 0 + (x \cdot y)$ (by 1 17i).
$0 + (x \cdot y) = x \cdot y$.
\end{proof}

\begin{proposition}[1 18 bb]
If $x > 0$ and $y < z$ then
$y \cdot x < z \cdot x$.
\end{proposition}


\begin{proposition}[1 18 d]
If $x \neq 0$ then $x \cdot x > 0$.
\end{proposition}
\begin{proof}
Let $x \neq 0$.
Case $x > 0$. Then thesis. end.
Case $x < 0$. Then $ -x > 0$ and $x \cdot x = (-x) \cdot (-x) > 0$. end.
\end{proof}

\begin{proposition}[1 18 dd]
$1 > 0$.
\end{proposition}

\begin{proposition}
$x < y$ iff $-x > -y$.
\end{proposition}
\begin{proof}
$x < y \iff x - y < 0$.
$x - y < 0 \iff (-y) + x < 0$.
$(-y) + x < 0 \iff (-y)+(-(-x)) < 0$.
$(-y)+(-(-x)) < 0 \iff (-y)-(-x) < 0$.
$(-y)-(-x) < 0 \iff -y < -x$.
\end{proof}

\begin{proposition}[1 18 c]
If $x < 0$ and $y < z$ then
$x \cdot y > x \cdot z$.
\end{proposition}
\begin{proof}
Let $x < 0$ and $y < z$.
$-x > 0$.
$(-x)\cdot y < (-x)\cdot z$ (by 1 18b).
$-(x\cdot y) < -(x\cdot z)$.
\end{proof}

\begin{proposition}[1 18 cc]
If $x < 0$ and $y < z$ then
$y \cdot x > z \cdot x$.
\end{proposition}

\begin{proposition}
$-1 < 0$.
\end{proposition}

\begin{proposition}
$x - 1 < x$.
\end{proposition}
\begin{proof}
$x - 1 < x + 0 = x$.
\end{proof}
[timelimit 10]
\begin{proposition}[1 18 e]
If $0 < y$ then $0 < 1/y$.
\end{proposition}
[timelimit 3]
\begin{proposition}[1 18 ee]
Assume $0 < x < y$.
Then $1/y < 1/x$.
\end{proposition}
\begin{proof}
Case $1/x < 1/y$.
Then
\[ 1 = x \cdot (1/x) = (1/x) \cdot x < (1/x) \cdot y =
y \cdot (1/x) < y \cdot (1/y) = 1. \]
Contradiction. end.

Case $1/x = 1/y$. Then
\[ 1 = x \cdot (1/x) < y \cdot (1/y) = 1. \]
Contradiction. end.

Hence $1/y < 1/x$ (by 1 5 i).
\end{proof}

\end{forthel}


\section{Upper and lower bounds}

\begin{forthel}

Let $E$ denote a subset of $\mathbb{R}$.

\begin{definition}[1 7]
An upper bound of $E$ is a
real number $b$ such that for all elements $x$ of $E$ $x \leq b$.
\end{definition}

\begin{definition}[1 7a]
$E$ is bounded above iff
$E$ has an upper bound.
\end{definition}

\begin{definition}[1 7b]
A lower bound of $E$ is a
real number $b$ such that for all elements $x$ of $E$ $x \geq b$.
\end{definition}

\begin{definition}[1 7c]
$E$ is bounded below iff
$E$ has a lower bound.
\end{definition}

\begin{definition}[1 8]
A least upper bound of $E$ is a real number $a$ such that
$a$ is an upper bound of $E$ and for all $x$ if $x < a$ then $x$
is not an upper bound of $E$.
\end{definition}

\begin{definition}[1 8a]
Let $E$ be bounded below.
A greatest lower bound of $E$ is a real number $a$ such that
$a$ is a lower bound of $E$ and for all $x$ if $x > a$ then $x$ is
not a lower bound of $E$.
\end{definition}

\begin{axiom}[1 19]
Assume that $E$ is nonempty and bounded above.
Then $E$ has a least upper bound.
\end{axiom}

\begin{definition}
$E^- = \{-x \in \mathbb{R} \mid x \in E\}$.
\end{definition}

\begin{lemma}
$E^-$ is a subset of $\mathbb{R}$.
\end{lemma}
[timelimit 10]
\begin{lemma}
$x$ is an upper bound of $E$ iff $-x$ is a lower bound of $E^-$.
\end{lemma}
[timelimit 3]

\begin{theorem}[1 11] Assume that $E$ is a nonempty subset of $\mathbb{R}$
such that $E$ is bounded below.
Then $E$ has a greatest lower bound.\end{theorem}
\begin{proof}
Take a lower bound $a$ of $E$.
$-a$ is an upper bound of $E^-$.
Take a least upper bound $b$ of $E^-$.
Let us show that $-b$ is a greatest lower bound of $E$.
$-b$ is a lower bound of $E$. Let $c$ be a lower bound of $E$.
Then $-c$ is an upper bound of $E^-$.
end.
\end{proof}

\end{forthel}


\section{The rational numbers}

\begin{forthel}

[synonym number/numbers]

\begin{signature}[1 19a]
A rational number is a real number.
\end{signature}

Let $p,q,r$ denote rational numbers.

\begin{definition}
$\mathbb{Q}$ is the collection of rational numbers.
\end{definition}

\begin{theorem}
$\mathbb{Q}$ is a set.
\end{theorem}
\begin{proof}
$\mathbb{Q}$ is a subset of $\mathbb{R}$.
\end{proof}

\end{forthel}


Theorem 1.19 of \cite{Rudin} states that $\mathbb{Q}$ is a
subfield of $\mathbb{R}$. We require this axiomatically.


\begin{forthel}

\begin{lemma}
$\mathbb{Q} \subseteq \mathbb{R}$.
\end{lemma}

\begin{axiom}
$p + q$, $p \cdot q$, $0$, $-p$, $1$ are rational numbers.
\end{axiom}

\begin{axiom}
Assume $p \neq 0$.
$1/p$ is a rational number.
\end{axiom}

\begin{axiom}
There exists a subset $A$ of $\mathbb{Q}$
such that $A$ is bounded above and
$x$ is a least upper bound of $A$.
\end{axiom}

\begin{theorem}
$\mathbb{R} = \{x \in \mathbb{R} \mid$ there exists $
A \subseteq \mathbb{Q}$ such that $
A$ is bounded above and $x$ is a least upper
bound of $A\}$.
\end{theorem}

\end{forthel}


\section{Integers}

\begin{forthel}

[synonym integer/-s]

\begin{signature}
An integer is a rational number.
\end{signature}
Let $a,b,i$ stand for integers.

\begin{definition}
$\mathbb{Z}$ is the collection of integers.
\end{definition}


\begin{theorem}
$\mathbb{Z}$ is a subset of $\mathbb{R}$.
\end{theorem}

\end{forthel}


$\mathbb{Z}$ is a discrete subring of $\mathbb{Q}$:


\begin{forthel}

\begin{axiom}
$a + b$, $a \cdot b$, $0$, $-a$, $1$ are integers.
\end{axiom}

\begin{axiom}
There is no integer $a$ such that $0 < a < 1$.
\end{axiom}

\begin{axiom}
There exist $a$ and $b$ such that
$a \neq 0 \wedge p = \frac{b}{a}$.
\end{axiom}

\begin{theorem}[Archimedes1]
$\mathbb{Z}$ is not bounded above.
\end{theorem}
\begin{proof}
Assume the contrary.
$\mathbb{Z}$ is nonempty. Indeed $0$ is an integer.
Take a least upper bound
$b$ of $\mathbb{Z}$.
Let us show that $b - 1$ is an upper bound of $\mathbb{Z}$.
Let $x \in \mathbb{Z}$. $x + 1 \in \mathbb{Z}$.
$x + 1 \leq b$.
$x = (x + 1) - 1 \leq b - 1$.
end.
\end{proof}

\begin{theorem}
$\mathbb{Z}$ is not bounded below.
\end{theorem}
\begin{proof}
Assume the contrary.
Take a real number $x$ that is a lower bound of $\mathbb{Z}$.
Then $-x$ is an upper bound of $\mathbb{Z}$.
Contradiction.
\end{proof}

\begin{theorem}[Archimedes2]
Let $x$ be a real number.
Then there is an integer $a$
such that $x < a$.\end{theorem}
\begin{proof}[by contradiction]
Assume the contrary.
Then $x$ is an upper bound of $\mathbb{Z}$.
Contradiction.
\end{proof}


\end{forthel}


\section{Division}

\begin{forthel}

Let $l,m,n$ denote integers.

\begin{definition}
$n$ divides $m$ iff for some $l$ $m = n \cdot l$.
\end{definition}

Let $x | y$ denote $x$ divides $y$.
Let a divisor of $x$ denote an integer that divides $x$.

\begin{lemma}
Assume $l | m | n$.
Then $l | n$.
\end{lemma}
\begin{proof}
Take integers $u,v$ such that $m = l \cdot u$ and $n = m \cdot v$.
Then $n = l \cdot (u \cdot v)$.
\end{proof}

\begin{lemma}
Let $l | m$ and $l | m + n$.
Then $l | n$.
\end{lemma}
\begin{proof}
Assume that $l$ is nonzero.
Take an integer $p$ such that $m = l \cdot p$.
Take an integer $q$ such that $m + n = l \cdot q$.
Take $r = q - p$.
We have $(l \cdot p) + (l \cdot r) = l \cdot q = m + n = (l \cdot p) + n$.
Hence $n = l \cdot r$.
\end{proof}


\end{forthel}


\section{The natural numbers}

\begin{forthel}

\begin{definition}
A natural number is a nonnegative integer.
\end{definition}

\begin{definition}
$\mathbb{N}$ is the collection
of natural numbers.
\end{definition}
Let $l,m,n$ stand for natural numbers.

\begin{lemma}
$\mathbb{N}$ is a subset of $\mathbb{R}$.
\end{lemma}

\begin{lemma} Let $n$ be a natural number. Then
$n = 0$ or $n = m + 1$ for some natural number $m$.
\end{lemma}
\begin{proof}
Case $n = 0$. Trivial.
$n \neq 0$. Let $m = n - 1$.
$n = m + 1$. $m$ is nonnegative.
\end{proof}

%\begin{lemma} n=n. \end{lemma}

\begin{lemma}
$\mathbb{Z} = \mathbb{N}^- \cup \mathbb{N}$.
\end{lemma}



\begin{lemma}
If $l + m = l + n$ or $m + l = n + l$ then $m = n$.
\end{lemma}

\begin{lemma}
Assume that $l$ is nonzero.
If $l \cdot m = l \cdot n$ or $m \cdot l = n \cdot l$ then $m = n$.
\end{lemma}

\begin{lemma}
If $m + n = 0$ then $m = 0$ and $n = 0$.
\end{lemma}

\begin{lemma}
Assume $m \leq n$. Then
there exists a natural number $l$ such that $m + l = n$.
\end{lemma}
\begin{proof}
$n - m$ is a natural number. $m + (n-m) = (n - m) + m = n$.
\end{proof}

\begin{lemma}
Assume there exists a natural number $l$ such 
that $m + l = n$. Then $m \leq n$.
\end{lemma}

\begin{lemma}
For every natural number $n$ $n = 0$ or $1 \leq n$.
\end{lemma}


\begin{lemma}
Let $m \neq 0$. Then $n \leq n \cdot m$.
\end{lemma}
\begin{proof}
$1 \leq m$.
\end{proof}

Let $x$ is nontrivial stand for $x \neq 0$ and $x \neq 1$.

\begin{definition}
$n$ is prime iff $n$ is nontrivial and
for every natural number $m$ that divides $n$ $m = 1$ or $m = n$.
\end{definition}

\end{forthel}

\section{The Principle of Mathematical Induction (Wiedijk \# 74)}

\Naproche provides a special binary relation
symbol $\prec$ for a universal inductive relation: if at any
point $m$ property $P$ is inherited at $m$ provided all
$\prec$-predecessors of $m$ satisfy $P$, then $P$ holds everywhere.
Induction along $<$ is ensured by:


\begin{forthel}

\begin{theorem}[Induction Theorem]
Assume $A \subseteq \mathbb{N}$
and $0 \in A$ and for all $n \in A$ $n + 1 \in A$.
Then $A = \mathbb{N}$.
\end{theorem}

\begin{proof}
Let us show that every element of $\mathbb{N}$ is an element of $A$.
	Let $n \in \mathbb{N}$.
	Assume the contrary.
	Define $F = \{  j \in \mathbb{N} \mid j \notin A\}$.
	$F$ is nonempty. $F$ is bounded below.
  Take a greatest lower bound $a$ of $F$.
	Let us show that $a+1$ is a lower bound of $F$.
		Let $x \in F$. $x - 1 \in \mathbb{Z}$.

		Case $x - 1 < 0$. Then $0 < x < 1$. Contradiction. end.

		Case $x - 1 = 0$. Then $x = 1$ and $1 \notin A$. Contradiction. end.

		Case $x - 1 > 0$. Then $x - 1 \in \mathbb{N}$.
      $x - 1 \notin A$. Indeed $(x - 1) + 1 = x \notin A$. $x - 1 \in F$.
			$a \leq x - 1$.
			$a + 1 \leq (x - 1) + 1 = x$.
		end.
	end.

	Then $a+1 > a$.
	Contradiction.
end.
\end{proof}

Let $m$ is inductively smaller than $n$ stand for $m \prec n$.

\begin{axiom} Let $m$ be a natural number. Then
$m$ is inductively smaller than $m+1$.
\end{axiom}

\begin{axiom}
If $n < m$ then $n \prec m$.
\end{axiom}




\begin{lemma}
There is an integer $m$ such that
$m -1 \leq x < m$.
\end{lemma}
\begin{proof}
Assume the contrary.
Then for all integers $m$ such that $x \geq m-1 $ we have $x \geq m$.
Take an integer $n$ such that $n < x$. Indeed $x$ is not a lower bound of $\mathbb{Z}$.
Define
\[ A = \{i \in \mathbb{N} \mid n + (i - 1) \leq x\}. \]
(1) $A = \mathbb{N}$.

Proof.
$A \subseteq \mathbb{N}$.
$1 \in A$.

For all $i \in A$ $i + 1 \in A$.

  Proof. Let $i \in A$. Then
  $n + (i -1) = (n + i) - 1 \leq x$ and
  $n + ((i + 1) - 1) = n + i \leq x$.
[timelimit 10]
  Hence $i + 1 \in A$.
[timelimit 3]
  qed.

qed.

(2) $x + 1$ is an upper bound of $\mathbb{Z}$.

Proof.
Let $j$ be an integer. Let us show that $j \leq x + 1$.

Case $j \leq n$. Then $j \leq x \leq x + 1$. Trivial.

Then $j > n$. Take a positive integer $i$
such that $j = n + i$. 
Indeed $j - n$ is a positive integer and  
$j = n + (j-n)$.
$i \in A$.
Hence $n + (i-1) \leq x$. $j = n+i = (n + (i-1)) + 1 \leq x + 1$.
qed.

qed.

Contradiction.
\end{proof}



\end{forthel}

\section{Exponentiation}

\begin{forthel}
Let $x$ denote real numbers.
Let $i$ denote integers.

\begin{signature} $x^{i}$ is a real number.
\end{signature}

\begin{axiom} $x^{0} = 1$. \end{axiom}

\begin{axiom} If $i \geq 0$ then $x^{i+1} = x^{i} \cdot x$.
\end{axiom}

\begin{lemma}
$x \cdot x^{m} = x^{m+1}$.
\end{lemma}

\begin{axiom}
Let $m$ be a natural number. 
Then $x^{m}$ is a real number and
$(1 - x^{m}) + (x^{m} - x^{m+1}) = 1 - x^{m+1}$.
\end{axiom}

\end{forthel}


\section{* Sum of a Geometric Series}

\newcommand{\sumgeom}[2]{\sum_{0 \leq i < #2} {#1}^i}

{\em Case construct in induction proof should have 
"Otherwise" to denote the remaining case. Beautify the 
[...] commands}

We treat the partial sums $\sumgeom{x}{n}$ of a geometric
series as a function in $x$ and $n$ which satisfies some
recursive axioms:

\begin{forthel}

\begin{lemma} 0=0. \end{lemma}

Let $x,y,z,w$ denote real numbers. 
Let $m,n$ denote natural number.


\begin{signature}
$\sumgeom{x}{n}$ is a real number.
\end{signature}

\begin{axiom}
$\sumgeom{x}{0} = 0$.
\end{axiom}

\begin{axiom} Let $x$ be a real number and $n$ be
a natural number. Then $n+1$ is a natural number and
$\sumgeom{x}{n+1} = (\sumgeom{x}{n}) + x^{n}$.
\end{axiom}

% That's the proof that can be found in the file
% W-Sum_of_a_geometric_series
% 
% Let $x$ denote real number. 
% 
% 
% \begin{theorem} Let $x \neq 1$. Then
% $$\sumgeom{x}{n} = \frac{1 - x^{n}}{1 - x}$$
% for all natural numbers $n$.
% \end{theorem}
% \begin{proof}[by induction on $n$]
% 
% $1 - x \neq 0$.
% 
% Let $n$ be a natural number.
% 
% Case $n = 0$. Trivial.
% 
% Take a natural number $m$ such that $m + 1 = n$. 
% $m$ is inductively smaller than $m+1$ and
% $\sumgeom{x}{m} = \frac{1 - x^{m}}{1 - x}$.
% 
% 
% Let $a = x^{m}$ and $b = x^{n}$ and $c=1-x$.
% Then $a,b,c$ are real numbers and
% $\sumgeom{x}{n} =
% \frac{1 - x^{m}}{1 - x} + x^{m} = 
% \frac{1 - x^{m}}{1 - x} + \frac{x^{m} \cdot (1-x)}{1 - x} =
% \frac{1 - x^{m}}{1 - x} + \frac{(x^{m} \cdot 1) + (x^{m} \cdot (-x))}{1 - x} = 
% \frac{1 - x^{m}}{1 - x} + \frac{x^{m} + ((-x) \cdot x^{m})}{1 - x} =
% \frac{1 - x^{m}}{1 - x} + \frac{x^{m} + (-(x \cdot x^{m}))}{1 - x} =
% \frac{1 - x^{m}}{1 - x} + \frac{x^{m} - x^{n}}{1 - x} = 
% \frac{1 - a}{c} + \frac{a - b}{c} =
% \frac{1 - b}{c} = 
% \frac{1 - x^{n}}{1 - x}$.
% 
% \end{proof}

\end{forthel}


\section{Finite Sequences and Products}

\begin{forthel}
[prover eprover]
\begin{definition}
$\Seq{m}{n}$ is the class of
natural numbers $i$ such that $m \leq i \leq n$.
\end{definition}

\begin{axiom}
$\Seq{m}{n}$ is a set.
\end{axiom}

\begin{axiom}
Assume $F$ is a function and $x \in \dom(F)$.
Then $F(x)$ is an object.
\end{axiom}

\begin{definition}
A sequence of length $n$ is a
function $F$ such that $\dom(F) = \Seq{1}{n}$.
\end{definition}

Let $F_{i}$ stand for $F(i)$.

\begin{definition}
Let $F$ be a sequence of length $n$.
$\FinSet{F}{1}{n} = \{ F_{i} | i \in \dom(F)\}$.
\end{definition}

\begin{signature}
Let $F$ be a sequence of length $n$
such that $\FinSet{F}{1}{n} \subseteq \mathbb{N}$.
$\Prod{F}{1}{n}$ is a natural number.
\end{signature}

\begin{axiom}[Factorproperty]
Let $F$ be a sequence of length $n$
such that $F(i)$ is a nonzero natural number for every $i \in \dom(F)$.
Then $\Prod{F}{1}{n}$ is nonzero and
$F(i)$ divides $\Prod{F}{1}{n}$ for every $i \in \dom(F)$.
\end{axiom}

\end{forthel}


\section{Finite and Infinite Sets}

\begin{forthel}

Let $S$ denote a class.

\begin{definition}
$S$ is finite iff
$S = \FinSet{s}{1}{n}$ for some natural number $n$ and some function $s$ that is
a sequence of length $n$.
\end{definition}

\begin{definition}
$S$ is infinite iff $S$ is not finite.
\end{definition}

\end{forthel}


\section{* The Infinitude of Primes}


%\begin{lemma} 1 = 1. \end{lemma}

%%%%%%Let $F_{i}$ stand for $F(i)$.

\begin{lemma}
Every nontrivial natural number $n$ has a divisor 
that is a prime natural number.
\end{lemma}
\begin{proof}[by induction on $n$]
Let $n$ be a nontrivial natural number. %Let $n$ be nontrivial.
Case $n$ is prime. Trivial.
Case $n$ is not prime.
Take a natural number $m$ such that $m$ divides $n$ 
and $m \neq 1$ and $m \neq n$. $m \leq n$.
$m$ is inductively smaller than $n$. $m$ is nontrivial.
Take a divisor $p$ of $m$ that is a prime natural number.
Then $p$ is a prime divisor of $n$.
end.
\end{proof}


\begin{signature}
$\Primes$ is the collection of prime natural numbers.
\end{signature}

\begin{theorem}[Euclid]
$\Primes$ is infinite.
\end{theorem}
\begin{proof}
Assume that $r$ is a natural number and
$p$ is a sequence of length $r$ and
$\FinSet{p}{1}{r}$ is a subclass of $\Primes$.
(1) $p_{i}$ is a nonzero natural number for every
$i \in  \dom(p)$.
Consider $n=\Prod{p}{1}{r}+1$.
$\Prod{p}{1}{r}$ is nonzero (by Factorproperty).
$n$ is nontrivial.
Take a divisor $q$ of $n$ that is a prime natural number.
Let us show that $q \neq p_{i}$ 
for all natural numbers $i$ such that $1 \leq i \leq r$.

Proof by contradiction.
Assume that $q=p_{i}$ for some natural number $i$ such that
$1 \leq i \leq r$.
Take a natural number $i$ such that $1 \leq i \leq r$
and $q=p_{i}$.
$q$ is a divisor of $n$.
$i \in \dom(p)$. $p_{i}$ is a divisor of $\Prod{p}{1}{r}$
(by Factorproperty, 1).
Thus $q$ divides $1$.
Contradiction. qed.

Hence $\FinSet{p}{1}{r}$ is not the class of prime natural numbers.
\end{proof}
\section{* Greatest Common Divisor Algorithm}

\begin{forthel}
Let $m,n$ denote integers.


\begin{definition}
The greatest common divisor of $m$ and $n$ is a natural number.
\end{definition}

\begin{axiom}
The greatest common divisor of $m$ and $n$ is a divisor of $m$ and a divisor of $n$.
\end{axiom}
\begin{axiom}
Let $d$ be a divisor of $m$ and a divisor of $n$.
Then $d$ is a divisor of the greatest common divisor of $m$ and $n$.
\end{axiom}


\begin{definition}
$\gcd(m,n)$ is a natural number.
\end{definition}

\begin{axiom}
If $m$ is not a natural number then $\gcd(m,n) = \gcd(-m,n)$ and
if $n$ is not a natural number then $\gcd(m,n) = \gcd(m,-n)$.
\end{axiom}
\begin{axiom}
If $m=0$ then $\gcd(m,n)=n$ and if $n=0$ then $\gcd(m,n)=m$.
\end{axiom}
\begin{axiom}
If $m \leq n$ then $\gcd(m,n) = \gcd(m,n-m)$ and if $n < m$ then $\gcd(m,n) = \gcd(m-n,n)$.
\end{axiom}


\begin{proposition}
The greatest common divisor of $m$ and $n$ is $\gcd(m,n)$.
\end{proposition}

\end{forthel}

\section{* Bezout's Theorem}


\begin{forthel}

Let $a,b$ denote integers.

\begin{proposition}
There exist integers $s,t$ such that $(a \cdot s) + (b \cdot t) = \gcd(a,b)$.
\end{proposition}

\end{forthel}

\section{* The Irrationality of the Square Root of 2}



\begin{forthel}

Let a prime number stand for a prime natural number.
Let $p$ denote a prime number.


\begin{definition}
    Let $n,m$ be integers.
    $n$ and $m$ are coprime iff $n$ and $m$ have no common divisor $d$ such that $d$ is a prime number.
\end{definition}

\begin{lemma}
    For all rational numbers $q$ there exist coprime integers $m,n$ such that $m \cdot q = n$.
\end{lemma}



Let $n$ denote an integer.
\begin{lemma}
    If $p$ divides $n^{2}$ then $p$ divides $n$.
\end{lemma}
%\begin{proof}
%    Assume $p$ divides $n^{2}$ and $p$ not divides $n$. $p$ not divides $n$ and $\gcd(p,n)=1$.
%    There exist integers $j,k$ such that $(k \cdot n) + (j \cdot p) = 1$.
%    Let $k,j$ be integers such that $(k \cdot n) + (j \cdot p) = 1$.
%    $n^{2} = n \cdot n^{1} = n \cdot n$.
%    $p$ divides $k \cdot (n \cdot n)$ and $p$ divides $(j \cdot p) \cdot n$.
%    Then $p$ divides $(k \cdot (n \cdot n)) + ((j \cdot p) \cdot n)$.
%    $(k \cdot n^{2}) + ((j \cdot p) \cdot n) = (k \cdot (n \cdot n)) + ((j \cdot p) \cdot n)$.
%    Then $((k \cdot n) \cdot n) + ((j \cdot p) \cdot n) = ((k \cdot n) + (j \cdot p)) \cdot n = 1 \cdot n = n$.
%    Thus $p$ divides $n$. 
%\end{proof}



% \begin{proposition}
    % $q^{2} = p$ for no rational number $q$.
% \end{proposition}
% \begin{proof}[by contradiction]
    % Assume the contrary.
    % Take a positive rational number $q$ such that $p = q^{2}$.
    % Take coprime natural numbers $m,n$ such that $m \cdot q = n$.
    % Then $(m \cdot q) \cdot n = n^{2}$. 
    % Then  $(m \cdot q) \cdot n = (m \cdot q) \cdot (m \cdot q) = ((m \cdot q) \cdot m) \cdot q$.
    % $((m \cdot q) \cdot m) \cdot q  = (m^{2} \cdot q) \cdot q = m^{2} \cdot q^{2}$.
    % $m^{2} \cdot q^{2} = n^{2}.$
    % Then $p \cdot m^{2} = n^{2}$.
    % $p$ divides $n^{2}$. Therefore $p$ divides $n$.
    % Take a natural number $k$ such that $n = k \cdot p$.
    % Then $p \cdot m^{2} = p \cdot (k \cdot n)$.
    % Therefore $m \cdot m$ is equal to $p \cdot k^{2}$.
    % Hence $p$ divides $m$.
    % Contradiction.
% \end{proof}
\end{forthel}

\section{* The Denumerability of the Rational Numbers}

\section{* The Non-Denumerability of the Continuum}

\section{Archimedian properties}

\begin{forthel}

\begin{theorem}[1 20 a]
Let $x > 0$.
Then there is a
positive integer $n$ such that \[ n \cdot x > y. \]
\end{theorem}
\begin{proof}
Take an integer $a$ such that $a > \frac{y}{x}$.
Take a positive integer $n$ such that $n > a$.
$n > \frac{y}{x}$ and $n \cdot x > \frac{y}{x} \cdot x = y$.
\end{proof}

\begin{theorem}[1 20 b]
Let $x < y$. Then there exists
$p \in \mathbb{Q}$ such that $x < p < y$.
\end{theorem}
\begin{proof}
We have $y - x > 0$.
Take a positive integer $n$ such that
$n\cdot (y-x) > 1$ (by 1 20 a).
Take an integer $m$ such that
\[ m -1 \leq n \cdot x < m. \]
[timelimit 10]
Then
\[ n \cdot x < m = (m - 1) + 1 \leq (n\cdot x) + 1 <
(n\cdot x) + (n\cdot (y-x)) = n \cdot (x + (y - x)) = n \cdot y. \]
$n > 0$ and
\[ x = \frac{n\cdot x}{n} < \frac{m}{n} < \frac{n\cdot y}{n} = y. \]
Let $p = \frac{m}{n}$. 
[timelimit 10]
Then $p \in \mathbb{Q}$ and $x < p < y$.
\end{proof}

\end{forthel}


\begin{thebibliography}{plain}
\bibitem{Wiedijk} 100 theorems
\bibitem{Naproche} Websit]

\bibitem{Rudin} Walter Rudin. \textit{Principles of Mathematical
Analysis}.
\end{thebibliography}

\end{document}
