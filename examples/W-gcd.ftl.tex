\documentclass{article}

\usepackage[utf8]{inputenc}
\usepackage[english]{babel}
\usepackage[foundations]{../lib/tex/naproche}


\title{Sum of an Arithmetic Series}
\author{}
\date{}

\begin{document}
\maketitle

\begin{forthel}
    [prove off][check off]

    [readtex \path{100_Theorems.ftl.tex}]

    [prove on][check on]
  \end{forthel}



\section{* Greatest Common Divisor Algorithm}

\begin{forthel}
Let $m,n$ denote integers.


\begin{definition}
The greatest common divisor of $m$ and $n$ is a natural number.
\end{definition}

\begin{axiom}
The greatest common divisor of $m$ and $n$ is a divisor of $m$ and a divisor of $n$.
\end{axiom}
\begin{axiom}
Let $d$ be a divisor of $m$ and a divisor of $n$.
Then $d$ is a divisor of the greatest common divisor of $m$ and $n$.
\end{axiom}


\begin{definition}
$\gcd(m,n)$ is a natural number.
\end{definition}

\begin{axiom}
If $m$ is not a natural number then $\gcd(m,n) = \gcd(-m,n)$ and
if $n$ is not a natural number then $\gcd(m,n) = \gcd(m,-n)$.
\end{axiom}
\begin{axiom}
If $m=0$ then $\gcd(m,n)=n$ and if $n=0$ then $\gcd(m,n)=m$.
\end{axiom}
\begin{axiom}
If $m \leq n$ then $\gcd(m,n) = \gcd(m,n-m)$ and if $n < m$ then $\gcd(m,n) = \gcd(m-n,n)$.
\end{axiom}


\begin{proposition}
The greatest common divisor of $m$ and $n$ is $\gcd(m,n)$.
\end{proposition}

\end{forthel}

\end{document}