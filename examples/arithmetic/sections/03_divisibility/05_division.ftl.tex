\documentclass[../../arithmetic.ftl.tex]{subfiles}

\begin{document}
  \begin{comment}
    \begin{forthel}
      % Uncomment for debugging:

      %[prove off][check off]
      %[readtex arithmetic/sections/01_arithmetic/01_peano-axioms.ftl.tex]
      %[readtex arithmetic/sections/01_arithmetic/02_addition.ftl.tex]
      %[readtex arithmetic/sections/01_arithmetic/03_multiplication.ftl.tex]
      %[readtex arithmetic/sections/01_arithmetic/04_exponentiation.ftl.tex]
      %[readtex arithmetic/sections/01_arithmetic/05_factorial.ftl.tex]
      %[readtex arithmetic/sections/02_ordering/01_ordering.ftl.tex]
      %[readtex arithmetic/sections/02_ordering/02_ordering-and-addition.ftl.tex]
      %[readtex arithmetic/sections/02_ordering/03_ordering-and-multiplication.ftl.tex]
      %[readtex arithmetic/sections/02_ordering/04_ordering-and-exponentiation.ftl.tex]
      %[readtex arithmetic/sections/02_ordering/05_induction.ftl.tex]
      %[readtex arithmetic/sections/02_ordering/06_standard-exercises.ftl.tex]
      %[readtex arithmetic/sections/02_ordering/07_subtraction.ftl.tex]
      %[readtex arithmetic/sections/03_divisibility/01_divisibility.ftl.tex]
      %[readtex arithmetic/sections/03_divisibility/02_euclidean-division.ftl.tex]
      %[readtex arithmetic/sections/03_divisibility/03_modular-arithmetic.ftl.tex]
      %[readtex arithmetic/sections/03_divisibility/04_primes.ftl.tex]
      %[prove on][check on]
    \end{forthel}
  \end{comment}


  \section{Division}

  \begin{forthel}
    [readtex \path{arithmetic/sections/01_arithmetic/05_factorial.ftl.tex}]
  \end{forthel}

  \begin{forthel}
    [readtex \path{arithmetic/sections/02_ordering/05_induction.ftl.tex}]
  \end{forthel}

  \begin{forthel}
    [readtex \path{arithmetic/sections/02_ordering/07_subtraction.ftl.tex}]
  \end{forthel}

  \begin{forthel}
    [readtex \path{arithmetic/sections/03_divisibility/01_divisibility.ftl.tex}]
  \end{forthel}

  \begin{forthel}
    Let $k,l,m,n$ denote natural numbers.
  \end{forthel}


  \noindent In this section we define fractions of natural numbers in order to get a (partial) inverse operation to multiplication.

  \begin{forthel}
    \begin{definition}
      Let $m$ be a nonzero divisor of $n$.
      $\frac{n}{m}$ is the natural number $k$ such that $k \cdot m = n$.
    \end{definition}

    Let the quotient of $n$ and $m$ stand for $\frac{n}{m}$.

    \begin{proposition}\label{Arithmetic_03_05_479904}
      \[ \frac{n}{1} = n. \]
    \end{proposition}
    \begin{proof}
      We have $\frac{n}{1} = \frac{n}{1} \cdot 1 = n$.
    \end{proof}

    \begin{proposition}\label{Arithmetic_03_05_471851}
      Let $m$ be a nonzero divisor of $n$.
      Then \[ \frac{n}{m} = 0 \iff n = 0. \]
    \end{proposition}
    \begin{proof}
      Case $\frac{n}{m} = 0$.
        Then $n = \frac{n}{m} \cdot m = 0 \cdot m = 0$.
      End.

      Case $n = 0$.
        Then $\frac{n}{m} \cdot m = n = 0$.
        Hence $\frac{n}{m} = 0$ or $m = 0$.
        $m$ is nonzero.
        Thus $\frac{n}{m} = 0$.
      End.
    \end{proof}

    \begin{proposition}\label{Arithmetic_03_05_363442}
      Let $k$ be nonzero.
      Assume $k \mid n,m$.
      Then \[ \frac{n + m}{k} = \frac{n}{k} + \frac{m}{k}. \]
    \end{proposition}
    \begin{proof}
      We have $\frac{n + m}{k} \cdot k = n + m$ and $\frac{n}{k} \cdot k = n$ and $\frac{m}{k} \cdot k = m$.
      Hence
      \[  \frac{n + m}{k} \cdot k
          = (\frac{n}{k} \cdot k) + (\frac{m}{k} \cdot k)
          = (\frac{n}{k} + \frac{m}{k}) \cdot k. \]
      Thus $\frac{n + m}{k} = \frac{n}{k} + \frac{m}{k}$.
    \end{proof}

    \begin{proposition}\label{Arithmetic_03_05_170600}
      Let $k$ be a nonzero divisor of $m$.
      Then \[ \frac{n \cdot m}{k} = n \cdot \frac{m}{k}. \]
    \end{proposition}
    \begin{proof}
      We have $\frac{n \cdot m}{k} \cdot k = n \cdot m$ and $\frac{m}{k} \cdot k = m$.
      Hence
      \[  \frac{n \cdot m}{k} \cdot k
          = n \cdot (\frac{m}{k} \cdot k)
          = (n \cdot \frac{m}{k}) \cdot k. \]
      Thus $\frac{n \cdot m}{k} = n \cdot \frac{m}{k}$.
    \end{proof}

    \begin{corollary}\label{Arithmetic_03_05_173047}
      Let $k$ be a nonzero divisor of $m$.
      Then \[ \frac{n \cdot m}{k} = \frac{m}{k} \cdot n. \]
    \end{corollary}

    \begin{corollary}\label{Arithmetic_03_05_401748}
      Let $k$ be a nonzero divisor of $n$.
      Then \[ \frac{n \cdot m}{k} = \frac{n}{k} \cdot m. \]
    \end{corollary}

    \begin{proposition}\label{Arithmetic_03_05_243924}
      Let $k$ be nonzero.
      Let $m$ be a nonzero divisor of $n$.
      Then \[ \frac{n \cdot k}{m \cdot k} = \frac{n}{m}. \]
    \end{proposition}
    \begin{proof}
      We have $\frac{n \cdot k}{m \cdot k} \cdot (m \cdot k) = n \cdot k$.
      Hence
      \[  (\frac{n \cdot k}{m \cdot k} \cdot m) \cdot k
          = \frac{n \cdot k}{m \cdot k} \cdot (m \cdot k)
          = n \cdot k. \]
      Thus $\frac{n \cdot k}{m \cdot k} \cdot m = n$.
      Therefore $\frac{n}{m} = \frac{n \cdot k}{m \cdot k}$.
    \end{proof}

    \begin{corollary}\label{Arithmetic_03_05_424330}
      Let $k$ be nonzero.
      Let $m$ be a nonzero divisor of $n$.
      Then \[ \frac{k \cdot n}{k \cdot m} = \frac{n}{m}. \]
    \end{corollary}

    \begin{proposition}\label{Arithmetic_03_05_469833}
      Let $k$ be nonzero and $n \geq m$.
      Then \[ \frac{k^{n}}{k^{m}} = k^{n - m}. \]
    \end{proposition}
    \begin{proof}
      We have $k^{n} = k^{(n - m) + m} = k^{n - m} \cdot k^{m}$.
      Hence
      \[  \frac{k^{n}}{k^{m}}
          = \frac{k^{n - m} \cdot k^{m}}{k^{m}}
          = \frac{k^{n - m} \cdot k^{m}}{1 \cdot k^{m}}
          = \frac{k^{n - m}}{1}
          = k^{n - m}. \]
    \end{proof}

    \begin{proposition}\label{Arithmetic_03_05_805525}
      $(m + k)!$ is divisible by $m!$ for any natural number $k$.
    \end{proposition}
    \begin{proof}
      Define $P = \class{k \in \mathbb{N} | m! \mid (m + k)!}$.

      (BASE CASE) $P$ contains $0$.
      Indeed $m! \mid (m!) = (m + 0)!$.

      (INDUCTION STEP) For all natural numbers $k$ we have $k \in P \implies k + 1 \in P$. \\
      Proof.
        Let $k$ be a natural number.
        Assume $k \in P$.
        We have
        \[  (((m + k) + 1)!)
            = ((m + (k + 1))!)
            = (((m + k) + 1)!)
            = ((m + k) + 1) \cdot ((m + k)!). \]
        $m!$ divides $(m + k)!$.
        Hence we can take a natural number $l$ such that $l \cdot (m!) = (m + k)!$.
        Then we have
        \[  (((m + k) + 1)!)
            = ((m + k) + 1) \cdot (l \cdot (m!))
            = (((m + k) + 1) \cdot l) \cdot (m!). \]
        Thus $m!$ divides $((m + k) + 1)!$.
      Qed.
    \end{proof}

    \begin{corollary}
      $(m + k)! \cdot n$ is divisible by $m!$.
    \end{corollary}
    \begin{proof}
      $(m + k)!$ is divisible by $m!$.
      Hence we can take a natural number $l$ such that $((m + k)!) = m! \cdot l$.
      Then $(m + k)! \cdot n = (m! \cdot l) \cdot n = m! \cdot (l \cdot n)$.
      Thus $(m + k)! \cdot n$ is divisible by $m!$.
    \end{proof}

    \begin{corollary}
      $(m + k)! \cdot n$ is divisible by $m! \cdot n$.
    \end{corollary}
    \begin{proof}
      We can take a natural number $l$ such that $m! \cdot l = ((m + k)!)$.
      Indeed $m!$ divides $(m + k)!$.
      Then $(m + k)! \cdot n = (m! \cdot l) \cdot n = m! \cdot (l \cdot n) = m! \cdot (n \cdot l) = (m! \cdot n) \cdot l$.
      Hence $m! \cdot n$ divides $(m + k)! \cdot n$.
    \end{proof}

    \begin{proposition}\label{Arithmetic_03_05_335719}
      $\frac{(m + k)!}{m!}$ is divisible by $k!$ for any natural numbers $m,k$.
    \end{proposition}
    \begin{proof}
      Let us show that for all natural numbers $m,k$ such that $m,k > 1$ we have
      \[ \frac{(m + k)!}{m!} = (k \cdot \frac{(m + (k - 1))!}{m!}) + \frac{((m - 1) + k)!}{(m - 1)!}. \]
        Let $m,k$ be natural numbers.
        Assume $m,k > 1$.

        %%%%%%%%%% Needed for ontological checking:
        $m,k$ are nonzero.
        $(m - 1) + k = (m + k) - 1 = m + (k - 1)$.
        $(((m - 1) + k)!) = (((m + k) - 1)!) = (m + (k - 1))!$.
        $(m - 1)!$ and $m!$ is a nonzero divisor of $((m - 1) + k)!$.
        $m!$ is a nonzero divisor of $((m - 1) + k)!$.
        $m!$ divides $((m + k) - 1)! \cdot (m + k)$.
        We have $(m - 1)! \cdot m = m!$.
        %%%%%%%%%%

        [prover vampire]

        (1) $k \cdot \frac{(m + (k - 1))!}{m!} = \frac{((m + k) - 1)!}{m!} \cdot k$.

        (2) $\frac{((m - 1) + k)!}{(m - 1)!} = \frac{((m + k) - 1)!}{m!} \cdot m$. \\
        Proof.
          \[ \frac{((m - 1) + k)!}{(m - 1)!} = \frac{((m - 1) + k)! \cdot m}{(m - 1)! \cdot m} \]
          (by \ref{Arithmetic_03_05_243924}).
          \[ \frac{((m - 1) + k)! \cdot m}{(m - 1)! \cdot m} = \frac{((m - 1) + k)! \cdot m}{m!}. \]
          \[ \frac{((m - 1) + k)! \cdot m}{m!} = \frac{((m - 1) + k)!}{m!} \cdot m \]
          (by \ref{Arithmetic_03_05_401748}).
          \[ \frac{((m - 1) + k)!}{m!} \cdot m = \frac{((m + k) - 1)!}{m!} \cdot m. \]
        Qed.

        (3) $(\frac{((m + k) - 1)!}{m!} \cdot k) + (\frac{((m + k) - 1)!}{m!} \cdot m) = \frac{(m + k)!}{m!}$. \\
        Proof.
          \[ (\frac{((m + k) - 1)!}{m!} \cdot k) + (\frac{((m + k) - 1)!}{m!} \cdot m) = \frac{((m + k) - 1)!}{m!} \cdot (k + m) \]
          (by \ref{Arithmetic_01_03_539933}).
          \[ \frac{((m + k) - 1)!}{m!} \cdot (k + m) = \frac{((m + k) - 1)!}{m!} \cdot (m + k). \]
          \[ \frac{((m + k) - 1)!}{m!} \cdot (m + k) = \frac{((m + k) - 1)! \cdot (m + k)}{m!} \]
          (by \ref{Arithmetic_03_05_401748}).
          \[ \frac{((m + k) - 1)! \cdot (m + k)}{m!} = \frac{(m + k)!}{m!}. \]
        Qed.

        Then we have the thesis (by 1, 2, 3).
      End.

      Define \[ P = \class{n \in \mathbb{N} | \classtext{for all $m,k \in \mathbb{N}$ such that $n = m + k$ we have $k! \mid \frac{(m + k)!}{m!}$}}. \]

      For all natural numbers $n$ if every predecessor of $n$ is contained in $P$ then $n \in P$. \\
      Proof.
        Let $n$ be a natural number.
        Assume that every predecessor of $n$ lies in $P$.

        Let us show that for all natural numbers $m,k$ such that $n = m + k$ we have $k! \mid \frac{(m + k)!}{m!}$.
          Let $m,k$ be natural numbers.
          Assume $n = m + k$.

          Case $m \leq 1$. Obvious.

          Case $k \leq 1$. Obvious.

          Case $m,k > 1$.
            Case $m + k < n$ or $m + k > n$. Obvious.

            Case $m + k = n$.
              We have \[ \frac{(m + k)!}{m!} = (k \cdot \frac{(m + (k - 1))!}{m!}) + \frac{((m - 1) + k)!}{(m - 1)!}. \]

              (1) $k! \mid k \cdot \frac{(m + (k - 1))!}{m!}$. \\
              Proof.
                We have $m + (k - 1) < n$.
                Hence $(k - 1)!$ divides $\frac{(m + (k - 1))!}{m!}$.
                Thus we can take a natural number $l$ such that $(k - 1)! \cdot l = \frac{(m + (k - 1))!}{m!}$.
                Then $k! \cdot l = (k \cdot (k - 1)!) \cdot l = k \cdot ((k - 1)! \cdot l) = k \cdot \frac{(m + (k - 1))!}{m!}$.
                Therefore $k!$ divides $k \cdot \frac{(m + (k - 1))!}{m!}$.
              Qed.

              [prover vampire]

              (2) $k! \mid \frac{((m - 1) + k)!}{(m - 1)!}$. \\
              Proof.
                We have $(m - 1) + k < n$.
                Indeed $(m - 1) + k = (m + k) - 1 = n - 1 < n$.
                Hence $k! \mid \frac{((m - 1) + k)!}{(m - 1)!}$.
              Qed.

              Then $k!$ divides $(k \cdot \frac{(m + (k - 1))!}{m!}) + \frac{((m - 1) + k)!}{(m - 1)!}$ (by \ref{Arithmetic_03_01_655411}, 1, 2).
            End.
          End.
        End.
      Qed.

      [prover vampire]
      Hence $P$ contains every natural number (by \ref{Arithmetic_02_05_167446}).
      Then we have the thesis.
    \end{proof}
  \end{forthel}
\end{document}
