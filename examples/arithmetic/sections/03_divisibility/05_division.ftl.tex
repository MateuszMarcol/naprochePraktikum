\documentclass[../../arithmetic.ftl.tex]{subfiles}

\begin{document}
  \section{Division}

  \begin{forthel}
    [readtex \path{arithmetic/sections/02_ordering/07_subtraction.ftl.tex}]
  \end{forthel}

  \begin{forthel}
    [readtex \path{arithmetic/sections/03_divisibility/01_divisibility.ftl.tex}]
  \end{forthel}

  \begin{forthel}
    Let $k,l,m,n$ denote natural numbers.
  \end{forthel}


  \noindent In this section we define fractions of natural numbers in order to get a (partial) inverse operation to multiplication.

  \begin{forthel}
    \begin{definition}
      Let $m$ be a nonzero divisor of $n$.
      $\frac{n}{m}$ is the natural number $k$ such that $k \cdot m = n$.
    \end{definition}

    Let the quotient of $n$ and $m$ stand for $\frac{n}{m}$.

    \begin{proposition}\label{Arithmetic_03_05_479904}
      \[ \frac{n}{1} = n. \]
    \end{proposition}
    \begin{proof}
      We have $\frac{n}{1} = \frac{n}{1} \cdot 1 = n$.
    \end{proof}

    \begin{proposition}\label{Arithmetic_03_05_471851}
      Let $m$ be a nonzero divisor of $n$.
      Then \[ \frac{n}{m} = 0 \iff n = 0. \]
    \end{proposition}
    \begin{proof}
      Case $\frac{n}{m} = 0$.
        Then $n = \frac{n}{m} \cdot m = 0 \cdot m = 0$.
      End.

      Case $n = 0$.
        Then $\frac{n}{m} \cdot m = n = 0$.
        Hence $\frac{n}{m} = 0$ or $m = 0$.
        $m$ is nonzero.
        Thus $\frac{n}{m} = 0$.
      End.
    \end{proof}

    \begin{proposition}\label{Arithmetic_03_05_363442}
      Let $k$ be nonzero.
      Assume $k \mid n,m$.
      Then \[ \frac{n + m}{k} = \frac{n}{k} + \frac{m}{k}. \]
    \end{proposition}
    \begin{proof}
      We have $\frac{n + m}{k} \cdot k = n + m$ and $\frac{n}{k} \cdot k = n$ and $\frac{m}{k} \cdot k = m$.
      Hence
      \[  \frac{n + m}{k} \cdot k
          = \left( \frac{n}{k} \cdot k \right) + \left( \frac{m}{k} \cdot k \right)
          = \left( \frac{n}{k} + \frac{m}{k} \right) \cdot k. \]
      Thus $\frac{n + m}{k} = \frac{n}{k} + \frac{m}{k}$.
    \end{proof}

    \begin{proposition}\label{Arithmetic_03_05_170600}
      Let $k$ be a nonzero divisor of $m$.
      Then \[ \frac{n \cdot m}{k} = n \cdot \frac{m}{k}. \]
    \end{proposition}
    \begin{proof}
      We have $\frac{n \cdot m}{k} \cdot k = n \cdot m$ and $\frac{m}{k} \cdot k = m$.
      Hence
      \[  \frac{n \cdot m}{k} \cdot k
          = n \cdot \left( \frac{m}{k} \cdot k \right)
          = \left( n \cdot \frac{m}{k} \right) \cdot k. \]
      Thus $\frac{n \cdot m}{k} = n \cdot \frac{m}{k}$.
    \end{proof}

    \begin{corollary}\label{Arithmetic_03_05_173047}
      Let $k$ be a nonzero divisor of $m$.
      Then \[ \frac{n \cdot m}{k} = \frac{m}{k} \cdot n. \]
    \end{corollary}

    \begin{corollary}\label{Arithmetic_03_05_401748}
      Let $k$ be a nonzero divisor of $n$.
      Then \[ \frac{n \cdot m}{k} = \frac{n}{k} \cdot m. \]
    \end{corollary}

    \begin{proposition}\label{Arithmetic_03_05_243924}
      Let $k$ be nonzero.
      Let $m$ be a nonzero divisor of $n$.
      Then \[ \frac{n \cdot k}{m \cdot k} = \frac{n}{m}. \]
    \end{proposition}
    \begin{proof}
      We have $\frac{n \cdot k}{m \cdot k} \cdot (m \cdot k) = n \cdot k$.
      Hence
      \[  \left( \frac{n \cdot k}{m \cdot k} \cdot m \right) \cdot k
          = \frac{n \cdot k}{m \cdot k} \cdot (m \cdot k)
          = n \cdot k. \]
      Thus $\frac{n \cdot k}{m \cdot k} \cdot m = n$.
      Therefore $\frac{n}{m} = \frac{n \cdot k}{m \cdot k}$.
    \end{proof}

    \begin{corollary}\label{Arithmetic_03_05_424330}
      Let $k$ be nonzero.
      Let $m$ be a nonzero divisor of $n$.
      Then \[ \frac{k \cdot n}{k \cdot m} = \frac{n}{m}. \]
    \end{corollary}

    \begin{proposition}\label{Arithmetic_03_05_469833}
      Let $k$ be nonzero and $n \geq m$.
      Then \[ \frac{k^{n}}{k^{m}} = k^{n - m}. \]
    \end{proposition}
    \begin{proof}
      We have $k^{n} = k^{(n - m) + m} = k^{n - m} \cdot k^{m}$.
      Hence
      \[  \frac{k^{n}}{k^{m}}
          = \frac{k^{n - m} \cdot k^{m}}{k^{m}}
          = \frac{k^{n - m} \cdot k^{m}}{1 \cdot k^{m}}
          = \frac{k^{n - m}}{1}
          = k^{n - m}. \]
    \end{proof}
  \end{forthel}
\end{document}
