\documentclass[../../arithmetic.tex]{subfiles}

\begin{document}
  \section{Primes}

  \begin{forthel}
    [readtex \path{arithmetic/sections/01_arithmetic/04_exponentiation.ftl.tex}]
  \end{forthel}

  \begin{forthel}
    [readtex \path{arithmetic/sections/02_ordering/04_ordering-and-exponentiation.ftl.tex}]
  \end{forthel}

  \begin{forthel}
    [readtex \path{arithmetic/sections/02_ordering/05_induction.ftl.tex}]
  \end{forthel}

  \begin{forthel}
    [readtex \path{arithmetic/sections/03_divisibility/01_divisibility.ftl.tex}]
  \end{forthel}

  \begin{forthel}
    [readtex \path{arithmetic/sections/03_divisibility/02_euclidean-division.ftl.tex}]
  \end{forthel}

  \begin{forthel}
    Let $k, l, m, n$ denote natural numbers.
  \end{forthel}


  \subsection{Definitions}

  Let us turn back to the notion of divisibility. We will now investigate
  natural numbers which cannot be decomposed into a product of two
  (non-trivial) smaller numbers. Such numbers are called \textit{prime}.

  \begin{forthel}
    \begin{definition}
      A trivial divisor of $n$ is a divisor $m$ of $n$ such that $m = 1$ or $m = n$.
    \end{definition}

    \begin{definition}
      A nontrivial divisor of $n$ is a divisor $m$ of $n$ such that $m \neq 1$ and $m \neq n$.
    \end{definition}

    \begin{definition}
      $n$ is prime iff $n > 1$ and $n$ has no nontrivial divisors.
    \end{definition}

    Let $n$ is compound stand for $n$ is not prime.
    Let a prime number stand for a natural number that is prime.

    \begin{definition}
      $n$ is composite iff $n > 1$ and $n$ has a nontrivial divisor.
    \end{definition}

    \begin{proposition}\label{Arithmetic_03_04_357744}
      Let $n > 1$.
      Then $n$ is prime iff every divisor of $n$ is a trivial divisor of $n$.
    \end{proposition}
  \end{forthel}

  \noindent Let us have a look at what the first few prime numbers are.

  \begin{forthel}
    \begin{proposition}\label{Arithmetic_03_04_175431}
      $2$, $3$, $5$ and $7$ are prime.
    \end{proposition}
    \begin{proof}
      Let us show that $2$ is prime.
        Let $k$ be a divisor of $2$.
        Then $0 < k \leq 2$.
        Hence $k = 1$ or $k = 2$.
        Thus $k$ is a trivial divisor of $2$.
      End.

      Let us show that $3$ is prime.
        Let $k$ be a divisor of $3$.
        Then $0 < k \leq 3$.
        Hence $k = 1$ or $k = 2$ or $k = 3$.
        $2$ does not divide $3$.
        Therefore $k = 1$ or $k = 3$.
        Thus $k$ is a trivial divisor of $3$.
      End.

      Let us show that $5$ is prime.
        Let $k$ be a divisor of $5$.
        Then $0 < k \leq 5$.
        Hence $k = 1$ or $k = 2$ or $k = 3$ or $k = 4$ or $k = 5$.
        $2$ does not divide $5$.
        $3$ does not divide $5$.
        $4$ does not divide $5$.
        Therefore $k = 1$ or $k = 5$.
        Thus $k$ is a trivial divisor of $5$.
      End.

      Let us show that $7$ is prime.
        Let $k$ be a divisor of $7$.
        Then $0 < k \leq 7$.
        Hence $k = 1$ or $k = 2$ or $k = 3$ or $k = 4$ or $k = 5$ or $k = 6$ or $k = 7$.
        $2$ does not divide $7$.
        $3$ does not divide $7$.
        $4$ does not divide $7$.
        $5$ does not divide $7$.
        $6$ does not divide $7$.
        Therefore $k = 1$ or $k = 7$.
        Thus $k$ is a trivial divisor of $7$.
      End.
    \end{proof}

    \begin{proposition}\label{Arithmetic_03_04_985728}
      $4$, $6$, $8$ and $9$ are compound.
    \end{proposition}
    \begin{proof}
      $4 = 2 \cdot 2$.
      Hence $2$ divides $4$.
      Thus $4$ is compound.

      $6 = 2 \cdot 3$.
      Hence $2$ divides $6$.
      Thus $6$ is compound.

      $8 = 2 \cdot 4$.
      Hence $2$ divides $8$.
      Thus $8$ is compound.

      $9 = 3 \cdot 3$.
      Hence $3$ divides $9$.
      Thus $9$ is compound.
    \end{proof}
  \end{forthel}

  \noindent An important fact about primes is that every natural number has a
  prime divisor. From this the \textit{fundamental theorem of arithmetic}
  can be derived, namely the assertion that every natural number has a unique
  decomposition into prime factors.

  \begin{forthel}
    \begin{proposition}\label{Arithmetic_03_04_130748}
      Every natural number that is greater than $1$ has a prime divisor.
    \end{proposition}
    \begin{proof}
      Define \[ P = \class{n \in \mathbb{N} | \text{if $n > 1$ then $n$ has a prime divisor}}. \]

      Let us show that (1) for every natural number $n$ if $P$ contains all predecessors of $n$ then $P$ contains $n$.
        Let $n$ be a natural number.
        Assume that $P$ contains all predecessors of $n$.
        $n = 0$ or $n = 1$ or $n$ is prime or $n$ is composite.

        Case $n = 0$ or $n = 1$. Trivial.

        Case $n$ is prime. Obvious.

        Case $n$ is composite.
          Take a nontrivial divisor $m$ of $n$.
          Then $1 < m < n$.
          $m$ is contained in $P$.
          Hence we can take a prime divisor $p$ of $m$.
          Then we have $p \mid m \mid n$.
          Thus $p \mid n$.
          Therefore $p$ is a prime divisor of $n$.
        End.
      End.

      Thus every natural number belongs to $P$ (by \ref{Arithmetic_02_05_167446}, 1).
    \end{proof}

    \begin{proposition}\label{Arithmetic_03_04_306779}
      Let $n$ be composite.
      Then $n$ has a nontrivial divisor $m$ such that $m^{2} \leq n$.
    \end{proposition}
    \begin{proof}
      Define \[ A = \class{m \in \mathbb{N} | \text{$m$ is a nontrivial divisor of $n$}}. \]
      $A$ contains some natural number.
      Hence we can take a least natural number $m$ of $A$.
      Consider a natural number $k$ such that $m \cdot k = n$.
      Then $m \leq k$.
      Indeed if $k < m$ then $k$ is the least natural number of $A$.
      Hence $m^{2} = m \cdot m \leq m \cdot k = n$.
      Therefore $m^{2} \leq n$.
    \end{proof}
  \end{forthel}

  \noindent Let us now have a look at natural numbers which have no
  (non-trivial) common divisor.
  Such numbers are called \textit{coprime}.

  \begin{forthel}
    \begin{definition}
      $n$ and $m$ are coprime iff for all nonzero natural numbers $k$ such that $k \mid n$ and $k \mid m$ we have $k = 1$.
    \end{definition}

    Let $n$ and $m$ are relatively prime stand for $n$ and $m$ are coprime.
    Let $n$ and $m$ are mutually prime stand for $n$ and $m$ are coprime.
    Let $n$ is prime to $m$ stand for $n$ and $m$ are coprime.

    \begin{proposition}\label{Arithmetic_03_04_356588}
      $n$ and $m$ are coprime iff for no prime number $p$ we have $p \mid n$ and $p \mid m$.
    \end{proposition}
    \begin{proof}
      Case $n$ and $m$ are coprime.
        Let $p$ be a prime number such that $p \mid n$ and $p \mid m$.
        Then $p$ is nonzero and $p \neq 1$.
        Contradiction.
      End.

      Case for no prime number $p$ we have $p \mid n$ and $p \mid m$.
        Let $k$ be a nonzero natural number such that $k \mid n$ and $k \mid m$.
        Assume that $k \neq 1$.
        Consider a prime divisor $p$ of $k$.
        Then $p \mid k \mid n,m$.
        Hence $p \mid n$ and $p \mid m$.
        Contradiction.
      End.
    \end{proof}

    \begin{proposition}\label{Arithmetic_03_04_691058}
      Let $p$ be a prime number.
      If $p$ does not divide $n$ then $p$ and $n$ are coprime.
    \end{proposition}
    \begin{proof}
      Assume $p \nmid n$.
      Suppose that $p$ and $n$ are not coprime.
      Take a nonzero natural number $k$ such that $k \mid p$ and $k \mid n$.
      Then $k = p$.
      Hence $p \mid n$.
      Contradiction.
    \end{proof}

    \begin{proposition}\label{Arithmetic_03_04_703692}
      Let $p$ be a prime number.
      Then \[ p \mid n \cdot m \implies (\text{$p \mid n$ or $p \mid m$}). \]
    \end{proposition}
    \begin{proof}
      Assume $p \mid n \cdot m$.

      Case $p \mid n$. Trivial.

      Case $p \nmid n$.
        Define \[ N = \class{x \in \mathbb{N} | \text{$x \neq 0$ and $p \mid x \cdot m$}}. \]
        We have $p \in N$ and $n \in N$.
        Hence $N$ contains some natural number.
        Thus we can take a least natural number $n'$ of $N$.

        Let us show that $n'$ divides all elements of $N$.
          Let $x \in N$.
          Take natural numbers $q,r$ such that $x = (n' \cdot q) + r$ and $r < n'$ (by \ref{Arithmetic_03_02_332233}).
          Indeed $n'$ is nonzero.
          Then $x \cdot m = ((q \cdot n') + r) \cdot m = ((q \cdot n') \cdot m) + (r \cdot m)$.
          We have $p \mid x \cdot m$.
          Hence $p \mid ((q \cdot n') \cdot m) + (r \cdot m)$.
          Thus $p \mid r \cdot m$ (by \ref{Arithmetic_03_01_695362}).
          Indeed $p \mid ((q \cdot n') \cdot m) = (q \cdot (n' \cdot m))$.
          Indeed $p \mid n' \cdot m$.
          Therefore $r = 0$.
          Indeed if $r \neq 0$ then $r$ is an element of $N$ that is less than $n'$.
          Hence $x = q \cdot n'$.
          Thus $n'$ divides $x$.
        End.

        Then we have $n' \mid p$ and $n' \mid n$.
        Hence $n' = p$ or $n' = 1$.
        Thus $n' = 1$.
        Indeed if $n' = p$ then $p \mid n$.
        Then $1 \in N$.
        Therefore $p \mid 1 \cdot m = m$.
      End.
    \end{proof}

    \begin{proposition}\label{Arithmetic_03_04_119851}
      Let $k$ be nonzero.
      Then for all nonzero $n,m$ if $k \cdot m^{2} = n^{2}$ then $k$ is compound.
    \end{proposition}
    \begin{proof}
      Case $k = 1$. Obvious.

      Case $k > 1$.
        (1) Define \[ P = \class{n \in \mathbb{N} | \classtext{for all natural numbers $m$ if $n$ and $m$ are nonzero and $k \cdot m^{2} = n^{2}$ then $k$ is compound}}. \]

        Let us show that for all natural numbers $n$ if $P$ contains all predecessors of $n$ then $P$ contains $n$.
          Let $n$ be a natural number.
          Presume that $P$ contains all predecessors of $n$.

          Let $m$ be a natural number.
          Assume that $n$ and $m$ are nonzero and $k \cdot m^{2} = n^{2}$.

          Suppose that $k$ is prime.
          $k$ is a nontrivial divisor of $n^{2}$.
          Hence $k$ divides $n$.
          Take a natural number $l$ such that $k \cdot l = n$.

          (2) Then $m^{2} = k \cdot l^{2}$ (by \ref{Arithmetic_01_03_169506}).
          Indeed $k \cdot m^{2} = (k \cdot l)^{2} = k \cdot (k \cdot l^{2})$.

          (3) $m$ is an element of $P$. \\
          Proof.
            We have $n^{2} > m^{2}$ (by \ref{Arithmetic_02_03_252473}).
            Indeed $k \cdot m^{2} = n^{2}$ and $k > 1$ and $m^{2} > 0$.
            Hence $m < n$.
            Indeed if $n \leq m$ then $n^{2} \leq m^{2}$.
            Thus $m \in P$.
          Qed.

          (4) $m$ is nonzero.
          Indeed $m = 0 \implies n^{2} = k \cdot 0^{2} = k \cdot 0 = 0$ and $n^{2} = 0 \implies n = 0$.

          (5) $l$ is nonzero.
          Indeed $l = 0 \implies m^{2} = k \cdot 0^{2} = k \cdot 0 = 0$ and $m^{2} = 0 \implies m = 0$.

          Therefore $k$ is compound (by 2, 3, 4, 5).
          Contradiction.
        End.

        Thus $P$ contains every natural number (by \ref{Arithmetic_02_05_167446}).
        Hence the thesis (by 1).
      End.
    \end{proof}
  \end{forthel}
\end{document}
