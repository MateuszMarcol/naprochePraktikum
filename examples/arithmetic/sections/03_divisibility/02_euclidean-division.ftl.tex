\documentclass[../../arithmetic.ftl.tex]{subfiles}

\begin{document}

  \section{Euclidean division}

  \begin{forthel}
    [readtex \path{arithmetic/sections/02_ordering/03_ordering-and-multiplication.ftl.tex}]
  \end{forthel}

  \begin{forthel}
    Let $k, l, m, n$ denote natural numbers.
  \end{forthel}


  \begin{forthel}
    \begin{proposition}\label{Arithmetic_03_02_332233}
      For all natural numbers $n,m$ such that $m$ is nonzero there exist natural numbers $q,r$ such that \[ n = (m \cdot q) + r \] and $r < m$.
    \end{proposition}
    \begin{proof}
      (1) Define $P = \class{n \in \mathbb{N} | \classtext{for all nonzero natural numbers $m$ there exist natural numbers $q,r$ such that $r < m$ and $n = (m \cdot q) + r$}}$.

      (BASE CASE) $P$ contains $0$.
      Proof.
        Take $q = 0$ and $r = 0$.
        Then for all nonzero natural numbers $m$ we have $r < m$ and $0 = (m \cdot q) + r$.
        Hence $0 \in P$.
      Qed.

      (INDUCTION STEP) For all natural numbers $n$: $n \in P \implies n + 1 \in P$.
      Proof.
        Let $n$ be a natural number.
        Assume $n \in P$.

        Let us show that for all nonzero natural numbers $m$ there exist natural numbers $q,r$ such that $r < m$ and $n + 1 = (m \cdot q) + r$.
          Let $m$ be a nonzero natural number.
          Take natural numbers $q',r'$ such that $r' < m$ and $n = (m \cdot q') + r'$ (by 1).
          Indeed $n \in P$.
          We have $r' + 1 < m$ or $r' + 1 = m$.

          Case $r' + 1 < m$.
            Take natural numbers $q,r$ such that $q = q'$ and $r = r' + 1$.
            Then $r < m$ and $n + 1 = ((q' \cdot m) + r') + 1 = (q' \cdot m) + (r' + 1) = (q \cdot m) + r$.
          End.

          Case $r' + 1 = m$.
            Take natural numbers $q,r$ such that $q = q' + 1$ and $r = 0$.
            Then $r < m$ and $n + 1 = ((q' \cdot m) + r') + 1 = (q' \cdot m) + (r' + 1) = (q' \cdot m) + m = (q' + 1) \cdot m = (q \cdot m) + r$.
          End.
        End.
      Qed.

      Then $P$ contains every natural number.
      Let $n,m$ be a natural numbers such that $m$ is nonzero.
      Then $n \in P$.
      Hence the thesis (by 1).
    \end{proof}

    \begin{proposition}\label{Arithmetic_03_02_531279}
      Let $m$ be nonzero.
      Let $q,q',r,r'$ be natural numbers such that $(m \cdot q) + r = n = (m \cdot q') + r'$ and $r,r' < m$.
      Then $q = q'$ and $r = r'$.
    \end{proposition}
    \begin{proof}
      We have $(m \cdot q) + r = (m \cdot q') + r'$.

      Case $q \geq q'$ and $r \geq r'$.
        Take natural numbers $q'',r''$ such that $q = q' + q''$ and $r = r' + r''$.
        Then $(m \cdot (q' + q'')) + (r' + r'') = (m \cdot q') + r'$.
        We have $(m \cdot (q' + q'')) + (r' + r'') = (m \cdot (q' + q'')) + (r'' + r') = ((m \cdot (q' + q'')) + r'') + r'$.
        Hence $((m \cdot (q' + q'')) + r'') + r' = (m \cdot q') + r'$.
        Thus $(m \cdot (q' + q'')) + r'' = m \cdot q'$.
        We have $m \cdot (q' + q'') = (m \cdot q') + (m \cdot q'')$.
        Hence $((m \cdot q') + (m \cdot q'')) + r'' = (m \cdot q') + ((m \cdot q'') + r'') = m \cdot q'$.
        Thus $(m \cdot q'') + r'' = 0$.
        Therefore $r'' = 0$ and $m \cdot q'' = 0$.
        Consequently $q'' = 0$.
        Indeed $m \neq 0$.
        Then we have $q = q' + 0 = q'$  and $r = r' + 0 = r'$.
      End.

      Case $q \geq q'$ and $r < r'$.
        Take a natural number $q''$ such that $q = q' + q''$.
        Take a nonzero natural number $r''$ such that $r' = r + r''$.
        Then $(m \cdot (q' + q'')) + r = (m \cdot q') + (r + r'')$.
        We have $(m \cdot q') + (r + r'') = (m \cdot q') + (r'' + r) = ((m \cdot q') + r'') + r$.
        Hence $(m \cdot (q' + q'')) + r = ((m \cdot q') + r'') + r$.
        Thus $m \cdot (q' + q'') = (m \cdot q') + r''$.
        We have $m \cdot (q' + q'') = (m \cdot q') + (m \cdot q'')$.
        Hence $(m \cdot q') + (m \cdot q'') = (m \cdot q') + r''$.
        Thus $m \cdot q'' = r'' < r' < m$.
        Therefore $q'' = 0$.
        Indeed if $q'' \geq 1$ then $m \cdot q'' \geq m$.
        Consequently $q = q' + 0 = q'$.
        Hence we have $(m \cdot q) + r = (m \cdot q) + r'$.
        Thus $r = r'$.
      End.

      Case $q < q'$ and $r \geq r'$.
        Take a nonzero natural number $q''$ such that $q' = q + q''$.
        Take a natural number $r''$ such that $r = r' + r''$.
        Then $(m \cdot q) + (r' + r'') = (m \cdot (q + q'')) + r'$.
        We have $(m \cdot q) + (r' + r'') = (m \cdot q) + (r'' + r') = ((m \cdot q) + r'') + r'$.
        Hence $((m \cdot q) + r'') + r' = (m \cdot (q + q'')) + r'$.
        Thus $(m \cdot q) + r'' = m \cdot (q + q'')$.
        We have $m \cdot (q + q'') = (m \cdot q) + (m \cdot q'')$.
        Hence $(m \cdot q) + r'' = (m \cdot q) + (m \cdot q'')$.
        Thus $m > r > r'' = m \cdot q''$.
        Therefore $q'' = 0$.
        Indeed if $q'' \geq 1$ then $m \cdot q'' \geq m$.
        Consequently $q' = q + 0 = q$.
        Hence we have $(m \cdot q) + r = (m \cdot q) + r'$.
        Thus $r = r'$.
      End.

      Case $q < q'$ and $r < r'$.
        Take nonzero natural numbers $q'',r''$ such that $q' = q + q''$ and $r' = r + r''$.
        Then $(m \cdot (q + q'')) + (r + r'') = (m \cdot q) + r$.
        We have $(m \cdot (q + q'')) + (r + r'') = (m \cdot (q + q'')) + (r'' + r) = ((m \cdot (q + q'')) + r'') + r$.
        Hence $((m \cdot (q + q'')) + r'') + r = (m \cdot q) + r$.
        Thus $(m \cdot (q + q'')) + r'' = m \cdot q$.
        We have $m \cdot (q + q'') = (m \cdot q) + (m \cdot q'')$.
        Hence $((m \cdot q) + (m \cdot q'')) + r'' = (m \cdot q) + ((m \cdot q'') + r'') = m \cdot q$.
        Thus $(m \cdot q'') + r'' = 0$.
        Therefore $r'' = 0$ and $m \cdot q'' = 0$.
        Consequently $q'' = 0$.
        Indeed $m \neq 0$.
        Then we have $q' = q + 0 = q$  and $r' = r + 0 = r$.
      End.
    \end{proof}

    \begin{definition}
      Let $m$ be nonzero.
      $n \mod m$ is the natural number $r$ such that $r < m$ and there exists a natural number $q$ such that $n = (m \cdot q) + r$.
    \end{definition}

    Let the remainder of $n$ over $m$ stand for $n \mod m$.

    \begin{definition}
      Let $m$ be nonzero.
      $n \div m$ is the natural number $q$ such that $n = (m \cdot q) + r$ for some natural number $r$ that is less than $m$.
    \end{definition}

    Let the quotient of $n$ over $m$ stand for $n \div m$.
  \end{forthel}
\end{document}
