\documentclass[../../arithmetic.tex]{subfiles}

\begin{document}
  \section{Binomial coefficients}

  \begin{forthel}
    [readtex \path{arithmetic/sections/01_arithmetic/05_factorial.ftl.tex}]
  \end{forthel}

  \begin{forthel}
    [readtex \path{arithmetic/sections/02_ordering/05_induction.ftl.tex}]
  \end{forthel}

  \begin{forthel}
    [readtex \path{arithmetic/sections/03_divisibility/05_division.ftl.tex}]
  \end{forthel}

  \begin{forthel}
    Let $k,l,m,n$ denote natural numbers.
  \end{forthel}


  \subsection{Definition}

  \begin{forthel}
    \begin{proposition}\label{Arithmetic_03_06_805525}
      $(m + k)!$ is divisible by $m!$ for any natural number $k$.
    \end{proposition}
    \begin{proof}
      Define $P = \class{k \in \Nat | m! \mid (m + k)!}$.

      (BASE CASE) $P$ contains $0$.
      Indeed $m! \mid (m!) = (m + 0)!$.

      (INDUCTION STEP) For all natural numbers $k$ we have $k \in P \implies k + 1 \in P$. \\
      Proof.
        Let $k$ be a natural number.
        Assume $k \in P$.
        We have
        \[  (((m + k) + 1)!)
            = ((m + (k + 1))!)
            = (((m + k) + 1)!)
            = ((m + k) + 1) \cdot ((m + k)!). \]
        $m!$ divides $(m + k)!$.
        Hence we can take a natural number $l$ such that $l \cdot (m!) = (m + k)!$.
        Then we have
        \[  (((m + k) + 1)!)
            = ((m + k) + 1) \cdot (l \cdot (m!))
            = (((m + k) + 1) \cdot l) \cdot (m!). \]
        Thus $m!$ divides $((m + k) + 1)!$.
      Qed.

      Then $P$ contains every natural number.
    \end{proof}

    \begin{corollary}\label{Arithmetic_03_06_797409}
      $(m + k)! \cdot n$ is divisible by $m!$.
    \end{corollary}
    \begin{proof}
      $(m + k)!$ is divisible by $m!$.
      Hence we can take a natural number $l$ such that $((m + k)!) = m! \cdot l$.
      Then $(m + k)! \cdot n = (m! \cdot l) \cdot n = m! \cdot (l \cdot n)$.
      Thus $(m + k)! \cdot n$ is divisible by $m!$.
    \end{proof}

    \begin{corollary}\label{Arithmetic_03_06_319523}
      $(m + k)! \cdot n$ is divisible by $m! \cdot n$.
    \end{corollary}
    \begin{proof}
      We can take a natural number $l$ such that $m! \cdot l = ((m + k)!)$.
      Indeed $m!$ divides $(m + k)!$.
      Then $(m + k)! \cdot n = (m! \cdot l) \cdot n = m! \cdot (l \cdot n) = m! \cdot (n \cdot l) = (m! \cdot n) \cdot l$.
      Hence $m! \cdot n$ divides $(m + k)! \cdot n$.
    \end{proof}

    \begin{proposition}\label{Arithmetic_03_06_335719}
      $\frac{(m + k)!}{m!}$ is divisible by $k!$ for any natural numbers $m,k$.
    \end{proposition}
    \begin{proof}
      Let us show that for all natural numbers $m,k$ such that $m,k > 1$ we have
      \[ \frac{(m + k)!}{m!} = \left( k \cdot \frac{(m + (k - 1))!}{m!} \right) + \frac{((m - 1) + k)!}{(m - 1)!}. \]
        Let $m,k$ be natural numbers.
        Assume $m,k > 1$.

        %%%%%%%%%% Needed for ontological checking:
        $m,k$ are nonzero.
        $(m - 1) + k = (m + k) - 1 = m + (k - 1)$.
        $(((m - 1) + k)!) = (((m + k) - 1)!) = (m + (k - 1))!$.
        $(m - 1)!$ and $m!$ is a nonzero divisor of $((m - 1) + k)!$.
        $m!$ is a nonzero divisor of $((m - 1) + k)!$.
        $m!$ divides $((m + k) - 1)! \cdot (m + k)$.
        We have $(m - 1)! \cdot m = m!$.
        %%%%%%%%%%

        [prover vampire]

        (1) $k \cdot \frac{(m + (k - 1))!}{m!} = \frac{((m + k) - 1)!}{m!} \cdot k$.

        (2) $\frac{((m - 1) + k)!}{(m - 1)!} = \frac{((m + k) - 1)!}{m!} \cdot m$. \\
        Proof.
          \[ \frac{((m - 1) + k)!}{(m - 1)!} = \frac{((m - 1) + k)! \cdot m}{(m - 1)! \cdot m} \]
          (by \ref{Arithmetic_03_05_243924}).
          \[ \frac{((m - 1) + k)! \cdot m}{(m - 1)! \cdot m} = \frac{((m - 1) + k)! \cdot m}{m!}. \]
          \[ \frac{((m - 1) + k)! \cdot m}{m!} = \frac{((m - 1) + k)!}{m!} \cdot m \]
          (by \ref{Arithmetic_03_05_401748}).
          \[ \frac{((m - 1) + k)!}{m!} \cdot m = \frac{((m + k) - 1)!}{m!} \cdot m. \]
        Qed.

        (3) $\left( \frac{((m + k) - 1)!}{m!} \cdot k \right) + \left( \frac{((m + k) - 1)!}{m!} \cdot m \right) = \frac{(m + k)!}{m!}$. \\
        Proof.
          \[ \left( \frac{((m + k) - 1)!}{m!} \cdot k \right) + \left( \frac{((m + k) - 1)!}{m!} \cdot m \right) = \frac{((m + k) - 1)!}{m!} \cdot (k + m) \]
          (by \ref{Arithmetic_01_03_539933}).
          \[ \frac{((m + k) - 1)!}{m!} \cdot (k + m) = \frac{((m + k) - 1)!}{m!} \cdot (m + k). \]
          \[ \frac{((m + k) - 1)!}{m!} \cdot (m + k) = \frac{((m + k) - 1)! \cdot (m + k)}{m!} \]
          (by \ref{Arithmetic_03_05_401748}).
          \[ \frac{((m + k) - 1)! \cdot (m + k)}{m!} = \frac{(m + k)!}{m!}. \]
        Qed.

        Then we have the thesis (by 1, 2, 3).
      End.

      Define \[ P = \class{n \in \Nat | \classtext{for all $m,k \in \Nat$ such that $n = m + k$ we have $k! \mid \frac{(m + k)!}{m!}$}}. \]

      For all natural numbers $n$ if every predecessor of $n$ is contained in $P$ then $n \in P$. \\
      Proof.
        Let $n$ be a natural number.
        Assume that every predecessor of $n$ lies in $P$.

        Let us show that for all natural numbers $m,k$ such that $n = m + k$ we have $k! \mid \frac{(m + k)!}{m!}$.
          Let $m,k$ be natural numbers.
          Assume $n = m + k$.

          Case $m \leq 1$. Obvious.

          Case $k \leq 1$. Obvious.

          Case $m,k > 1$.
            Case $m + k < n$ or $m + k > n$. Obvious.

            Case $m + k = n$.
              We have \[ \frac{(m + k)!}{m!} = \left( k \cdot \frac{(m + (k - 1))!}{m!} \right) + \frac{((m - 1) + k)!}{(m - 1)!}. \]

              (1) $k! \mid k \cdot \frac{(m + (k - 1))!}{m!}$. \\
              Proof.
                We have $m + (k - 1) < n$.
                Hence $(k - 1)!$ divides $\frac{(m + (k - 1))!}{m!}$.
                Thus we can take a natural number $l$ such that $(k - 1)! \cdot l = \frac{(m + (k - 1))!}{m!}$.
                Then $k! \cdot l = (k \cdot (k - 1)!) \cdot l = k \cdot ((k - 1)! \cdot l) = k \cdot \frac{(m + (k - 1))!}{m!}$.
                Therefore $k!$ divides $k \cdot \frac{(m + (k - 1))!}{m!}$.
              Qed.

              [prover vampire]

              (2) $k! \mid \frac{((m - 1) + k)!}{(m - 1)!}$. \\
              Proof.
                We have $(m - 1) + k < n$.
                Indeed $(m - 1) + k = (m + k) - 1 = n - 1 < n$.
                Hence $k! \mid \frac{((m - 1) + k)!}{(m - 1)!}$.
              Qed.

              Then $k!$ divides $\left( k \cdot \frac{(m + (k - 1))!}{m!} \right) + \frac{((m - 1) + k)!}{(m - 1)!}$ (by \ref{Arithmetic_03_01_655411}, 1, 2).
            End.
          End.
        End.
      Qed.

      [prover vampire]
      Hence $P$ contains every natural number (by \ref{Arithmetic_02_05_167446}).
      Then we have the thesis.
    \end{proof}

    \begin{lemma}\label{Arithmetic_03_06_504192}
      Let $n \geq k$.
      Then $k! \cdot ((n - k)!)$ divides $n!$.
    \end{lemma}
    \begin{proof}
      We have \[ (n!) = ((k + (n - k))!) = k! \cdot \frac{(k + (n - k))!}{k!}. \]
      $k!$ is divisible by $k!$ and $\frac{(k + (n - k))!}{k!}$ is divisible by $(n - k)!$.
      Hence $k! \cdot ((n - k)!)$ divides $n!$.
    \end{proof}

    \begin{definition}
      Let $n \geq k$.
      \[ \binom{n}{k} = \frac{n!}{k! \cdot ((n - k)!)}. \]
    \end{definition}
  \end{forthel}


  \subsection{Basic properties}

  \begin{forthel}
    \begin{proposition}\label{Arithmetic_03_06_502271}
      Let $n \geq k$.
      Then \[ \binom{n}{k} \geq 1. \]
    \end{proposition}
    \begin{proof}
      We have $n! \neq 0$.
      Hence \[ \binom{n}{k} = \frac{n!}{k! \cdot ((n - k)!)} \neq 0. \]
      Therefore the thesis.
    \end{proof}

    \begin{proposition}\label{Arithmetic_03_06_766613}
      \[ \binom{n}{0} = 1 = \binom{n}{n}. \]
    \end{proposition}
    \begin{proof}
      \[ \binom{n}{0}
          = \frac{n!}{0! \cdot ((n - 0)!)}
          = \frac{n!}{1 \cdot (n!)}
          = \frac{n!}{n!}
          = 1. \]

      \[ \binom{n}{n}
          = \frac{n!}{n! \cdot ((n - n)!)}
          = \frac{n!}{n! \cdot (0!)}
          = \frac{n!}{n! \cdot 1}
          = \frac{n!}{n!}
          = 1. \]
    \end{proof}

    \begin{proposition}\label{Arithmetic_03_06_877389}
      Let $n$ be nonzero.
      Then \[ \binom{n}{1} = n = \binom{n}{n - 1}. \]
    \end{proposition}
    \begin{proof}
      Let us show that $\frac{n!}{(n - 1)!} = n$.
        $1 \cdot (n - 1)!$ divides $n \cdot (n - 1)!$.
        Indeed $n \cdot (1 \cdot ((n - 1)!)) = (n \cdot 1) \cdot ((n - 1)!) = n \cdot ((n - 1)!)$.
        Therefore
        \[ \frac{n!}{(n - 1)!}
           = \frac{n \cdot (n - 1)!}{1 \cdot (n - 1)!}
           = \frac{n}{1}
           = n. \]
      End.

      \[ \binom{n}{1}
          = \frac{n!}{1! \cdot ((n - 1)!)}
          = \frac{n!}{1 \cdot ((n - 1)!)}
          = \frac{n!}{(n - 1)!}
          = n. \]

      \[ \binom{n}{n - 1}
          = \frac{n!}{(n - 1)! \cdot ((n - (n - 1))!)}
          = \frac{n!}{(n - 1)! \cdot (((n - n) + 1)!)} \]
      \[  = \frac{n!}{(n - 1)! \cdot (1!)}
          = \frac{n!}{(n - 1)!}
          = n. \]
    \end{proof}

    \begin{proposition}\label{Arithmetic_03_06_360382}
      Let $n \geq k$.
      Then \[ \binom{n}{k} = \binom{n}{n - k}. \]
    \end{proposition}
    \begin{proof}
      \[ \binom{n}{n - k}
          = \frac{n!}{(n - k)! \cdot ((n - (n - k))!)} \]
      \[  = \frac{n!}{(n - k)! \cdot (k!)}
          = \frac{n!}{k! \cdot ((n - k)!)}
          = \binom{n}{k}. \]
    \end{proof}

    \begin{proposition}[Pascal's rule]\label{Arithmetic_03_06_639359}
      Let $n > k$.
      Then \[ \binom{n + 1}{k + 1} = \binom{n}{k} + \binom{n}{k + 1}. \]
    \end{proposition}
    \begin{proof}
      %%%%%%%%%% Needed for ontological checking:
      (1) $((k + 1)! \cdot ((n - (k + 1))!)) \cdot (n - k)$ divides $n! \cdot (n - k)$. \\
      Proof.
        $(k + 1)! \cdot ((n - (k + 1))!)$ divides $n!$.
        Hence we can take a natural number $l$ such that $(n!) = l \cdot ((k + 1)! \cdot ((n - (k + 1))!))$.
        Then \[ n! \cdot (n - k) = (l \cdot ((k + 1)! \cdot ((n - (k + 1))!))) \cdot (n - k). \]
        [prover vampire]
        We have \[ (l \cdot ((k + 1)! \cdot ((n - (k + 1))!))) \cdot (n - k) = l \cdot (((k + 1)! \cdot ((n - (k + 1))!)) \cdot (n - k)) \]
        (by \ref{Arithmetic_01_03_299637}).
        Hence \[ n! \cdot (n - k) = l \cdot (((k + 1)! \cdot ((n - (k + 1))!)) \cdot (n - k)). \]
        Then we have the thesis (by \ref{Arithmetic_03_01_926592}).
      Qed.

      (2) $(k + 1)! \cdot ((n - k)!)$ divides $(k + 1) \cdot (n!)$. \\
      Proof.
        $k! \cdot ((n - k)!)$ divides $n!$.
        Hence we can take a natural number $l$ such that $(k! \cdot ((n - k)!)) \cdot l = n!$.
        Then $(k + 1) \cdot ((k! \cdot ((n - k)!)) \cdot l) = (k + 1) \cdot (n!)$.
        Thus $((k + 1)! \cdot ((n - k)!)) \cdot l =
        (k + 1)! \cdot (((n - k)!) \cdot l) =
        ((k + 1) \cdot (k!)) \cdot (((n - k)!) \cdot l) =
        (k + 1) \cdot (k! \cdot (((n - k)!) \cdot l)) =
        (k + 1) \cdot ((k! \cdot ((n - k)!)) \cdot l) =
        (k + 1) \cdot (n!)$.
        Indeed $k! \cdot (((n - k)!) \cdot l) = (k! \cdot ((n - k)!)) \cdot l$.
      Qed.

      (3) $(k + 1)! \cdot (n - k)!$ divides $n! \cdot (n - k)$.

      (4) $(k + 1)! \cdot ((n - k)!)$ divides $n! \cdot ((k + 1) + (n - k))$.
      %%%%%%%%%%

      Let us show that $\binom{n}{k} = \frac{(k + 1) \cdot (n!)}{(k + 1)! \cdot ((n - k)!)}$.
        [prover vampire]
        We have $(k + 1) \cdot (k! \cdot ((n - k)!)) =
        ((k + 1) \cdot (k!)) \cdot ((n - k)!) =
        (k + 1)! \cdot ((n - k)!)$.
        Hence
        \[ \binom{n}{k} \]
        \[ = \frac{n!}{k! \cdot ((n - k)!)} \]
        \[ = \frac{(k + 1) \cdot (n!)}{(k + 1) \cdot (k! \cdot ((n - k)!))} \]
        \[ = \frac{(k + 1) \cdot (n!)}{(k + 1)! \cdot ((n - k)!)}. \]
      End.

      Let us show that $\binom{n}{k + 1} = \frac{n! \cdot (n - k)}{(k + 1)! \cdot (n - k)!}$.
        \[ \binom{n}{k + 1} = \frac{n!}{(k + 1)! \cdot ((n - (k + 1))!)}. \]
        \[ \frac{n!}{(k + 1)! \cdot ((n - (k + 1))!)} = \frac{n! \cdot (n - k)}{((k + 1)! \cdot ((n - (k + 1))!)) \cdot (n - k)} \]
        (by \ref{Arithmetic_03_05_243924}).
        Indeed $(k + 1)! \cdot ((n - (k + 1))!)$ is a nonzero divisor of $n!$.
        We have $((k + 1)! \cdot ((n - (k + 1))!)) \cdot (n - k) =
        (k + 1)! \cdot ((n - (k + 1))! \cdot (n - k)) =
        (k + 1)! \cdot (((n - k) - 1)! \cdot (n - k)) =
        (k + 1)! \cdot ((n - k)!)$.
        Hence \[ \frac{n! \cdot (n - k)}{((k + 1)! \cdot ((n - (k + 1))!)) \cdot (n - k)} = \frac{n! \cdot (n - k)}{(k + 1)! \cdot (n - k)!}. \]
      End.

      Then we have
      \[   \binom{n}{k} + \binom{n}{k + 1} \]
      \[ = \frac{(k + 1) \cdot (n!)}{(k + 1)! \cdot ((n - k)!)} + \frac{n! \cdot (n - k)}{(k + 1)! \cdot (n - k)!} \]
      \[ = \frac{((k + 1) \cdot (n!)) + (n! \cdot (n - k))}{(k + 1)! \cdot ((n - k)!)} \]
      (by \ref{Arithmetic_03_05_363442}).
      Indeed $(k + 1)! \cdot ((n - k)!)$ is a nonzero divisor of $(k + 1) \cdot (n!)$ and a nonzero divisor of $n! \cdot (n - k)$.

      \[   \frac{((k + 1) \cdot (n!)) + (n! \cdot (n - k))}{(k + 1)! \cdot ((n - k)!)} \]
      \[ = \frac{n! \cdot ((k + 1) + (n - k))}{(k + 1)! \cdot ((n - k)!)} \]
      \[ = \frac{n! \cdot (n + 1)}{(k + 1)! \cdot ((n - k)!)} \]
      \[ = \frac{(n + 1)!}{(k + 1)! \cdot ((n - k)!)} \]
      \[ = \frac{(n + 1)!}{(k + 1)! \cdot (((n + 1) - (k + 1))!)} \]
      \[ = \binom{n + 1}{k + 1}. \]
    \end{proof}
  \end{forthel}
\end{document}
