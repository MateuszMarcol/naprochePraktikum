\documentclass[../../arithmetic.ftl.tex]{subfiles}

\begin{document}
  \section{Peano Arithmetic}

  \begin{forthel}
    [readtex \path{vocabulary.ftl.tex}]
  \end{forthel}

  \begin{forthel}
    [readtex \path{macros.ftl.tex}]
  \end{forthel}


  \subsection{The Peano axioms}

  This arithmetic is based on the notion of natural numbers.
  These are introduced as some sort of elements that is equiped with a a unary
  function $\succ$ (which maps any natural number to its direct successor)
  and that contains a constant $0$ (the unique least natural number).

  \begin{forthel}
    \begin{signature}
      A natural number is an element.
    \end{signature}

    Let $k,l,m,n$ denote natural numbers.

    \begin{definition}
      $\mathbb{N}$ is the class of natural numbers.
    \end{definition}

    \begin{signature}
      $0$ is a natural number.
    \end{signature}

    Let $n$ is nonzero stand for $n \neq 0$.

    \begin{signature}
      $\succ(n)$ is a natural number.
    \end{signature}

    Let the direct successor of $n$ stand for $\succ(n)$.
  \end{forthel}

  \noindent The natural numbers are characterized by the following so-called
  Peano axioms.

  \begin{forthel}
    \begin{axiom}[\nth{1} Peano axiom]\label{Arithmetic_01_01_011520}
      If $\succ(n) = \succ(m)$ then $n = m$.
    \end{axiom}

    \begin{axiom}[\nth{2} Peano axiom]\label{Arithmetic_01_01_795282}
      $0$ is not the direct successor of any natural number.
    \end{axiom}

    \begin{axiom}[\nth{3} Peano axiom]\label{Arithmetic_01_01_350822}
      Let $P$ be a class.
      Assume $0 \in P$ and for all natural numbers $n$ we have $n \in P \implies \succ(n) \in P$.
      Then every natural number is an element of $P$.
    \end{axiom}
  \end{forthel}


  \subsection{Immediate consequences}

  \noindent The \nth{3} Peano axiom (also called the \textit{induction axiom})
  allows us to prove that the signature $(0, \succ)$ captures the whole class
  of natural numbers in the sense that every natural number is either zero or a
  successor:

  \begin{forthel}
    \begin{proposition}\label{Arithmetic_01_01_178800}
      For all $n$ we have $n = 0$ or $n = \succ(m)$ for some natural number $m$.
    \end{proposition}
    \begin{proof}
      Define \[ P = \class{n \in \mathbb{N} | \classtext{$n = 0$ or $n = \succ(m)$ for some natural number $m$}}. \]

      $0 \in P$ and for all natural numbers $n$ we have $n \in P \implies \succ(n) \in P$.
      Hence the thesis (by \nameref{Arithmetic_01_01_350822}).
    \end{proof}
  \end{forthel}

  \noindent This allows us to define the direct predecessor of a non-zero
  natural number as follows.

  \begin{forthel}
    \begin{definition}
      Let $n$ be nonzero.
      $\pred(n)$ is the natural number $m$ such that $\succ(m) = n$.
    \end{definition}

    Let the direct predecessor of $n$ stand for $\pred(n)$.
  \end{forthel}

  \noindent Note that direct predecessors must be unique by the \nth{2} Peano
  axiom.
  Moreover, we can show that no natural number is its own successor.

  \begin{forthel}
    \begin{proposition}\label{Arithmetic_01_01_670417}
      For no natural number $n$ we have $n = \succ(n)$.
    \end{proposition}
    \begin{proof}
      Define \[ P = \class{n \in \mathbb{N} | n \neq \succ(n)}. \]

      (BASE CASE) $0$ belongs to $P$.

      (INDUCTION STEP) For all $n$ we have $n \in P \implies \succ(n) \in P$. \\
      Proof.
        Let $n$ be a natural number.
        Assume that $n \in P$.
        Then $n \neq \succ(n)$.
        If $\succ(n) = \succ(\succ(n))$ then $n = \succ(n)$.
        Thus it is wrong that $\succ(n) = \succ(\succ(n))$.
        Hence $\succ(n) \in P$.
      Qed.

      Therefore every natural number is an element of $P$.
      Then we have the thesis.
    \end{proof}
  \end{forthel}


  \subsection{Additional constants}

  \noindent Let us end this section by introducing new constant symbols for the
  first few successors of $0$.

  \begin{forthel}
    \begin{definition}
      $1 = \succ(0)$.
    \end{definition}

    \begin{definition}
      $2 = \succ(1)$.
    \end{definition}

    \begin{definition}
      $3 = \succ(2)$.
    \end{definition}

    \begin{definition}
      $4 = \succ(3)$.
    \end{definition}

    \begin{definition}
      $5 = \succ(4)$.
    \end{definition}

    \begin{definition}
      $6 = \succ(5)$.
    \end{definition}

    \begin{definition}
      $7 = \succ(6)$.
    \end{definition}

    \begin{definition}
      $8 = \succ(7)$.
    \end{definition}

    \begin{definition}
      $9 = \succ(8)$.
    \end{definition}
  \end{forthel}
\end{document}
