\documentclass[../../arithmetic.ftl.tex]{subfiles}

\begin{document}
  \section{Factorial}

  \begin{forthel}
    [readtex \path{arithmetic/sections/01_arithmetic/03_multiplication.ftl.tex}]
  \end{forthel}

  \begin{forthel}
    Let $k, l, m, n$ denote natural numbers.
  \end{forthel}

  \noindent An operation rather rarely mentioned together with (formal) Peano
  arithmetic is the factorial operation which we are going to define now.

  \begin{forthel}
    \begin{signature}
      $n!$ is a natural number.
    \end{signature}

    \begin{axiom}[\nth{1} factorial axiom]\label{Arithmetic_01_05_169222}
      $(0!) = 1$.
    \end{axiom}

    \begin{axiom}[\nth{2} factorial axiom]\label{Arithmetic_01_05_539010}
      $((n + 1)!) = n! \cdot (n + 1)$.
    \end{axiom}
  \end{forthel}

  \noindent \textcolor{gray}{Note that we have to put the LHS of any expression
  of the form \enquote{$x! = y$} in parentheses, because such an expression can
  either be understood as \enquote{$x$ factorial is equal to $y$} or as
  \enquote{$x$ is not equal to $y$} by Naproche since it treats the combination
  of an exclamation mark followed by an equality sign as a synonym for
  \enquote{$\neq$}.}

  \begin{forthel}
    \begin{proposition}\label{Arithmetic_01_05_473272}
      $n!$ is nonzero for any natural number $n$.
    \end{proposition}
    \begin{proof}
      Define \[ P = \class{n \in \mathbb{N} | n! \neq 0}. \]

      (BASE CASE) $P$ contains $0$.
      Indeed $(0!) = 1 \neq 0$.

      (INDUCTION STEP) For every natural number $n$ we have $n \in P \implies n + 1 \in P$. \\
      Proof.
        Let $n$ be a natural number.
        Assume $n \in P$.
        We have $((n + 1)!) = (n + 1) \cdot (n!)$.
        $n + 1$ and $n!$ are nonzero.
        Hence $(n + 1)!$ is nonzero.
      Qed.

      Thus $P$ contains every natural number.
    \end{proof}
  \end{forthel}
\end{document}
