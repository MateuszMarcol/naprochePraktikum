\documentclass[../../arithmetic.ftl.tex]{subfiles}

\begin{document}

  \section{Multiplication}

  \begin{forthel}
    [readtex \path{arithmetic/sections/01_arithmetic/02_addition.ftl.tex}]
  \end{forthel}

  \begin{forthel}
    Let $k, l, m, n$ denote natural numbers.
  \end{forthel}


  \subsection{Axioms}

  Having introduced addition in the last section, we now define a multiplication
  operation on the natural numbers.

  \begin{forthel}
    \begin{signature}
      $n \cdot m$ is a natural number.
    \end{signature}

    Let the product of $n$ and $m$ stand for $n \cdot m$.

    \begin{axiom}[\nth{1} multiplication axiom]\label{Arithmetic_01_03_374176}
      $n \cdot 0 = 0$.
    \end{axiom}

    \begin{axiom}[\nth{2} multiplication axiom]\label{Arithmetic_01_03_667207}
      $n \cdot (m + 1) = (n \cdot m) + n$.
    \end{axiom}
  \end{forthel}


  \subsection{Computation laws}

  Let us show some basic computation laws for it.


  \subsubsection*{Associativity}

  \begin{forthel}
    \begin{proposition}\label{Arithmetic_01_03_539933}
      For all $n,m,k$ we have \[ n \cdot (m + k) = (n \cdot m) + (n \cdot k). \]
    \end{proposition}
    \begin{proof}
      Define $P = \class{k \in \mathbb{N} | \classtext{$n \cdot (m + k) = (n \cdot m) + (n \cdot k)$ for all natural numbers $n,m$}}$.

      (BASE CASE) $0$ is an element of $P$.
      Indeed for all natural numbers $n,m$ we have $n \cdot (m + 0) = n \cdot m = (n \cdot m) + 0 = (n \cdot m) + (n \cdot 0)$.

      (INDUCTION STEP) For all natural numbers $k$ we have $k \in P \implies k + 1 \in P$. \\
      Proof.
        Let $k$ be a natural number.
        Assume $k \in P$.

        For all natural numbers $n,m$ we have $n \cdot (m + (k + 1)) = (n \cdot m) + (n \cdot (k + 1))$. \\
        Proof.
          Let $n,m$ be natural numbers.

          \[   n \cdot (m + (k + 1)) \]
          \[ = n \cdot ((m + k) + 1) \]             % 2nd addition axiom
          \[ = (n \cdot (m + k)) + n \]             % 2nd multiplication axiom
          \[ = ((n \cdot m) + (n \cdot k)) + n \]   % k \in P
          \[ = (n \cdot m) + ((n \cdot k) + n) \]   % associativity of addition
          \[ = (n \cdot m) + (n \cdot (k + 1)). \]  % 2nd multiplication axiom

          Hence the thesis.
        Qed.
      Qed.

      Therefore every natural number is contained in $P$.
    \end{proof}
  \end{forthel}


  \subsubsection*{Distributivity:}

  \begin{forthel}
    \begin{proposition}\label{Arithmetic_01_03_322712}
      For all $n,m,k$ we have \[ (n + m) \cdot k = (n \cdot k) + (m \cdot k). \]
    \end{proposition}
    \begin{proof}
      Define $P = \class{k \in \mathbb{N} | \classtext{$(n + m) \cdot k = (n \cdot k) + (m \cdot k)$ for all natural numbers $n,m$}}$.

      (BASE CASE) $0$ belongs to $P$.
      Indeed $(n + m) \cdot 0 = 0 = 0 + 0 = (n \cdot 0) + (m \cdot 0)$ for all natural numbers $n,m$.

      (INDUCTION STEP) For all natural numbers $k$ we have $k \in P \implies k + 1 \in P$. \\
      Proof.
        Let $k$ be a natural number.
        Assume $k \in P$.

        $(n + m) \cdot (k + 1) = (n \cdot (k + 1)) + (m \cdot (k + 1))$ for all
        natural numbers $n,m$. \\
        Proof.
          Let $n,m$ be natural numbers.
          We have $((n \cdot k) + ((m \cdot k) + n)) + m =
          (((n \cdot k) + n) + (m \cdot k)) + m$.
          Hence
          \[ (n + m) \cdot (k + 1) \]
          % 2nd multiplication axiom:
          \[ = ((n + m) \cdot k) + (n + m) \]
          % k \in P:
          \[ = ((n \cdot k) + (m \cdot k)) + (n + m) \]
          % assoziativity of addition:
          \[ = (((n \cdot k) + (m \cdot k)) + n) + m \]
          % assoziativity of addition:
          \[ = ((n \cdot k) + ((m \cdot k) + n)) + m \]
          \[ = (((n \cdot k) + n) + (m \cdot k)) + m \]
          % associativity of addition:
          \[ = ((n \cdot k) + n) + ((m \cdot k) + m) \]
          % 2nd multiplication axiom:
          \[ = (n \cdot (k + 1)) + (m \cdot (k + 1)). \]
        Qed.
      Qed.

      Thus every natural number is an element of $P$.
    \end{proof}


    \begin{proposition}\label{Arithmetic_01_03_866630}
      $n \cdot 1 = n$.
    \end{proposition}
    \begin{proof}
      $ n \cdot 1
      = n \cdot (0 + 1)   % commutativity of addition, 1st addition axiom
      = (n \cdot 0) + n   % 2nd multiplication axiom
      = 0 + n             % 1st multiplication axiom
      = n$.               % commutativity of addition, 1st addition axiom
    \end{proof}


    \begin{corollary}\label{Arithmetic_01_03_302621}
    $n \cdot 2 = n + n$.
    \end{corollary}
    \begin{proof}
      $ n \cdot 2
      = n \cdot (1 + 1)   % Definition of 2
      = (n \cdot 1) + n   % 2nd multiplication axiom
      = n + n$.           % 1 is neutral wrt. multiplication
    \end{proof}


    \begin{proposition}\label{Arithmetic_01_03_299637}
      For all $n,m,k$ we have \[ n \cdot (m \cdot k) = (n \cdot m) \cdot k. \]
    \end{proposition}
    \begin{proof}
      Define $P = \class{k \in \mathbb{N} | \classtext{$n \cdot (m \cdot k) = (n \cdot m) \cdot k$ for all natural numbers $n,m$}}$.

      (BASE CASE) $0$ is contained in $P$.
      Indeed for all natural numbers $n,m$ we have $n \cdot (m \cdot 0) = n \cdot 0 = 0 = (n \cdot m) \cdot 0$.

      (INDUCTION STEP) For all natural numbers $k$ we have $k \in P \implies k + 1 \in P$. \\
      Proof.
        Let $k$ be a natural number.
        Assume $k \in P$.

        For all natural numbers $n,m$ we have $n \cdot (m \cdot (k + 1)) = (n \cdot m) \cdot (k + 1)$. \\
        Proof.
          Let $n,m$ be natural numbers.
          \[ n \cdot (m \cdot (k + 1)) \]
          % 2nd multiplication axiom:
          \[ = n \cdot ((m \cdot k) + m) \]
          % left distributivity:
          \[ = (n \cdot (m \cdot k)) + (n \cdot m) \]
          % k \in P:
          \[ = ((n \cdot m) \cdot k) + (n \cdot m) \]
          % 1 is neutral wrt. multiplication:
          \[ = ((n \cdot m) \cdot k) + ((n \cdot m) \cdot 1) \]
          % left distributivity:
          \[ = (n \cdot m) \cdot (k + 1). \]
        Qed.
      Qed.

      Hence every natural number is contained in $P$.
    \end{proof}
  \end{forthel}


  \subsubsection*{Commutativity:}

  \begin{forthel}
    \begin{proposition}\label{Arithmetic_01_03_850937}
      For all $n,m$ we have \[ n \cdot m = m \cdot n. \]
    \end{proposition}
    \begin{proof}
      Define $P = \class{m \in \mathbb{N} | \text{$n \cdot m = m \cdot n$ for all natural numbers $n$}}$.

      (BASE CASE 1) $0$ is contained in $P$. \\
      Proof.

        For all natural numbers $n$ we have $n \cdot 0 = 0 \cdot n$. \\
        Proof.
          Define $Q = \class{n \in \mathbb{N} | n \cdot 0 = 0 \cdot n}$.

          $0$ is contained in $Q$.

          For all natural numbers $n$ we have $n \in Q \implies n + 1 \in Q$. \\
          Proof.
            Let $n$ be a natural number.
            Assume $n \in Q$.
            Then
            \[
              (n + 1) \cdot 0
            = 0                 % 1st multiplication axiom
            = n \cdot 0         % 1st multiplication axiom
            = 0 \cdot n         % n \in Q
            = (0 \cdot n) + 0   % 1st addition axiom
            = 0 \cdot (n + 1).  % 2nd multiplication axiom
            \]
          Qed.
        Qed.
      Qed.

      (BASE CASE 2) $1$ belongs to $P$. \\
      Proof.
        Let us show that for all natural numbers $n$ we have $n \cdot 1 = 1 \cdot n$.
          Define $Q = \class{n \in \mathbb{N} | n \cdot 1 = 1 \cdot n}$.

          $0$ is contained in $Q$.

          For all natural numbers $n$ we have $n \in Q \implies n + 1 \in Q$. \\
          Proof.
            Let $n$ be a natural number.
            Assume $n \in Q$.
            Then
            \[   (n + 1) \cdot 1 \]
            \[ = (n \cdot 1) + 1 \]   % right-distributivity,
            \[ = (1 \cdot n) + 1 \]   % 1 \in Q
            \[ = 1 \cdot (n + 1). \]  % left-distributivity
          Qed.

          Thus every natural number is contained in $Q$.
          Hence the thesis.
        End.
      Qed.

      (INDUCTION STEP) For all natural numbers $m$ we have $m \in P \implies m + 1 \in P$. \\
      Proof.
        Let $m$ be a natural number.
        Assume $m \in P$.

        For all natural numbers $n$ we have $n \cdot (m + 1) = (m + 1) \cdot n$. \\
        Proof.
          Let $n$ be a natural number.
          Then
          \[ n \cdot (m + 1) \]
          % left distributivity:
          \[ = (n \cdot m) + (n \cdot 1) \]
          % 1,m \im P:
          \[ = (m \cdot n) + (1 \cdot n) \]
          % commutativity of addition:
          \[ = (1 \cdot n) + (m \cdot n) \]
          % right distributivity:
          \[ = (1 + m) \cdot n \]
          % commutativity of addition:
          \[ = (m + 1) \cdot n. \]
        Qed.
      Qed.

      Hence every natural number is contained in $P$.
    \end{proof}
  \end{forthel}


  \subsubsection*{Non-existence of zero-divisors:}

  \begin{forthel}
    \begin{proposition}\label{Arithmetic_01_03_692941}
      For all $n,m$ we have \[ n \cdot m = 0 \implies (\text{$n = 0$ or $m = 0$}). \]
    \end{proposition}
    \begin{proof}
      Let $n,m$ be natural numbers.
      Assume $n \cdot m = 0$.

      If $n$ and $m$ are not equal to $0$ then we have a contradiction. \\
      Proof.
        Assume $n,m \neq 0$.
        Take natural numbers $n',m'$ such that $n = (n' + 1)$ and $m = (m' + 1)$.
        Then
        \[   0 \]
        \[ = n \cdot m \]
        \[ = (n' + 1) \cdot (m' + 1) \]
        \[ = ((n' + 1) \cdot m') + (n' + 1) \]
        \[ = (((n' + 1) \cdot m') + n') + 1. \]
        Hence $0 = k + 1$ for some natural number $k$.
        Contradiction.
      Qed.

      Thus $n = 0$ or $m = 0$.
    \end{proof}
  \end{forthel}


  \subsubsection*{Cancellation:}

  \begin{forthel}
    \begin{proposition}\label{Arithmetic_01_03_799692}
      Assume $k \neq 0$.
      Then for all $n,m$ we have \[ n \cdot k = m \cdot k \implies n = m. \]
    \end{proposition}
    \begin{proof}
      Define $P = \class{n \in \mathbb{N} | \classtext{for all natural numbers $m$ if $n \cdot k = m \cdot k$ and $k \neq 0$ then $n = m$}}$.

      (BASE CASE) $0$ is contained in $P$. \\
      Proof.
        Let us show that for all natural numbers $m$ if $0 \cdot k = m \cdot k$ and $k \neq 0$ then $0 = m$.
          Let $m,k$ be natural numbers.
          Assume that $0 \cdot k = m \cdot k$ and $k \neq 0$.
          Then $m \cdot k = 0$.
          Hence $m = 0$ or $k = 0$.
          Thus $m = 0$.
        End.
      Qed.

      (INDUCTION STEP) For all natural numbers $n$ we have $n \in P \implies n + 1 \in P$. \\
      Proof.
        Let $n$ be a natural number.
        Assume $n \in P$.

        For all natural numbers $m$ if $(n + 1) \cdot k = m \cdot k$ and $k \neq 0$ then $n + 1 = m$. \\
        Proof.
          Let $m$ be natural numbers.
          Assume $(n + 1) \cdot k = m \cdot k$ and $k \neq 0$.

          Case $m = 0$.
            Then $(n + 1) \cdot k = 0$.
            Hence $n + 1 = 0$.
            Contradiction.
          End.

          Case $m \neq 0$.
            Take a natural number $m'$ such that $m = m' + 1$.
            Then $(n + 1) \cdot k = (m' + 1) \cdot k$.
            Hence $(n \cdot k) + k = (m' \cdot k) + k$.
            Thus $n \cdot k = m' \cdot k$ (by \ref{Arithmetic_01_02_882987}).
            Then we have $n = m'$.
            Therefore $n + 1 = m' + 1 = m$.
          End.
        Qed.
      Qed.

      Thus every natural number is contained in $P$.
    \end{proof}


    \begin{corollary}\label{Arithmetic_01_03_169506}
      Assume $k \neq 0$.
      Then for all $n,m$ we have \[ k \cdot n = k \cdot m \implies n = m. \]
    \end{corollary}
    \begin{proof}
      Let $n,m$ be natural numbers.
      Assume $k \cdot n = k \cdot m$.
      We have $k \cdot n = n \cdot k$ and $k \cdot m = m \cdot k$.
      Hence $n \cdot k = m \cdot k$.
      Thus $n = m$.
    \end{proof}
  \end{forthel}
\end{document}
