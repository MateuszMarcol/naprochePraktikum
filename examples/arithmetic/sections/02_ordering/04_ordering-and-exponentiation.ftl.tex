\documentclass[../../natural-numbers.ftl.tex]{subfiles}

\begin{document}

  \section{Ordering and exponentiation}

  \begin{forthel}
    [readtex \path{arithmetic/sections/01_arithmetic/04_exponentiation.ftl.tex}]
  \end{forthel}

  \begin{forthel}
    [readtex \path{arithmetic/sections/02_ordering/03_ordering-and-multiplication.ftl.tex}]
  \end{forthel}

  \begin{forthel}
    Let $k, l, m, n$ denote natural numbers.
  \end{forthel}


  \begin{forthel}
    \begin{proposition}[NN 02 04 770958]
      Assume $k \neq 0$.
      Then for all $n,m$ we have \[ n < m \iff n^{k} < m^{k}. \]
    \end{proposition}
    \begin{proof}
      Define $P = \class{k' \in \mathbb{N} | \classtext{for all natural numbers $n,m$ if $n < m$ and $k' > 1$ then $n^{k'} < m^{k'}$}}$.

      Let us show that every natural number is contained in $P$.
        (BASE CASE 1) $P$ contains $0$.

        (BASE CASE 2) $P$ contains $1$.

        (BASE CASE 3) $P$ contains $2$. \\
        Proof.
          Let us show that for all natural numbers $n,m$ if $n < m$ then $n^{2} < m^{2}$.
            Let $n,m$ be natural numbers.
            Assume $n < m$.

            Case $n = 0$ or $m = 0$. Obvious.

            Case $n,m \neq 0$.
              Then $n \cdot n < n \cdot m < m \cdot m$.
              Hence $n^{2} = n \cdot n < n \cdot m < m \cdot m = m^{2}$.
            End.
          End.
        Qed.

        (INDUCTION STEP) For all natural numbers $k'$ we have $k' \in P \implies k' + 1 \in P$. \\
        Proof.
          Let $k'$ be a natural number.
          Assume $k' \in P$.

          For all natural numbers $n,m$ if $n < m$ and $k' + 1 > 1$ then $n^{k' + 1} < m^{k' + 1}$. \\
          Proof.
            Let $n,m$ be natural numbers.
            Assume $n < m$ and $k' + 1 > 1$.
            Then $n^{k'} < m^{k'}$.
            Indeed $k' \neq 0$ and $if k' = 1$ then $n^{k'} < m^{k'}$.

            Case $k' \leq 1$.
              Then $k' = 0$ or $k' = 1$.
              Hence $k' + 1 = 1$ or $k' + 1 = 2$.
              Thus $k' + 1 \in P$.
              Therefore $n^{k' + 1} < m^{k' + 1}$.
            End.

            Case $k' > 1$.
              Case $n = 0$.
                Then $m \neq 0$.
                Hence $n^{k' + 1} = 0 < m^{k'} \cdot m = m^{k' + 1}$.
                Thus $n^{k' + 1} < m^{k' + 1}$.
              End.

              Case $n \neq 0$.
                Then $n^{k'} \cdot n < m^{k'} \cdot n < m^{k'} \cdot m$.
                Indeed $m^{k'} \neq 0$.
                Hence $n^{k' + 1} = n^{k'} \cdot n < m^{k'} \cdot n < m^{k'} \cdot m = m^{k' + 1}$.
                Thus $n^{k' + 1} < m^{k' + 1}$ (by NN 02 01 662806).
              End.
            End.

            Hence the thesis.
            Indeed $k' \leq 1$ or $k' > 1$.
          Qed.

          $k' + 1 \in P$.
        Qed.

        Therefore every natural number is contained in $P$.
      End.


      Define $Q = \class{k' \in \mathbb{N} | \classtext{for all natural numbers $n,m$ if $n \geq m$ then $n^{k'} \geq m^{k'}$}}$.

      Let us show that every natural number is contained in $Q$.
        (BASE CASE) $Q$ contains $0$.

        (INDUCTION STEP) For all natural numbers $k'$ we have $k' \in Q \implies k' + 1 \in Q$. \\
        Proof.
          Let $k'$ be a natural number.
          Assume $k' \in Q$.

          For all natural numbers $n,m$ if $n \geq m$ then $n^{k' + 1} \geq m^{k' + 1}$. \\
          Proof.
            Let $n,m$ be natural numbers.
            Assume $n \geq m$.
            Then $n^{k'} \geq m^{k'}$.
            Hence $n^{k'} \cdot n \geq m^{k'} \cdot n \geq m^{k'} \cdot m$.
            Thus $n^{k' + 1} = n^{k'} \cdot n \geq m^{k'} \cdot n \geq m^{k'} \cdot m = m^{k' + 1}$.
            Therefore $n^{k' + 1} \geq m^{k' + 1}$ (by NN 02 01 394529).
          Qed.

          Hence the thesis.
          Indeed $k' + 1$ is a natural number.
        Qed.

        Thus every natural number is contained in $Q$.
      End.


      Let $n,m$ be natural numbers.

      Case $n < m$.
        Case $k = 1$. Obvious.

        Case$k \neq 1$.
          Then $k > 1$.
          Indeed $k < 1$ or $k = 1$ or $k > 1$.
          Hence $n^{k} < m^{k}$.
          Indeed $n$ and $m$ belong to $P$.
        End.
      End.

      Case $n^{k} < m^{k}$.
        Then $n^{k} \ngeq m^{k}$.
        Hence $n \ngeq m$.
        Indeed $n$ and $m$ are contained in $Q$.
        Thus $n < m$.
      End.
    \end{proof}


    \begin{corollary}[NN 02 04 537812]
      Assume $k \neq 0$.
      Then \[ n^{k} = m^{k} \implies n = m. \]
    \end{corollary}
    \begin{proof}
      Assume $n \neq m$.
      Then $n < m$ or $m < n$.
      If $n < m$ then $n^{k} < m^{k}$.
      If $m < n$ then $m^{k} < n^{k}$.
      Thus $n^{k} \neq m^{k}$.
      Hence the thesis.
    \end{proof}


    \begin{corollary}[NN 02 04 707319]
      Assume $k \neq 0$.
      Then \[ n^{k} \leq m^{k} \iff n \leq m. \]
    \end{corollary}
    \begin{proof}
      If $n^{k} < m^{k}$ then $n < m$.
      If $n^{k} = m^{k}$ then $n = m$.

      If $n < m$ then $n^{k} < m^{k}$.
      If $n = m$ then $n^{k} = m^{k}$.
    \end{proof}


    \begin{proposition}[NN 02 04 274623]
      Assume $k > 1$.
      Then for all $n,m$ we have \[ n < m \iff k^{n} < k^{m}. \]
    \end{proposition}
    \begin{proof}
      Define $P = \class{m \in \mathbb{N} | \classtext{for all natural numbers $n$ if $k > 1$ and $n < m$ then $k^{n} < k^{m}$}}$.

      Let us show that every natural number is contained in $P$.

        (BASE CASE) $P$ contains $0$.

        (INDUCTION STEP) For all natural numbers $m$ we have $m \in P \implies m + 1 \in P$. \\
        Proof.
          Let $m$ be a natural number.
          Assume $m \in P$.

          For all natural numbers $n$ if $k > 1$ and $n < m + 1$ then $k^{n} < k^{m + 1}$. \\
          Proof.
            Let $n$ be natural numbers such that $k > 1$ and $n < m + 1$.
            Then $n \leq m$.
            We have $k^{m} \cdot 1 < k^{m} \cdot k$.
            Indeed $k^{m} \neq 0$.
            Case $n = m$.
              Then $k^{n} = k^{m} < k^{m} \cdot k = k^{m + 1}$.
            End.

            Case $n < m$.
              Then $k^{n} < k^{m} < k^{m} \cdot k = k^{m + 1}$.
            End.
          Qed.
        Qed.

        Hence every natural number is contained in $P$.
      End.


      Define $Q = \class{n \in \mathbb{N} | \classtext{for all natural numbers $m$ if $n \geq m$ then $k^{n} \geq k^{m}$ or $k \leq 1$}}$.

      Let us show that every natural number is contained in $Q$.

        (BASE CASE) $0 \in Q$.

        (INDUCTION STEP) For all natural numbers $n$ we have $n \in Q \implies n + 1 \in Q$. \\
        Proof.
          Let $n$ be a natural number.
          Assume $n \in Q$.

          For all natural numbers $m$ if $n + 1 \geq m$ then $k^{n + 1} \geq k^{m}$ or $k \leq 1$. \\
          Proof.
            Let $m$ be natural numbers.
            Assume $n + 1 \geq m$.

            Case $n + 1 = m$. Obvious.

            Case $n + 1 > m$.
              Then $n \geq m$.
              Hence $k^{n} \geq k^{m}$ or $k \leq 1$.

              Case $k \leq 1$. Obvious.

              Case $k^{n} \geq k^{m}$.
                We have $k^{n} \cdot 1 \leq k^{n} \cdot k$.
                Indeed $1 \leq k$ and $k^{n} \neq 0$.
                Hence $k^{m} \leq k^{n} = k^{n} \cdot 1 \leq k^{n} \cdot k = k^{n + 1}$.
              End.
            End.
          Qed.
        Qed.

        Thus every natural number is contained in $Q$.
      End.


      Let $n,m$ be natural numbers.

      Case $n < m$.
        Then $k^{n} < k^{m}$.
        Indeed $n$ and $m$ are contained in $P$.
      End.

      Case $k^{n} < k^{m}$.
        Then it is wrong that $k^{n} \geq k^{m}$ or $k \leq 1$.
        Hence $n \ngeq m$.
        Indeed $n$ and $m$ are contained in $Q$.
        Thus $n < m$.
      End.
    \end{proof}


    \begin{corollary}[NN 02 04 837306]
      Assume $k > 1$.
      Then \[ k^{n} = k^{m} \implies n = m. \]
    \end{corollary}
    \begin{proof}
      Assume $n \neq m$.
      Then $n < m$ or $m < n$.
      If $n < m$ then $k^{n} < k^{m}$.
      If $m < n$ then $k^{m} < k^{n}$.
      Thus $k^{n} \neq k^{m}$.
      Hence the thesis.
    \end{proof}


    \begin{corollary}[NN 02 04 734298]
      Assume $k > 1$.
      Then \[ n \leq m \iff k^{n} \leq k^{m}. \]
    \end{corollary}
  \end{forthel}
\end{document}
