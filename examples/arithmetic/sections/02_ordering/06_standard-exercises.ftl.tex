\documentclass[../../arithmetic.ftl.tex]{subfiles}

\begin{document}
  \begin{comment}
    \begin{forthel}
      % Uncomment for debugging:

      %[prove off][check off]
      %[readtex arithmetic/sections/01_arithmetic/01_peano-axioms.ftl.tex]
      %[readtex arithmetic/sections/01_arithmetic/02_addition.ftl.tex]
      %[readtex arithmetic/sections/01_arithmetic/03_multiplication.ftl.tex]
      %[readtex arithmetic/sections/01_arithmetic/04_exponentiation.ftl.tex]
      %[readtex arithmetic/sections/01_arithmetic/05_factorial.ftl.tex]
      %[readtex arithmetic/sections/02_ordering/01_ordering.ftl.tex]
      %[readtex arithmetic/sections/02_ordering/02_ordering-and-addition.ftl.tex]
      %[readtex arithmetic/sections/02_ordering/03_ordering-and-multiplication.ftl.tex]
      %[readtex arithmetic/sections/02_ordering/04_ordering-and-exponentiation.ftl.tex]
      %[readtex arithmetic/sections/02_ordering/05_induction.ftl.tex]
      %[prove on][check on]
    \end{forthel}
  \end{comment}


  \section{Standard exercises}

  \begin{forthel}
    [readtex \path{arithmetic/sections/01_arithmetic/04_exponentiation.ftl.tex}]
  \end{forthel}

  \begin{forthel}
    [readtex \path{arithmetic/sections/01_arithmetic/05_factorial.ftl.tex}]
  \end{forthel}

  \begin{forthel}
    [readtex \path{arithmetic/sections/02_ordering/04_ordering-and-exponentiation.ftl.tex}]
  \end{forthel}

  \begin{forthel}
    [readtex \path{arithmetic/sections/02_ordering/05_induction.ftl.tex}]
  \end{forthel}

  \begin{forthel}
    Let $k, l, m, n$ denote natural numbers.
  \end{forthel}

  \noindent In this section we will have a look some standard text book
  exercises on induction and prove them within our arithmetic.

  \begin{forthel}
    \begin{proposition}\label{Arithmetic_02_06_276270}
      We have \[ (n + 1)^{2} = (n^{2} + (2 \cdot n)) + 1. \]
    \end{proposition}
    \begin{proof}
      We have
      \[   (n + 1)^{2} \]
      \[ = (n + 1) \cdot (n + 1) \]
      \[ = ((n + 1) \cdot n) + (n + 1) \]
      \[ = ((n \cdot n) + n) + (n + 1) \]
      \[ = (n^{2} + n) + (n + 1) \]
      \[ = ((n^{2} + n) + n) + 1 \]
      \[ = (n^{2} + (n + n)) + 1 \]
      \[ = (n^{2} + (2 \cdot n)) + 1. \]
    \end{proof}


    \begin{proposition}\label{Arithmetic_02_06_671464}
      For all $n$ if $n \geq 3$ then \[ n^{2} > (2 \cdot n) + 1. \]
    \end{proposition}
    \begin{proof}
      Define \[ P = \class{n \in \mathbb{N} | n^{2} > (2 \cdot n) + 1}. \]

      (BASE CASE) $P$ contains $3$.

      (INDUCTION STEP) For all natural numbers $n$ such that $n \geq 3$ we have $n \in P \implies n + 1 \in P$. \\
      Proof.
        Let $n$ be a natural number.
        Suppose $n \geq 3$.
        Assume $n \in P$.

        $(n^{2} + (2 \cdot n)) + 1 > (((2 \cdot n) + 1) + (2 \cdot n)) + 1$.
        Indeed $n^{2} + (2 \cdot n) > ((2 \cdot n) + 1) + (2 \cdot n)$.

        $(2 \cdot (n + n)) + 1 > (2 \cdot (n + 1)) + 1$.
        Indeed $2 \cdot (n + n) > 2 \cdot (n + 1)$.
        Indeed $n + n > n + 1$ and $2 \neq 0$.

        Hence
        \[   (n + 1)^{2} \]
        \[ = (n^{2} + (2 \cdot n)) + 1 \]
        \[ > (((2 \cdot n) + 1) + (2 \cdot n)) + 1 \]
        \[ > ((2 \cdot n) + (2 \cdot n)) + 1 \]
        \[ = (2 \cdot (n + n)) + 1 \]
        \[ > (2 \cdot (n + 1)) + 1. \]

        Thus $(n + 1)^{2} > (2 \cdot (n + 1)) + 1$ (by \ref{Arithmetic_02_01_662806}).
      Qed.

      Therefore $P$ contains every natural number $n$ such that $n \geq 3$ (by \ref{Arithmetic_02_05_497603}).
    \end{proof}


    \begin{proposition}\label{Arithmetic_02_06_205395}
      For all $n$ if $n \geq 5$ then \[ 2^{n} > n^{2}. \]
    \end{proposition}
    \begin{proof}
      Define \[ P = \class{n \in \mathbb{N} | 2^{n} > n^{2}}. \]

      (BASE CASE) $P$ contains $5$.
      Indeed $2^{5} = 2 \cdot (2 \cdot (2 \cdot (2 \cdot 2))) = (5 \cdot 5) + 7 > 5 \cdot 5 = 5^{2}$.
      Indeed $((5 \cdot 5) + 7) > 5 \cdot 5$.

      (INDUCTION STEP) For all natural numbers $n$ such that $n \geq 5$ we have $n \in P \implies n + 1 \in P$. \\
      Proof.
        Let $n$ be a natural number.
        Suppose $n \geq 5$.
        Assume $n \in P$.
        Then $2^{n} > n^{2}$.

        (1) $2^{n} \cdot 2 > n^{2} \cdot 2$ (by \ref{Arithmetic_02_03_496205}).
        Indeed $2 \neq 0$.

        (2) $n^{2} \cdot 2 = n^{2} + n^{2}$.

        (3) $n^{2} + n^{2} > n^{2} + ((2 \cdot n) + 1)$ (by \ref{Arithmetic_02_02_316437}).
        Indeed $n^{2} > (2 \cdot n) + 1$.

        (4) $n^{2} + ((2 \cdot n) + 1) = (n + 1)^{2}$.

        Hence
        \[   2^{n + 1} \]
        \[ = 2^{n} \cdot 2 \]
        \[ > n^{2} \cdot 2 \]
        \[ = n^{2} + n^{2} \]
        \[ > n^{2} + ((2 \cdot n) + 1) \]
        \[ = (n + 1)^{2}. \]

        Thus $2^{n + 1} > (n + 1)^{2}$.
      Qed.

      Therefore $P$ contains every natural number $n$ such that $n \geq 5$ (by \ref{Arithmetic_02_05_497603}).
    \end{proof}


    \begin{proposition}\label{Arithmetic_02_06_527159}
      For all $n$ if $n \geq 2$ then \[ n^{n} > n!. \]
    \end{proposition}
    \begin{proof}
      Define \[ P = \class{n \in \mathbb{N} | n^{n} > n!}. \]

      (BASE CASE) $P$ contains $2$.

      (INDUCTION STEP) For all natural numbers $n$ such that $n \geq 2$ we have $n \in P \implies n + 1 \in P$. \\
      Proof.
        Let $n$ be a natural number.
        Suppose $n \geq 2$.
        Assume $n \in P$.

        (1) $(n + 1)^{n} \cdot (n + 1) > n^{n} \cdot (n + 1)$. \\
        Proof.
          We have $n + 1 > n$ and $n \neq 0$.
          Thus $(n + 1)^{n} > n^{n}$ (by \ref{Arithmetic_02_04_770958}).
          $n + 1$ is nonzero.
          Hence the thesis (by \ref{Arithmetic_02_03_496205}).
        Qed.

        (2) $n^{n} \cdot (n + 1) > n! \cdot (n + 1)$ (by \ref{Arithmetic_02_03_496205}).
        Indeed $n^{n} > n!$ and $n + 1 \neq 0$.

        Hence
        \[   (n + 1)^{n + 1} \]
        \[ = (n + 1)^{n} \cdot (n + 1) \]
        \[ > n^{n} \cdot (n + 1) \]
        \[ > n! \cdot (n + 1) \]
        \[ = (n + 1)!. \]

        Thus $(n + 1)^{n + 1} > (n + 1)!$.
      Qed.

      Therefore $P$ contains every natural number $n$ such that $n \geq 2$ (by \ref{Arithmetic_02_05_497603}).
    \end{proof}


    \begin{proposition}\label{Arithmetic_02_06_493411}
      For all $n$ if $n \geq 4$ then \[ n! > 2^{n}. \]
    \end{proposition}
    \begin{proof}
      Define \[ P = \class{n \in \mathbb{N} | n! > 2^{n}}. \]

      (BASE CASE) $P$ contains $4$. \\
      Proof.
        \[   (4!) \]
        \[ = 4 \cdot (3 \cdot 2) \]
        \[ = 2 \cdot (2 \cdot (3 \cdot 2)) \]
        \[ = 3 \cdot (2 \cdot (2 \cdot 2)) \]
        \[ > 2 \cdot (2 \cdot (2 \cdot 2)) \]
        \[ = 2^{4}. \]
      Qed.

      (INDUCTION STEP) For all natural numbers $n$ such that $n \geq 4$ we have $n \in P \implies n + 1 \in P$. \\
      Proof.
        Let $n$ be a natural number.
        Suppose $n \geq 4$.
        Assume $n \in P$.
        Then $n! > 2^{n}$.

        (1) $0 \neq n + 1 > 2$.
        Indeed $n > 1$.

        (2) $n! \cdot (n + 1) > 2^{n} \cdot (n + 1)$ (by \ref{Arithmetic_02_03_496205}).

        (3) $2^{n} \cdot (n + 1) > 2^{n} \cdot 2$ (by \ref{Arithmetic_02_03_332119}).
        Indeed $2^{n} \neq 0$.

        Hence
        \[   ((n + 1)!) \]
        \[ = n! \cdot (n + 1) \]
        \[ > 2^{n} \cdot (n + 1) \]
        \[ > 2^{n} \cdot 2 \]
        \[ = 2^{n + 1}. \]

        Thus $(n + 1)! > 2^{n + 1}$.
      Qed.

      Therefore $P$ contains every natural number $n$ such that $n \geq 4$ (by \ref{Arithmetic_02_05_497603}).
    \end{proof}
  \end{forthel}
\end{document}
