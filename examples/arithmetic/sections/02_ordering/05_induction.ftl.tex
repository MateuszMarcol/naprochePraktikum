\documentclass[../../arithmetic.ftl.tex]{subfiles}

\begin{document}

  \section{Induction}

  \begin{forthel}
    [readtex \path{arithmetic/sections/02_ordering/01_ordering.ftl.tex}]
  \end{forthel}

  \begin{forthel}
    Let $k, l, m, n$ denote natural numbers.
  \end{forthel}

  When we introduced the Peano axioms we came across an induction axiom which
  gives us a method to prove universal assertions about the natural numbers.
  In this section we will give some reformulations of this induction
  principle.


  \subsection{Least natural numbers}

  As a first example of such a reformulation we will show in this paragraph that
  every collection of natural numbers admits a smallest element.

  \begin{forthel}
    Let $P$ denote a class.

    \begin{definition}
      A least natural number of $P$ is a natural number $n$ such that $n \in P$ and no natural number that is less than $n$ belongs to $P$.
    \end{definition}

    \begin{lemma}
      Let $n,m$ be least natural numbers of $P$.
      Then $n = m$.
    \end{lemma}
    \begin{proof}
      Assume $n \neq m$.
      Then $n < m$ or $m < n$.
      If $n < m$ then $n \notin P$ and if $m < n$ then $m \notin P$.
      Contradiction.
      Therefore $n = m$.
    \end{proof}


    \begin{theorem}\label{Arithmetic_02_05_124228}
      Assume that $P$ contains some natural number.
      Then $P$ has a least natural number.
    \end{theorem}
    \begin{proof}
      Assume the contrary.
      Define \[ Q = \class{n \in \mathbb{N} | \text{n is less than any natural number $m$ such that $m \in P$}}. \]

      Let us show that every natural number belongs to $Q$.

        (BASE CASE) $Q$ contains $0$. \\
        Proof.
          If $P$ contains $0$ then $0$ is the least natural number of $P$.
          Hence $0$ is less than any natural number $m$ such that $m \in P$.
          Therefore $Q$ contains $0$.
        Qed.

        For all natural numbers $n$ we have $n \in Q \implies n + 1 \in Q$. \\
        Proof.
          Let $n$ be a natural number.
          Assume $n \in Q$.
          Then $n$ is less than any natural number $m$ such that $m \in P$.
          Assume that $Q$ does not contain $n + 1$.
          Then we can take a natural number $m$ such that $m \in P$ and $n + 1 \nless m$.
          Hence $n < m \leq n + 1$.
          Thus $m = n + 1$.
          Then $n + 1$ is the least natural number of $P$.
          Contradiction.
        Qed.
      End.

      Then every natural number is less than any natural number $n$ such that $n \in P$.
      Hence there is no natural number $n$ such that $n \in P$.
      Contradiction.
    \end{proof}
  \end{forthel}


  \subsection{Induction via predecessors}

  Next we will see how to merge the base and induction step of a proof by
  induction into a single step. This yields a new induction principle.

  \begin{forthel}
    \begin{theorem}\label{Arithmetic_02_05_167446}
      Assume for all natural numbers $n$ if $P$ contains all predecessors of $n$ then $P$ contains $n$.
      Then $P$ contains every natural number.
    \end{theorem}
    \begin{proof}
      Assume the contrary.
      Take a natural number $n$ such that $P$ does not contain $n$.
      Define $Q = \class{k \in \mathbb{N} | k \notin P}$.
      Then $Q$ contains $n$.
      Thus we can take a least natural number $m$ of $Q$.
      Hence $Q$ does not contain any predecessor of $m$.
      Therefore $P$ contains all predecessors of $m$.
      Thus $P$ contains $m$.
      Contradiction.
    \end{proof}
  \end{forthel}


  \subsection{Induction above a certain number}

  In our induction principle given by the \nth{3} Peano axiom we considered the
  number $0$ as the starting point of an inductive proof. But we can as well
  start at any arbitrary number $k$ to prove that a statement holds for all
  natural numbers from $k$ on.

  \begin{forthel}
    \begin{theorem}\label{Arithmetic_02_05_497603}
      Let $k$ be a natural number such that $k \in P$.
      Suppose that for all natural numbers $n$ such that $n \geq k$ we have $n \in P \implies n + 1 \in P$.
      Then for every natural number $n$ such that $n \geq k$ we have $n \in P$.
    \end{theorem}
    \begin{proof}
      Define $Q = \class{n \in \mathbb{N} | \text{if $n \geq k$ then $n \in P$}}$.

      Let us show that every natural number belongs to $Q$.

        (BASE CASE) We have $0 \in Q$.

        (INDUCTION STEP) For all natural numbers $n$ we have $n \in Q \implies n + 1 \in Q$. \\
        Proof.
          Let $n$ be a natural number.
          Assume $n \in Q$.

          If $n + 1 \geq k$ then $n + 1 \in P$. \\
          Proof.
            Assume $n + 1 \geq k$.

            Case $n < k$.
              Then $n + 1 = k$.
              Hence $n + 1 \in P$.
            End.

            Case $n \geq k$.
              Then $n \in P$.
              Hence $n + 1 \in P$.
            End.
          Qed.

          Thus we have $n + 1 \in Q$.
        Qed.
      End.

      Therefore $Q$ contains every natural number.
      Hence the thesis.
    \end{proof}
  \end{forthel}
\end{document}
