\documentclass[../../arithmetic.ftl.tex]{subfiles}

\begin{document}

  \section{Ordering and multiplication}

  \begin{forthel}
    [readtex \path{arithmetic/sections/01_arithmetic/03_multiplication.ftl.tex}]
  \end{forthel}

  \begin{forthel}
    [readtex \path{arithmetic/sections/02_ordering/02_ordering-and-addition.ftl.tex}]
  \end{forthel}

  \begin{forthel}
    Let $k, l, m, n$ denote natural numbers.
  \end{forthel}


  \begin{forthel}
    \begin{proposition}\label{Arithmetic_02_03_496205}
      Assume $k \neq 0$.
      Then for all $n,m$ we have \[ n < m \iff n \cdot k < m \cdot k. \]
    \end{proposition}
    \begin{proof}
      Define $P = \class{n \in \mathbb{N} | \classtext{for all natural numbers $m$ if $n \cdot k < m \cdot k$ then $n < m$}}$.

      Let us show that every natural number is contained in $P$.
        (BASE CASE) $P$ contains $0$.

        (INDUCTION STEP) For all natural numbers $n$ we have $n \in P \implies n + 1 \in P$.
        Proof.
          Let $n$ be a natural number.
          Assume $n \in P$.

          For all natural numbers $m$ if $(n + 1) \cdot k < m \cdot k$ then $n + 1 < m$. \\
          Proof.
            Let $m$ be a natural number.
            Assume $(n + 1) \cdot k < m \cdot k$.
            Then $(n \cdot k) + k < m \cdot k$.
            Hence $n \cdot k < m \cdot k$.
            Thus $n < m$.
            Then $n + 1 \leq m$.
            If $n + 1 = m$ then $(n + 1) \cdot k = m \cdot k$.
            Hence the thesis.
          Qed.
        Qed.

        Therefore every natural number is contained in $P$.
      End.

      Let $n,m$ be natural numbers.

      Case $n < m$.
        Take a positive natural number $l$ such that $m = n + l$.
        Then $m \cdot k = (n + l) \cdot k = (n \cdot k) + (l \cdot k)$.
        $l \cdot k$ is positive.
        Hence $n \cdot k < m \cdot k$.
      End.

      Case $n \cdot k < m \cdot k$.
        Then $n < m$.
        Indeed $n$ and $m$ are contained in $P$.
      End.
    \end{proof}


    \begin{corollary}\label{Arithmetic_02_03_332119}
      Assume $k \neq 0$.
      Then \[ n < m \iff k \cdot n < k \cdot m. \]
    \end{corollary}
    \begin{proof}
      We have $k \cdot n = n \cdot k$ and $k \cdot m = m \cdot k$.
      Hence $k \cdot n < k \cdot m$ iff $n \cdot k < m \cdot k$.
    \end{proof}


    \begin{proposition}\label{Arithmetic_02_03_319805}
      For all $n,m$ we have \[ n,m > k \implies n \cdot m > k. \]
    \end{proposition}
    \begin{proof}
      Define $P = \class{n \in \mathbb{N} | \classtext{for all natural numbers $m$ if $n,m > k$ then $n \cdot m > k$}}$.

      (BASE CASE) $P$ contains $0$.

      (INDUCTION STEP) For all natural numbers $n$ we have $n \in P \implies n + 1 \in P$. \\
      Proof.
        Let $n$ be a natural number.
        Assume $n \in P$.

        For all natural numbers $m$ if $n + 1, m > k$ then $(n + 1) \cdot m > k$. \\
        Proof.
          Let $m$ be a natural number.
          Assume $n + 1, m > k$.
          Then $(n + 1) \cdot m = (n \cdot m) + m$.
          If $n = 0$ then $(n \cdot m) + m = 0 + m = m > k$.
          If $n \neq 0$ then $(n \cdot m) + m > m > k$.
          Indeed if $n \neq 0 then n \cdot m > 0$.
          Indeed $m > 0$.
          Hence $(n + 1) \cdot m > k$.
        Qed.
      Qed.

      Hence every natural number is contained in $P$.
    \end{proof}


    \begin{corollary}\label{Arithmetic_02_03_496763}
      We have \[ n \leq m \implies k \cdot n \leq k \cdot m. \]
    \end{corollary}

    \begin{corollary}\label{Arithmetic_02_03_575338}
      Assume $k \neq 0$.
      Then \[ k \cdot n \leq k \cdot m \implies n \leq m. \]
    \end{corollary}

    \begin{corollary}\label{Arithmetic_02_03_419208}
      We have \[ n \leq m \implies n \cdot k \leq m \cdot k. \]
    \end{corollary}

    \begin{corollary}\label{Arithmetic_02_03_582576}
      Assume $k \neq 0$.
      Then \[ n \cdot k \leq m \cdot k \implies n \leq m. \]
    \end{corollary}

    \begin{proposition}\label{Arithmetic_02_03_252473}
      Let $k > 1$ and $m > 0$.
      Then $k \cdot m > m$.
    \end{proposition}
    \begin{proof}
      Take a natural number $l$ such that $k = l + 2$.
      Then
      \[      k \cdot m \]
      \[ =    (l + 2) \cdot m \]
      \[ =    (l \cdot m) + (2 \cdot m) \]
      \[ =    (l \cdot m) + (m + m) \]
      \[ =    ((l \cdot m) + m) + m \]
      \[ =    ((l + 1) \cdot m) + m \]
      \[ \geq 1 + m \]
      \[ >    m. \]
    \end{proof}
  \end{forthel}
\end{document}
