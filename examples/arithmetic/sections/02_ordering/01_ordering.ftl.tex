\documentclass[../../arithmetic.ftl.tex]{subfiles}

\begin{document}
  \begin{comment}
    \begin{forthel}
      % Uncomment for debugging:

      %[prove off][check off]
      %[readtex arithmetic/sections/01_arithmetic/01_peano-axioms.ftl.tex]
      %[readtex arithmetic/sections/01_arithmetic/02_addition.ftl.tex]
      %[readtex arithmetic/sections/01_arithmetic/03_multiplication.ftl.tex]
      %[readtex arithmetic/sections/01_arithmetic/04_exponentiation.ftl.tex]
      %[readtex arithmetic/sections/01_arithmetic/05_factorial.ftl.tex]
      %[prove on][check on]
    \end{forthel}
  \end{comment}


  \section{Ordering}

  \begin{forthel}
    [readtex \path{arithmetic/sections/01_arithmetic/02_addition.ftl.tex}]
  \end{forthel}

  \begin{forthel}
    Let $k, l, m, n$ denote natural numbers.
  \end{forthel}

  In this section we will establish an order on the natural
  numbers.


  \subsection{Definitions and immediate consequences}

  A natural number $m$ should be greater than a natural number $n$ if $m$ can
  be reached from $n$ by iteratively applying the successor operation to $n$.
  Or, in other words, if $m$ can be reached by adding a nonzero natural number
  to $m$.

  \begin{forthel}
    \begin{definition}
      $n < m$ iff there exists a nonzero natural number $k$ such that $m = n + k$.
    \end{definition}

    Let $n$ is less than $m$ stand for $n < m$.
    Let $n > m$ stand for $m < n$.
    Let $n$ is greater than $m$ stand for $n > m$.
    Let $n \nless m$ stand for $n$ is not less than $m$.
    Let $n \ngtr m$ stand for $n$ is not greater than $m$.
    Let $n$ is positive stand for $n > 0$.

    \begin{definition}
      $n \leq m$ iff there exists a natural number $k$ such that $m = n + k$.
    \end{definition}

    Let $n$ is less than or equal to $m$ stand for $n \leq m$.
    Let $n \geq m$ stand for $m \leq n$.
    Let $n$ is greater than or equal to $m$ stand for $n \geq m$.
    Let $n \nleq m$ stand for $n$ is not less than or equal to $m$.
    Let $n \ngeq m$ stand for $n$ is not greater than or equal to $m$.

    \begin{proposition}\label{Arithmetic_02_01_206749}
      $n \leq m$ iff $n < m$ or $n = m$.
    \end{proposition}
    \begin{proof}
      Case $n \leq m$.
        Take a natural number $k$ such that $m = n + k$.
        If $k = 0$ then $n = m$. If $k \neq 0$ then $n < m$.
      End.

      Case $n < m$ or $n = m$.
        If $n < m$ then there is a positive natural number $k$ such that $m = n + k$.
        If $n = m$ then $m = n + 0$.
        Thus if $n < m$ then there is a natural number $k$ such that $m = n + k$.
        Hence the thesis.
      End.
    \end{proof}
  \end{forthel}

  This relation enables us to to generalize the notions of direct
  predecessors and successors:

  \begin{forthel}
    \begin{definition}
      A predecessor of $n$ is a natural number that is less than $n$.
    \end{definition}

    \begin{definition}
      A successor of $n$ is a natural number that is greater than $n$.
    \end{definition}
  \end{forthel}

  A direct consequence of the definition of our ordering relation
  is that the terms \enquote{positive} and \enquote{non-zero}
  coincide on the natural numbers.

  \begin{forthel}
    \begin{proposition}\label{Arithmetic_02_01_115117}
      $n$ is positive iff $n$ is nonzero.
    \end{proposition}
    \begin{proof}
      Case $n$ is positive.
        Take a positive natural number $k$ such that $n = 0 + k = k$.
        Then we have $n \neq 0$.
      End.

      Case $n$ is nonzero.
        Take a natural number $k$ such that $n = k + 1$.
        Then $n = 0 + (k + 1)$.
        $k + 1$ is positive.
        Hence $0 < n$.
      End.
    \end{proof}
  \end{forthel}


  \subsection{Basic properties}

  Let us now prove some basic relational properties of the
  ordering.

  \begin{forthel}
    \begin{proposition}\label{Arithmetic_02_01_659871}
      $n \nless n$.
    \end{proposition}
    \begin{proof}
      Assume the contrary.
      Then we can take a positive natural number $k$ such that $n = n + k$.
      Then we have $0 = k$.
      Contradiction.
    \end{proof}


    \begin{proposition}\label{Arithmetic_02_01_679789}
      If $n < m$ then $n \neq m$.
    \end{proposition}
    \begin{proof}
      Assume $n < m$.
      Take a positive natural number $k$ such that $m = n + k$.
      If $n = m$ then $k = 0$.
      Hence $n \neq m$.
    \end{proof}


    \begin{proposition}\label{Arithmetic_02_01_710123}
      If $n \leq m$ and $m \leq n$ then $n = m$.
    \end{proposition}
    \begin{proof}
      Assume $n \leq m$ and $m \leq n$.
      Take natural numbers $k,l$ such that $m = n + k$ and $n = m + l$.
      Then $m = (m + l) + k = m + (l + k)$.
      Hence $l + k = 0$.
      Therefore $l = 0 = k$.
      Then we have the thesis.
    \end{proof}


    \begin{proposition}\label{Arithmetic_02_01_662806}
      If $n < m < k$ then $n < k$.
    \end{proposition}
    \begin{proof}
      Assume $n < m < k$.
      Take a positive natural number $a$ such that $m = n + a$.
      Take a positive natural number $b$ such that $k = m + b$.
      Then $k = (n + a) + b = n + (a + b)$.
      $a + b$ is positive.
      Hence $n < k$.
    \end{proof}


    \begin{proposition}\label{Arithmetic_02_01_394529}
      If $n \leq m \leq k$ then $n \leq k$.
    \end{proposition}
    \begin{proof}
      Case $n = m = k$. Obvious.

      Case $n = m < k$. Obvious.

      Case $n < m = k$. Obvious.

      Case $n < m < k$. Obvious.
    \end{proof}


    \begin{proposition}\label{Arithmetic_02_01_161701}
      If $n \leq m < k$ then $n < k$.
    \end{proposition}
    \begin{proof}
      Assume $n \leq m < k$.
      If $n = m$ then $n < k$.
      If $n < m$ then $n < k$.
    \end{proof}


    \begin{proposition}\label{Arithmetic_02_01_807366}
      If $n < m \leq k$ then $n < k$.
    \end{proposition}
    \begin{proof}
      Assume $n < m \leq k$.
      If $m = k$ then $n < k$.
      If $m < k$ then $n < k$.
    \end{proof}


    \begin{proposition}\label{Arithmetic_02_01_802467}
      If $n < m$ then $n + 1 \leq m$.
    \end{proposition}
    \begin{proof}
      Assume $n < m$.
      Take a positive natural number $k$ such that $m = n + k$.

      Case $k = 1$.
        Then $m = n + 1$.
        Hence $n + 1 \leq m$.
      End.

      Case $k \neq 1$.
        Then we can take a natural number $l$ such that $k = l + 1$.
        Then $m = n + (l + 1) = (n + l) + 1 = (n + 1) + l$.
        $l$ is positive.
        Thus $n + 1 < m$.
      End.
    \end{proof}


    \begin{proposition}\label{Arithmetic_02_01_299356}
      For all $n,m$ we have $n < m$ or $n = m$ or $n > m$.
    \end{proposition}
    \begin{proof}
      Define \[ P = \class{m \in \mathbb{N} | \classtext{for all natural numbers $n$ we have $n < m$ or $n = m$ or $n > m$}}. \]

      (BASE CASE) $P$ contains $0$.

      (INDUCTION STEP) For all natural numbers $m$ we have $m \in P \implies m + 1 \in P$. \\
      Proof.
        Let $m$ be a natural number.
        Assume $m \in P$.

        For all natural numbers $n$ we have $n < m + 1$ or $n = m + 1$ or $n > m + 1$. \\
        Proof.
          Let $n$ be a natural number.

          Case $n < m$. Obvious.

          Case $n = m$. Obvious.

          Case $n > m$.
            Take a positive natural number $k$ such that $n = m + k$.

            Case $k = 1$. Obvious.

            Case$k \neq 1$.
              Take a natural number $l$ such that $n = (m + 1) + l$.
              Hence $n > m + 1$.
              Indeed $l$ is positive.
            End.
          End.
        Qed.
      Qed.

      Thus every natural number is contained in $P$.
    \end{proof}


    \begin{proposition}\label{Arithmetic_02_01_112345}
      $n \nless m$ iff $n \geq m$.
    \end{proposition}
    \begin{proof}
      Case $n \nless m$.
        Then $n = m$ or $n > m$.
        Hence $n \geq m$.
      End.

      Case $n \geq m$.
        Assume $n < m$.
        Then $n \leq m$.
        Hence $n = m$.
        Contradiction.
      End.
    \end{proof}
  \end{forthel}


  \subsection{Ordering and successors}

  We end this section by showing that there are no natural numbers
  between $n$ and $n + 1$.

  \begin{forthel}
    \begin{proposition}\label{Arithmetic_02_01_203608}
      If $n < m \leq n + 1$ then $m = n + 1$.
    \end{proposition}
    \begin{proof}
      Assume $n < m \leq n + 1$.
      Take a positive natural number $k$ such that $m = n + k$.
      Take a natural number $l$ such that $n + 1 = m + l$.
      Then $n + 1 = m + l = (n + k) + l = n + (k + l)$.
      Hence $k + l = 1$.

      We have $l = 0$. \\
      Proof.
        Assume the contrary.
        Then $k,l > 0$.

        Case $k,l = 1$.
          Then $k + l = 2 \neq 1$.
          Contradiction.
        End.

        Case $k = 1 and l \neq 1$.
          Then $l > 1$.
          Hence $k + l > 1 + l > 1$.
          Contradiction.
        End.

        Case $k \neq 1 and l = 1$.
          Then $k > 1$.
          Hence $k + l > k + 1 > 1$.
          Contradiction.
        End.

        Case $k,l \neq 1$.
          Take natural numbers $a,b$ such that $k = a + 1$ and $l = b + 1$.
          Indeed $k,l \neq 0$.
          Hence $k = a + 1$ and $l = b + 1$.
          Thus $k,l > 1$. Indeed $a,b$ are positive.
        End.
      Qed.

      Then we have $n + 1 = m + l = m + 0 = m$.
    \end{proof}


    \begin{proposition}\label{Arithmetic_02_01_126729}
      If $n \leq m < n + 1$ then $n = m$.
    \end{proposition}
    \begin{proof}
      Assume $n \leq m < n + 1$.

      Case $n = m$. Obvious.

      Case $n < m$.
        Then $n < m \leq n + 1$.
        Hence $n = m$.
      End.
    \end{proof}


    \begin{corollary}\label{Arithmetic_02_01_666910}
      There is no natural number $m$ such that $n < m < n + 1$.
    \end{corollary}
    \begin{proof}
      Assume the contrary.
      Take a natural number $m$ such that $n < m < n + 1$.
      Then $n < m \leq n + 1$ and $n \leq m < n + 1$.
      Hence $m = n + 1$ and $m = n$ (by \ref{Arithmetic_02_01_203608}, \ref{Arithmetic_02_01_126729}).
      Hence $n = n + 1$.
      Contradiction.
    \end{proof}


    \begin{proposition}\label{Arithmetic_02_01_408119}
      $n + 1 \geq 1$.
    \end{proposition}
    \begin{proof}
      Case $n = 0$. Obvious.

      Case $n \neq 0$.
        Then $n > 0$.
        Hence $n + 1 > 0 + 1 = 1$.
      End.
    \end{proof}
  \end{forthel}
\end{document}
