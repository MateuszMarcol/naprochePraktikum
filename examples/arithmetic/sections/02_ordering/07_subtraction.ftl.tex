\documentclass[../../arithmetic.ftl.tex]{subfiles}

\begin{document}

  \section{Subtraction}

  \begin{forthel}
    [readtex \path{arithmetic/sections/01_arithmetic/03_multiplication.ftl.tex}]
  \end{forthel}

  \begin{forthel}
    [readtex \path{arithmetic/sections/02_ordering/01_ordering.ftl.tex}]
  \end{forthel}

  \begin{forthel}
    Let $k, l, m, n$ denote natural numbers.
  \end{forthel}


  \begin{forthel}
    \begin{definition}
      Let $n \geq m$.
      $n - m$ is the natural number $k$ such that $n = m + k$.
    \end{definition}

    Let the difference of $n$ and $m$ stand for $n - m$.

    \begin{proposition}\label{Arithmetic_02_07_297505}
      Let $n \geq m$.
      Then $n - m = 0$ iff $n = m$.
    \end{proposition}
    \begin{proof}
      Case $n - m = 0$.
        Then $n = (n - m) + m = 0 + m = m$.
      End.

      Case $n = m$.
        We have $(n - m) + m = n = m = 0 + m$.
        Hence $n - m = 0$.
      End.
    \end{proof}

    \begin{corollary}\label{Arithmetic_02_07_239083}
      $n - n = 0$.
    \end{corollary}

    \begin{proposition}\label{Arithmetic_02_07_151829}
      $n - 0 = n$.
    \end{proposition}
    \begin{proof}
      We have $n = (n - 0) + 0 = n - 0$.
    \end{proof}

    \begin{proposition}\label{Arithmetic_02_07_236650}
      Let $n \geq m$.
      Then $n - m \leq n$.
    \end{proposition}
    \begin{proof}
      We have $(n - m) + m = n$.
      Hence $n - m \leq n$.
    \end{proof}

    \begin{proposition}\label{Arithmetic_02_07_554898}
      Let $n$ be nonzero.
      $n - 1$ is the direct predecessor of $n$.
    \end{proposition}
    \begin{proof}
      We have $(n - 1) + 1 = n = \pred(n) + 1$.
      Hence $n - 1 = \pred(n)$.
    \end{proof}

    \begin{proposition}\label{Arithmetic_02_07_654395}
      Let $n > m$.
      Assume $m \neq 0$.
      Then $n - m < n$.
    \end{proposition}
    \begin{proof}
      We have $(n - m) + m = n$.
      Assume $n - m = n$.
      Then $n + m = (n - m) + m = n = n + 0$.
      Hence $m = 0$.
      Contradiction.
    \end{proof}

    \begin{proposition}\label{Arithmetic_02_07_588928}
      Assume $n \geq m$.
      Then \[(n - m) + k = (n + k) - m. \]
    \end{proposition}
    \begin{proof}
      Assume $n \geq m$.
      We have
      \[   ((n - m) + k) + m \]
      \[ = ((n - m) + m) + k \]
      \[ = n + k \]
      \[ = ((n + k) - m) + m. \]

      Hence $(n - m) + k = (n + k) - m$.
    \end{proof}

    \begin{proposition}\label{Arithmetic_02_07_251141}
      Assume $n \geq m + k$.
      Then \[ (n - m) - k = n - (m + k). \]
    \end{proposition}
    \begin{proof}
      We have
      \[   ((n - m) - k) + (m + k) \]
      \[ = (((n - m) - k) + k) + m \]
      \[ = (n - m) + m \]
      \[ = n \]
      \[ = (n - (m + k)) + (m + k). \]

      Hence $(n - m) - k = n - (m + k)$.
    \end{proof}

    \begin{proposition}\label{Arithmetic_02_07_446298}
      Let $n \geq m$.
      Then \[ (n - m) \cdot k = (n \cdot k) - (m \cdot k). \]
    \end{proposition}
    \begin{proof}
      We have
      \[   ((n - m) \cdot k) + (m \cdot k) \]
      \[ = ((n - m) + m) \cdot k \]
      \[ = n \cdot k \]
      \[ = ((n \cdot k) - (m \cdot k)) + (m \cdot k). \]

      Hence $(n - m) \cdot k = (n \cdot k) - (m \cdot k)$.
    \end{proof}
  \end{forthel}
\end{document}
