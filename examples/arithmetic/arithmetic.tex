\documentclass{article}

\usepackage[english]{babel}
\usepackage{amssymb}
\usepackage{xurl}
\usepackage{hyperref}
\usepackage{subfiles}
\usepackage[super]{nth}
\usepackage{csquotes}
\usepackage[arithmetic]{../../lib/tex/naproche}

\hypersetup{
  colorlinks=true,
  linkcolor=blue,
  urlcolor=blue
}

\title{Arithmetic}
\author{Marcel Schütz}
\date{2021}

\begin{document}
  \maketitle

  \begin{abstract}
    This is a formalization of Peano Arithmetic with addition, multiplication,
    exponentiation and factorial.
    These operations are introduced axiomatically and are accompanied with
    detailed proofs of their common computation laws.
    Moreover, the standard ordering on the natural numbers is given, together
    with proofs of its behaviour with respect to the mentioned operations.
    Furthermore, the notion of divisibility is introduced which finally leads
    to some results of basic number theory.

    This text can be seen as a collection of basic results from undergraduate
    mathematics or serve as a foundation for more sophisticated formalizations.
  \end{abstract}

  \tableofcontents

  \newpage
  \part{Arithmetic}

  \subfile{sections/01_arithmetic/01_peano-axioms.ftl.tex}

  \subfile{sections/01_arithmetic/02_addition.ftl.tex}

  \subfile{sections/01_arithmetic/03_multiplication.ftl.tex}

  \subfile{sections/01_arithmetic/04_exponentiation.ftl.tex}

  \subfile{sections/01_arithmetic/05_factorial.ftl.tex}


  \newpage
  \part{Ordering}

  \subfile{sections/02_ordering/01_ordering.ftl.tex}

  \subfile{sections/02_ordering/02_ordering-and-addition.ftl.tex}

  \subfile{sections/02_ordering/03_ordering-and-multiplication.ftl.tex}

  \subfile{sections/02_ordering/04_ordering-and-exponentiation.ftl.tex}

  \subfile{sections/02_ordering/05_induction.ftl.tex}

  \subfile{sections/02_ordering/06_standard-exercises.ftl.tex}

  \subfile{sections/02_ordering/07_subtraction.ftl.tex}


  \newpage
  \part{Divisibility}

  \subfile{sections/03_divisibility/01_divisibility.ftl.tex}

  \subfile{sections/03_divisibility/02_euclidean-division.ftl.tex}

  \subfile{sections/03_divisibility/03_modular-arithmetic.ftl.tex}

  \subfile{sections/03_divisibility/04_primes.ftl.tex}

  \subfile{sections/03_divisibility/05_division.ftl.tex}

  \subfile{sections/03_divisibility/06_binomial-coefficient.ftl.tex}

  \subfile{sections/03_divisibility/07_even-and-odd-numbers.ftl.tex}
\end{document}
