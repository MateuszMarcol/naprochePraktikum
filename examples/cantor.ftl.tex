\documentclass{article}

\usepackage[utf8]{inputenc}
\usepackage[english]{babel}
\usepackage{../lib/tex/naproche}

\title{Cantor's theorem}
\author{}
\date{}

\begin{document}
  \pagenumbering{gobble}

  \maketitle

  In this document we prove Cantor's theorem:

  \begin{quotedtheorem}
    There is no surjection defined on a set $M$ that surjects onto the powerset of $M$.
  \end{quotedtheorem}

  \begin{forthel}
    [synonym subset/-s]
    [synonym surject/-s]

    Let $M$ denote a set. Let $f$ denote a function.

    \begin{axiom}
        $M$ is setsized.
    \end{axiom}

    \begin{axiom}
        Let $x$ be an element of $M$. Then $x$ is setsized.
    \end{axiom}

    Let the value of $f$ at $x$ stand for $f(x)$.
    Let $f$ is defined on $M$ stand for $\dom(f) = M$.
    Let the domain of $f$ stand for $\dom(f)$.


    \begin{axiom}
      The value of $f$ at any element of the domain of $f$ is a set.
    \end{axiom}

    \begin{definition}[Subset]
      A subset of $M$ is a set $N$ such that every element of $N$ is an element of $M$.
    \end{definition}

    \begin{definition}
      The powerset of $M$ is the class of subsets of $M$.
    \end{definition}

    \begin{axiom}
      The powerset of $M$ is a set.
    \end{axiom}

    \begin{definition}
      $f$ surjects onto $M$ iff every element of $M$ is equal to the value of $f$ at some element of the domain of $f$.
    \end{definition}

    \begin{theorem}[Cantor]
      No function that is defined on $M$ surjects onto the powerset of $M$.
    \end{theorem}
    \begin{proof}
      Proof by contradiction. Assume the contrary.
      Take a function $f$ that is defined on $M$ and surjects onto the powerset of $M$.
      Define $N = \{ x "in" M \mid x \notin f(x) \}$.
      Then for all sets $x$ we have $x\in N$ if and only if $x\notin f(x) = N$.
      Contradiction.
    \end{proof}

  \end{forthel}

\end{document}
