\documentclass{article}

\usepackage[utf8]{inputenc}
\usepackage[english]{babel}
\usepackage{../lib/tex/naproche}

\title{Knaster-Tarski fixed point theorem}
\author{Andrei Paskevich et al.}
\date{}

\begin{document}
  \pagenumbering{gobble}

  \maketitle

  The Knaster-Tarski theorem is a result from lattice theory about fixed points of monotone functions.
  Bronis\l aw Knaster and Alfred Tarski established it in 1928 for the special case of power set lattices
  (Knaster, B. with Tarski, A.: Un th\' eor\`eme sur les fonctions d'ensembles, Ann. Soc. Polon. Math.
  vol. 6 (1928), 133-134). This more general result was stated by Tarski in 1955 (Tarski, A.: A lattice-theoretical
  fixpoint theorem and its applications, Pacific Journal of Mathematics vol. 5 (1955), no. 2, 285-309).

  The formulation and proof of the theorem require a minimal axiomatic setup on posets and functions.

  \section{ForTheL Setup}

  \begin{forthel}
    [synonym set/-s] [synonym subset/-s] [synonym element/-s]

    [synonym belong/-s] [synonym bound/-s] [synonym supremum/suprema]

    [synonym infimum/infima] [synonym lattice/-s] [synonym function/-s]

    [synonym point/-s]

    [read macros.ftl]
  \end{forthel}

  \section{Sets and Elements}

  \begin{forthel}

    \begin{signature}
      An element is a notion.
    \end{signature}

    Let $S,T$ denote sets.
    Let $x,y,z,u,v,w$ denote elements.

    \begin{axiom}
      $S$ is setsized.
    \end{axiom}

    \begin{axiom}
      $x$ is setsized.
    \end{axiom}

    \begin{axiom}
      Every element of $S$ is an element.
    \end{axiom}

    \begin{definition}[DefEmpty]
      $S$ is empty iff $S$ has no elements.
    \end{definition}

    \begin{definition}[DefSub]
      A subset of $S$ is a class $T$ such that every $x \in T$ belongs to $S$.
    \end{definition}

    \begin{axiom}
      Let C be a subset of T. Then C is a set.
    \end{axiom}
  \end{forthel}

  \section{Partial Order, Supremum and Infimum}

  \begin{forthel}

    \begin{signature}[LessRel]
      $x \leq y$ is an atom.
    \end{signature}

    \begin{axiom}[ARefl]
      $x \leq x$.
    \end{axiom}

    \begin{axiom}[ASymm]
      $x \leq y \leq x \implies x = y$.
    \end{axiom}

    \begin{axiom}[ATrans]
      $x \leq y \leq z \implies x \leq z$.
    \end{axiom}

    \begin{definition}[DefLB]
      Let $S$ be a subset of $T$. A lower bound of $S$ in $T$ is an element $u$ of $T$ such that
      $u \leq x$ for every $x \in S$.
    \end{definition}

    \begin{definition}[DefUB]
      Let $S$ be a subset of $T$. An upper bound of $S$ in $T$ is an element $u$ of $T$ such that
      $x \leq u$ for every $x \in S$.
    \end{definition}

    \begin{definition}[DefInf]
      Let $S$ be a subset of $T$. An infimum of $S$ in $T$ is an element $u$ of $T$ such that
      $u$ is a lower bound of $S$ in $T$ and for every lower bound $v$ of $S$ in $T$ we have $v \leq u$.
    \end{definition}

    \begin{definition}[DefSup]
      Let $S$ be a subset of $T$. A supremum of $S$ in $T$ is an element $u$ of $T$ such that
      $u$ is a upper bound of $S$ in $T$ and for every upper bound $v$ of $S$ in $T$ we have $u \leq v$.
    \end{definition}

    \begin{lemma}[SupUn]
      Let $S$ be a subset of $T$. Let $u,v$ be suprema of $S$ in $T$. Then $u = v$.
    \end{lemma}

    \begin{lemma}[InfUn]
      Let $S$ be a subset of $T$. Let $u,v$ be infima of $S$ in $T$. Then $u = v$.
    \end{lemma}

    \begin{definition}[DefCLat]
      A complete lattice is a set $S$ such that every subset of $S$ has an infimum in $S$ and a supremum in $S$.
    \end{definition}
  \end{forthel}

  \section{Functions}

  \begin{forthel}

    Let $f$ stand for a function.

    \begin{axiom}
      $Dom(f)$ is a set.
    \end{axiom}

    \begin{signature}[RanSort]
      $Ran(f)$ is a set.
    \end{signature}

    \begin{definition}[DefDom]
      $f$ is on $S$ iff $Dom(f) = Ran(f) = S$.
    \end{definition}

    \begin{axiom}[ImgSort]
      Let $x$ belong to $Dom(f)$. $f(x)$ is an element of $Ran(f)$.
    \end{axiom}

    \begin{definition}[DefFix]
      A fixed point of $f$ is an element $x$ of $Dom(f)$ such that $f(x) = x$.
    \end{definition}

    \begin{definition}[DefMonot]
      $f$ is monotone iff for all $x,y \in Dom(f)$ $x \leq y \implies f(x) \leq f(y)$.
    \end{definition}
  \end{forthel}

  \section{Knaster-Tarski theorem}

  \begin{forthel}

    \begin{theorem}[KnasterTarski]
      Let $U$ be a complete lattice and $f$ be a monotone function on $U$.
      Let $S$ be the class of fixed points of $f$. Then $S$ is a complete lattice.
    \end{theorem}
    \begin{proof}
      Let $T$ be a subset of $S$.
      Let us show that $T$ has a supremum in $S$. \\
        Define
        \[
          P = \{x \in U \mid f(x) \leq x "and" x "is an upper bound of" T "in" U\} .
        \]
        Take an infimum $p$ of $P$ in $U$. $f(p)$ is a lower bound of $P$ in $U$ and an upper bound of $T$ in $U$.
        Hence $p$ is a fixed point of $f$ and a supremum of $T$ in $S$.
      end. \\
      Let us show that $T$ has an infimum in $S$. \\
        Define
        \[
          Q = \{x \in U \mid x \leq f(x) "and" x "is a lower bound of" T "in" U\} .
        \]
        Take a supremum $q$ of $Q$ in $U$. $f(q)$ is an upper bound of $Q$ in $U$ and a lower bound of $T$ in $U$.
        Hence $q$ is a fixed point of $f$ and an infimum of $T$ in $S$.
      end.
    \end{proof}
  \end{forthel}

\end{document}
