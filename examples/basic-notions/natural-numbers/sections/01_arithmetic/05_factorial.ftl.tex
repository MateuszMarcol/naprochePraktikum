\documentclass[../../natural-numbers.ftl.tex]{subfiles}

\begin{document}

  \typeout{
    \begin{forthel}
      [readtex basic-notions/natural-numbers/sections/01_arithmetic/03_multiplication.ftl.tex]

      [checkconsistency off]

      Let $k, l, m, n$ denote natural numbers.
    \end{forthel}
  }


  \section{Factorial}

  \begin{forthel}
    \begin{signature}
      $n!$ is a natural number.
    \end{signature}

    \begin{axiom}[1st factorial axiom]
      $(0!) = 1$.
    \end{axiom}

    \begin{axiom}[2nd factorial axiom]
      $((n + 1)!) = n! \cdot (n + 1)$.
    \end{axiom}

    % Note that we cannot write "x! = y" since this can either be understood as
    % "x! = y" ("x factorial is equal to y") or as "x != y" ("x is not equal to
    % y").

    \begin{proposition}
      $n!$ is nonzero for any natural number $n$.
    \end{proposition}
    \begin{proof}
      Define $P = \class{n \in \mathbb{N} | n! \neq 0}$.

      (BASE CASE) $P$ contains $0$.
      Indeed $(0!) = 1 \neq 0$.

      (INDUCTION STEP) For every natural number $n$ we have $n \in P \implies n + 1 \in P$. \\
      Proof.
        Let $n$ be a natural number.
        Assume $n \in P$.
        We have $((n + 1)!) = (n + 1) \cdot (n!)$.
        $n + 1$ and $n!$ are nonzero.
        Hence $(n + 1)!$ is nonzero.
      Qed.

      Thus $P$ contains every natural number.
    \end{proof}
  \end{forthel}

\end{document}
