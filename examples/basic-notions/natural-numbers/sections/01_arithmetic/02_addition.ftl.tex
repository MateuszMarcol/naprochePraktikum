\documentclass[../../natural-numbers.ftl.tex]{subfiles}


\begin{document}

  \section{Addition}

  \begin{forthel}
    [readtex \path{basic-notions/natural-numbers/sections/01_arithmetic/01_peano-axioms.ftl.tex}]
  \end{forthel}

  \begin{forthel}
    Let $k, l, m, n$ denote natural numbers.
  \end{forthel}


  \subsection{Axioms}

  \begin{forthel}
    \begin{signature}
      $n + m$ is a natural number.
    \end{signature}

    Let the sum of $n$ and $m$ stand for $n + m$.

    \begin{axiom}[1st addition axiom]
      $n + 0 = n$.
    \end{axiom}

    \begin{axiom}[2nd addition axiom]
      $n + \succ(m) = \succ(n + m)$.
    \end{axiom}
  \end{forthel}


  \subsection{Immediate consequences}

  \begin{forthel}
    \begin{lemma}
      $\succ(n) = n + 1$.
    \end{lemma}

    \begin{corollary}[1st Peano axiom]
      If $n + 1 = m + 1$ then $n = m$.
    \end{corollary}

    \begin{corollary}[2nd Peano axiom]
      For no n we have $n + 1 = 0$.
    \end{corollary}

    \begin{corollary}[3rd Peano axiom]
      Let $P$ be a class.
      Assume $0 \in P$ and for all $n$: $n \in P \implies n + 1 \in P$.
      Then every natural number is an element of $P$.
    \end{corollary}

    \begin{corollary}[2nd addition axiom]
      $n + (m + 1) = (n + m) + 1$.
    \end{corollary}
  \end{forthel}


  \subsection{Computation laws}

  \begin{forthel}
    \begin{proposition}[NN 01 02 468785]
      For all $n,m,k$ we have \[ n + (m + k) = (n + m) + k. \]
    \end{proposition}
    \begin{proof}
      Define $P = \class{ k \in \mathbb{N} | \text{for all $n,m$: $n + (m + k) = (n + m) + k$}}$.

      (BASE CASE) $0$ is contained in $P$.
      Indeed $n + (m + 0) = n + m = (n + m) + 0$ for all natural numbers $n,m$.

      (INDUCTION STEP) For all $k$ we have $k \in P \implies k + 1 \in P$. \\
      Proof.
        Let $k$ be a natural number.
         Assume $k \in P$.

        Let us show that $n + (m + (k + 1)) = (n + m) + (k + 1)$ for all natural numbers $n,m$. \\
          Let $n,m$ be natural numbers.
          Then $n + m$ is a natural number.

          \[   n + (m + (k + 1)) \]
          \[ = n + ((m + k) + 1) \]  % 2nd addition axiom
          \[ = (n + (m + k)) + 1 \]  % 2nd addition axiom
          \[ = ((n + m) + k) + 1 \]  % Induction hypothesis
          \[ = (n + m) + (k + 1). \] % by 2nd addition axiom

          Hence the thesis.
        End.

        Therefore we have $k + 1 \in P$.
      Qed.

      Thus every natural number is an element of $P$.
    \end{proof}


    \begin{proposition}[NN 01 02 273100]
      For all $n,m$ we have \[ n + m = m + n. \]
    \end{proposition}
    \begin{proof}
      Define $P = \class{m \in \mathbb{N} | \text{$n + m = m + n$ for all natural numbers $n$}}$.

      (BASE CASE 1) $0$ is an element of $P$. \\
      Proof.
        Define $Q = \class{n \in \mathbb{N} | n + 0 = 0 + n}$.

        $0$ belongs to $Q$.

        For all $n$ we have $n \in Q \implies n + 1 \in Q$. \\
        Proof.
          Let $n$ be a natural number.
          Assume $n \in Q$.

          \[   (n + 1) + 0 \]
          \[ = n + 1 \]        % 1st addition axiom
          \[ = (n + 0) + 1 \]  % 1st addition axiom
          \[ = (0 + n) + 1 \]  % Induction hypothesis
          \[ = 0 + (n + 1). \] % 2nd addition axiom
        Qed.

        Thus every natural number belongs to $Q$.
        Therefore $0$ is an element of $P$.
      Qed.

      (BASE CASE 2) $1$ is contained in $P$. \\
      Proof.
        Define $Q = \class{n \in \mathbb{N} | n + 1 = 1 + n}$.

        $0$ is an element of $Q$.

        For all natural numbers $n$ we have $n \in Q \implies n + 1 \in Q$. \\
        Proof.
          Let $n$ be a natural number.
          Assume that $n$ is contained in $Q$.

          \[   (n + 1) + 1 \]
          \[ = (1 + n) + 1 \]  % Induction hypothesis
          \[ = 1 + (n + 1). \] % 2nd addition axiom
        Qed.

        Thus every natural number belongs to $Q$.
        Therefore $1$ is an element of $P$.
      Qed.

      (INDUCTION STEP) For all natural numbers $n$ we have $n \in P \implies n + 1 \in P$. \\
      Proof.
        Let $n$ be an natural number.
        Assume $n \in P$.

        $(n + 1) + m = m + (n + 1)$ for all natural numbers $m$. \\
        Proof.
          Let $m$ be a natural number.

          \[   (n + 1) + m \]
          \[ = n + (1 + m) \]  % Associativity of addition
          \[ = (1 + m) + n \]  % Induction hypothesis
          \[ = (m + 1) + n \]  % BASE CASE 2
          \[ = m + (n + 1). \] % Associativity of addition and by BASE CASE 2
        Qed.
      Qed.

      Hence every natural number is an element of $P$.
    \end{proof}


    \begin{proposition}[NN 01 02 882987]
      For all natural numbers $n,m,k$ we have \[ n + k = m + k \implies n = m. \]
    \end{proposition}
    \begin{proof}
      Define $P = \class{k \in \mathbb{N} | \classtext{for all natural numbers $n,m$ if $n + k = m + k$ then $n = m$}}$.

      (BASE CASE) $0$ is an element of $P$.

      (INDUCTION STEP) For all $k$ we have $k \in P \implies k + 1 \in P$. \\
      Proof.
        Let $k$ be a natural number.
        Assume $k \in P$.

        For all natural numbers $n,m$ we have $n + (k + 1) = m + (k + 1) \implies n = m$. \\
        Proof.
          Let $n,m$ be natural numbers.
          Assume $n + (k + 1) = m + (k + 1)$.
          Then $(n + k) + 1 = (m + k) + 1$.
          Hence $n + k = m + k$.
          Thus $n = m$.
        Qed.

        Hence the thesis.
      Qed.

      Therefore every natural number is an element of $P$.
    \end{proof}


    \begin{corollary}[NN 01 02 402018]
      For all $n,m,k$ we have \[ k + n = k + m \implies n = m. \]
    \end{corollary}
    \begin{proof}
      Let $n,m,k$ be natural numbers.
      Assume $k + n = k + m$.
      We have $k + n = n + k$ and $k + m = m + k$.
      Hence $n + k = m + k$.
      Thus $n = m$.
    \end{proof}
  \end{forthel}
\end{document}
