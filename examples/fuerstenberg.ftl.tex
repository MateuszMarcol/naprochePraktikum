\documentclass{article}

\usepackage[utf8]{inputenc}
\usepackage[english]{babel}
\usepackage{../lib/tex/naproche}

\title{Fürstenberg's proof of the infinitude of primes}
\author{Andrei Paskevich}
\date{}


\begin{document}
  \pagenumbering{gobble}

  \maketitle

  \section{Integers}

  \begin{forthel}
    [unfoldlow on]
    [synonym integer/-s]

    \begin{signature}[Integers]
      An integer is a notion.
    \end{signature}

    Let $a,b,c,d,i,j,k,l,m,n$ stand for integers.

    \begin{axiom}
      $a$ is setsized.
    \end{axiom}

    \begin{signature}[IntZero]
      $0$ is an integer.
    \end{signature}

    \begin{signature}[IntOne]
      $1$ is an integer.
    \end{signature}

    \begin{signature}[IntNeg]
      $-a$ is an integer.
    \end{signature}

    \begin{signature}[IntPlus]
      $a + b$ is an integer.
    \end{signature}

    \begin{signature}[IntMult]
      $a * b$ is an integer.
    \end{signature}

    Let $a - b$ stand for $a + (-b)$.

    \begin{axiom}[AddAsso]
      $a + (b + c) = (a + b) + c$.
    \end{axiom}

    \begin{axiom}[AddComm]
      $a + b = b + a$.
    \end{axiom}

    \begin{axiom}[AddZero]
      $a + 0 = a = 0 + a$.
    \end{axiom}

    \begin{axiom}[AddNeg]
      $a - a = 0 = -a + a$.
    \end{axiom}

    \begin{axiom}[MulAsso]
      $a * (b * c) = (a * b) * c$.
    \end{axiom}

    \begin{axiom}[MulComm]
      $a * b = b * a$.
    \end{axiom}

    \begin{axiom}[MulOne]
      $a * 1 = a = 1 * a$.
    \end{axiom}

    \begin{axiom}[Distrib]
      $a * (b + c) = (a*b) + (a*c)$ and$ (a + b) * c = (a*c) + (b*c)$.
    \end{axiom}

    \begin{lemma}[MulZero]
      $a * 0 = 0 = 0 * a$.
    \end{lemma}

    \begin{lemma}[MulMinOne]
      $-1 * a = -a = a * -1$.
    \end{lemma}
    \begin{proof}
      $(-1 * a) + a = 0$.
    \end{proof}

    \begin{axiom}[ZeroDiv]
      $a != 0 /\ b != 0 => a * b != 0$.
    \end{axiom}

    \begin{lemma}
      $--a$ is an integer.
    \end{lemma}

    Let $a$ is nonzero stand for $a != 0$.
    Let $p,q$ stand for nonzero integers.

    [synonym divisor/-s] [synonym divide/-s]

    \begin{definition}[Divisor]
      A divisor of $b$ is a nonzero integer $a$ such that for some $n$ $(a * n = b)$.
    \end{definition}

    Let $a$ divides $b$ stand for $a$ is a divisor of $b$.
    Let $a | b$ stand for $a$ is a divisor of $b$.

    \begin{definition}[EquMod]
      $a = b (mod q)$ iff $q | a-b$.
    \end{definition}

    \begin{lemma}[EquModRef]
      $a = a (mod q)$.
    \end{lemma}

    \begin{lemma}[EquModSym]
      $a = b (mod q) => b = a (mod q)$.
    \end{lemma}
    \begin{proof}
      Assume that $a = b (mod q)$.

      (1) Take $n$ such that $q * n = a - b$.

      $q * -n .= (-1) * (q * n)$ (by MulMinOne, MulAsso,MulComm) $.= (-1) * (a - b)$ (by 1).
    \end{proof}

    \begin{lemma}[EquModTrn]
      $a = b (mod q) /\ b = c (mod q) => a = c (mod q)$.
    \end{lemma}
    \begin{proof}
      Assume that $a = b (mod q) /\ b = c (mod q)$. Take $n$ such that $q * n = a - b$. Take $m$ such that $q * m = b - c$. We have $q * (n + m) = a - c$.
    \end{proof}

    \begin{lemma}[EquModMul]
      $a = b (mod p * q) => a = b (mod p) /\ a = b (mod q)$.
    \end{lemma}
    \begin{proof}
      Assume that $a = b (mod p * q)$. Take $m$ such that $(p * q) * m = a - b$. We have $p * (q * m) = a - b = q * (p * m)$.
    \end{proof}

    \begin{signature}[Prime]
      $a$ is prime is an atom.
    \end{signature}

    Let a prime stand for a prime nonzero integer.

    \begin{axiom}[PrimeDivisor]
      $n$ has a prime divisor iff $n != 1 /\ n != -1$.
    \end{axiom}
  \end{forthel}


  \section{Generic sets}

  \begin{forthel}
    [synonym belong/-s] [synonym subset/-s]
    [read ZFC.ftl]

    Let $S,T$ stand for sets.

    Let $x$ belongs to $S$ stand for $x$ is an element of $S$.
    Let $x << S$ stand for $x$ is an element of $S$.

    \begin{definition}[Subset]
      A subset of $S$ is a set $T$ such that every element of $T$ belongs to $S$.
    \end{definition}

    Let $S [= T$ stand for $S$ is a subset of $T$.

    \begin{signature}[FinSet]
      $S$ is finite is an atom.
    \end{signature}

    Let $x$ is infinite stand for $x$ is not finite.
  \end{forthel}


  \section{Sets of integers}

  \begin{forthel}
    \begin{definition}
      $INT$ is the class of integers.
    \end{definition}

    \begin{axiom}
      $INT$ is a set.
    \end{axiom}

    Let $A,B,C,D$ stand for subsets of $INT$.

    \begin{definition}[Union]
      $A \cup B = \{ "integer" x | x << A \vee x << B \}$.
    \end{definition}

    \begin{definition}[Intersection]
      $A \cap B = \{ "integer" x | x << A \wedge x << B \}$.
    \end{definition}

    \begin{definition}[IntegerSets]
      A family of integer sets is a set $S$ such that every element of $S$ is a subset of $INT$.
    \end{definition}

    \begin{definition}[UnionSet]
      Let $S$ be a family of integer sets.
      $\cup S = \{ "integer" x | x "belongs to some element of" S \}$.
    \end{definition}

    \begin{lemma}
      Let $S$ be a family of integer sets. $\cup S$ is a subset of $INT$.
    \end{lemma}

    \begin{definition}[Complement]
      $~ A = { "integer" x | x "does not belong to" A }$.
    \end{definition}

    \begin{lemma}
      $~ A$ is a subset of $INT$.
    \end{lemma}
   \end{forthel}


  \section{Introducing topology}

  \begin{forthel}

    \begin{definition}[ArSeq]
      $ArSeq(a,q) = { "integer" b | b = a (mod q) }$.
    \end{definition}

    \begin{lemma}
      $ArSeq(a,q)$ is a set.
    \end{lemma}

    \begin{definition}[Open]
      $A$ is open iff for any $a << A$ there exists $q$ such that $ArSeq(a,q) [= A$.
    \end{definition}

    \begin{definition}[Closed]
      $A$ is closed iff $~A$ is open.
    \end{definition}

    \begin{definition}[OpenIntegerSets]
      An open family is a family of integer sets $S$ such that every element of $S$ is open.
    \end{definition}

    \begin{lemma}[UnionOpen]
      Let $S$ be an open family. $\cup S$ is open.
    \end{lemma}
    \begin{proof}
      Let $x << \cup S$. Take a set $M$ such that ($M$ is an element of $S$ and $x << M$). Take $q$ such that $ArSeq(x,q) [= M$. Then $ArSeq(x,q) [= \cup S$.
    \end{proof}

    \begin{lemma}[InterOpen]
      Let $A,B$ be open subsets of $INT$. Then $A \cap B$ is a subset of $INT$ and $A \cap B$ is open.
    \end{lemma}
    \begin{proof}
      $A \cap B$ is a subset of $INT$. Let $x << A \cap B$. Then $x$ is an integer. Take $q$ such that $ArSeq(x,q) [= A$. Take $p$ such that $ArSeq(x,p) [= B$.

      Let us show that $p*q$ is a nonzero integer and $ArSeq(x, p * q) [= A \cap B$.
        $p*q$ is a nonzero integer. Let $a << ArSeq(x, p * q)$.

        $a << ArSeq (x, p)$ and $a << ArSeq (x, q)$.
        proof.
          $x$ is an integer and $a = x (mod p * q)$. $a = x (mod p)$ and $a = x (mod q)$ (by EquModMul).
        end.

        Therefore $a << A$ and $a << B$. Hence $a << A \cap B$.
      end.
    \end{proof}

    \begin{lemma}[UnionClosed]
      Let $A,B$ be closed subsets of $INT$. $A \cup B$ is closed.
    \end{lemma}
    \begin{proof}
      We have $~A,~B [= INT$. $~(A \cup B) = ~A \cap ~B$.
    \end{proof}

    \begin{axiom}[UnionSClosed]
      Let $S$ be a finite family of integer sets such that all elements of $S$ are closed subsets of $INT$. $\cup S$ is closed.
    \end{axiom}

    \begin{lemma}[ArSeqClosed]
      $ArSeq(a,q)$ is a closed subset of $INT$.
    \end{lemma}
    \begin{proof}
      Proof by contradiction. $ArSeq(a,q)$ is a subset of $INT$. Let $b << ~ArSeq (a,q)$.

      Let us show that $ArSeq(b,q) [= ~ArSeq(a,q)$. Let $c << ArSeq(b,q)$.
        Assume not $c << ~ArSeq(a,q)$. Then $c = b (mod q)$ and $a = c (mod q)$. Hence $b = a (mod q)$. Therefore $b << ArSeq(a,q)$. Contradiction.
      end.
    \end{proof}

    \begin{theorem}[Fuerstenberg]
      Let $S = {ArSeq(0,r) | r "is a prime"}$. $S$ is infinite.
    \end{theorem}
    \begin{proof}
      Proof by contradiction. $S$ is a family of integer sets.

      We have $~ \cup S = {1, -1}$.
      proof.
        Let us show that for any integer $n$ $n$ belongs to $\cup S$ iff $n$ has a prime divisor.
          Let $n$ be an integer.

          If $n$ has a prime divisor then $n$ belongs to $\cup S$.
          proof.
            Assume $n$ has a prime divisor. Take a prime divisor $p$ of $n$. $ArSeq(0,p)$ is setsized. $ArSeq(0,p) << S$. $n << ArSeq(0,p)$.
          end.

          If $n$ belongs to $\cup S$ then $n$ has a prime divisor.
          proof.
            Assume $n$ belongs to $\cup S$. Take a prime $r$ such that $n << ArSeq(0,r)$. Then $r$ is a prime divisor of $n$.
          end.
        end.
      end.

      Assume that $S$ is finite. Then $\cup S$ is closed and $~ \cup S$ is open.

      Take $p$ such that $ArSeq(1,p) [= ~ \cup S$.

      $ArSeq(1,p)$ has an element $x$ such that neither $x = 1$ nor $x = -1$.
      proof.
        $1 + p$ and $1 - p$ are integers.
        $1 + p$ and $1 - p$ belong to $ArSeq(1,p)$. $1 + p !=  1 /\ 1 - p !=  1$. $1 + p != -1 \/ 1 - p != -1$.
        Indeed $1 + p = 1 (mod p)$ and $1 - p = 1 (mod p)$.
      end.

      We have a contradiction.
    \end{proof}
  \end{forthel}

\end{document}
