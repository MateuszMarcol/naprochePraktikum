\documentclass{article}

\usepackage[utf8]{inputenc}
\usepackage[british]{babel}
\usepackage[type={CC},modifier={zero},version={1.0},imagemodifier=-80x15]{doclicense}
\usepackage{amsfonts}

\usepackage{../lib/tex/naproche}

\renewcommand{\mod}{\text{mod }}
\newcommand{\Int}{\mathbb{Z}}

\title{Fürstenberg's proof of the infinitude of primes}
\author{Andrei Paskevich et. al.}
\date{??? - 2021}


\begin{document}
  \maketitle

  Fürstenberg's proof of the infinitude of primes is a topological proof of the fact that there are infinitely many primes. It was published 1955 while Fürstenberg was still an undergraduate student.

  \section{Integers}

  First we introduce the integers $\Int$ as an integral domain. We take two constants, $0$ and $1$, a unary operation $-$ and two binary operations $+$ and $\cdot$.

  \begin{forthel}
    [unfoldlow on]
    [synonym integer/-s]

    \begin{signature}[Integers]
      An integer is a notion.
    \end{signature}

    Let $a,b,c,d,i,j,k,l,m,n$ stand for integers.

    \begin{axiom}
      $a$ is setsized.
    \end{axiom}

    \begin{signature}[IntZero]
      $0$ is an integer.
    \end{signature}

    \begin{signature}[IntOne]
      $1$ is an integer.
    \end{signature}

    \begin{signature}[IntNeg]
      $-a$ is an integer.
    \end{signature}

    \begin{signature}[IntPlus]
      $a + b$ is an integer.
    \end{signature}

    \begin{signature}[IntMult]
      $a \cdot b$ is an integer.
    \end{signature}

    Let $a - b$ stand for $a + (-b)$.
  \end{forthel}

  Moreover, we assume $(\Int, 0, +, -)$ to be a commutative group and $(\Int, 1, \cdot)$ to be a commutative monoid which satisfy the distribution laws.

  \begin{forthel}
    \begin{axiom}[AddAsso]
      $a + (b + c) = (a + b) + c$.
    \end{axiom}

    \begin{axiom}[AddComm]
      $a + b = b + a$.
    \end{axiom}

    \begin{axiom}[AddZero]
      $a + 0 = a = 0 + a$.
    \end{axiom}

    \begin{axiom}[AddNeg]
      $a - a = 0 = -a + a$.
    \end{axiom}

    \begin{axiom}[MulAsso]
      $a \cdot (b \cdot c) = (a \cdot b) \cdot c$.
    \end{axiom}

    \begin{axiom}[MulComm]
      $a \cdot b = b \cdot a$.
    \end{axiom}

    \begin{axiom}[MulOne]
      $a \cdot 1 = a = 1 \cdot a$.
    \end{axiom}

    \begin{axiom}[Distrib]
      $a \cdot (b + c) = (a \cdot b) + (a \cdot c)$ and$ (a + b) \cdot c = (a \cdot c) + (b \cdot c)$.
    \end{axiom}

    \begin{lemma}[MulZero]
      $a \cdot 0 = 0 = 0 \cdot a$.
    \end{lemma}

    \begin{lemma}[MulMinOne]
      $-1 \cdot a = -a = a \cdot -1$.
    \end{lemma}
    \begin{proof}
      $(-1 \cdot a) + a = 0$.
    \end{proof}
  \end{forthel}

  Furthermore we assume that there are no non-trivial zero divisors.

  \begin{forthel}
    \begin{axiom}[ZeroDiv]
      $a \neq 0 \wedge b \neq 0 \implies a \cdot b \neq 0$.
    \end{axiom}

    \begin{lemma}
      $-(-a)$ is an integer.
    \end{lemma}
  \end{forthel}

Note that we did not state that the ring of integers is non-trivial, yet. Let us continue with the notion of divisors and congruency.

  \begin{forthel}
    Let $a$ is nonzero stand for $a \neq 0$.
    Let $p,q$ stand for nonzero integers.

    [synonym divisor/-s] [synonym divide/-s]

    \begin{definition}[Divisor]
      A divisor of $b$ is a nonzero integer $a$ such that for some $n$ $(a \cdot n = b)$.
    \end{definition}

    Let $a$ divides $b$ stand for $a$ is a divisor of $b$.
    Let $a \mid b$ stand for $a$ is a divisor of $b$.

    \begin{definition}[EquMod]
      $a = b ~(\mod q)$ iff $q \mid a-b$.
    \end{definition}

    \begin{lemma}[EquModRef]
      $a = a ~(\mod q)$.
    \end{lemma}

    \begin{lemma}[EquModSym]
      $a = b ~(\mod q) \implies b = a ~(\mod q)$.
    \end{lemma}
    \begin{proof}
      Assume that $a = b ~(\mod q)$.

      (1) Take $n$ such that $q \cdot n = a - b$.

      $q \cdot -n .= (-1) \cdot (q \cdot n)$ (by MulMinOne, MulAsso,MulComm) $.= (-1) \cdot (a - b)$ (by 1).
    \end{proof}

    \begin{lemma}[EquModTrn]
      $a = b ~(\mod q) \wedge b = c ~(\mod q) \implies a = c ~(\mod q)$.
    \end{lemma}
    \begin{proof}
      Assume that $a = b ~(\mod q) \wedge b = c ~(\mod q)$. Take $n$ such that $q \cdot n = a - b$. Take $m$ such that $q \cdot m = b - c$. We have $q \cdot (n + m) = a - c$.
    \end{proof}

    \begin{lemma}[EquModMul]
      $a = b ~(\mod p \cdot q) \implies a = b ~(\mod p) \wedge a = b ~(\mod q)$.
    \end{lemma}
    \begin{proof}
      Assume that $a = b ~(\mod p \cdot q)$. Take $m$ such that $(p \cdot q) \cdot m = a - b$. We have $p \cdot (q \cdot m) = a - b = q \cdot (p \cdot m)$.
    \end{proof}
  \end{forthel}

  Now we introduce prime integers. All we need to know about them is that every integer $n$ has a prime divisor iff $n \neq 1$ and $n \neq -1$.

  \begin{forthel}
    \begin{signature}[Prime]
      $a$ is prime is an atom.
    \end{signature}

    Let a prime stand for a prime nonzero integer.

    \begin{axiom}[PrimeDivisor]
      $n$ has a prime divisor iff $n \neq 1 \wedge n \neq -1$.
    \end{axiom}
  \end{forthel}


  \section{Generic sets}

  Another important notion is that of finite subsets of $\Int$. We leave the characterization of what it means for a set to be finite for the next sections.

  \begin{forthel}
    [synonym belong/-s] [synonym subset/-s]
    [read ZFC.ftl]

    Let $S,T$ stand for sets.

    Let $x$ belongs to $S$ stand for $x$ is an element of $S$.

    \begin{definition}[Subset]
      A subset of $S$ is a set $T$ such that every element of $T$ belongs to $S$.
    \end{definition}

    Let $S \subseteq T$ stand for $S$ is a subset of $T$.

    \begin{signature}[FinSet]
      $S$ is finite is an atom.
    \end{signature}

    Let $x$ is infinite stand for $x$ is not finite.
  \end{forthel}


  \section{Sets of integers}

  Since Fürstenberg's proof is a topological proof we have to define unions, intersections and also complements of sets.

  \begin{forthel}
    \begin{definition}
      $\Int$ is the class of integers.
    \end{definition}

    \begin{axiom}
      $\Int$ is a set.
    \end{axiom}

    Let $A,B,C,D$ stand for subsets of $\Int$.

    \begin{definition}[Union]
      $A \cup B = \{ "integer" x \mid x \in A \vee x \in B \}$.
    \end{definition}

    \begin{definition}[Intersection]
      $A \cap B = \{ "integer" x \mid x \in A \wedge x \in B \}$.
    \end{definition}

    \begin{definition}[IntegerSets]
      A family of integer sets is a set $S$ such that every element of $S$ is a subset of $\Int$.
    \end{definition}

    \begin{definition}[UnionSet]
      Let $S$ be a family of integer sets.
      $\bigcup S = \{ "integer" x \mid x "belongs to some element of" S \}$.
    \end{definition}

    \begin{lemma}
      Let $S$ be a family of integer sets. $\bigcup S$ is a subset of $\Int$.
    \end{lemma}

    \begin{definition}[Complement]
      $\overline{A} = \{ "integer" x \mid x "does not belong to" A \}$.
    \end{definition}

    \begin{lemma}
      $\overline{A}$ is a subset of $\Int$.
    \end{lemma}
   \end{forthel}


  \section{Introducing topology}

  \begin{forthel}
    \begin{definition}[ArSeq]
      $q \Int + a = \{ "integer" b \mid b = a ~(\mod q) \}$.
    \end{definition}

    \begin{lemma}
      $q \Int + a$ is a set.
    \end{lemma}
  \end{forthel}

  This allows us to define the so-called \textit{evenly spaced integer topology} where its open sets are defined as follows:

  \begin{forthel}
    \begin{definition}[Open]
      $A$ is open iff for any $a \in A$ there exists $q$ such that $q \Int + a \subseteq A$.
    \end{definition}
  \end{forthel}

  Note that we have declared $q$ as a \textit{non-zero} integer. Otherwise every set $A$ would be open because then $0 \Int + a = \emptyset \subseteq A$ for every $a \in A$.

  \begin{forthel}
    \begin{definition}[Closed]
      $A$ is closed iff $\overline{A}$ is open.
    \end{definition}

    \begin{definition}[OpenIntegerSets]
      An open family is a family of integer sets $S$ such that every element of $S$ is open.
    \end{definition}
  \end{forthel}

  We can easisly check that the open sets really form a topology on $\Int$.

  \begin{forthel}
    \begin{lemma}[UnionOpen]
      Let $S$ be an open family. $\bigcup S$ is open.
    \end{lemma}
    \begin{proof}
      Let $x \in \bigcup S$. Take a set $M$ such that ($M$ is an element of $S$ and $x \in M$). Take $q$ such that $q \Int + x \subseteq M$. Then $q \Int + x \subseteq \bigcup S$.
    \end{proof}

    \begin{lemma}[InterOpen]
      Let $A,B$ be open subsets of $\Int$. Then $A \cap B$ is a subset of $\Int$ and $A \cap B$ is open.
    \end{lemma}
    \begin{proof}
      $A \cap B$ is a subset of $\Int$. Let $x \in A \cap B$. Then $x$ is an integer. Take $q$ such that $q \Int + x \subseteq A$. Take $p$ such that $p \Int + x \subseteq B$.

      Let us show that $p \cdot q$ is a nonzero integer and $(p \cdot q) \Int + x \subseteq A \cap B$.
        $p \cdot q$ is a nonzero integer. Let $a \in (p \cdot q) \Int + x$.

        $a \in p \Int + x$ and $a \in q \Int + x$. \\
        Proof.
          $x$ is an integer and $a = x ~(\mod p \cdot q)$. $a = x ~(\mod p)$ and $a = x ~(\mod q)$ (by EquModMul).
        Qed.

        Therefore $a \in A$ and $a \in B$. Hence $a \in A \cap B$.
      End.
    \end{proof}

    \begin{lemma}[UnionClosed]
      Let $A,B$ be closed subsets of $\Int$. $A \cup B$ is closed.
    \end{lemma}
    \begin{proof}
      We have $\overline{A}, \overline{B} \subseteq \Int$. $\overline{A \cup B} = \overline{A} \cap \overline{B}$.
    \end{proof}
  \end{forthel}

  Now we eventually give a condition for a set to be finite, namely that unions of finite families of closed sets are closed.

  \begin{forthel}
    \begin{axiom}[UnionSClosed]
      Let $S$ be a finite family of integer sets such that all elements of $S$ are closed subsets of $\Int$. $\bigcup S$ is closed.
    \end{axiom}
  \end{forthel}

  This characterization allows us to prove that a family $S$ of closed sets is infinite by exploiting that the assumption that $S$ is finite yields that $\bigcup S$ is closed which eventually ends up in a contradiction. In Fürstenberg's proof we will use this method to show that the family $\{ r \Int \mid r \text{ is prime} \}$ is infinite. To use the above argument we thus have to prove that any $q \Int + a$ is closed.

  \begin{forthel}
    \begin{lemma}[ArSeqClosed]
      $q \Int + a$ is a closed subset of $\Int$.
    \end{lemma}
    \begin{proof}
      Proof by contradiction. $q \Int + a$ is a subset of $\Int$. Let $b \in \overline{q \Int + a}$.

      Let us show that $q \Int + b \subseteq \overline{q \Int + a}$. Let $c \in q \Int + b$.
        Assume not $c \in \overline{q \Int + a}$. Then $c = b ~(\mod q)$ and $a = c ~(\mod q)$. Hence $b = a ~(\mod q)$. Therefore $b \in q \Int + a$. Contradiction.
      End.
    \end{proof}
  \end{forthel}

  To prove that there are infinitely many primes we identify a prime number $r$ with the set $r \Int$ and show that the set $S = \{r \Int \mid r \textrm{ is a prime} \}$ is infinite. It is easy to see that $\bigcup S = \{ \text{integer } n \mid n \text{ has a prime divisor} \} = \Int \setminus \{ 1, -1 \}$. So if $S$ is finite then $\bigcup S$ is closed (by \textit{UnionClosed}) and hence $\{ 1, -1 \}$ is open. But then some arithmetic sequence $p \Int + 1$ is contained in $\{ 1, -1 \}$ which obviously cannot be the case.

  \begin{forthel}
    \begin{theorem}[Fuerstenberg]
      Let $S = \{ r \Int + 0 \mid r "is a prime" \}$. $S$ is infinite.
    \end{theorem}
    \begin{proof}
      Proof by contradiction. $S$ is a family of integer sets.

      We have $\overline{\bigcup S} = \{ 1, -1 \}$. \\
      Proof.
        Let us show that for any integer $n$ $n$ belongs to $\bigcup S$ iff $n$ has a prime divisor.
          Let $n$ be an integer.

          If $n$ has a prime divisor then $n$ belongs to $\bigcup S$. \\
          Proof.
            Assume $n$ has a prime divisor. Take a prime divisor $p$ of $n$. $p \Int + 0$ is setsized. $p \Int + 0 \in S$. $n \in p \Int + 0$.
          Qed.

          If $n$ belongs to $\bigcup S$ then $n$ has a prime divisor. \\
          Proof.
            Assume $n$ belongs to $\bigcup S$. Take a prime $r$ such that $n \in r \Int + 0$. Then $r$ is a prime divisor of $n$.
          Qed.
        End.
      Qed.

      Assume that $S$ is finite. Then $\bigcup S$ is closed and $\overline{\bigcup S}$ is open.

      Take $p$ such that $p \Int + 1 \subseteq \overline{\bigcup S}$.

      $p \Int + 1$ has an element $x$ such that neither $x = 1$ nor $x = -1$. \\
      Proof.
        $1 + p$ and $1 - p$ are integers.
        $1 + p$ and $1 - p$ belong to $p \Int + 1$. $1 + p \neq  1 \wedge 1 - p \neq  1$. $1 + p \neq -1 \vee 1 - p \neq -1$. Indeed $1 + p = 1 ~(\mod p)$ and $1 - p = 1 ~(\mod p)$.
      Qed.

      We have a contradiction.
    \end{proof}
  \end{forthel}

  Note that we cannot define $q \Int$ as $q \Int + 0$ in our formalization since then any term of the form $q \Int + a$ would be ambigue: It could either be interpreted as $q \Int + a$ or as $(q \Int + 0) + a$. This is a result of some kind of overloading of the symbol $+$. We use $+$ on the one hand to denote integer addition and on the other hand it is part of the operator $- \Int + -$.

  \vfill\doclicenseThis
\end{document}
